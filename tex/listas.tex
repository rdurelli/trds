\section{Abreviaturas e Siglas}

A classe \textit{icmc} implementa a criação da lista de abreviaturas e siglas com o pacote \textit{nomencl}. A inserção de abreviaturas e siglas na lista é realizada com o comando \comando{sigla\{A\}\{B\}} que também insere o conteúdo da sigla no local do documento onde a mesma foi definida. Os parâmetros utilizados são: \textit{A} que é a sigla e \textit{B} que é o nome por extenso. Caso deseja-se inserir a sigla apenas na lista, pode-se utilizar o comando \comando{sigla*\{A\}\{B\}}.

Para se gerar a lista de siglas na parte pre-textual do documento é preciso incluir o comando \comando{incluidelistasiglas} antes do início do documento. Além disto, a compilação do documento deve conter o comando \textit{makeindex} após duas compilações com o \textit{pdflatex}. Por exemplo, supondo que o documento principal tenha o nome de \textit{thesis}, podemos usar a seguinte sequência de comandos:

\begin{verbatim}
pdflatex thesis.tex
pdflatex thesis.tex
makeindex thesis.nlo -s nomencl.ist -o thesis.nls
pdflatex thesis.tex
\end{verbatim}

No \autoref{chapter:ferramentas-uteis} serão apresentadas algumas ferramentas que podem facilitar o processo de compilação do documento. Em especial, o ShareLaTeX não necessita de um processo de compilação especial para gerar a lista de abreviaturas e siglas.


\section{Símbolos}

A definição de símbolos é semelhante a definição de siglas, porém deve ser usado o comando \comando{simbolo\{S\}\{DS\}}, onde \textit{S} é o símbolo e \textit{DS} é a descrição do símbolo. Como exemplo definimos os símbolos $\mathbb{X}$\simbolo{\mathbb{X}}{Variável X} e $\mathsf{I\!R}$\simbolo{\mathsf{I\!R}}{Conjunto dos números reais}. Para incluir a lista de símbolos, basta usar o comando \comando{incluidelistasimbolos} antes do início do documento.
