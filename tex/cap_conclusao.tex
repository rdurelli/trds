\section{Considerações Finais do Projeto de Doutorado}

Neste projeto de doutorado o objetivo foi buscar soluções que facilitam a aplicação e o reuso de refatorações para o metamodelo KDM. Para isso foi definido uma abordagem para a criação e disponibilização de refatorações para o KDM, bem como um apoio ferramental que permite aplicá-las em diagramas de classe da UML. A abordagem possui dois principais passos: (\textit{i}) o primeiro envolve diretrizes que apoiam o engenheiro de modernização durante a implementação de refatorações para o KDM; e (\textit{ii}) o segundo consiste na especificação das refatorações por meio da criação de instâncias do metamodelo SRM e posterior disponibilização delas em um repositório.

O apoio ferramental, denominado KDM-RE, é composto por três plug-ins do Eclipse: (\textit{i}) o primeiro consiste em um conjunto de \textit{Wizards} que apoia o engenheiro de software na aplicação das refatorações em diagramas de classe UML; (\textit{ii}) o segundo consiste em um módulo de propagação de mudanças que permite manter modelos internos do KDM sincronizados e; (\textit{iii}) o terceiro consiste em um apoio à importação e reuso de refatorações disponíveis no repositório.

Como parte da pesquisa, dois experimentos foram conduzidos com o objetivo de testar e avaliar as vantagens de utilizar a abordagem e a ferramenta KDM-RE. Os resultados mostram que a abordagem pode trazer benefícios para o engenheiro de software durante a atividade de aplicação de refatorações em sistemas representadas pelo metamodelo KDM.

Nas demais seções deste capítulo são apresentados os seguintes tópicos: na Seção X, trabalhos relacionados são apresentados; na Seção X. as contribuições; na Seção Y, as limitações; na Seção U, os trabalhos futuros; e na seção Y, os artigos publicados durante o período de doutorado \change{mudar as seções}. 

\subsection{Publicações Resultantes}
Durante este projeto de doutorado foi possível publicar um conjunto de artigos científicos. Tais artigos são resultados de trabalhos totalmente relacionado a este projeto de doutorado, bem como publicações realizadas em parcerias com outros pesquisadores. Acredita-se que estas também são importantes como parte da experiência de uma pesquisa em nível de doutorado.

\begin{itemize}
	
	\item \textbf{2 trabalhos completos publicado como capítulo de livro}
		\begin{enumerate}
			
			\item Viana, Matheus ; Penteado, Rosângela ; Prado, Antônio do ; \textbf{Durelli, Rafael} . \aspas{\textit{Developing Frameworks from Extended Feature Models}}. Advances in Intelligent Systems and Computing. 1ed.: Springer International Publishing, 2014, v. 263, p. 263-284.
			
			\begin{itemize}
			\item Nível de contribuição: Médio - Auxiliou-se na escrita e na estrutura do artigo;
			\end{itemize}
			
			\item Júnior, Paulo Afonso Parreira ; Penteado, Rosângela Dellosso ; Viana, Matheus Carvalho ; \textbf{Durelli, Rafael Serapilha} ; DE CAMARGO, VALTER VIEIRA ; Costa, Heitor Augustus Xavier . \aspas{\textit{Reengineering of Object-Oriented Software into Aspect-Oriented Ones Supported by Class Models}}. Lecture Notes in Business Information Processing. 1ed.: Springer International Publishing, 2014, v. 190, p. 296-313.
			    \begin{itemize}
			    \item Nível de contribuição: Médio - Auxiliou-se na escrita e na estrutura do artigo;
			    \end{itemize}
			
		\end{enumerate}
	
	\item \textbf{2 trabalhos completos publicados em revistas}
		\begin{enumerate}
			\item GOTTARDI, THIAGO ; \textbf{DURELLI, RAFAEL} ; LÓPEZ, ÓSCAR ; DE CAMARGO, VALTER . \aspas{\textit{Model-based reuse for crosscutting frameworks: assessing reuse and maintenance effort}}. Journal of Software Engineering Research and Development, v. 1, p. 4, 2013.
			    \begin{itemize}
			        \item Nível de contribuição: Alto - Auxiliou-se na elaboração do apoio ferramental, bem como na escrita e na estrutura do artigo;
			    \end{itemize}
			    
			\item SANTIBÁÑEZ, DANIEL S. M. ; \textbf{DURELLI, RAFAEL} ; DE CAMARGO, VALTER . \aspas{\textit{A Combined Approach for Concern Identification in KDM models}}. Journal of the Brazilian Computer Society, v. 1, p. 4, 2015.
			  \begin{itemize}
			        \item Nível de contribuição: Alto - Auxiliou-se na elaboração do apoio ferramental, bem como na escrita e na estrutura do artigo;
			    \end{itemize}
		\end{enumerate}
	\item \textbf{18 trabalhos completos publicados em anais de congressos/\textit{workshops}}
	\begin{enumerate}
	    
	    \item \textbf{DURELLI, R. S.}; DURELLI, V. H. S. . \aspas{\textit{A Systematic Mapping Study on Formal Methods Applied to Crosscutting Concerns Mining}}. In: IX Experimental Software Engineering Latin American Workshop (ESELAW), 2012, Buenos Aires. IX Experimental Software Engineering Latin American Workshop (ESELAW), 2012.
	            \begin{itemize}
			        \item Nível de contribuição: Alto - O doutorando é o principal investigador e conduziu a elaboração do artigo com ajuda de colaboradores;
			    \end{itemize}
	 	
	 	\item \textbf{DURELLI, R. S.}; DURELLI, V. H. S. . \aspas{\textit{F2MoC: A Preliminary Product Line DSL for Mobile Robots}}. In: Simpósio Brasileiro de Sistemas de Informação (SBSI), 2012, São Paulo. Simpósio Brasileiro de Sistemas de Informação (SBSI), 2012.
	 	        \begin{itemize}
			        \item Nível de contribuição: Alto - O doutorando é o principal investigador e conduziu a elaboração do artigo com ajuda de colaboradores;
			    \end{itemize}
	 	
	 	
	 	\item Gottardi; \textbf{DURELLI, R. S.} ; PASTOR, O. L. ; CAMARGO, V. V. . Model-Based Reuse for Crosscutting Frameworks: Assessing Reuse and Maintainability Effort. In: Simpósio Brasileiro de Engenharia de Software, 2012, Natal. Simpósio Brasileiro de Engenharia de Software, 2012.
	 	     \begin{itemize}
			        \item Nível de contribuição: Alto - Auxiliou-se na elaboração do apoio ferramental, bem como na escrita e na estrutura do artigo;
			    \end{itemize}
	 	
	 	\item \textbf{DURELLI, R. S.} ; Gottardi ; CAMARGO, V. V. . CrossFIRE: An Infrastructure for Storing Crosscutting Framework Families and Supporting their Model-Based Reuse. In: XXVI Simpósio Brasileiro de Engenharia de Software - XXVI Sessão de Ferramenta, 2012, Natal. Simpósio Brasileiro de Engenharia de Software, 2012. v. 6. p. 1-6.
	 	            
	 	            \begin{itemize}
			        \item Nível de contribuição: Alto - O doutorando é o principal investigador e conduziu a elaboração do artigo com ajuda de colaboradores;
			    \end{itemize}
	 	
	 	\item \textbf{DURELLI, R. S.}; SANTIBANEZ, D. S. M. ; ANQUETIL, N. ; DELAMARO, M. E. ; CAMARGO, V. V. . A Systematic Review on Mining Techniques for Crosscutting Concerns. In: ACM SAC 2013, 2012, Coimbra. ACM SAC Software Engineering (SE) Track, 2013. v. 28th.
	 	
	 	    \begin{itemize}
			        \item Nível de contribuição: Alto - O doutorando é o principal investigador e conduziu a elaboração do artigo com ajuda de colaboradores;
			    \end{itemize}
	
	 	\item PARREIRA JUNIOR, P. A.; VIANA, M. C. ; \textbf{DURELLI, R. S.} ; CAMARGO, V. V. ; COSTA, H. A. X. ; PENTEADO, R. A. D. . Concern-Based Refactorings Supported by Class Models to Reengineer Object-Oriented Software into Aspect-Oriented Ones. In: International Conference on Enterprise Information Systems (ICEIS), 2013, ANGERS/FR. XV International Conference on Enterprise Information Systems, 2013.
	 	
	 	    \begin{itemize}
			    \item Nível de contribuição: Médio - Auxiliou-se na escrita e na estrutura do artigo;
			    \end{itemize}
		
		\item VIANA, M. C. ; \textbf{DURELLI, R. S.} ; PENTEADO, R. A. D. ; PRADO, A. F. . F3: From features to frameworks.. In: International Conference on Enterprise Information Systems (ICEIS), 2013, ANGERS/FR. XV International Conference on Enterprise Information Systems, 2013.
		    
		    \begin{itemize}
			    \item Nível de contribuição: Médio - Auxiliou-se na escrita e na estrutura do artigo;
			    \end{itemize}
		
		\item VIANA, M. C. ; PENTEADO, R. A. D. ; PRADO, A. F. ; \textbf{DURELLI, R. S}. . An Approach to Develop Frameworks from Feature Models. In: International Conference on Information Reuse and Integration, 2013, San Francisco. An Approach to Develop Frameworks from Feature Models, 2013.
		        \begin{itemize}
			    \item Nível de contribuição: Médio - Auxiliou-se na escrita e na estrutura do artigo;
			    \end{itemize}
		
		\item VIANA, M. C. ; PENTEADO, R. A. D. ; PRADO, A. F. ; \textbf{DURELLI, RAFAEL S}. . F3T: From Features to Frameworks Tool. In: XXVII Simpósio Brasileiro de Engenharia de Software (SBES 2013), 2013, Brasília. F3T: From Features to Frameworks Tool, 2013.
		    \begin{itemize}
			    \item Nível de contribuição: Médio - Auxiliou-se na escrita e na estrutura do artigo;
			    \end{itemize}
		
		\item SANTIBÁÑEZ, DANIEL S. M. ; \textbf{DURELLI, RAFAEL S.} ; CAMARGO, V. V. . CCKDM - A Concern Mining Tool for Assisting in the Architecture-Driven Modernization Process. In: XXVII Simpósio Brasileiro de Engenharia de Software - XXVII Sessão de Ferramenta, 2013, Brasília. Simpósio Brasileiro de Engenharia de Software, 2013.
		    \begin{itemize}
			        \item Nível de contribuição: Alto - Auxiliou-se na elaboração do apoio ferramental, bem como na escrita e na estrutura do artigo;
			    \end{itemize}
		
		\item SANTIBANEZ, D. S. M. ; \textbf{DURELLI, RAFAEL S.} ; CAMARGO, V. V. . A Combined Approach for Concern Identification in KDM models. In: Latin American Workshop on Aspect-Oriented Software Development (LA-WASP), 2013, Brasília. Congresso Brasileiro de Software: Teoria e Prática (CBSoft), 2013.
		
		    \begin{itemize}
			        \item Nível de contribuição: Alto - Auxiliou-se na elaboração do apoio ferramental, bem como na escrita e na estrutura do artigo;
			    \end{itemize}
		
		
		\item PINTO, Victor Hugo S. C. ; \textbf{DURELLI, R. S.} ; OLIVEIRA, A. L. ; CAMARGO, V. V. . Evaluating the Effort for Modularizing Multiple-Domain Frameworks towards Framework Product Lines with Aspect-Oriented Programming and Model-Driven Development. In: International Conference on Enterprise Information Systems (ICEIS), 2014, Lisboa. International Conference on Enterprise Information Systems (ICEIS), 2014.
		
		    \begin{itemize}
			        \item Nível de contribuição: Baixo - Auxiliou-se na elaboração do experimento;
			    \end{itemize}
		
		\item DIAS, D. R. C. ; \textbf{DURELLI, R. S.} ; BREGA, J. R. F. ; GNECCO, B. B ; TREVELIN, L. C. ; GUIMARAES, M. P. . Data Network in Development of 3D Collaborative Virtual Environments: A Systematic Review. In: The 14th International Conference on Computational Science and Applications (ICCSA 2014), 2014, Guimarães. The 14th International Conference on Computational Science and Applications (ICCSA 2014), 2014.
		
		    \begin{itemize}
			        \item Nível de contribuição: Baixo - Auxiliou-se na elaboração do protocolo da revisão sistemática;
			    \end{itemize}
		
		\item \textbf{DURELLI, R. S.} ; SANTIBANEZ, D. S. M. ; DELAMARO, MÁRCIO E. ; CAMARGO, V. V. . Towards a Refactoring Catalogue for Knowledge Discovery Metamodel. In: IEEE International Conference on Information Reuse and Integration, 2014, San Francisco. IEEE International Conference on Information Reuse and Integration, 2014. p. 1-8.
		
		    \begin{itemize}
			        \item Nível de contribuição: Alto - O doutorando é o principal investigador e conduziu a elaboração do artigo com ajuda de colaboradores;
			    \end{itemize}
		
		\item \textbf{DURELLI, R. S.} ; SANTIBANEZ, D. S. M. ; MARINHO, B. S. ; HONDA, R. R. ; DELAMARO, M. E. ; ANQUETIL, N. ; CAMARGO, V. V. . A Mapping Study on Architecture-Driven Modernization. In: IEEE International Conference on Information Reuse and Integration, 2014, San Francisco. IEEE International Conference on Information Reuse and Integration, 2014. p. 1-8.
		    \begin{itemize}
			        \item Nível de contribuição: Alto - O doutorando é o principal investigador e conduziu a elaboração do artigo com ajuda de colaboradores;
			    \end{itemize}
		
		\item MARINHO, B. S. ; CAMARGO, V. V. ; HONDA, R. R. ; \textbf{DURELLI, R. S.} . KDM-AO: An Aspect-Oriented Extension of the Knowledge Discovery Metamodel. In: 28th Brazilian Symposium on Software Engineering (SBES), 2014, Maceió. 28th Brazilian Symposium on Software Engineering (SBES), 2014. p. 1-10.
		
		    \begin{itemize}
			        \item Nível de contribuição: Alto - Auxiliou-se na elaboração da extensão, bem como na escrita e na estrutura do artigo;
			    \end{itemize}
		
		\item MARINHO, B. S. ; \textbf{DURELLI, RAFAEL S.} ; HONDA, R. R. ; CAMARGO, V. V. . Investigating Lightweight and Heavyweight KDM Extensions for Aspect-Oriented Modernization. In: 11th Workshop on Software Modularity (WMod) -- Brazilian Conference on Software: theory and practice, 2014, maceio. 11th Workshop on Software Modularity (WMod) -- Brazilian Conference on Software: theory and practice, 2014.
		
		\begin{itemize}
			        \item Nível de contribuição: Alto - Auxiliou-se na elaboração da extensão, bem como na escrita e na estrutura do artigo;
			    \end{itemize}
		
		\item \textbf{DURELLI, RAFAEL S.} ; MARINHO, B. S. ; HONDA, R. R. ; DELAMARO, MÁRCIO E. ; CAMARGO, V. V. . KDM-RE: A Model-Driven Refactoring Tool for KDM.. In: II Workshop on Software Visualization, Evolution and Maintenance -- Brazilian Conference on Software: theory and practice, 2014, Maceio. II Workshop on Software Visualization, Evolution and Maintenance -- Brazilian Conference on Software: theory and practice, 2014. p. 1-8.
		
		    \begin{itemize}
			        \item Nível de contribuição: Alto - O doutorando é o principal investigador e conduziu a elaboração do artigo com ajuda de colaboradores;
			    \end{itemize}
	

\end{enumerate}
\end{itemize}