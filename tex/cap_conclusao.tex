\section{Considerações Finais do Projeto de Doutorado}
Neste projeto de doutorado, almejou-se buscar soluções para facilitar a aplicação e o reúso de refatorações para o metamodelo KDM. Para isso, foi definida uma abordagem com o foco na criação e disponibilização de refatorações para o KDM, bem como um apoio ferramental que permite aplicá-las em diagramas de classe da UML. A abordagem possui dois principais passos: (\textit{i}) o primeiro envolve diretrizes que apoiam o engenheiro de modernização durante a implementação de refatorações para o KDM; (\textit{ii}) o segundo consiste na especificação das refatorações por meio da criação de instâncias do metamodelo SRM e posterior disponibilização delas em um repositório.

Além disso, um apoio computacional, denominado KDM-RE, também foi desenvolvido para auxiliar o engenheiro de software durante o processo de modernização. Esse apoio computacional é um \textit{plug-in} do Ambiente de Desenvolvimento Eclipse, o qual foi dividido em três principais módulos: (\textit{i}) o primeiro engloba um conjunto de \textit{Wizards} que apoia o engenheiro de software na aplicação das refatorações em diagramas de classe UML; (\textit{ii}) o segundo consiste em um apoio à importação e reúso de refatorações disponíveis no repositório; (\textit{iii}) o terceiro compreende um módulo de propagação de mudanças que permite manter modelos internos do KDM sincronizados.

Como parte da pesquisa, um experimento foi conduzido com o objetivo de testar e avaliar as vantagens de utilizar a abordagem aqui apresentada. Os resultados mostram que a abordagem pode trazer benefícios para o engenheiro de software durante a atividade de aplicação de refatorações em sistemas, representados pelo metamodelo KDM.

%Como parte da pesquisa, dois experimentos\unsure{mudar} foram conduzidos com o objetivo de testar e avaliar as vantagens de utilizar a abordagem e a ferramenta KDM-RE. Os resultados mostram que a abordagem pode trazer benefícios para o engenheiro de software durante a atividade de aplicação de refatorações em sistemas, representados pelo metamodelo KDM.

Nas demais seções deste capítulo, são apresentados os seguintes tópicos: na Seção~\ref{sec:contribuicoes_desta_tese}, as contribuições desta tese são descritas; na Seção~\ref{sec:limitacoes_trabalho}, as limitações são destacadas; na Seção~\ref{sec:trabalhos_futuros_tese}, os possíveis trabalhos futuros são salientados.%; e na Seção~\ref{sec:publicacoes_resultantes}, as publicações resultantes durante o período de doutorado são destacados\change{mudar as seções}. 

\section{Contribuições desta Tese}\label{sec:contribuicoes_desta_tese}

A principal contribuição desta tese é fornecer para a ADM soluções para a criação, aplicação e reúso de refatorações para o metamodelo KDM. As contribuições mais específicas são os passos propostos para criar refatorações e restrições para o metamodelo KDM, o metamodelo SRM que fornece uma terminologia comum, padronizada e independente de linguagem para a especificação de refatorações e o apoio computacional que permite aplicar refatorações em diagramas UML que são representações gráficas de modelos KDM. 

A abordagem para criar refatorações torna a criação de refatorações para o metamodelo KDM um processo sistemático e guiado, facilitando a tarefa do engenheiro de modernização e procurando garantir que as refatorações desenvolvidas estejam estruturadas corretamente com base nas metaclasses do KDM. Acredita-se que a abordagem apresentada tem potencial para conduzir engenheiros de modernização durante a criação de refatorações para o metamodelo KDM. Embora nenhuma avaliação tenha sido realiza neste sentido, a dificuldade na escrita de refatorações (ATL) e restrições (OCL) são evidentes, uma vez que, engenheiros não estão familiarizados com linguagens de transformação de modelo e linguagens de restrição~\cite{Sendall_2003_rafael_final}, assim, o fornecimento de passos sistemáticos e \textit{templates} auxiliam e guiam engenheiros de modernização durante a criação de refatorações e restrições para o KDM.

Sobre o metamodelo SRM, é possível concluir que ele fornece uma terminologia comum, padronizada e independente de linguagem para a especificação de refatorações. A ideia é que ferramentas de modernização adotem esse metamodelo como base para suas refatorações, permitindo assim o reúso de instâncias de refatorações entre ferramentas. Esse metamodelo possui um conjunto de metaclasses que define meta-atributos específicos para representar informações (metadados) de refatoração, auxiliando, assim, o compartilhamento das refatorações de forma intuitiva entre os engenheiros. Dessa forma, o SRM permite a interoperabilidade entre ferramentas de modernização, desde que elas adotem o metamodelo supracitado e viabiliza a especificação de refatorações em um formato padronizado por parte dos engenheiros de modernização. O SRM segue a mesma proposta de outros metamodelos definidos na ADM e está totalmente integrado com o metamodelo KDM. Uma instância do metamodelo de refatoração contém metadados que representam uma refatoração escrita para ser executada em uma instância do metamodelo KDM. O algoritmo de refatoração, bem como as restrições podem ser especificadas por meio do SRM. Além disso, se ferramentas de modernização utilizarem internamente o SRM elas podem possuir \textit{templates} que são capazes de gerar automaticamente algoritmos de refatoração para o KDM. %Assim, instâncias do metamodelo SRM são enviadas para um repositório único, fazendo com que as refatorações definidas por engenheiros de modernização possam ser reutilizadas. % colocar...l

Outra contribuição são os produtos gerados pela pesquisa realizada. A ferramenta KDM-RE foi implementada como um conjunto de \textit{plug-ins} para o Ambiente de Desenvolvimento Eclipse IDE. Cada \textit{plug-in} representa um módulo da KDM-RE: (\textit{i}) módulo de refatoração, (\textit{ii}) módulo do SRM e (\textit{iii}) módulo de sincronização. O primeiro módulo permite a aplicação de refatorações graficamente por meio de diagramas de classe da UML, porém, as refatorações são realizadas transparentemente no metamodelo KDM e, posteriormente, replicadas nos diagramas de classe da UML. O segundo módulo contém uma DSL que foi desenvolvida para facilitar a instanciação do metamodelo SRM. Optou-se por desenvolver uma DSL, pois o processo de instanciação do SRM não é trivial, engenheiros de modernização devem estar familiarizados como as particularidades das refatorações (por exemplo, qual(is) é (são) o(s) pré-requisito(s) para a execução de uma refatoração) e como/onde utilizar e programar tais refatorações. No entanto, com o uso da DSL, engenheiros de modernização são forçados a seguir a semântica correta para a instanciação do metamodelo SRM. Adicionalmente, o segundo módulo da KDM-RE também permite que instâncias do metamodelo SRM sejam enviadas para um repositório, fazendo com que as refatorações definidas por engenheiros de modernização possam ser reutilizadas. O terceiro módulo é responsável por manter consistente e propagar mudanças após a aplicação de refatorações em instâncias do metamodelo KDM. Regras pré-definidas em ATL foram criadas e são disparadas após a aplicação de refatorações em instâncias do metamodelo KDM. Pretende-se disponibilizar a KDM-RE em um repositório de projetos de software, como por exemplo o Sourceforge e GitHub.








%Assim, foi criada uma abordagem para criar e disponibilizar refatorações para o metamodelo KDM e um apoio ferramental que permite aplicá-las em diagramas de classe da UML. %Além disso, após a aplicação de refatorações um módulo de programação de mudança é executado para manter os modelos internos (visões) do KDM sincronizados. 
%
\begin{comment}
Em particular, destacam-se as seguintes contribuições deste trabalho:

\begin{enumerate}

\item \textbf{Uma abordagem para criar refatorações para o KDM}: foi definida uma abordagem para tornar a criação de refatorações para o metamodelo KDM um processo sistemático e guiado. Para isso, foi especificado um conjunto de passos para auxiliar a criação de refatorações para o metamodelo KDM~\cite{durelli_catalogo}. Mais especificamente, essa abordagem possui seis principais passos para guiar a criação de refatorações para o KDM. Acredita-se que a abordagem apresentada no Capítulo~\ref{chapter:catalogo_refactoring_KDM} tem potencial para conduzir engenheiros de modernização durante a criação de refatorações para o metamodelo KDM. Embora nenhuma avaliação tenha sido realizada neste sentido, a dificuldade na escrita de refatorações (ATL) e restrições (OCL) é evidente, assim, o fornecimento de passos sistemático e \textit{templates} guiam o engenheiro a criar refatorações e restrições para o metamodelo KDM.   

\item \textbf{Um metamodelo para prover o reúso de refatorações para o KDM}: foi proposto um metamodelo denominado SRM para viabilizar a disponibilização de refatorações para o KDM de forma que possam ser mais facilmente especificadas, disponibilizadas e reutilizadas. O SRM contém 14 metaclasses e quatro enumerações que permitem armazenar metadados relacionados com refatorações. O objetivo amplo desse metamodelo é permitir o reúso de refatorações e auxiliar o engenheiro de modernização a definir refatorações representativas em forma de metadados. Em seguida, as instâncias desse metamodelo (arquivos em XMI) são enviadas para um repositório e são reutilizadas por engenheiros de software por meio de um apoio computacional. Assim, é possível concluir que o SRM facilita a interoperabilidade entre ferramentas de modernização. Pretende-se que o metamodelo SRM seja utilizado por outros pesquisadores para prover o reúso e a disponibilização de refatorações no contexto do metamodelo KDM;

%\item \textbf{Abordagem KDM-SInc}: foi proposta uma abordagem para manter uma determinada instância do metamodelo KDM sincronizada e consistente após a aplicação de refatorações. A abordagem KDM-SInc possui três passos. O primeiro passo denominado \textit{Diff} utiliza o \textit{framework} EMFCompare  para comparar duas instâncias do KDM; uma instância original (\aspas{KDM esquerdo}) e uma instância refatorada (\aspas{KDM direito}). Em seguida o segundo passo utiliza um motor de busca que implementa o algoritmo DFS juntamente com um conjunto de expressões definidas em XPath. Por fim, o terceiro passo possui um motor de propagação que utiliza um conjunto de transformações pré-definidas em ATL para executar as propagações e sincronização.

\item \textbf{Apoio Computacional}: o apoio computacional apresentado no Capítulo~\ref{chapter:ferramenta_kdm_re} foi definido para automatizar a atividade de aplicação e reutilização de refatorações em sistemas representados pelo KDM. Para auxiliar o engenheiro de software, as refatorações podem ser aplicadas diretamente em diagramas de classes UML, porém, cada refatoração é de fato realizada de forma subjacente no metamodelo KDM e, posteriormente, replicada automaticamente nos diagramas de classes UML. Adicionalmente, após a aplicação de refatorações em sistemas representados pelo KDM, é de suma importância manter todos os pacotes/artefatos sincronizados e consistentes. Dessa forma, esse apoio computacional também contém um \textit{plug-in} responsável por aplicar regras de propagações que são realizadas em instância do metamodelo KDM. O intuito desse \textit{plug-in} é manter todos os artefatos sincronizados e consistentes de acordo com a refatoração aplicada;

\item \textbf{Realização de experimento}: sete sistemas foram escolhidos para aplicar um conjunto de refatorações utilizando o apoio computacional KDM-RE. O principal objetivo desse experimento foi verificar se após a aplicação de um conjunto de refatorações com base em \textit{bad-smells} já identificados as refatorações adaptadas para o metamodelo KDM melhoram os sistemas em termos de atributos de qualidade. Quatro atributos de qualidade foram considerados no experimento: \aspas{reusabilidade}, \aspas{flexibilidade}, \aspas{facilidade de compreensão} e \aspas{eficácia}. Os resultados mostraram que para todos os atributos de qualidade, exceto \aspas{eficácia}, uma melhora foi obtida após a aplicação de um conjunto de refatorações.

\end{enumerate}
\end{comment}


Com relação à arquitetura da KDM-RE, optou-se por permitir que o engenheiro de software aplique refatorações nos diagramas UML, mas ao invés dessas refatorações serem executadas na instância UML correspondente, são diretamente executadas na instância do KDM que está sendo exibido graficamente. Pode-se imaginar que uma estratégia diferente seja aplicar as refatorações na instância UML e depois transformá-la para uma instância do KDM. Entretanto, a estratégia adotada foi escolhida por causa da semelhança das metaclasses da UML e do KDM, permitindo que um mapeamento implícito fosse mantido durante as transformações. A geração da instância UML refatorada é então feita com o apoio de um \textit{plug-in} existente fornecido pelo MoDisco~\cite{Bruneliere_2010MODISCO}.


Durante a condução desta pesquisa também foi possível concluir que a UML mostra-se limitada mediante a tarefa de representar graficamente do KDM. Idealmente, deveriam existir visualizações gráficas mais adequadas que permitissem uma análise mais clara e detalhada de todas as visões do KDM. Porém, quando engenheiros de software estiverem focados em refatorações de baixo nível, não há problema em usar os diagramas UML. 

Outra contribuição desta tese é o experimento conduzido, no qual sete sistemas foram escolhidos para aplicar um conjunto de refatorações utilizando o apoio computacional KDM-RE. O principal objetivo desse experimento foi verificar se após a aplicação de um conjunto de refatorações com base em \textit{bad-smells} já identificados as refatorações criadas para o metamodelo KDM melhoram os sistemas em termos de atributos de qualidade. Quatro atributos de qualidade foram considerados no experimento: \aspas{reusabilidade}, \aspas{flexibilidade}, \aspas{facilidade de compreensão} e \aspas{eficácia}. Os resultados mostraram que para todos os atributos de qualidade, exceto \aspas{eficácia}, uma melhora foi obtida após a aplicação de um conjunto de refatorações.


Com base em todo o trabalho realizado foi possível perceber que o ideal do OMG com a ADM é possível de ser atingido mas ainda há vários desafios a serem superados. A existência de ferramentas de modernização que adotem o KDM e os demais metamodelos da ADM como metamodelos base depende bastante de evidências sobre a qualidade do mesmo. Além disso, organizações que já possuem suas próprias ferramentas com modelos proprietários precisam ter evidências muito claras dos benefícios de trocar pelo KDM.  Um dos fatores primordiais para que organizações passem a utilizar o KDM em suas ferramentas é a quantidade de recursos disponíveis para esse metamodelo. Quanto maior for a quantidade de repositórios de recursos para o KDM, maior será a motivação das organizações em adotarem esse metamodelo. Porém, acredita-se, que as contribuições enfatizadas nesta tese contribuem para o avanço do ideal do OMG durante o processo de modernização, principalmente, no contexto da ADM e do KDM. Em especial, a abordagem aqui apresentada, contribui com as perspectivas do engenheiro de modernização e o apoio computacional auxilia o engenheiro de software. O engenheiro de modernização é capaz de utilizar a abordagem e o metamodelo SRM para criar e disponibilizar refatorações no contexto do metamodelo KDM e, possivelmente, aumentar o nível de reusabilidade e interoperabilidade de refatorações em nível de modelo. O engenheiro de software é beneficiado porque pode utilizar as facilidades providas pelo apoio computacional KDM-RE e, com isso, aplicar refatorações graficamente por meio de diagramas UML e reutilizar refatorações para o KDM.


%As contribuições mais específicas são uma abordagem para tornar a criação de refatorações para o metamodelo KDM um processo sistemático e guiado, o metamodelo SRM para viabilizar a disponibilização de refatorações para o metamodelo KDM de forma que possam ser mais facilmente especificadas, disponibilizadas e reusadas e um apoio ferramental denominado KDM-RE que apoio a abordagem proposta nesta Tese.

%A abordagem para criar refatorações para o metamodelo KDM~\cite{durelli_catalogo} envolve diretrizes que apoiam o engenheiro de modernização durante a implementação de refatorações para o KDM. Mais especificadamente essa abordagem possui cinco principais atividades que o engenheiro de modernização deve seguir para criar refatorações para o KDM. 

%O metamodelo SRM, juntamente com a refatoração criada, constituem uma forma de viabilizar a disponibilização de refatorações para o metamodelo KDM de forma que possam ser mais facilmente especificadas, disponibilizas e reutilizadas. O SRM contém metaclasses que permitem armazenar metadados relacionadas com refatorações. O objetivo amplo desse metamodelo é permitir e aumentar o reúso de refatorações para um amplo domínio e auxiliar o engenheiro de modernização a definir refatorações representativas em forma de metadados. Em seguida as instâncias desse metamodelo (XMI) são enviadas para um repositório e são reutilizadas por engenheiros de software por meio de um apoio computacional. Pretende-se que o metamodelo SRM seja utilizado por outros pesquisadores para prover o reúso e disponibilização de refatorações no contexto do KDM.

%Outra contribuição é o apoio computacional apresentado no Capítulo~\ref{chapter:ferramenta_kdm_re}. O apoio computacional foi definido para automatizar a atividade de aplicação e reutilização de refatorações em sistemas representados pelo KDM. Para auxiliar o engenheiro de software, as refatorações podem ser aplicadas diretamente em diagramas de classes UML, porém, a refatoração é de fato realizada transparentemente no metamodelo KDM e posteriormente replicada nos diagramas de classes UML. Adicionalmente, após a aplicação de refatorações em sistemas representados pelo KDM é de suma importância manter todos os pacotes/artefatos sincronizados e consistentes. Dessa forma, esse apoio computacional também contém um plug-in responsável por aplicar regras de propagações que são realizadas em instância do metamodelo KDM. O intuito desse plug-in é manter todos os artefatos sincronizados e consistentes de acordo com a refatoração aplicada.

\section{Limitações}\label{sec:limitacoes_trabalho}

A abordagem e o apoio computacional desenvolvidos nesta tese possuem as seguintes limitações:

\begin{itemize}

\item A abordagem de criação de refatorações para o KDM descreve como criar refatorações para o metamodelo KDM. Contudo, a forma como o mecanismo, bem como as pré- e pós-condições dependem de \textit{templates} e devem ser criados totalmente manualmente. Neste projeto de doutorado, foram realizados exemplos práticos com a utilização de ATL e OCL para implementar o mecanismo e as asserções (pré- e pós-condições), respectivamente;

\item O metamodelo SRM viabiliza o reúso das refatorações dentro de ferramentas de modernização, propiciando a interoperabilidade entre as mesmas. No entanto, como o SRM é baseado no KDM, podem-se apenas especificar refatorações para os elementos estruturais representados pelas metaclasses contidas no KDM;  


\item O apoio computacional KDM-RE contém um módulo que provê uma DSL para auxiliar a instanciação do metamodelo SRM. Essa DSL aumenta o nível de abstração do processo de instanciação de refatorações, escondendo determinadas complexidades. O uso dessa DSL guia o engenheiro de modernização durante a especificação de uma determinada refatoração, norteando quais metaclasses devem ser especificadas. No entanto, a semântica de uma determinada refatoração utilizando essa DSL continua sendo responsabilidade do engenheiro de modernização;

%\item Refatorações são aplicadas apenas utilizando o diagrama de classe da UML por meio do apoio computacional KDM-RE. 

\item O apoio computacional KDM-RE foi desenvolvido para ser utilizado no ambiente de desenvolvimento Eclipse, portanto, o conhecimento sobre esse ambiente é um pré-requisito necessário para o manuseio desse apoio computacional.
\end{itemize}

\section{Sugestões de Trabalhos Futuros}\label{sec:trabalhos_futuros_tese}

Refatoração, especialmente refatoração em nível de modelo, é uma disciplina relativamente nova e uma área de pesquisa ativa de acordo com o OMG~\cite{OMG_OMG, ADM:OMG}. Esta tese, considera a aplicação de refatorações no contexto da ADM e do metamodelo KDM de forma integrada, subjacente e automatizada. Assim, o trabalho desenvolvido, nesta tese, deve passar por vários ajustes com base na experiência substancial de aplicações práticas para obter relevância na indústria. Com base no MS conduzido e apresentado no Capítulo~\ref{chapter:mapeamento_sistematico} e no trabalho apresentado nesta tese, há muitas pesquisas que podem ser realizadas no futuro pelo autor e por outros pesquisadores do grupo de pesquisa em engenharia de software do ICMC.

Uma dessas pesquisas seria realizar mais estudos de caso com outros programas, visando comprovar se os mesmos resultados, ou resultados semelhantes, serão obtidos. Além disso, os passos para criar refatorações para o metamodelo KDM, os quais foram definidos no Capítulo~\ref{chapter:catalogo_refactoring_KDM}, necessitam de estudos de caso que envolvessem a criação de um conjunto de refatorações. No contexto do metamodelo SRM, também seria importante aplicar estudos de caso envolvendo uma população de novas refatorações para verificar se o mesmo pode ser utilizado para prover a especificação e a reutilização de refatorações de forma eficiente. Também seria interessante como trabalho futuro a construção de uma ferramenta de apoio a abordagem apresentada no Capítulo~\ref{chapter:catalogo_refactoring_KDM}. Essa ferramenta poderia auxiliar no acompanhamento da abordagem, na documentação e criação de refatorações para o metamodelo KDM de forma totalmente automática.

Outro trabalho futuro seria implementar algum mecanismo de suporte à evolução do metamodelo SRM. Isso se mostra necessário quando novas refatorações muito específicas precisam ser criadas. Também pode-se implementar uma forma de identificar automaticamente \textit{bad-smells} utilizando o metamodelo KDM como base. %O apoio computacional aqui apresentado não identifica os \textit{bad-smells}, sendo de inteira responsabilidade do engenheiro de software detectar falhas de projeto.


A ferramenta KDM-RE utiliza apenas o diagrama de classe da UML como interface (\textit{front-end}) para a aplicação de refatorações. Outros diagramas da UML são raramente utilizados na literatura durante atividades de refatorações. O uso combinado de múltiplos diagramas da UML durante a atividade de refatoração poderiam auxiliar o engenheiro de software a entender melhor a estrutura do sistema, bem como compreender também seu comportamento. Por exemplo, além do diagrama de classe da UML, diagramas de caso de uso e sequencia poderiam ser utilizados durante a atividade de refatoração como interface.

\begin{comment}

\section{Publicações Resultantes}\label{sec:publicacoes_resultantes}
Durante este projeto de doutorado foi possível publicar um conjunto de artigos científicos. Tais artigos são resultados de trabalhos totalmente relacionado a este projeto de doutorado, bem como publicações realizadas em parcerias com outros pesquisadores. Acredita-se que estas também são importantes como parte da experiência de uma pesquisa em nível de doutorado. Dessa forma, nessa seção os artigos científicos são separados na seguinte orderm: (\textit{i}) totalmente relacionado a este projeto, bem como ADM e KDM e (\textit{ii}) publicações realizadas em parcerias.

\subsection{Publicações totalmente relacionado a este projeto, bem como ADM e KDM.}

Nesta seção, as publicações diretamente relacionadas a este projeto são listadas. Além dessas publicações, aqui também são destacadas as publicações realizadas com parceiros durante o doutorado e que são totalmente relacionados com o tema ADM e KDM. Tais publicações foram importante para auxiliar o doutorando a entender e aprimorar o seu conhecimento sobre ADM e KDM e realizar sua pesquisa de forma satisfatória.

\begin{itemize}
	\item \textbf{Trabalho completo publicado em revista}
			
			\begin{enumerate}
			    \item SANTIBÁÑEZ, DANIEL S. M. ; \textbf{DURELLI, RAFAEL} ; DE CAMARGO, VALTER . \aspas{\textit{A Combined Approach for Concern Identification in KDM models}}. \textbf{Journal of the Brazilian Computer Society}, 2015.
			  \begin{itemize}
			        \item Nível de contribuição: Alto - Auxiliou-se na elaboração do apoio ferramental, bem como na escrita e na estrutura do artigo;
			    \end{itemize}
			\end{enumerate}
	\item \textbf{8 Trabalhos completos publicados em anais de congressos/workshops}
	\begin{enumerate}
	 	
	 	\item \textbf{DURELLI, R. S.}; SANTIBANEZ, D. S. M. ; ANQUETIL, N. ; DELAMARO, M. E. ; CAMARGO, V. V. \aspas{\textit{A Systematic Review on Mining Techniques for Crosscutting Concerns}}. \textbf{The ACM Symposium on Applied Computing (SAC)}, 2013, Coimbra.
	 	
	 	    \begin{itemize}
			        \item Nível de contribuição: Alto - O doutorando é o principal investigador e conduziu a elaboração do artigo com ajuda de colaboradores;
			    \end{itemize}
		
		\item SANTIBÁÑEZ, DANIEL S. M. ; \textbf{DURELLI, RAFAEL S.}; CAMARGO, V. V. \aspas{\textit{CCKDM - A Concern Mining Tool for Assisting in the Architecture-Driven Modernization Process}}. \textbf{XXVII Simpósio Brasileiro de Engenharia de Software - XXVII Sessão de Ferramenta}, 2013, Brasília.
		    \begin{itemize}
			        \item Nível de contribuição: Alto - Auxiliou-se na elaboração do apoio ferramental, bem como na escrita e na estrutura do artigo;
			    \end{itemize}
		
		\item SANTIBANEZ, D. S. M. ; \textbf{DURELLI, RAFAEL S.}; CAMARGO, V. V. \aspas{\textit{A Combined Approach for Concern Identification in KDM models}}. \textbf{Latin American Workshop on Aspect-Oriented Software Development (LA-WASP)}, 2013, Brasília. Congresso Brasileiro de Software: Teoria e Prática (CBSoft), 2013.
		
		    \begin{itemize}
			        \item Nível de contribuição: Alto - Auxiliou-se na elaboração do apoio ferramental, bem como na escrita e na estrutura do artigo;
			    \end{itemize}
		
		\item \textbf{DURELLI, R. S.} ; SANTIBANEZ, D. S. M. ; DELAMARO, MÁRCIO E. ; CAMARGO, V. V. \aspas{\textit{Towards a Refactoring Catalogue for Knowledge Discovery Metamodel}}. \textbf{IEEE International Conference on Information Reuse and Integration}, 2014, São Francisco.
		
		    \begin{itemize}
			        \item Nível de contribuição: Alto - O doutorando é o principal investigador e conduziu a elaboração do artigo com ajuda de colaboradores;
			    \end{itemize}
		
		\item \textbf{DURELLI, R. S.} ; SANTIBANEZ, D. S. M. ; MARINHO, B. S. ; HONDA, R. R. ; DELAMARO, M. E. ; ANQUETIL, N. ; CAMARGO, V. V. \aspas{\textit{A Mapping Study on Architecture-Driven Modernization}}. \textbf{IEEE International Conference on Information Reuse and Integration}, 2014, São Francisco.
		    \begin{itemize}
			        \item Nível de contribuição: Alto - O doutorando é o principal investigador e conduziu a elaboração do artigo com ajuda de colaboradores;
			    \end{itemize}
		
		\item MARINHO, B. S. ; CAMARGO, V. V. ; HONDA, R. R. ; \textbf{DURELLI, R. S.} . \aspas{\textit{KDM-AO: An Aspect-Oriented Extension of the Knowledge Discovery Metamodel}}. \textbf{28th Brazilian Symposium on Software Engineering (SBES)}, 2014, Maceió.
		
		    \begin{itemize}
			        \item Nível de contribuição: Alto - Auxiliou-se na elaboração da extensão, bem como na escrita e na estrutura do artigo;
			    \end{itemize}
		
		\item MARINHO, B. S. ; \textbf{DURELLI, RAFAEL S.} ; HONDA, R. R. ; CAMARGO, V. V. . \aspas{\textit{Investigating Lightweight and Heavyweight KDM Extensions for Aspect-Oriented Modernization}}. \textbf{11th Workshop on Software Modularity (WMod) -- Brazilian Conference on Software: theory and practice}, 2014, Maceió.
		
		\begin{itemize}
			        \item Nível de contribuição: Alto - Auxiliou-se na elaboração da extensão, bem como na escrita e na estrutura do artigo;
			    \end{itemize}
		
		\item \textbf{DURELLI, RAFAEL S.} ; MARINHO, B. S. ; HONDA, R. R. ; DELAMARO, MÁRCIO E. ; CAMARGO, V. V. . \aspas{\textit{KDM-RE: A Model-Driven Refactoring Tool for KDM}}. \textbf{II Workshop on Software Visualization, Evolution and Maintenance -- Brazilian Conference on Software: theory and practice}, 2014, Maceió.
		
		    \begin{itemize}
			        \item Nível de contribuição: Alto - O doutorando é o principal investigador e conduziu a elaboração do artigo com ajuda de colaboradores;
			    \end{itemize}
\end{enumerate}
\end{itemize}

\subsection{Publicações realizadas em parcerias}

Nesta seção, os trabalhos relacionados realizados em parcerias durante o doutorado são apresentados. Embora tais artigos não estejam diretamente relacionado com o tema ADM e KDM, todos foram de alguma forma importantes para a formação do doutorando, bem como para a criação de colaboração.

\begin{itemize}
	
	\item \textbf{2 trabalhos completos publicado como capítulo de livro}
		\begin{enumerate}
			
			\item Viana, Matheus ; Penteado, Rosângela ; Prado, Antônio do ; \textbf{Durelli, Rafael} . \aspas{\textit{Developing Frameworks from Extended Feature Models}}. Advances in Intelligent Systems and Computing. 1ed.: Springer International Publishing, 2014, v. 263, p. 263-284.
			
			\begin{itemize}
			\item Nível de contribuição: Médio - Auxiliou-se na escrita e na estrutura do artigo;
			\end{itemize}
			
			\item Júnior, Paulo Afonso Parreira ; Penteado, Rosângela Dellosso ; Viana, Matheus Carvalho ; \textbf{Durelli, Rafael Serapilha} ; DE CAMARGO, VALTER VIEIRA ; Costa, Heitor Augustus Xavier . \aspas{\textit{Reengineering of Object-Oriented Software into Aspect-Oriented Ones Supported by Class Models}}. Lecture Notes in Business Information Processing. 1ed.: Springer International Publishing, 2014, v. 190, p. 296-313.
			    \begin{itemize}
			    \item Nível de contribuição: Médio - Auxiliou-se na escrita e na estrutura do artigo;
			    \end{itemize}
			
		\end{enumerate}
	
	\item \textbf{Trabalho completo publicado em revista}
		\begin{enumerate}
			\item GOTTARDI, THIAGO ; \textbf{DURELLI, RAFAEL} ; LÓPEZ, ÓSCAR ; DE CAMARGO, VALTER . \aspas{\textit{Model-based reuse for crosscutting frameworks: assessing reuse and maintenance effort}}. \textbf{Journal of Software Engineering Research and Development}, 2013.
			    \begin{itemize}
			        \item Nível de contribuição: Alto - Auxiliou-se na elaboração do apoio ferramental, bem como na escrita e na estrutura do artigo;
			    \end{itemize}
		\end{enumerate}
	\item \textbf{10 trabalhos completos publicados em anais de congressos/\textit{workshops}}
	\begin{enumerate}
	    
	    \item \textbf{DURELLI, R. S.}; DURELLI, V. H. S. . \aspas{\aspas{\textit{A Systematic Mapping Study on Formal Methods Applied to Crosscutting Concerns Mining}}}. \textbf{IX Experimental Software Engineering Latin American Workshop (ESELAW)}, 2012, Buenos Aires.
	            \begin{itemize}
			        \item Nível de contribuição: Alto - O doutorando é o principal investigador e conduziu a elaboração do artigo com ajuda de colaboradores;
			    \end{itemize}
	 	
	 	\item \textbf{DURELLI, R. S.}; DURELLI, V. H. S. . \aspas{\aspas{\textit{F2MoC: A Preliminary Product Line DSL for Mobile Robots}}}. \textbf{Simpósio Brasileiro de Sistemas de Informação (SBSI)}, 2012, São Paulo.
	 	        \begin{itemize}
			        \item Nível de contribuição: Alto - O doutorando é o principal investigador e conduziu a elaboração do artigo com ajuda de colaboradores;
			    \end{itemize}
	 	
	 	
	 	\item Gottardi; \textbf{DURELLI, R. S.} ; PASTOR, O. L. ; CAMARGO, V. V. . \aspas{\textit{Model-Based Reuse for Crosscutting Frameworks: Assessing Reuse and Maintainability Effort}}. \textbf{Simpósio Brasileiro de Engenharia de Software (SBES)}, 2012, Natal.
	 	     \begin{itemize}
			        \item Nível de contribuição: Alto - Auxiliou-se na elaboração do apoio ferramental, bem como na escrita e na estrutura do artigo;
			    \end{itemize}
	 	
	 	\item \textbf{DURELLI, R. S.}; Gottardi ; CAMARGO, V. V. . \aspas{\textit{CrossFIRE: An Infrastructure for Storing Crosscutting Framework Families and Supporting their Model-Based Reuse}}. \textbf{XXVI Simpósio Brasileiro de Engenharia de Software - XXVI Sessão de Ferramenta}, 2012, Natal.
	 	       \begin{itemize}
			        \item Nível de contribuição: Alto - O doutorando é o principal investigador e conduziu a elaboração do artigo com ajuda de colaboradores;
			   \end{itemize}
	 	
	 	\item PARREIRA JUNIOR, P. A.; VIANA, M. C. ; \textbf{DURELLI, R. S.} ; CAMARGO, V. V. ; COSTA, H. A. X. ; PENTEADO, R. A. D. \aspas{\textit{Concern-Based Refactorings Supported by Class Models to Reengineer Object-Oriented Software into Aspect-Oriented Ones}}. \textbf{International Conference on Enterprise Information Systems (ICEIS)}, 2013, ANGERS-França.
	 	
	 	    \begin{itemize}
			    \item Nível de contribuição: Baixo - Auxiliou-se na estrutura do artigo;
			    \end{itemize}
		
		\item VIANA, M. C. ; \textbf{DURELLI, R. S.} ; PENTEADO, R. A. D. ; PRADO, A. F. \aspas{\textit{F3: From features to frameworks}}. \textbf{International Conference on Enterprise Information Systems (ICEIS)}, 2013, ANGERS-França.
		    
		    \begin{itemize}
			    \item Nível de contribuição: Médio - Auxiliou-se na escrita e na estrutura do artigo;
			    \end{itemize}
		
		\item VIANA, M. C. ; PENTEADO, R. A. D. ; PRADO, A. F. ; \textbf{DURELLI, R. S}. . \aspas{\textit{An Approach to Develop Frameworks from Feature Models}}. \textbf{International Conference on Information Reuse and Integration}, 2013, São Francisco.
		        \begin{itemize}
			    \item Nível de contribuição: Médio - Auxiliou-se na escrita e na estrutura do artigo;
			    \end{itemize}
		
		\item VIANA, M. C. ; PENTEADO, R. A. D. ; PRADO, A. F. ; \textbf{DURELLI, RAFAEL S}. \aspas{\textit{F3T: From Features to Frameworks Tool}}. \textbf{XXVII Simpósio Brasileiro de Engenharia de Software (SBES)}, 2013, Brasília.
		    \begin{itemize}
			    \item Nível de contribuição: Médio - Auxiliou-se na escrita e na estrutura do artigo;
			    \end{itemize}
		
		\item PINTO, Victor Hugo S. C. ; \textbf{DURELLI, R. S.} ; OLIVEIRA, A. L. ; CAMARGO, V. V. \aspas{\textit{Evaluating the Effort for Modularizing Multiple-Domain Frameworks towards Framework Product Lines with Aspect-Oriented Programming and Model-Driven Development}}. \textbf{International Conference on Enterprise Information Systems (ICEIS)}, 2014, Lisboa.
		
		    \begin{itemize}
			        \item Nível de contribuição: Baixo - Auxiliou-se na elaboração do experimento;
			    \end{itemize}
		
		\item DIAS, D. R. C. ; \textbf{DURELLI, R. S.} ; BREGA, J. R. F. ; GNECCO, B. B ; TREVELIN, L. C. ; GUIMARAES, M. P. \aspas{\textit{Data Network in Development of 3D Collaborative Virtual Environments: A Systematic Review}}. \textbf{The 14th International Conference on Computational Science and Applications (ICCSA)}, 2014, Guimarães.
		
		    \begin{itemize}
			        \item Nível de contribuição: Baixo - Auxiliou-se na elaboração do protocolo da revisão sistemática;
			    \end{itemize}

\end{enumerate}
\end{itemize}

\end{comment}