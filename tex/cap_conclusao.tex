\section{Considerações Finais do Projeto de Doutorado}

Neste projeto de doutorado o objetivo foi buscar soluções que facilitam a aplicação e o reúso de refatorações para o metamodelo KDM. Para isso foi definido uma abordagem para a criação e disponibilização de refatorações para o KDM, bem como um apoio ferramental que permite aplicá-las em diagramas de classe da UML. A abordagem possui dois principais passos: (\textit{i}) o primeiro envolve diretrizes que apoiam o engenheiro de modernização durante a implementação de refatorações para o KDM; e (\textit{ii}) o segundo consiste na especificação das refatorações por meio da criação de instâncias do metamodelo SRM e posterior disponibilização delas em um repositório.

O apoio ferramental, denominado KDM-RE, é composto por três plug-ins do Eclipse: (\textit{i}) o primeiro consiste em um conjunto de \textit{Wizards} que apoia o engenheiro de software na aplicação das refatorações em diagramas de classe UML; (\textit{ii}) o segundo consiste em um módulo de propagação de mudanças que permite manter modelos internos do KDM sincronizados e; (\textit{iii}) o terceiro consiste em um apoio à importação e reúso de refatorações disponíveis no repositório.

Como parte da pesquisa, dois experimentos foram conduzidos com o objetivo de testar e avaliar as vantagens de utilizar a abordagem e a ferramenta KDM-RE. Os resultados mostram que a abordagem pode trazer benefícios para o engenheiro de software durante a atividade de aplicação de refatorações em sistemas representadas pelo metamodelo KDM.

Nas demais seções deste capítulo são apresentados os seguintes tópicos: na Seção X as contribuições desta Tese são apresentadas; na Seção Y, as limitações são destacadas; na Seção X, os possíveis trabalhos futuros são apresentados; e na Seção~\ref{sec:publicacoes_resultantes}, as publicações resultantes durante o período de doutorado são destacados\change{mudar as seções}. 

\section{Publicações Resultantes}\label{sec:publicacoes_resultantes}
Durante este projeto de doutorado foi possível publicar um conjunto de artigos científicos. Tais artigos são resultados de trabalhos totalmente relacionado a este projeto de doutorado, bem como publicações realizadas em parcerias com outros pesquisadores. Acredita-se que estas também são importantes como parte da experiência de uma pesquisa em nível de doutorado. Dessa forma, nessa seção os artigos científicos são separados na seguinte orderm: (\textit{i}) totalmente relacionado a este projeto, bem como ADM e KDM e (\textit{ii}) publicações realizadas em parcerias.

\subsection{Publicações totalmente relacionado a este projeto, bem como ADM e KDM.}

Nessa seção as publicações diretamente relacionadas a este projeto são listadas. Além dessas publicações, aqui também são destacadas as publicações realizadas com parceiros durante o doutorado e que são totalmente relacionados com o tema ADM e KDM. Tais publicações foram importante para auxiliar o doutorando a entender e aprimorar o seu conhecimento sobre esse tema para realizar sua pesquisa de forma satisfatória.

\begin{itemize}
	\item \textbf{Trabalho completo publicado em revista}
			
			\begin{enumerate}
			    \item SANTIBÁÑEZ, DANIEL S. M. ; \textbf{DURELLI, RAFAEL} ; DE CAMARGO, VALTER . \aspas{\textit{A Combined Approach for Concern Identification in KDM models}}. \textbf{Journal of the Brazilian Computer Society}, 2015.
			  \begin{itemize}
			        \item Nível de contribuição: Alto - Auxiliou-se na elaboração do apoio ferramental, bem como na escrita e na estrutura do artigo;
			    \end{itemize}
			\end{enumerate}
	\item \textbf{8 Trabalhos completos publicados em anais de congressos/workshops}
	\begin{enumerate}
	 	
	 	\item \textbf{DURELLI, R. S.}; SANTIBANEZ, D. S. M. ; ANQUETIL, N. ; DELAMARO, M. E. ; CAMARGO, V. V. \aspas{\textit{A Systematic Review on Mining Techniques for Crosscutting Concerns}}. \textbf{The ACM Symposium on Applied Computing (SAC)}, 2013, Coimbra.
	 	
	 	    \begin{itemize}
			        \item Nível de contribuição: Alto - O doutorando é o principal investigador e conduziu a elaboração do artigo com ajuda de colaboradores;
			    \end{itemize}
		
		\item SANTIBÁÑEZ, DANIEL S. M. ; \textbf{DURELLI, RAFAEL S.}; CAMARGO, V. V. \aspas{\textit{CCKDM - A Concern Mining Tool for Assisting in the Architecture-Driven Modernization Process}}. \textbf{XXVII Simpósio Brasileiro de Engenharia de Software - XXVII Sessão de Ferramenta}, 2013, Brasília.
		    \begin{itemize}
			        \item Nível de contribuição: Alto - Auxiliou-se na elaboração do apoio ferramental, bem como na escrita e na estrutura do artigo;
			    \end{itemize}
		
		\item SANTIBANEZ, D. S. M. ; \textbf{DURELLI, RAFAEL S.}; CAMARGO, V. V. \aspas{\textit{A Combined Approach for Concern Identification in KDM models}}. \textbf{Latin American Workshop on Aspect-Oriented Software Development (LA-WASP)}, 2013, Brasília. Congresso Brasileiro de Software: Teoria e Prática (CBSoft), 2013.
		
		    \begin{itemize}
			        \item Nível de contribuição: Alto - Auxiliou-se na elaboração do apoio ferramental, bem como na escrita e na estrutura do artigo;
			    \end{itemize}
		
		\item \textbf{DURELLI, R. S.} ; SANTIBANEZ, D. S. M. ; DELAMARO, MÁRCIO E. ; CAMARGO, V. V. \aspas{\textit{Towards a Refactoring Catalogue for Knowledge Discovery Metamodel}}. \textbf{IEEE International Conference on Information Reuse and Integration}, 2014, São Francisco.
		
		    \begin{itemize}
			        \item Nível de contribuição: Alto - O doutorando é o principal investigador e conduziu a elaboração do artigo com ajuda de colaboradores;
			    \end{itemize}
		
		\item \textbf{DURELLI, R. S.} ; SANTIBANEZ, D. S. M. ; MARINHO, B. S. ; HONDA, R. R. ; DELAMARO, M. E. ; ANQUETIL, N. ; CAMARGO, V. V. \aspas{\textit{A Mapping Study on Architecture-Driven Modernization}}. \textbf{IEEE International Conference on Information Reuse and Integration}, 2014, São Francisco.
		    \begin{itemize}
			        \item Nível de contribuição: Alto - O doutorando é o principal investigador e conduziu a elaboração do artigo com ajuda de colaboradores;
			    \end{itemize}
		
		\item MARINHO, B. S. ; CAMARGO, V. V. ; HONDA, R. R. ; \textbf{DURELLI, R. S.} . \aspas{\textit{KDM-AO: An Aspect-Oriented Extension of the Knowledge Discovery Metamodel}}. \textbf{28th Brazilian Symposium on Software Engineering (SBES)}, 2014, Maceió.
		
		    \begin{itemize}
			        \item Nível de contribuição: Alto - Auxiliou-se na elaboração da extensão, bem como na escrita e na estrutura do artigo;
			    \end{itemize}
		
		\item MARINHO, B. S. ; \textbf{DURELLI, RAFAEL S.} ; HONDA, R. R. ; CAMARGO, V. V. . \aspas{\textit{Investigating Lightweight and Heavyweight KDM Extensions for Aspect-Oriented Modernization}}. \textbf{11th Workshop on Software Modularity (WMod) -- Brazilian Conference on Software: theory and practice}, 2014, Maceió.
		
		\begin{itemize}
			        \item Nível de contribuição: Alto - Auxiliou-se na elaboração da extensão, bem como na escrita e na estrutura do artigo;
			    \end{itemize}
		
		\item \textbf{DURELLI, RAFAEL S.} ; MARINHO, B. S. ; HONDA, R. R. ; DELAMARO, MÁRCIO E. ; CAMARGO, V. V. . \aspas{\textit{KDM-RE: A Model-Driven Refactoring Tool for KDM}}. \textbf{II Workshop on Software Visualization, Evolution and Maintenance -- Brazilian Conference on Software: theory and practice}, 2014, Maceió.
		
		    \begin{itemize}
			        \item Nível de contribuição: Alto - O doutorando é o principal investigador e conduziu a elaboração do artigo com ajuda de colaboradores;
			    \end{itemize}
\end{enumerate}
\end{itemize}

\subsection{Publicações realizadas em parcerias}

Nessa seção os trabalhos relacionados em parcerias durante o doutorado são apresentados. Embora tais artigos não estejam diretamente relacionado com o tema ADM e KDM, todos foram de alguma forma importantes para a formação do doutorando, bem como para a criação de colaboração.

\begin{itemize}
	
	\item \textbf{2 trabalhos completos publicado como capítulo de livro}
		\begin{enumerate}
			
			\item Viana, Matheus ; Penteado, Rosângela ; Prado, Antônio do ; \textbf{Durelli, Rafael} . \aspas{\textit{Developing Frameworks from Extended Feature Models}}. Advances in Intelligent Systems and Computing. 1ed.: Springer International Publishing, 2014, v. 263, p. 263-284.
			
			\begin{itemize}
			\item Nível de contribuição: Médio - Auxiliou-se na escrita e na estrutura do artigo;
			\end{itemize}
			
			\item Júnior, Paulo Afonso Parreira ; Penteado, Rosângela Dellosso ; Viana, Matheus Carvalho ; \textbf{Durelli, Rafael Serapilha} ; DE CAMARGO, VALTER VIEIRA ; Costa, Heitor Augustus Xavier . \aspas{\textit{Reengineering of Object-Oriented Software into Aspect-Oriented Ones Supported by Class Models}}. Lecture Notes in Business Information Processing. 1ed.: Springer International Publishing, 2014, v. 190, p. 296-313.
			    \begin{itemize}
			    \item Nível de contribuição: Médio - Auxiliou-se na escrita e na estrutura do artigo;
			    \end{itemize}
			
		\end{enumerate}
	
	\item \textbf{Trabalho completo publicado em revista}
		\begin{enumerate}
			\item GOTTARDI, THIAGO ; \textbf{DURELLI, RAFAEL} ; LÓPEZ, ÓSCAR ; DE CAMARGO, VALTER . \aspas{\textit{Model-based reuse for crosscutting frameworks: assessing reuse and maintenance effort}}. \textbf{Journal of Software Engineering Research and Development}, 2013.
			    \begin{itemize}
			        \item Nível de contribuição: Alto - Auxiliou-se na elaboração do apoio ferramental, bem como na escrita e na estrutura do artigo;
			    \end{itemize}
		\end{enumerate}
	\item \textbf{10 trabalhos completos publicados em anais de congressos/\textit{workshops}}
	\begin{enumerate}
	    
	    \item \textbf{DURELLI, R. S.}; DURELLI, V. H. S. . \aspas{\aspas{\textit{A Systematic Mapping Study on Formal Methods Applied to Crosscutting Concerns Mining}}}. \textbf{IX Experimental Software Engineering Latin American Workshop (ESELAW)}, 2012, Buenos Aires.
	            \begin{itemize}
			        \item Nível de contribuição: Alto - O doutorando é o principal investigador e conduziu a elaboração do artigo com ajuda de colaboradores;
			    \end{itemize}
	 	
	 	\item \textbf{DURELLI, R. S.}; DURELLI, V. H. S. . \aspas{\aspas{\textit{F2MoC: A Preliminary Product Line DSL for Mobile Robots}}}. \textbf{Simpósio Brasileiro de Sistemas de Informação (SBSI)}, 2012, São Paulo.
	 	        \begin{itemize}
			        \item Nível de contribuição: Alto - O doutorando é o principal investigador e conduziu a elaboração do artigo com ajuda de colaboradores;
			    \end{itemize}
	 	
	 	
	 	\item Gottardi; \textbf{DURELLI, R. S.} ; PASTOR, O. L. ; CAMARGO, V. V. . \aspas{\textit{Model-Based Reuse for Crosscutting Frameworks: Assessing Reuse and Maintainability Effort}}. \textbf{Simpósio Brasileiro de Engenharia de Software (SBES)}, 2012, Natal.
	 	     \begin{itemize}
			        \item Nível de contribuição: Alto - Auxiliou-se na elaboração do apoio ferramental, bem como na escrita e na estrutura do artigo;
			    \end{itemize}
	 	
	 	\item \textbf{DURELLI, R. S.}; Gottardi ; CAMARGO, V. V. . \aspas{\textit{CrossFIRE: An Infrastructure for Storing Crosscutting Framework Families and Supporting their Model-Based Reuse}}. \textbf{XXVI Simpósio Brasileiro de Engenharia de Software - XXVI Sessão de Ferramenta}, 2012, Natal.
	 	       \begin{itemize}
			        \item Nível de contribuição: Alto - O doutorando é o principal investigador e conduziu a elaboração do artigo com ajuda de colaboradores;
			   \end{itemize}
	 	
	 	\item PARREIRA JUNIOR, P. A.; VIANA, M. C. ; \textbf{DURELLI, R. S.} ; CAMARGO, V. V. ; COSTA, H. A. X. ; PENTEADO, R. A. D. \aspas{\textit{Concern-Based Refactorings Supported by Class Models to Reengineer Object-Oriented Software into Aspect-Oriented Ones}}. \textbf{International Conference on Enterprise Information Systems (ICEIS)}, 2013, ANGERS-França.
	 	
	 	    \begin{itemize}
			    \item Nível de contribuição: Baixo - Auxiliou-se na estrutura do artigo;
			    \end{itemize}
		
		\item VIANA, M. C. ; \textbf{DURELLI, R. S.} ; PENTEADO, R. A. D. ; PRADO, A. F. \aspas{\textit{F3: From features to frameworks}}. \textbf{International Conference on Enterprise Information Systems (ICEIS)}, 2013, ANGERS-França.
		    
		    \begin{itemize}
			    \item Nível de contribuição: Médio - Auxiliou-se na escrita e na estrutura do artigo;
			    \end{itemize}
		
		\item VIANA, M. C. ; PENTEADO, R. A. D. ; PRADO, A. F. ; \textbf{DURELLI, R. S}. . \aspas{\textit{An Approach to Develop Frameworks from Feature Models}}. \textbf{International Conference on Information Reuse and Integration}, 2013, São Francisco.
		        \begin{itemize}
			    \item Nível de contribuição: Médio - Auxiliou-se na escrita e na estrutura do artigo;
			    \end{itemize}
		
		\item VIANA, M. C. ; PENTEADO, R. A. D. ; PRADO, A. F. ; \textbf{DURELLI, RAFAEL S}. \aspas{\textit{F3T: From Features to Frameworks Tool}}. \textbf{XXVII Simpósio Brasileiro de Engenharia de Software (SBES)}, 2013, Brasília.
		    \begin{itemize}
			    \item Nível de contribuição: Médio - Auxiliou-se na escrita e na estrutura do artigo;
			    \end{itemize}
		
		\item PINTO, Victor Hugo S. C. ; \textbf{DURELLI, R. S.} ; OLIVEIRA, A. L. ; CAMARGO, V. V. \aspas{\textit{Evaluating the Effort for Modularizing Multiple-Domain Frameworks towards Framework Product Lines with Aspect-Oriented Programming and Model-Driven Development}}. \textbf{International Conference on Enterprise Information Systems (ICEIS)}, 2014, Lisboa.
		
		    \begin{itemize}
			        \item Nível de contribuição: Baixo - Auxiliou-se na elaboração do experimento;
			    \end{itemize}
		
		\item DIAS, D. R. C. ; \textbf{DURELLI, R. S.} ; BREGA, J. R. F. ; GNECCO, B. B ; TREVELIN, L. C. ; GUIMARAES, M. P. \aspas{\textit{Data Network in Development of 3D Collaborative Virtual Environments: A Systematic Review}}. \textbf{The 14th International Conference on Computational Science and Applications (ICCSA)}, 2014, Guimarães.
		
		    \begin{itemize}
			        \item Nível de contribuição: Baixo - Auxiliou-se na elaboração do protocolo da revisão sistemática;
			    \end{itemize}

\end{enumerate}
\end{itemize}

\section{Contribuições desta Tese}\label{sec:contribuicoes_desta_tese}

A principal contribuição desta Tese é criar soluções que facilitam a aplicação e o reúso de refatorações no contexto do metamodelo KDM por meio do fornecimento de uma abordagem para a criação e disponibilização de refatorações para o KDM e um apoio ferramental que permite aplicá-las em diagramas de classe da UML. %Além disso, após a aplicação de refatorações um módulo de programação de mudança é executado para manter os modelos internos (visões) do KDM sincronizados. 
%
Em particular, destacam-se as seguintes contribuições deste trabalho:

\begin{enumerate}

\item \textbf{Uma abordagem para criar de refatorações para o KDM}: foi definido uma abordagem para tornar a criação de refatorações para o metamodelo KDM um processo sistemático e guiado. Para isso foi especificado um conjunto de atividades para auxiliar a criação de refatorações para o metamodelo KDM~\cite{durelli_catalogo}. Mais especificadamente essa abordagem possui cinco principais atividades que o engenheiro de modernização deve seguir para criar refatorações para o KDM;

\item \textbf{Um metamodelo para prover o reúso de refatorações para o KDM}: foi proposto um metamodelo denominado SRM para viabilizar a disponibilização de refatorações para o KDM de forma que possam ser mais facilmente especificadas, disponibilizadas e reutilizadas. O SRM contém metaclasses que permitem armazenar metadados relacionadas com refatorações. O objetivo amplo desse metamodelo é permitir e aumentar o reúso de refatorações para um amplo domínio e auxiliar o engenheiro de modernização a definir refatorações representativas em forma de metadados. Em seguida as instâncias desse metamodelo (XMI) são enviadas para um repositório e são reutilizadas por engenheiros de software por meio de um apoio computacional. Pretende-se que o metamodelo SRM seja utilizado por outros pesquisadores para prover o reúso e disponibilização de refatorações no contexto do KDM;

\item \textbf{Apoio Computacional}: o apoio computacional apresentado no Capítulo~\ref{chapter:ferramenta_kdm_re} foi definido para automatizar a atividade de aplicação e reutilização de refatorações em sistemas representados pelo KDM. Para auxiliar o engenheiro de software, as refatorações podem ser aplicadas diretamente em diagramas de classes UML, porém, a refatoração é de fato realizada de forma subjacente no metamodelo KDM e posteriormente replicada automaticamente nos diagramas de classes UML. Adicionalmente, após a aplicação de refatorações em sistemas representados pelo KDM é de suma importância manter todos os pacotes/artefatos sincronizados e consistentes. Dessa forma, esse apoio computacional também contém um plug-in responsável por aplicar regras de propagações que são realizadas em instância do metamodelo KDM. O intuito desse plug-in é manter todos os artefatos sincronizados e consistentes de acordo com a refatoração aplicada;

\item \textbf{Realização de experimento}: sete sistemas foram escolhidos para aplicar um conjunto de refatorações utilizando o apoio computacional KDM-RE. O principal objetivo desse experimento foi verificar se após a aplicação de um conjunto de refatorações com base em \textit{bad-smells} já identificados as refatorações adaptadas para o metamodelo KDM melhoram os sistemas em termos de atributos de qualidade. Quatro atributos de qualidade foram considerados: \aspas{reusabilidade}, \aspas{flexibilidade}, \aspas{facilidade de compreensão} e \aspas{eficácia}. Os resultados mostraram que para todos os atributos de qualidade exceto eficácia uma melhora foi obtido após a aplicação de um conjunto de refatorações.

\end{enumerate}

Acredita-se, portanto, que as contribuições trazidas por esta Tese contribuem para o avanço da área de refatoração em nível de modelo, principalmente no contexto da ADM e KDM. Em especial, a abordagem contribui com as perspectiva do engenheiro de modernização e do engenheiro de software. O engenheiro de modernização é capaz de utilizar abordagem e o metamodelo SRM para criar e disponibilizar refatorações no contexto do metamodelo KDM e possivelmente aumentar o nível de reusabilidade de refatorações em nível de modelo. O engenheiro de software é beneficiado porque pode utilizar as facilidades providas pelo apoio computacional KDM-RE e com isso pode aplicar e reutilizar refatorações para o KDM.


%As contribuições mais específicas são uma abordagem para tornar a criação de refatorações para o metamodelo KDM um processo sistemático e guiado, o metamodelo SRM para viabilizar a disponibilização de refatorações para o metamodelo KDM de forma que possam ser mais facilmente especificadas, disponibilizadas e reusadas e um apoio ferramental denominado KDM-RE que apoio a abordagem proposta nesta Tese.

%A abordagem para criar refatorações para o metamodelo KDM~\cite{durelli_catalogo} envolve diretrizes que apoiam o engenheiro de modernização durante a implementação de refatorações para o KDM. Mais especificadamente essa abordagem possui cinco principais atividades que o engenheiro de modernização deve seguir para criar refatorações para o KDM. 

%O metamodelo SRM, juntamente com a refatoração criada, constituem uma forma de viabilizar a disponibilização de refatorações para o metamodelo KDM de forma que possam ser mais facilmente especificadas, disponibilizas e reutilizadas. O SRM contém metaclasses que permitem armazenar metadados relacionadas com refatorações. O objetivo amplo desse metamodelo é permitir e aumentar o reúso de refatorações para um amplo domínio e auxiliar o engenheiro de modernização a definir refatorações representativas em forma de metadados. Em seguida as instâncias desse metamodelo (XMI) são enviadas para um repositório e são reutilizadas por engenheiros de software por meio de um apoio computacional. Pretende-se que o metamodelo SRM seja utilizado por outros pesquisadores para prover o reúso e disponibilização de refatorações no contexto do KDM.

%Outra contribuição é o apoio computacional apresentado no Capítulo~\ref{chapter:ferramenta_kdm_re}. O apoio computacional foi definido para automatizar a atividade de aplicação e reutilização de refatorações em sistemas representados pelo KDM. Para auxiliar o engenheiro de software, as refatorações podem ser aplicadas diretamente em diagramas de classes UML, porém, a refatoração é de fato realizada transparentemente no metamodelo KDM e posteriormente replicada nos diagramas de classes UML. Adicionalmente, após a aplicação de refatorações em sistemas representados pelo KDM é de suma importância manter todos os pacotes/artefatos sincronizados e consistentes. Dessa forma, esse apoio computacional também contém um plug-in responsável por aplicar regras de propagações que são realizadas em instância do metamodelo KDM. O intuito desse plug-in é manter todos os artefatos sincronizados e consistentes de acordo com a refatoração aplicada.

\section{Limitações}

A abordagem e o apoio computacional desenvolvidos nesta Tese possuem as seguintes limitações:

\begin{itemize}

\item A abordagem de criação de refatorações para o KDM descreve como criar refatorações para o metamodelo KDM. Contudo, a forma como o mecanismo, bem como as pré- e pós-condições dependem do conhecimento do engenheiro de modernização. Neste projeto de doutorado foram realizados exemplos práticos com a utilização de ATL e OCL para implementar o mecanismo e as asserções (pré- e pós-condições), respectivamente;

\item O apoio computacional KDM-RE contém um módulo que provê uma DSL para auxiliar a instanciação do metamodelo SRM. Essa DSL aumenta o nível de abstração do processo de instanciação de refatorações, escondendo determinadas complexidades. O uso dessa DSL guia o engenheiro de modernização durante a especificação de uma determinada refatoração, norteando qualis metaclasses deve) ser especificadas. No entanto, a semântica de uma determinada refatoração utilizando essa DSL continua sendo responsabilidade do engenheiro de modernização;

%\item Refatorações são aplicadas apenas utilizando o diagrama de classe da UML por meio do apoio computacional KDM-RE. 

\item O apoio computacional KDM-RE foi desenvolvida para ser utilizada no ambiente de desenvolvimento Eclipse, portanto, o conhecimento sobre esse ambiente é um pré-requisito necessário para o manuseio desse apoio computacional.
\end{itemize}

\section{Sugestões de Trabalhos Futuros}\label{sec:trabalhos_futuros_tese}

Refatoração, especialmente refatoração em nível de modelo é uma disciplina relativamente nova e uma área de pesquisa ativa de acordo com o OMG~\cite{OMG_OMG, ADM:OMG}. Esta Tese considera a aplicação de refatorações no contexto da ADM e do metamodelo KDM de forma integrada, subjacente e automatizada. Assim, o trabalho desenvolvido nesta Tese deve passar por vários ajustes com base na experiência substancial de aplicações práticas para obter relevância na indústria. Com base no MS conduzido e apresentado no Capítulo~\ref{chapter:mapeamento_sistematico} e o trabalho apresentado nesta Tese, há muitas pesquisas que podem ser realizadas no futuro pelo autor e por outros pesquisadores do grupo de pesquisa em engenharia de software do ICMC.

Um desses trabalhos seria realizar mais estudos de caso com outros programas com o objetivo de comprovar se os mesmos resultados, ou resultados semelhantes, são obtidos. Além disso, tanto para os passos para criar refatorações para o metamodelo KDM definidos no Capítulo X quanto para o metamodelo SRM definido no Capítulo X, também seriam interessantes mais estudos de caso que envolvessem a criação e reutilização de um conjunto de refatorações. Para o metamodelo SRM, também seriam importantes mais estudos de caso envolvendo uma população de novas refatorações para verificar se o mesmo pode ser utilizado para prover a utilização de forma eficiente. \change{mudar os capítulos}

Um outro trabalho seria implementar algum mecanismo de suporte à evolução do metamodelo SRM. Isso se mostra necessário quando novas refatorações muito específicas precisam ser criadas. Também pode-se implementar uma forma de identificar automaticamente \textit{bad-smells} utilizando o metamodelo KDM como base. O apoio computacional aqui apresentado não identifica os \textit{bad-smells}, sendo de inteira responsabilidade do engenheiro de software detectar falhas de projeto.


A ferramenta KDM-RE utiliza apenas o diagrama de classe da UML como interface (\textit{front-end}) para a aplicação de refatorações, outros diagramas da UML são raramente utilizados na literatura durante atividades de refatorações. O uso combinado de múltiplos diagramas da UML durante a atividade de refatoração poderiam auxiliar o engenheiro de software a entender melhor o estrutura do sistema, bem como entender também seu comportamento. Por exemplo, além do diagrama de classe da UML, diagramas como estado e objetos poderiam ser utilizados durante atividades de refatorações como interface. 

