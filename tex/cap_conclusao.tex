\section{Considerações Finais do Projeto de Doutorado}
Neste projeto de doutorado o objetivo foi buscar soluções que facilitam a aplicação e o reúso de refatorações para o metamodelo KDM. Para isso, foi definido uma abordagem para a criação e disponibilização de refatorações para o KDM, bem como um apoio ferramental que permite aplicá-las em diagramas de classe da UML. A abordagem possui dois principais passos: (\textit{i}) o primeiro envolve diretrizes que apoiam o engenheiro de modernização durante a implementação de refatorações para o KDM; e (\textit{ii}) o segundo consiste na especificação das refatorações por meio da criação de instâncias do metamodelo SRM e posterior disponibilização delas em um repositório. Além disso, um apoio computacional, denominado KDM-RE, também foi desenvolvido. Esse apoio computacional é composto por três \textit{plug-ins} do Eclipse: (\textit{i}) o primeiro consiste em um conjunto de \textit{Wizards} que apoia o engenheiro de software na aplicação das refatorações em diagramas de classe UML; (\textit{ii}) o segundo consiste em um apoio à importação e reúso de refatorações disponíveis no repositório; (\textit{iii}) o terceiro consiste em um módulo de propagação de mudanças que permite manter modelos internos do KDM sincronizados.

Como parte da pesquisa, dois experimentos\unsure{mudar} foram conduzidos com o objetivo de testar e avaliar as vantagens de utilizar a abordagem e a ferramenta KDM-RE. Os resultados mostram que a abordagem pode trazer benefícios para o engenheiro de software durante a atividade de aplicação de refatorações em sistemas representadas pelo metamodelo KDM.

Nas demais seções deste capítulo são apresentados os seguintes tópicos: na Seção~\ref{sec:contribuicoes_desta_tese}, as contribuições desta Tese são apresentadas; na Seção~\ref{sec:limitacoes_trabalho}, as limitações são destacadas; na Seção~\ref{sec:trabalhos_futuros_tese}, os possíveis trabalhos futuros são apresentados.%; e na Seção~\ref{sec:publicacoes_resultantes}, as publicações resultantes durante o período de doutorado são destacados\change{mudar as seções}. 

\section{Contribuições desta Tese}\label{sec:contribuicoes_desta_tese}

A principal contribuição desta tese é criar soluções que facilitam a aplicação e o reúso de refatorações no contexto do metamodelo KDM. Assim, foi criado uma abordagem para criar e disponibilizar refatorações para o metamodelo KDM e um apoio ferramental que permite aplicá-las em diagramas de classe da UML. %Além disso, após a aplicação de refatorações um módulo de programação de mudança é executado para manter os modelos internos (visões) do KDM sincronizados. 
%
Em particular, destacam-se as seguintes contribuições deste trabalho:

\begin{enumerate}

\item \textbf{Uma abordagem para criar de refatorações para o KDM}: foi definido uma abordagem para tornar a criação de refatorações para o metamodelo KDM um processo sistemático e guiado. Para isso foi especificado um conjunto de passos para auxiliar a criação de refatorações para o metamodelo KDM~\cite{durelli_catalogo}. Mais especificadamente essa abordagem possui cinco principais passos que o engenheiro de modernização deve seguir para criar refatorações para o KDM;

\item \textbf{Um metamodelo para prover o reúso de refatorações para o KDM}: foi proposto um metamodelo denominado SRM para viabilizar a disponibilização de refatorações para o KDM de forma que possam ser mais facilmente especificadas, disponibilizadas e reutilizadas. O SRM contém 14 metaclasses e subclasses enumerações que permitem armazenar metadados relacionadas com refatorações. O objetivo amplo desse metamodelo é permitir e aumentar o reúso de refatorações para um amplo domínio e auxiliar o engenheiro de modernização a definir refatorações representativas em forma de metadados. Em seguida as instâncias desse metamodelo, arquivos em XMI, são enviadas para um repositório e são reutilizadas por engenheiros de software por meio de um apoio computacional. Pretende-se que o metamodelo SRM seja utilizado por outros pesquisadores para prover o reúso e disponibilização de refatorações no contexto do KDM;

%\item \textbf{Abordagem KDM-SInc}: foi proposta uma abordagem para manter uma determinada instância do metamodelo KDM sincronizada e consistente após a aplicação de refatorações. A abordagem KDM-SInc possui três passos. O primeiro passo denominado \textit{Diff} utiliza o \textit{framework} EMFCompare  para comparar duas instâncias do KDM; uma instância original (\aspas{KDM esquerdo}) e uma instância refatorada (\aspas{KDM direito}). Em seguida o segundo passo utiliza um motor de busca que implementa o algoritmo DFS juntamente com um conjunto de expressões definidas em XPath. Por fim, o terceiro passo possui um motor de propagação que utiliza um conjunto de transformações pré-definidas em ATL para executar as propagações e sincronização.

\item \textbf{Apoio Computacional}: o apoio computacional apresentado no Capítulo~\ref{chapter:ferramenta_kdm_re} foi definido para automatizar a atividade de aplicação e reutilização de refatorações em sistemas representados pelo KDM. Para auxiliar o engenheiro de software, as refatorações podem ser aplicadas diretamente em diagramas de classes UML, porém, a refatoração é de fato realizada de forma subjacente no metamodelo KDM e posteriormente replicada automaticamente nos diagramas de classes UML. Adicionalmente, após a aplicação de refatorações em sistemas representados pelo KDM é de suma importância manter todos os pacotes/artefatos sincronizados e consistentes. Dessa forma, esse apoio computacional também contém um \textit{plug-in} responsável por aplicar regras de propagações que são realizadas em instância do metamodelo KDM. O intuito desse \textit{plug-in} é manter todos os artefatos sincronizados e consistentes de acordo com a refatoração aplicada;

\item \textbf{Realização de experimento}: sete sistemas foram escolhidos para aplicar um conjunto de refatorações utilizando o apoio computacional KDM-RE. O principal objetivo desse experimento foi verificar se após a aplicação de um conjunto de refatorações com base em \textit{bad-smells} já identificados as refatorações adaptadas para o metamodelo KDM melhoram os sistemas em termos de atributos de qualidade. Quatro atributos de qualidade foram considerados: \aspas{reusabilidade}, \aspas{flexibilidade}, \aspas{facilidade de compreensão} e \aspas{eficácia}. Os resultados mostraram que para todos os atributos de qualidade exceto eficácia uma melhora foi obtido após a aplicação de um conjunto de refatorações.

\end{enumerate}

Acredita-se, portanto, que as contribuições trazidas por esta tese contribuem para o avanço da área de refatoração em nível de modelo, principalmente no contexto da ADM e do KDM. Em especial, a abordagem aqui apresentada, contribui com as perspectiva do engenheiro de modernização e o apoio computacional auxilia o engenheiro de software. O engenheiro de modernização é capaz de utilizar a abordagem e o metamodelo SRM para criar e disponibilizar refatorações no contexto do metamodelo KDM e possivelmente aumentar o nível de reusabilidade de refatorações em nível de modelo. O engenheiro de software é beneficiado porque pode utilizar as facilidades providas pelo apoio computacional KDM-RE e, com isso, pode aplicar e reutilizar refatorações para o KDM.


%As contribuições mais específicas são uma abordagem para tornar a criação de refatorações para o metamodelo KDM um processo sistemático e guiado, o metamodelo SRM para viabilizar a disponibilização de refatorações para o metamodelo KDM de forma que possam ser mais facilmente especificadas, disponibilizadas e reusadas e um apoio ferramental denominado KDM-RE que apoio a abordagem proposta nesta Tese.

%A abordagem para criar refatorações para o metamodelo KDM~\cite{durelli_catalogo} envolve diretrizes que apoiam o engenheiro de modernização durante a implementação de refatorações para o KDM. Mais especificadamente essa abordagem possui cinco principais atividades que o engenheiro de modernização deve seguir para criar refatorações para o KDM. 

%O metamodelo SRM, juntamente com a refatoração criada, constituem uma forma de viabilizar a disponibilização de refatorações para o metamodelo KDM de forma que possam ser mais facilmente especificadas, disponibilizas e reutilizadas. O SRM contém metaclasses que permitem armazenar metadados relacionadas com refatorações. O objetivo amplo desse metamodelo é permitir e aumentar o reúso de refatorações para um amplo domínio e auxiliar o engenheiro de modernização a definir refatorações representativas em forma de metadados. Em seguida as instâncias desse metamodelo (XMI) são enviadas para um repositório e são reutilizadas por engenheiros de software por meio de um apoio computacional. Pretende-se que o metamodelo SRM seja utilizado por outros pesquisadores para prover o reúso e disponibilização de refatorações no contexto do KDM.

%Outra contribuição é o apoio computacional apresentado no Capítulo~\ref{chapter:ferramenta_kdm_re}. O apoio computacional foi definido para automatizar a atividade de aplicação e reutilização de refatorações em sistemas representados pelo KDM. Para auxiliar o engenheiro de software, as refatorações podem ser aplicadas diretamente em diagramas de classes UML, porém, a refatoração é de fato realizada transparentemente no metamodelo KDM e posteriormente replicada nos diagramas de classes UML. Adicionalmente, após a aplicação de refatorações em sistemas representados pelo KDM é de suma importância manter todos os pacotes/artefatos sincronizados e consistentes. Dessa forma, esse apoio computacional também contém um plug-in responsável por aplicar regras de propagações que são realizadas em instância do metamodelo KDM. O intuito desse plug-in é manter todos os artefatos sincronizados e consistentes de acordo com a refatoração aplicada.

\section{Limitações}\label{sec:limitacoes_trabalho}

A abordagem e o apoio computacional desenvolvidos nesta tese possuem as seguintes limitações:

\begin{itemize}

\item A abordagem de criação de refatorações para o KDM descreve como criar refatorações para o metamodelo KDM. Contudo, a forma como o mecanismo, bem como as pré- e pós-condições dependem de \textit{templates} e devem ser criados totalmente manualmente. Neste projeto de doutorado foram realizados exemplos práticos com a utilização de ATL e OCL para implementar o mecanismo e as asserções (pré- e pós-condições), respectivamente;

\item O metamodelo SRM proporciona a interoperabilidade de refatorações para um amplo domínio. No entanto, como o SRM é baseado no KDM pode-se apenas especificar refatorações para os elementos estruturais representados pelas metaclasses contidas no KDM;  


\item O apoio computacional KDM-RE contém um módulo que provê uma DSL para auxiliar a instanciação do metamodelo SRM. Essa DSL aumenta o nível de abstração do processo de instanciação de refatorações, escondendo determinadas complexidades. O uso dessa DSL guia o engenheiro de modernização durante a especificação de uma determinada refatoração, norteando qualis metaclasses deve) ser especificadas. No entanto, a semântica de uma determinada refatoração utilizando essa DSL continua sendo responsabilidade do engenheiro de modernização;

%\item Refatorações são aplicadas apenas utilizando o diagrama de classe da UML por meio do apoio computacional KDM-RE. 

\item O apoio computacional KDM-RE foi desenvolvida para ser utilizada no ambiente de desenvolvimento Eclipse, portanto, o conhecimento sobre esse ambiente é um pré-requisito necessário para o manuseio desse apoio computacional.
\end{itemize}

\section{Sugestões de Trabalhos Futuros}\label{sec:trabalhos_futuros_tese}

Refatoração, especialmente refatoração em nível de modelo é uma disciplina relativamente nova e uma área de pesquisa ativa de acordo com o OMG~\cite{OMG_OMG, ADM:OMG}. Esta tese, considera à aplicação de refatorações no contexto da ADM e do metamodelo KDM de forma integrada, subjacente e automatizada. Assim, o trabalho desenvolvido nesta tese deve passar por vários ajustes com base na experiência substancial de aplicações práticas para obter relevância na indústria. Com base no MS conduzido e apresentado no Capítulo~\ref{chapter:mapeamento_sistematico} e o trabalho apresentado nesta tese, há muitas pesquisas que podem ser realizadas no futuro pelo autor e por outros pesquisadores do grupo de pesquisa em engenharia de software do ICMC.

Um desses trabalhos seria realizar mais estudos de caso com outros programas com o objetivo de comprovar se os mesmos resultados, ou resultados semelhantes, são obtidos. Além disso, os passos para criar refatorações para o metamodelo KDM definidos no Capítulo~\ref{chapter:catalogo_refactoring_KDM}, também seriam interessantes mais estudos de caso que envolvessem a criação de um conjunto de refatorações. Para o metamodelo SRM, também seriam importantes mais estudos de caso envolvendo uma população de novas refatorações para verificar se o mesmo pode ser utilizado para prover a utilização de forma eficiente. Também seria interessante como trabalho futuro a construção de uma ferramenta de apoio a abordagem apresentada no Capítulo~\ref{chapter:catalogo_refactoring_KDM}. Essa ferramenta poderia auxiliar no acompanhamento da abordagem, na documentação e criação de refatorações para o metamodelo KDM;

Um outro trabalho seria implementar algum mecanismo de suporte à evolução do metamodelo SRM. Isso se mostra necessário quando novas refatorações muito específicas precisam ser criadas. Também pode-se implementar uma forma de identificar automaticamente \textit{bad-smells} utilizando o metamodelo KDM como base. O apoio computacional aqui apresentado não identifica os \textit{bad-smells}, sendo de inteira responsabilidade do engenheiro de software detectar falhas de projeto.


A ferramenta KDM-RE utiliza apenas o diagrama de classe da UML como interface (\textit{front-end}) para a aplicação de refatorações, outros diagramas da UML são raramente utilizados na literatura durante atividades de refatorações. O uso combinado de múltiplos diagramas da UML durante a atividade de refatoração poderiam auxiliar o engenheiro de software a entender melhor o estrutura do sistema, bem como entender também seu comportamento. Por exemplo, além do diagrama de classe da UML, diagramas de caso de uso e sequencia poderiam ser utilizados durante a atividade de refatoração como interface. 