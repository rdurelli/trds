
\section{Contextualização}\label{sec:contexto}


Para lidar com mudanças e crescentes necessidades de negócios, sistemas de software estão em constante evolução. Atividades de evolução do software podem abranger desde a manutenção até a substituição total do sistema [1]. De acordo com [2] a manutenção de software é a atividade mais custosa no ciclo de vida do sistema de software [2]. Segundo o padrão ISO/IEC 14764, o processo de manutenção de software inclui um conjunto de tarefas necessárias para modificar o software existente, e ainda deve-se preservar sua integridade [3]. As tarefas de manutenção de software podem ser vistas como modificações incrementais que adicionam ou atualizam um conjunto de funcionalidades, ou corrigem falhas do projeto. Usualmente, com o passar do tempo, a integridade conceitual do sistema tende a diminuir, o que afeta a sua qualidade. Essa deterioração é conhecida na literatura como o fenômeno de envelhecimento [4], lidar com esse fenômeno não é uma atividade trivial e barata.

Uma técnica comum e amplamente utilizada para lidar com este problema é a aplicação de um processo de reestruturação de software com o intuito de melhorar sua estrutura e o seu \emph{design}. O processo de reestruturação de software que segue o paradigma orientado a objeto é comumente chamado de refatoração [5]. A refatoração foi primeiramente proposta por Opdyke [ref] em 1992 como uma metodologia para reestruturar programas. Seguindo a mesma linha de pensamento, pesquisadores como o Fowler tornaram a refatoração uma disciplina comumente conhecida e aplicada. De acordo com Fowler [4], a refatoração é um processo disciplinado que é utilizado para melhorar a estrutura de software, preservando seu comportamento. Com o apoio ferramental adequado, a refatoração pode ser uma maneira eficiente e eficaz para ajudar a melhorar o \emph{design} do software, tornar o software mais fácil de entender, e para auxiliar na identificação de erros. Na literatura é possível identificar um conjunto de ferramentas, automáticas ou semiautomáticas, que auxiliam na aplicação de refatorações (refs). 


Uma outra linha de pesquisa que vem se popularizando é a Engenharia de Software Dirigida à Modelos (do inglês - \sigla{MDSE}{\textit{Model-Driven Software Engineering}}). Usualmente, pesquisas encontradas na literatura sobre refatorações estão mais focadas em criar ferramentas e abordagens para refatorações que ocorrem em nível de código-fonte. No entanto, com o surgimento da MDSE aumentou o interesse e a necessidade de ferramentas de apoio à refatorações em nível de modelo. Na verdade, de acordo com Mens and Tourwe[10] a adaptação de ferramentas de refatoração para dar total apoio a aplicação de refatorações em nível de modelo pode ser de grande utilidade. Algumas motivações podem ser destacadas para a realização dessas adaptações: (\emph{i}) modelos podem fornecer uma visão abstrata do sistema; assim, visualizações de mudanças estruturais são mais fáceis de serem visualizadas e detectadas; (\emph{ii}) problemas descoberto ainda em nível de \emph{design} podem ser solucionados diretamente no modelo aumentando a produtividade; e (\emph{iii}) a capacidade de explorar caminhos alternativos de decisões é muito mais barato em nível de modelo, uma vez que não se faz necessário a alteração direta do código-fonte, fazendo com que o sistema ainda opere enquanto mudanças são testadas pelo modernizador.


O processo de refatorações em nível de modelos tende a ser mais complexo do que refatorações aplicadas em código-fonte (ref), uma vez que além das refatorações é necessário também a realização de atividades para preservar o comportamento interno do modelo, verificar a consistência do modelo, manter a sincronização do modelo e suas visões, etc. De acordo com Van Gordp, desenvolvedores de software utilizam refatorações em nível de \emph{design}, assim, é intuitivo explorar os conceitos de MDSE utilizando a UML para a aplicação de refatorações (altas refs de UML refatoracaos). Nesse sentido, vários pesquisadores iniciaram pesquisas com o objetivo de implementar refatorações no contexto da UML (refs todos DE UML). Uma das vantagens em se utilizar refatorações em nível de modelos, UML, é que os desenvolvedores de software não precisam se preocupar com características especificas de linguagens de programação (Java, C++, C\#, etc). Além disso, a utilização de um diagrama de classe fornece uma visão abstrata do sistema, assim, o engenheiro pode facilmente visualizar e verificar quais refatorações devem ser aplicadas no sistema. No entanto, de acordo com P. Van Gorp utilizar apenas diagramas da UML não é uma abordagem adequada para manter e representar todo um sistema {ref}. Por exemplo, isso ocorre principalmente porque a UML não consegue representar todas as construções de um determinado sistema, por exemplo, declarações internas de um método não são consideradas no contexto do metamodelo da UML. Além disso, a UML não contêm um conjunto de meta-classes para representar todos os artefatos de um sistema, por exemplo, com a UML não é possível representar desde os níveis mais baixos de abstração de um sistema, como o código-fonte, até níveis mais altos, como a arquitetura do sistema e regras de negócios. Adicionalmente, utilizando o metamodelo da UML poucas informações do código-fonte são representadas, por exemplo, nome da classe, nome do método e seus parâmetros, atributos e seus tipos.  

Com o objetivo de mitigar essa limitação em 2003 o \sigla{OMG}{Object Management Group} criou uma força tarefa para analisar e evoluir os tradicionais processos de reengenharia, formalizando-os e fazendo com que eles fossem totalmente apoiado por modelos~\cite{ADM:OMG}. Logo, o termo Modernização Dirigida à Arquitetura (do inglês - \sigla{ADM}{\textit{Architecture-Driven Modernization}}) surgiu como uma solução para os problemas de padronização. A ADM é um processo de modernização de sistemas legados que utiliza um conjunto de metamodelos para representar completamente um sistema por meio de diferentes representações arquiteturais. Esses modelos são então submetidos à refatorações e otimizações e posteriormente o código-fonte pode ser então gerado novamente por meio de atividades de engenharia avante (ref). Durante a modernização de um sistema são gerados vários modelos de acordo com os metamodelos da ADM, que representam diferentes partes/visões do sistema, como: fluxo de dados, banco de dados, elementos de programação (métodos, classes, tipos de dados, etc.) e arquitetura (et al., 2011a, 2011b,2011c).

O \sigla{KDM}{\textit{Knowledge Discovery Meta-model}} é o principal metamodelo da ADM com uma ampla quantidade de meta-classes para representar desde os níveis mais baixos de abstração de um sistema, como o código-fonte, até níveis mais altos (arquitetura do sistema), permitindo a representação de conceitos de qualquer domínio~\cite{KDM:specification,KDM:ISO}. A ideia principal da ADM é que a comunidade comece a desenvolver ferramentas que atuem apenas sobre instancias do metamodelo KDM, ao invés de serem dependentes de plataformas e linguagens especificas. Por exemplo, um catálogo de refatorações (CITAR eu) para o KDM tem o poder de reestruturar um sistema independentemente da linguagem de programação que foi usada em seu desenvolvimento, já que as refatorações ocorrem nos modelos.

Diferentemente de metamodelos existentes, como a UML, o KDM mantêm todas as visões/representações do sistema em uma única instância do KDM, dessa forma, o KDM pode ser considerado como uma família de metamodelos, uma vez que o mesmo compartilha a consistência e terminologia homogenia. No entanto, embora a ADM tenha investido esforços para fornecer metamodelos para auxiliar o engenheiro a conduzir a modernização de um sistema legado seguindo todas as diretrizes de MDSE, a ADM não fornece instruções de como criar, reusar e/ou aplicar refatorações no metamodelo KDM, nem mesmo como manter uma determinada instância do KDM sincronizada após a aplicação de refatorações. Assim, tais particularidades devem ser tratadas pelos modernizadores de software. Isto faz com que os desenvolvedores criem suas próprias soluções, porém, usualmente tais soluções tendem a ser proprietárias e especificas de ferramentas dificultando a interoperabilidade entre diferentes ferramentas. Assim, um dos principais objetivos dessa Tese é auxiliar e fornecer os primeiros passos para a aplicação de refatorações para o contexto da ADM e principalmente para o metamodelo KDM.



\section{Motivação}\label{sec:justificativa_e_motivacao}


No contexto da \sigla{POO}{Programação Orientada a Objeto}, refatorações são técnicas utilizadas para melhorar a estrutura do software[2]. Hoje em dia é evidente que a refatoração é de suma importância para melhorar a qualidade do código-fonte, e assim, melhorar a sua manutenibilidade. Embora a refatoração dirigidas a modelo (do inglês - \sigla{MDRef}{\textit{Model-Driven Refactoring}} tenha alcançado bastante reconhecimento e aceitação[colocar refs], ainda se faz necessário pesquisas nessa área (ref mapeamento meu e do cara). %Nesse sentido, vários pesquisadores iniciaram pesquisas com o objetivo de implementar refatorações no contexto da UML (refs todos DE UML). 
%
Embora a ADM, e principalmente o KDM tenham sido propostos para auxiliar todo o processo da modernização de sistemas, até esse momento existe uma ausência de abordagens e ferramentas que auxiliem os modernizadores de software à aplicar refatorações de forma consistente para o metamodelo KDM. Dessa forma, usualmente os modernizadores de software precisam desenvolver suas próprias ferramentas para refatorar diversos sistemas. Tais soluções geralmente tendem a serem proprietárias e consequentemente torna-se difícil a reutilização e a interoperabilidade entre ferramentas.


É comumente observado que a atividade de refatoração é pertinente a qualquer processo de modernização. Dessa forma, quando um sistema é representado utilizando diferentes visões conceituais para representar níveis de abstração do sistema (por exemplo, visão arquitetural, visão de código-fonte, visão do banco de dados, etc), um acidente comum que surge durante atividades de refatorações é a dessincronização das instâncias do metamodelo, resultando em visões inconsistente após a aplicação de uma refatoração. Dessa forma, no contexto do metamodelo KDM existe uma carência em abordagens e ferramentas que auxiliam a sincronizar tais mudanças após a aplicação de um conjunto de refatorações no KDM. Pesquisas recentes sugerem que a aplicação de técnicas de propagação de mudanças podem auxiliar na identificação e atualização de todas as instâncias/visões do KDM, permitindo assim manter todas as visões/instâncias do metamodelo KDM sincronizadas [7]–[10]. 

%Usualmente, o processo de refatorações em modelos é mais complexo do que refatoções em código-fonte (ref), uma vez que além das refatorações é necessário também a realização de atividades para verificar a consistência do modelo, manter a sincronição do modelo e suas visões, etc. Em consequência disso, poucos avanços significativos foram conseguidos em relação a definição de refatoração para o metamodelo KDM.

Dessa forma, pode-se resumir e destacar as principais motivações para aplicar refatorações em nível de modelos, principalmente utilizando o KDM:

\begin{itemize}
	\item O KDM é um metamodelo que possui um conjunto de meta-classes complementares para representar diferentes visões conceituais de um mesmo sistema, por exemplo, é possível representar desde o código-fonte até mesmo a arquitetura de um sistema utilizando um sub-conjunto de meta-classes especificas do KDM. Dessa forma, quando refatorações são aplicadas em uma visão conceitual, por exemplo, em uma instância do KDM que representa o código-fonte do sistema, todas as outras visões conceituais do sistema deveriam manter a inconsistente e a sincronização entre si. No contexto da ADM, e principalmente do metamodelo KDM existe uma carência de abordagens que sincronizam tais refatorações. Pesquisas recentes sugerem que a aplicação de técnicas de propagação de mudanças podem auxiliar na identificação e atualização de todas as instâncias/visões do KDM, permitindo assim manter todas as visões/instâncias do metamodelo KDM sincronizadas [7]–[10].  

	%Existe um relacionamento complementar entre todas as visões conceituais do metamodelo KDM. Dessa forma, quando refatorações são aplicadas em uma visão conceitual, por exemplo, em uma instância do KDM que representa o código-fonte do sistema, e não é sem considerar a sincronização e consistência de outras visões 
	\item Embora a refatoração dirigidas a modelo (do inglês - \sigla{MDRef}{\textit{Model-Driven Refactoring}} tenha alcançado bastante reconhecimento e aceitação, até o presente momento apenas trabalhos desenvolvido no contexto do metamodelo \sigla{UML}{\textit{Unified Modeling Language}} foram encontrados e consolidados na literatura (ref mapeamento meu e do cara). Dessa forma, existe uma ausência de abordagens para o metamodelo KDM, uma vez que o mesmo é um metamodelo mais novo quando comparado a UML. A refatoração de um sistema em geral tende a ser uma atividade complexa; modificações manuais, sem qualquer catálogo de refatoração, bem como um ambiente integrado, pode resultar em efeitos colaterais indesejados e acarretar em um processo tedioso. Neste contexto, existe uma necessidade para elaborar/adaptar um catálogo de refatoração para o metamodelo KDM e também criar um ambiente que auxilie a aplicação de refatorações diretamente no KDM, assim, os engenheiros podem reduzir o tempo e esforço durante a refatoração de sistemas legados e ainda respeitando a interoperabilidade fornecida pelo metamodelo KDM.

	\item Na literatura é possível identificar um conjunto de refatorações já validadas e que são usualmente aplicadas em código-fonte, por exemplo, \textit{Extract Class}, \textit{Move Method}, \textit{Move Attribute}, etc. Essas são apenas alguns exemplos de refatorações úteis que não são facilmente reutilizadas na prática durante a condução de modernização de um determinado sistema (fowler, ADM refactoring). Essa limitação pode ser atribuída devido a ausência de um meio padronizado de disponibilizar refatorações. Embora a ADM forneça um conjunto de metamodelos para auxiliar o engenheiro de software a conduzir MDRE até esse momento a ADM não provê instruções para auxiliar o engenheiro a promover o reuso de refatorações juntamente com os seus metamodelos padronizados (por exemplo, KDM) durante o processo de modernização. Essa limitação faz com que o engenheiro crie suas próprias soluções/refatorações, resultando em um possível atraso no processo de modernização. Contudo, as soluções/refatorações  definidas não são facilmente reutilizadas pois tendem a ser proprietárias. Uma abordagem promissora é lidar com a refatoração de forma independente da linguagem – aumentando assim as possibilidades de reutilização de refatorações. Dessa forna, existe uma necessidade de criar uma metamodelo para auxiliar o engenheiro de modernização a promover o reuso de refatorações no contexto da ADM e principalmente de forma integrada com o metamodelo KDM. 

\end{itemize}

\section{Objetivos}\label{sec:objetivos}

Como salientado anteriormente, embora pesquisas sobre \sigla{MDRef}{\textit{Model-Driven Refactoring}} possam ser identificadas (ref UML) na literatura, até este momento nenhum trabalho ou esforço têm sido investigado para auxiliar a OMG e a ADM a definir soluções e padronizações para a aplicação de refatorações para o metamodelo KDM. Em 2012 a ADM realizou um \emph{call for proposal} onde o principal objetivo era a necessidade da criação de soluções e padronizações para auxiliar a aplicação de refatorações para o metamodelo KDM. Dessa forma, o objetivo mais amplo dessa tese de doutorado é preencher a ausência de abordagens, padronizações e ferramentas que auxiliem a ADM na definição de refatorações para o contexto do metamodelo KDM. Assim, esta tese de doutorado aborda as seguintes questões de pesquisas:

\begin{itemize}
 	\item Qual é o estado da arte de refatorações no contexto da ADM, e principalmente para o metamodelo KDM?;
 	\item Como especificar refatorações para que elas possam ser reutilizadas de forma independente de linguagem e plataforma de programação?;
 	\item Como automatizar o processo de aplicação de refatorações dentro do contexto da ADM e KDM?;
 	\item Como manter todas as visões do metamodelo KDM sincronizado após a aplicação de um conjunto de refatorações?
 \end{itemize} 


A tese subjacente a este trabalho é de que é possível e benéfico o uso de refatorações para o contexto da ADM, principalmente para o metamodelo KDM. Além disso, pretende-se viabilizar a reutilização e padronizações de refatorações por meio da utilização de um metamodelo de refatorações. Adicionalmente, planeja-se verificar a possibilidade de manter todas as visões do metamodelo KDM sincronizada e consistentes após a aplicação de um conjunto de refatorações. %e de técnicas de propagação de mudanças para manter todas as visões do KDM sincronizadas e consistêntes após a aplicação de uma refatoração. 
Neste contexto, esta tese de doutorado cobre os seguintes aspectos: 


\begin{itemize}
	\item a criação/adaptação de um conjunto de refatorações para o metamodelo KDM;
	\item uma ferramenta totalmente integrada para auxiliar o engenheiro de modernização a aplicar um conjunto de refatorações de forma gráfica, por meio da utilização de diagramas;
	\item a definição, criação e integração de um metamodelo padronizado de refatoração para auxiliar engenheiros de modernização a compartilhar, criar e reutilizar refatorações no contexto da ADM e KDM;
	\item a elaboração de uma linguagem específica de domínio (do inglês - \sigla{DSL}{\textit{Domain-Specific Language}}). Essa DSL possui duas finalidades, a saber: (\textit{i}) auxiliar o engenheiro de modernização a instanciar o metamodelo de refatoração proposto e (\textit{ii}) facilitar a criação de um conjunto de refatorações de forma guiada e automática;
	\item a definição de um ambiente \emph{Web} para também auxiliar a instanciação do metamodelo de refatoração proposto;
	\item a concepção de um repositório totalmente integrado com a ferramenta e com o ambiente \emph{web} para facilitar o compartilhamento e o reuso de refatorações que estão em conformidade com o metamodelo de refatoração proposto.
\end{itemize}


\section{Contribuições}\label{sec:contribuicoes}

A principal contribuição dessa tese de doutorado é entender como refatorações tradicionais, ou seja, refatorações comumente utilizadas em sistemas implementados com o paradigma orientado a objeto, podem ser adaptadas, aplicadas, padronizadas e reutilizadas no contexto da ADM e principalmente para o metamodelo KDM. Em outras palavras, esta pesquisa pode ser entendida como uma incursão inicial para auxiliar a OMG e a ADM na definição de padronizações e soluções ferramentais para facilitar o engenheiro de modernização durante o uso de refatorações para o KDM. Pontualmente, as principais contribuições dessa tese são:

\begin{itemize}
	\item um ambiente para auxiliar o engenheiro de modernização durante a aplicação de refatorações para o metamodelo KDM;
	\item a investigação e definição de um metamodelo de refatoração para auxilar os engenheiros de modernização a compartilhar, criar e reutilizar refatorações no contexto da ADM e KDM;
	\item a definição de uma DSL para (\textit{i}) auxiliar o engenheiro de modernização a instanciar o metamodelo de refatoração proposto e (\textit{ii}) facilitar a criação de um conjunto de refatorações de forma guiada e automática;
	\item a elaboração de um ambiente \emph{Web} para também auxiliar a instanciação do metamodelo de refatoração proposto;
	\item a concepção de um repositório totalmente integrado com a ferramenta e com o ambiente \emph{web} para facilitar o compartilhamento e o reuso de refatorações que estão em conformidade com o metamodelo de refatoração proposto.
\end{itemize}
    
\section{Convenções adotadas nesta Tese}\label{sec:convencoes}

Ao longo desta Tese, \textit{Itálico} é utilizado para dar ênfases, introduzir novos termos e para destacar palavras em inglês. \texttt{Typewriter} é utilizado para operador Java, operador da DSL, palavras chaves, nome de métodos, variáveis e URL que aparecem no texto. Símbolos \ding{202}, \ding{203}, \ding{204}, \ding{205} ou \textcircled{a}, \textcircled{b}, \textcircled{c}, \textcircled{d}, são utilizados para chamar a atenção do leitor para informações importantes em figuras e códigos.

\section{Grupo de Pesquisa}

Esse trabalho é uma contribuição para o grupo de pesquisa do Departamento de Ciência de Computação e Estatística do Instituto de Ciências Matemáticas e de Computação (ICMC) da Universidade de São Paulo (campus São Carlos/SP). Além disso, esse trabalho também foi conduzido em parceria com o grupo de pesquisa AdvanSE (\textit{Advanced Research on Software Engineering}), da Universidade Federal de São Carlos (UFSCar). O grupo possui pesquisas em andamento sobre extensões, refatorações, mineração, métricas e validações de arquitetura utilizando a ADM e o metamodelo KDM.

\section{Estrutura da Tese}

......... A tese....