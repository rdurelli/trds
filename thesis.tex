%!TEX TS-program = pdflatex
% ------------------------------------------------------------------------
% ------------------------------------------------------------------------
% ICMC: Modelo de Trabalho Acadêmico (tese de doutorado, dissertação de
% mestrado e trabalhos monográficos em geral) em conformidade com 
% ABNT NBR 14724:2011: Informação e documentação - Trabalhos acadêmicos -
% Apresentação
% ------------------------------------------------------------------------
% ------------------------------------------------------------------------

% Opções: 
%   Qualificação         = qualificacao 
%   Curso                = doutorado/mestrado
%   Situação do trabalho = pre-defesa/pos-defesa (exceto para qualificação)
% -- opções do pacote babel --
% Idioma padrão = brazil
    %spanish,           % idioma adicional para hifenização
    %english,           % idioma adicional para hifenização
    %brazil             % o último idioma é o principal do documento
\documentclass[doutorado, pre-defesa, spanish, english, brazil]{packages/icmc}

% ---
% Pacotes Opcionais
% ---
\usepackage{rotating}           % Usado para rotacionar o texto
\usepackage{pifont}
\usepackage[all,knot,arc,import,poly]{xy}   % Pacote para desenhos gráficos
% Este pacote pode conflitar com outros pacotes gráficos como o ``pictex''
% Então é necessário usar apenas um dos pacotes conflitantes
\usepackage[colorinlistoftodos]{todonotes}
\usepackage{array}
\usepackage{multirow}
\usepackage{amsmath}
\usepackage{rotating}
\usepackage{longtable}
\usepackage{xcolor}
\usepackage{lipsum}                     % Dummytext
\usepackage{subfigure}


%---------------------------



\usepackage{xargs}                      % Use more than one optional parameter in a new commands
\usepackage[pdftex,dvipsnames]{xcolor}  % Coloured text etc.
% 
\usepackage[colorinlistoftodos,prependcaption,textsize=tiny]{todonotes}
\newcommandx{\unsure}[2][1=]{\todo[linecolor=red,backgroundcolor=red!25,bordercolor=red,#1]{#2}}
\newcommandx{\change}[2][1=]{\todo[linecolor=blue,backgroundcolor=blue!25,bordercolor=blue,#1]{#2}}
\newcommandx{\info}[2][1=]{\todo[linecolor=OliveGreen,backgroundcolor=OliveGreen!25,bordercolor=OliveGreen,#1]{#2}}
\newcommandx{\improvement}[2][1=]{\todo[linecolor=Plum,backgroundcolor=Plum!25,bordercolor=Plum,#1]{#2}}
\newcommandx{\thiswillnotshow}[2][1=]{\todo[disable,#1]{#2}}


\newcommand{\VerbL}{0.52\textwidth}
\newcommand{\LatL}{0.42\textwidth} 

\lstdefinelanguage{ATL}{
  morekeywords={true,false,
   Bag,Set,OrderedSet,Sequence,Tuple,Integer,Real,Boolean,String,TupleType,
  not,and,or,xor,implies,module,create,from,uses,helper,def,context,
  rule,using,derived,to,mapsTo,distinct,
  foreach,in,do,if,then,else,endif,let,
  library,query,for,div,refining,entrypoint, drop, thisModule},
 keywordstyle=[2]{\textbf},
 morecomment=[l]{--},
 morestring=[b]{'}}
 
\lstdefinelanguage{Xtext}{
 morekeywords={grammar, with, hidden, generate, as, import, ID, , STRING, returns, current, terminal, enum},
 keywordstyle=[2]{\textbf},
 morecomment=[l]{//}, 
 morecomment=[s]{/*}{*/}, 
 morestring=[b]",
 tabsize=4}
% ----------------------------------------------------------------
% Informações de dados para CAPA e FOLHA DE ROSTO
% ---An infrastructure for Creating KDM refactorings that deals with static synchronization and behavior preservation
\titulo{Um abordagem para a criação, adaptação e uso de refatorações para o metamodelo KDM.}
\autor[Durelli, R. S.]{Rafael Serapilha Durelli}
\orientador{Prof. Dr.}{Márcio Eduardo Delamaro}
%\coorientador{Prof. Dr.}{Fulano de Tal}
\curso{CCMC}
\data{5}{5}{2015} % Data do depósito 
% ---


% ---
% RESUMOS
% ---

% Resumo em português
% conter no máximo 500 palavras
\textoresumo{
A Modernização Dirigida à Arquitetura (do inglês - \sigla{ADM}{\textit{Architecture-Driven Modernization}}) tem como objetivo estabelecer metamodelos padronizados para auxiliar todo o processo da reengenharia de software, como por exemplo o \sigla{KDM}{\emph{Knowledge Discovery Meta-model}}. %Este metamodelo, padronizado, é utilizado para representar artefatos de software, seus elementos, associações, e ambientes operacionais. Embora a refatoração tenha alcançado bastante reconhecimento e aceitação na literatura, refatorações para modelos tende a ser mais multifacetada e desafiadora do que refatorações tradicionais. 
Até o presente momento existe uma carência de abordagens e ferramentas que auxiliam todo o processo de refatoração para a ADM e principalmente para o metamodelo KDM. Nesta dissertação defende-se a tese de que refatorações podem ser aplicadas no contexto da ADM e KDM. Para demonstrar esta tese, é apresentada uma abordagem para a criação, adaptação e uso de refatorações para o KDM. A abordagem possui dois principais passos: (\textit{i}) o primeiro passo consiste na definição de diretrizes que guiam o engenheiro de modernização durante a criação e adaptação de refatorações para o KDM; e (\textit{ii}) o segundo passo consiste na disponibilização de refatorações por meio de um metamodelo. Um apoio computacional, denominado KDM-RE, também foi desenvolvida para apoiar o uso da abordagem, fornecendo editores para a aplicação de refatorações, sincronização dos artefatos e reutilização de refatorações. Foi realizado dois experimentos que buscaram reproduzir os cenários em que engenheiros de modernização tivessem que realizar refatorações em um modelo KDM. Os resultados mostram que a abordagem pode trazer benefícios para o engenheiro de modernização durante a aplicação de refatorações em relação ao atributo de qualidade \sigla{QMOOD}{\textit{Quality Model for Object-Oriented Design}}.


%apresenta uma abordagem que consiste de três principais passos: (\textit{i}) o primeiro passo, consiste na definição de diretrizes para auxiliar o engenheiro de modernização na criação e adaptação de refatorações para o KDM; (\textit{ii}) o segundo passo consiste na definição de regras de propagações que são realizadas em instância do metamodelo KDM para manter todos os artefatos sincronizados e consistentes de acordo com a refatoração aplicada; e (\textit{iii}) o terceiro passo da abordagem consiste na reutilização de refatorações por meio de um metamodelo de refatorações. Uma ferramenta, denominada KDM-RE, também foi desenvolvida para apoiar o uso da abordagem, fornecendo editores para a aplicação de refatorações, sincronização dos artefatos e reutilização de refatorações. Foi realizado dois experimentos que buscaram reproduzir os cenários em que engenheiros de modernização tivessem que realizar refatorações em um modelo KDM. Os dados dos experimentos foram avaliados em relação ao atributo de qualidade \sigla{QMOOD}{\textit{Quality Model for Object-Oriented Design}}.


%Além de viável a aplicação refatoração, sincronização dos artefatos e reutilização de refatoração para o metamodelo KDM, experimentos realizados ao longo deste projeto mostraram que a abordagem torna o 


%O objetivo é permitir a interoperabilidade de refatorações para um amplo domínio e auxiliar o engenheiro de modernização a definir refatorações representativas em forma de metadados.


%O KDM é um metamodelo que possui um conjunto de meta-classes complementares para representar diferentes visões conceituais de um mesmo sistema. Assim, o segundo passo da abordagem se preocupa em sincronizar e propagar as refatorações por todo o KDM para manter sua instância consistente. 




%são elaboradas/adaptadas catálogos de refatoração para o meta-modelo KDM, além disso, também é criado um ambiente que auxilie a aplicação de refatorações diretamente no KDM, assim, os engenheiros podem reduzir o tempo e esforço durante a refatoração de sistemas legados. 
%
%O KDM é um meta-modelo que possui um conjunto de meta-classes complementares para representar diferentes visões conceituais de um mesmo sistema, por exemplo, é possível representar desde o código-fonte até mesmo a arquitetura de um sistema utilizando um sub-conjunto de meta-classes especificas do KDM. Dessa forma, quando refatorações são aplicadas em uma visão conceitual, por exemplo, em uma instância do KDM que representa o código-fonte do sistema, todas as outras visões conceituais do sistema deveriam manter a consistente e a sincronização entre sí. No contexto da ADM, e principalmente do meta-modelo KDM existe uma carência em abordagens que sincronizam tais mudanças (ou seja, refatorações) de forma automática. Dessa forma, nesta Tese técnica de propagação de mudanças para auxiliar na identificação e atualização de todas as instâncias/visões do KDM é proposta, assim, é possível manter todas as visões/instâncias do meta-modelo KDM sincronizadas e consistentes.
%
%Durante a condução da Tese foi percebido que a ADM fornece um conjunto de meta-modelos para auxiliar o engenheiro de software a conduzir MDRE. Porém, a ADM não provê instruções ou ate mesmo um meta-modelo, para auxiliar o engenheiro a promover o reuso de refatorações juntamente com os seus meta-modelos padronizados (por exemplo, KDM) durante o processo de modernização. Essa limitação faz com que o engenheiro crie suas próprias soluções/refatorações, resultando em um possível atraso no processo de modernização. Contudo, as soluções/refatorações definidas não são facilmente reutilizadas. Dessa forna, um meta-modelo para auxiliar o engenheiro de modernização a promover o reuso de refatorações no contexto do meta-modelo KDM também é proposto. Posteriormente, estudos de casos e experimentos são apresentados e discutidos.
}{ADM, KDM, Engenheiro de Modernização, Experimento,}

% ---
% resumo em inglês
% ---
\textoresumo[english]{
    This paper is a brief model for writing qualification monographs, dissertations and thesis using \LaTeX environment, in accordance with the standards required by the Institute of Mathematics and Computer Sciences (ICMC), University of São Paulo (USP). For making this model, the latest version (1.9.2) \textit{abnTeX2} classes package was used. This package follow the rules of the Brazilian Association of Technical Standards. A drafting a monograph, dissertation or thesis can be done by overwriting the contents of this model.
    }{Template, Qualification monograph, Dissertation, Thesis, Latex}

% ---
% resumo em espanhol
% ---
%\textoresumo[spanish]{
 %   Este trabajo es un breve modelo para la redacción de monografías de cualificación, disertaciones y tesis utilizando el ambiente LaTeX, de acuerdo a las normas exigidas por el Instituto de Ciencias Matemáticas y Computación (ICMC) de la Universidad de São Paulo (USP). Para la confección de este modelo fue utilizada la última versión (1.9.1) del paquete de clases abnTeX2 que sigue las normas de la Asociación Brasilera de Normas Técnicas. La elaboración de una monografía, disertación o tesis puede ser realizada reescribiendo el contenido de este modelo.
%    }{Template, Monografías de cualificación, Disertacion, Tesis, Latex}
    
    
% ---
% Configurações de aparência do PDF final
% ---
% alterando o aspecto da cor azul
\definecolor{blue}{RGB}{41,5,195}

% informações do PDF
\makeatletter
\hypersetup{
        pagebackref=true,
        pdftitle={\@title}, 
        pdfauthor={\@author},
        pdfsubject={\imprimirpreambulo},
        pdfcreator={LaTeX with abnTeX2/ICMC-USP},
        pdfkeywords={\palavraschave}, 
        colorlinks=true,            % false: boxed links; true: colored links
        linkcolor=blue,             % color of internal links
        citecolor=blue,             % color of links to bibliography
        filecolor=magenta,          % color of file links
        urlcolor=blue,
        bookmarksdepth=4
}
\makeatother
% --- 

% ----------------------------------------------------------
% ELEMENTOS PRÉ-TEXTUAIS
% ----------------------------------------------------------

% Inserir a ficha catalográfica
%\incluifichacatalografica*{tex/fichaCatalografica.pdf}
\incluifichacatalografica{634} % Código Cutter: número atribuído ao sobrenome do autor. Para obtê-lo, consulte a tabela Cutter Sanborn (em http://www.davignon.qc.ca/cutter1.html), procure pelo sobrenome ou forma mais próxima ao sobrenome completo e coloque o número indicado como parâmetro.


% DEDICATÓRIA / AGRADECIMENTO / EPÍGRAFE
\textodedicatoria*{tex/pre-textual/dedicatoria}
\textoagradecimentos*{tex/pre-textual/agradecimentos}
\textoepigrafe*{tex/pre-textual/epigrafe}

% Inclui a lista de figuras
\incluilistadefiguras

% Inclui a lista de tabelas
\incluilistadetabelas

% Inclui a lista de quadros
\incluilistadequadros

% Inclui a lista de algoritmos
\incluilistadealgoritmos

% Inclui a lista de códigos
\incluilistadecodigos

% Inclui a lista de siglas e abreviaturas
\incluilistadesiglas

% Inclui a lista de símbolos
\incluilistadesimbolos

% ----
% Início do documento
% ----
\begin{document}
% ----------------------------------------------------------
% ELEMENTOS TEXTUAIS
% ----------------------------------------------------------
\textual

\chapter{Introdução}
\label{chapter:introducao}

\section{Contextualização}\label{sec:contexto}


Para lidar com mudanças e crescentes necessidades de negócios, sistemas de software estão em constante evolução. Atividades de evolução do software podem abranger desde a manutenção até a substituição total do sistema~\cite{seacord_2003}. De acordo com~\citeonline{Lientz_1978} a manutenção de software é a atividade mais custosa no ciclo de vida do sistema de software. O processo de manutenção de software inclui um conjunto de tarefas necessárias para modificar o software existente, e ainda deve-se preservar sua integridade. As tarefas de manutenção de software podem ser vistas como modificações incrementais que adicionam ou atualizam um conjunto de funcionalidades, ou corrigem falhas do projeto. Usualmente, com o passar do tempo, a integridade conceitual do sistema tende a diminuir, o que afeta a sua qualidade. Essa deterioração é conhecida na literatura como o fenômeno de envelhecimento~\cite{Fowler1999}, lidar com esse fenômeno não é uma atividade trivial e barata.

Uma técnica comum e amplamente utilizada para lidar com este problema é a aplicação de um processo de reestruturação de software com o intuito de melhorar sua estrutura e o seu \emph{design}. O processo de reestruturação de software que segue o paradigma orientado a objeto é comumente chamado de refatoração~\cite{OPDYKE_1992, Fowler1999, Mens04}. A refatoração foi primeiramente proposta por~\citeonline{OPDYKE_1992} como uma metodologia para reestruturar programas. Seguindo a mesma linha de pensamento, pesquisadores como o \citeonline{Fowler1999} tornaram a refatoração uma disciplina comumente conhecida e aplicada. A refatoração é um processo disciplinado que é utilizado para melhorar a estrutura de software, preservando seu comportamento~\cite{Fowler1999}. Com o apoio ferramental adequado, a refatoração pode ser uma maneira eficiente e eficaz para ajudar a melhorar o \emph{design} do software, tornar o software mais fácil de entender, e para auxiliar na identificação de erros. %Na literatura é possível identificar um conjunto de ferramentas, automáticas ou semiautomáticas, que auxiliam na aplicação de refatorações (refs). 


Outra linha de pesquisa que vem se popularizando é a Engenharia Dirigida por Modelos (do inglês - \sigla{MDD}{\textit{Model-Driven Development}}). Usualmente, pesquisas encontradas na literatura sobre refatorações estão mais focadas em criar ferramentas e abordagens para refatorações que ocorrem em nível de código-fonte. No entanto, com o surgimento da MDD aumentou o interesse e a necessidade de ferramentas de apoio à refatorações em nível de modelo. Na verdade, de acordo com~\citeonline{Mens_and_Tourwe_2004} a adaptação de ferramentas de refatoração para fornecer apoio à aplicação de refatorações em nível de modelo pode ser de grande utilidade. Algumas motivações podem ser destacadas para a realização dessas adaptações: (\emph{i}) modelos podem fornecer uma visão abstrata do sistema; assim, visualizações de mudanças estruturais são mais fáceis de serem visualizadas e detectadas; (\emph{ii}) problemas descoberto ainda em nível de \emph{design} podem ser solucionados diretamente no modelo aumentando a produtividade; e (\emph{iii}) a capacidade de explorar caminhos alternativos de decisões é muito mais barato em nível de modelo, uma vez que não se faz necessário a alteração direta do código-fonte, fazendo com que o sistema ainda opere enquanto mudanças são testadas pelo modernizador.


O processo de refatorações em nível de modelos tende a ser mais complexo do que refatorações aplicadas em código-fonte~\cite{Mens_2006}, uma vez que além das refatorações é necessária também a realização de atividades para verificar a consistência do modelo, manter a sincronização do modelo e suas visões, etc~\cite{KolahdouzRahimi20145}. Segundo~\citeonline{Gorp}, desenvolvedores de software utilizam refatorações em nível de \emph{design}, assim, é intuitivo explorar os conceitos de MDD utilizando a UML para a aplicação de refatorações~\cite{Salem_2008, Gorp, Egyed_2008, Briand_2006, staron2004implementing}. Nesse sentido, vários pesquisadores iniciaram pesquisas com o objetivo de implementar refatorações no contexto da UML~\cite{revisao_sistematica_uml_refactoring}. Uma das vantagens em se utilizar refatorações em nível de modelos, tais como a UML, é que os desenvolvedores de software não precisam se preocupar com características especificas de linguagens de programação (Java, C++, C\#, etc). Além disso, a utilização de um diagrama de classe fornece uma visão abstrata do sistema, assim, o engenheiro pode facilmente visualizar e verificar quais refatorações devem ser aplicadas no sistema. 

No entanto, de acordo com~\citeonline{Gorp} utilizar apenas os diagramas da UML não é uma abordagem adequada para manter e representar todos os artefatos de um sistema de software. Isso ocorre principalmente porque a UML não consegue representar todas as construções de um determinado sistema, por exemplo, declarações internas de um método não são consideradas no contexto do metamodelo da UML. Além disso, a UML não contém um conjunto de metaclasses para representar todos os artefatos de um sistema. Com a UML não é possível representar níveis mais baixos de abstração de um sistema, como o código-fonte, nem mesmo representar níveis mais altos, como a arquitetura do sistema e regras de negócios. Utilizando o metamodelo da UML poucas informações do código-fonte são representadas, por exemplo, nome da classe, nome do método e seus parâmetros, atributos e seus tipos.  

Com o objetivo de mitigar essa limitação em 2003 o \textit{Object Management Group} (OMG) criou uma força tarefa para analisar e evoluir os tradicionais processos de reengenharia, formalizando-os e fazendo com que eles fossem totalmente apoiados por modelos~\cite{ADM:OMG}. Logo, o termo Modernização Dirigida à Arquitetura (do inglês - \sigla{ADM}{\textit{Architecture-Driven Modernization}}) surgiu como uma solução para os problemas de padronização. A ADM é um processo de modernização de sistemas legados que utiliza um conjunto de metamodelos para representar completamente um sistema por meio de diferentes representações arquiteturais. Esses modelos são então submetidos à refatorações, otimizações e posteriormente o código-fonte pode ser então gerado novamente por meio de atividades de engenharia avante. Durante a modernização de um sistema são gerados vários modelos de acordo com os metamodelos da ADM, que representam diferentes partes/visões do sistema, como: fluxo de dados, banco de dados, elementos de programação (métodos, classes, tipos de dados, etc.) e arquitetura~\cite{PerezCastillo20121370}.

O Knowledge Discovery Metamodel (KDM) é o principal metamodelo da ADM com uma ampla quantidade de metaclasses para representar desde os níveis mais baixos de abstração de um sistema (código-fonte), até níveis mais altos (arquitetura do sistema), permitindo a representação de conceitos de qualquer domínio~\cite{KDM:specification,KDM:ISO}. Diferentemente de metamodelos existentes, como a UML, o KDM mantêm todas as visões/representações do sistema em uma única instância do KDM; o KDM pode ser considerado como uma família de metamodelos, uma vez que o mesmo compartilha a consistência e terminologia homogenia. A ideia principal da ADM é que a comunidade comece a desenvolver ferramentas que atuem apenas sobre instancias do metamodelo KDM, ao invés de serem dependentes de plataformas e linguagens especificas. Por exemplo, um catálogo de refatorações~\cite{durelli_catalogo} para o KDM tem o poder de reestruturar um sistema independentemente da linguagem de programação que foi usada em seu desenvolvimento, já que as refatorações ocorrem nos modelos. Outro exemplo seria a aplicação de técnicas de mineração de interesses transversais utilizando como base o metamodelo KDM~\cite{Durelli:2013_ACM, dani_san, daniel_san_journal}.


No entanto, embora a ADM tenha investido esforços para fornecer metamodelos para auxiliar o engenheiro de modernização a conduzir a reengenharia de um sistema seguindo todas as diretrizes de MDD, a ADM ainda não fornece instruções de como criar, reusar e/ou aplicar refatorações no metamodelo KDM, nem mesmo como manter uma determinada instância do KDM sincronizada após a aplicação de refatorações. Assim, tais particularidades devem ser tratadas pelos engenheiros de modernização. Isto faz com que os engenheiros criem suas próprias soluções. Porém, usualmente tais soluções são proprietárias e especificas o que pode dificultar a interoperabilidade entre diferentes soluções. Dessa forma, nesta Tese é apresentada uma abordagem para criação e disponibilização de refatorações para o metamodelo KDM, bem como um apoio ferramental que permite aplicá-las em diagramas de classe da UML. 

%Esta tese de doutorado aborda as seguintes questões de pesquisas (QP):

%\begin{itemize}
% 	\item \textbf{QP$_1$}: Qual é o estado da arte de refatorações no contexto da ADM, e principalmente para o metamodelo KDM?;
% 	\item \textbf{QP$_2$}: Como manter todas as visões do metamodelo KDM sincronizado após a aplicação de um conjunto de refatorações?;
 %	\item \textbf{QP$_3$}: Como especificar refatorações para que elas possam ser disponibilizadas/reutilizadas de forma independente de linguagem e plataforma de programação?;
 %	\item \textbf{QP$_4$}: Como automatizar o processo de aplicação de refatorações dentro do contexto da ADM e KDM?.
 %\end{itemize} 


\section{Motivações}\label{sec:justificativa_e_motivacao}

%Refatorações são técnicas utilizadas para melhorar a estrutura do software~\cite{Fowler1999}. Hoje em dia é evidente que a refatoração é de suma importância para melhorar a qualidade do código-fonte. Embora a refatoração para modelos tenha alcançado bastante reconhecimento e aceitação~\cite{Salem_2008, Gorp, Egyed_2008, Briand_2006, staron2004implementing}, ainda se faz necessárias pesquisas nessa área~\cite{revisao_sistematica_uml_refactoring, durelli_systematic_mapping}. 

%Embora a ADM e o KDM tenham sido propostos para auxiliar todo o processo da modernização de sistemas por meio de modelos, ainda hoje existe uma ausência de abordagens e apoios computacionais para auxiliar os engenheiros de modernização durante a aplicação de refatorações de forma consistente em instâncias do metamodelo KDM\footnote{É importante ressaltar que para facilitar a leitura e o entendimento  desta Tese em vários pontos o termo \aspas{refatorações para o KDM} será utilizada de forma intercambiável a \aspas{refatorações aplicadas em instâncias do metamodelo KDM}.}. Dessa forma, usualmente os engenheiros de modernização precisam desenvolver suas próprias abordagens e apoios ferramentais para refatorar diversos sistemas. Tais soluções tendem a ser proprietárias e consequentemente tornam-se difíceis de serem reutilizadas e dificulta a interoperabilidade entre outras ferramentas. Assim, é possível concluir que este é um campo atual e promissor para novas pesquisas. Nesse contexto, a pesquisa apresentada nesta tese visa a contribuir para a criação de abordagens e apoios computacionais para auxiliar os engenheiros de software durante a aplicação de refatorações para o KDM. No contexto desta Tese, as principais motivações são:

As motivações que levaram ao desenvolvimento desta Tese foram:

\begin{enumerate}

\item A escassez de abordagens para criar refatorações para o KDM. Isto é, quais são as diretrizes que permitem o engenheiro de modernização criar refatorações tradicionais para o metamodelo KDM. A ausência de diretrizes desse tipo dificulta a criação de refatorações para o KDM, fazendo com que engenheiros de modernização tenham que criar suas próprias soluções;

\item Ausência de uma infraestrutura de suporte ao reúso de refatorações para o KDM. A ausência de uma infraestrutura desse tipo faz com que engenheiros de modernização criem suas próprias soluções para especificar, armazenar e disponibilizar refatorações, dificultando bastante o reúso delas em outros contextos. 



%\item Ausência de um metamodelo que reúna as principais informações relacionadas com refatorações. Refatorações são disponibilizadas por meio de linguagem natural~\cite{Fowler1999}, porém não são facilmente reutilizadas. Uma abordagem promissora é lidar com a refatoração de forma independente da linguagem – aumentando assim as possibilidades de reutilização das refatorações;

%\item Embora a ADM forneça um conjunto de metamodelos para auxiliar o engenheiro de modernização a conduzir modernização até esse momento a ADM não provê instruções para auxiliar o engenheiro a promover o reúso de refatorações juntamente com os seus metamodelos padronizados (por exemplo, KDM) durante o processo de modernização. Essa limitação faz com que o engenheiro de modernização crie suas próprias soluções/refatorações, resultando em um possível atraso no processo de modernização. Contudo, as soluções/refatorações  definidas não são facilmente reutilizadas, pois tendem a ser proprietárias e específicas de linguagem e plataforma. Uma solução promissora é lidar com a refatoração de forma independente da linguagem – aumentando assim as possibilidades de reutilização de refatorações. Dessa forma, existe a motivação de criar uma metamodelo para auxiliar o engenheiro de modernização a promover o reúso de refatorações no contexto da ADM e principalmente de forma integrada com o metamodelo KDM. 


\item Carência de apoios computacionais efetivos para auxiliar o engenheiro de software durante a aplicação, disponibilização e reúso de refatorações para o KDM. A escassez de apoio computacional efetivo pode também dificultar a condução adequada de refatorações para o KDM.

\end{enumerate}

Nesse contexto, a pesquisa apresentada nesta tese visa a contribuir para a definição de uma abordagem para a criação, disponibilização e aplicação de refatorações para o KDM.


%É comumente observado que a atividade de refatoração é pertinente a qualquer processo de modernização. Dessa forma, quando um sistema é representado utilizando diferentes visões conceituais para representar níveis de abstração do sistema (por exemplo, visão arquitetural, visão de código-fonte, visão do banco de dados, etc), um acidente comum que surge durante atividades de refatorações é a dessincronização das instâncias do metamodelo, resultando em visões inconsistente após a aplicação de uma refatoração. Dessa forma, no contexto do metamodelo KDM existe uma carência em abordagens e ferramentas que auxiliam a sincronizar tais mudanças após a aplicação de um conjunto de refatorações no KDM. Pesquisas recentes sugerem que a aplicação de técnicas de propagação de mudanças podem auxiliar na identificação e atualização de todas as instâncias/visões do KDM, permitindo assim manter todas as visões/instâncias do metamodelo KDM sincronizadas [7]–[10]. 
%Usualmente, o processo de refatorações em modelos é mais complexo do que refatoções em código-fonte (ref), uma vez que além das refatorações é necessário também a realização de atividades para verificar a consistência do modelo, manter a sincronição do modelo e suas visões, etc. Em consequência disso, poucos avanços significativos foram conseguidos em relação a definição de refatoração para o metamodelo KDM.
%
%Pode-se destacar as principais motivações para aplicar refatorações em nível de modelos, principalmente utilizando o KDM:

%\begin{itemize}

    %\item Embora a refatoração para modelos tenha alcançado bastante reconhecimento e aceitação, até o presente momento apenas trabalhos desenvolvido no contexto do metamodelo \sigla{UML}{\textit{Unified Modeling Language}} foram encontrados e consolidados na literatura~\cite{revisao_sistematica_uml_refactoring, durelli_systematic_mapping}. 
    
 %   \item Ausência de abordagens para adaptar e especificar refatorações para o KDM. %Até o presente momento apenas trabalhos desenvolvidos no contexto metamodelo \sigla{UML}{\textit{Unified Modeling Language}} foram encontrados e consolidados na literatura~\cite{revisao_sistematica_uml_refactoring, durelli_systematic_mapping}. Neste contexto, existe uma motivação para criar diretrizes de como adaptar refatorações para o KDM.
    
    %Dessa forma, existe uma ausência de abordagens para o metamodelo KDM, uma vez que o mesmo é um metamodelo mais novo quando comparado a UML. A refatoração de um sistema em geral tende a ser uma atividade complexa; modificações manuais, sem qualquer catálogo de refatoração, bem como um ambiente integrado, pode resultar em efeitos colaterais indesejados e acarretar em um processo tedioso. Neste contexto, existe uma necessidade para elaborar/adaptar um catálogo de refatoração para o metamodelo KDM e também criar um ambiente que auxilie a aplicação de refatorações diretamente no KDM, assim, os engenheiros podem reduzir o tempo e esforço durante a refatoração de sistemas legados e ainda respeitando a interoperabilidade fornecida pelo metamodelo KDM.

%	\item Carência de abordagens e apoios computacionais para manter o metamodelo KDM consistente e sincronizado após a aplicação de refatorações;
	
	%O KDM é um metamodelo que possui um conjunto de metaclasses complementares para representar diferentes artefatos e visões conceituais de um mesmo sistema. Dessa forma, quando refatorações são aplicadas em uma visão conceitual todas as outras visões conceituais do sistema deveriam manter-se consistentes e sincronizadas. No contexto da ADM, e principalmente do metamodelo KDM existe uma carência de abordagens que sincronizam tais refatorações. A aplicação de técnicas de propagação de mudanças podem auxiliar na identificação e atualização de todas as instâncias/visões do KDM, permitindo assim manter todas as visões/instâncias do metamodelo KDM sincronizadas.  

	%Existe um relacionamento complementar entre todas as visões conceituais do metamodelo KDM. Dessa forma, quando refatorações são aplicadas em uma visão conceitual, por exemplo, em uma instância do KDM que representa o código-fonte do sistema, e não é sem considerar a sincronização e consistência de outras visões 
	%\item Embora a refatoração para modelos tenha alcançado bastante reconhecimento e aceitação, até o presente momento apenas trabalhos desenvolvido no contexto do metamodelo \sigla{UML}{\textit{Unified Modeling Language}} foram encontrados e consolidados na literatura~\cite{revisao_sistematica_uml_refactoring, durelli_systematic_mapping}. Dessa forma, existe uma ausência de abordagens para o metamodelo KDM, uma vez que o mesmo é um metamodelo mais novo quando comparado a UML. A refatoração de um sistema em geral tende a ser uma atividade complexa; modificações manuais, sem qualquer catálogo de refatoração, bem como um ambiente integrado, pode resultar em efeitos colaterais indesejados e acarretar em um processo tedioso. Neste contexto, existe uma necessidade para elaborar/adaptar um catálogo de refatoração para o metamodelo KDM e também criar um ambiente que auxilie a aplicação de refatorações diretamente no KDM, assim, os engenheiros podem reduzir o tempo e esforço durante a refatoração de sistemas legados e ainda respeitando a interoperabilidade fornecida pelo metamodelo KDM.


    %\item Na literatura é possível identificar um conjunto de refatorações já validadas e que são usualmente aplicadas em código-fonte, por exemplo, \textit{Extract Class}, \textit{Move Method}, \textit{Move Attribute}, etc. Essas são apenas alguns exemplos de refatorações úteis que não são facilmente reutilizadas na prática durante a condução de modernização de um determinado sistema (fowler, ADM refactoring). Essa limitação pode ser atribuída devido a ausência de um meio padronizado de disponibilizar refatorações. Embora a ADM forneça um conjunto de metamodelos para auxiliar o engenheiro de software a conduzir MDRE até esse momento a ADM não provê instruções para auxiliar o engenheiro a promover o reúso de refatorações juntamente com os seus metamodelos padronizados (por exemplo, KDM) durante o processo de modernização. Essa limitação faz com que o engenheiro crie suas próprias soluções/refatorações, resultando em um possível atraso no processo de modernização. Contudo, as soluções/refatorações  definidas não são facilmente reutilizadas pois tendem a ser proprietárias. Uma abordagem promissora é lidar com a refatoração de forma independente da linguagem – aumentando assim as possibilidades de reutilização de refatorações. Dessa forna, existe uma necessidade de criar uma metamodelo para auxiliar o engenheiro de modernização a promover o reúso de refatorações no contexto da ADM e principalmente de forma integrada com o metamodelo KDM. 
    
    %\item Escassez de meios para disponibilizar e promover o reúso de refatorações no contexto da ADM e KDM. 

%	\item Refatorações são metodologias bem consolidadas e amplamente utilizadas tanto academicamente quanto industrialmente. Usualmente, refatorações são compartilhas e distribuídas por meio de catálogos escritos em linguagem natural.
	%Na literatura é possível identificar um conjunto de refatorações já validadas e que são usualmente aplicadas em código-fonte. 
%	No entanto, refatorações não são facilmente reutilizadas na prática. Essa limitação pode ser atribuída devido a ausência de um meio padronizado de disponibilizar refatorações. Embora a ADM forneça um conjunto de metamodelos para representar diversos artefatos de sistemas de software até esse momento a ADM não provê instruções para auxiliar de como prover o reúso de refatorações. Uma abordagem promissora é lidar com a refatoração de forma independente da linguagem – aumentando assim as possibilidades de reutilização de refatorações. Dessa forna, existe uma necessidade de criar um metamodelo para auxiliar o engenheiro de modernização a promover o reúso de refatorações no contexto da ADM e principalmente de forma integrada com o metamodelo KDM. 

%\end{itemize}


\section{A Abordagem Desenvolvida em Resumo}\label{sec:introducao:a_abordagem_desenvolvida}

Nessa Tese desenvolveu-se uma abordagem para auxiliar o engenheiro de modernização a criar, reutilizar e aplicar refatorações para o metamodelo KDM. Na Figura~\ref{fig:abordagem_kdm_tese_processo} uma visão geral da abordagem definida nesta Tese é apresentada. A abordagem desenvolvida nesta Tese contém dois principais passos. O passo \ding{182} consiste na criação de refatorações para o metamodelo KDM. Esse passo é apoiado por cinco diretrizes, as quais o engenheiro de modernização segue para criar refatorações para o metamodelo KDM. A primeira diretriz consiste em identificar os elementos estruturais entre o paradigma orientado a objeto e o metamodelo KDM. Em seguida, o engenheiro de modernização escolhe qual refatoração almeja-se criar para o KDM. Então, a refatoração é implementada por meio da linguagem de transformação ATL. As restrições (pré- e pós-condições) da refatoração também são implementadas em OCL. E então o engenheiro de modernização documenta a refatoração por meio de duas especificações: especificação informal e formal.



%A criação de refatorações para o KDM é apoiada por cinco diretrizesO objetivo é criar diretrizes para que outros engenheiros de modernização possam criar refatorações para o KDM.

%O metamodelo KDM consiste em um conjunto de pacotes para representar diversos artefatos existentes de um sistema de software. Assim, após a aplicação de refatorações é de suma importância manter todos os pacotes/artefatos sincronizados e consistentes. Dessa forma, no passo \ding{183} regras de propagações são realizadas em instância do metamodelo KDM para manter todos os artefatos sincronizados e consistentes de acordo com a refatoração aplicada. 
%O passo \ding{183} consiste na disponibilização de refatorações por meio de um metamodelo aqui definido. Esse metamodelo contém metaclasses que permitem armazenar informações relacionadas com refatorações, desde seu nome até seu mecanismo. O objetivo é permitir e aumentar a interoperabilidade de refatorações para um amplo domínio e auxiliar o engenheiro de modernização a definir refatorações representativas em forma de metadados. Posteriormente as instâncias desse metamodelo são enviadas para um repositório e são reutilizadas por engenheiros de software por meio de um apoio computacional. O apoio computacional \ding{184} também foi definido para auxiliar o engenheiro de software a aplicar refatorações em sistemas representados pelo KDM. Após a aplicação de refatorações é de suma importância manter todos os pacotes/artefatos sincronizados e consistentes. Dessa forma, esse apoio computacional contém um plug-in responsável por implementar regras de propagações que são realizadas em instância do metamodelo KDM. Essas regras mantêm todos os artefatos sincronizados e consistentes de acordo com a refatoração aplicada. 

\begin{figure}[h]
	\centering
	% Requires \usepackage{graphicx}
	\caption{Macro visão da abordagem proposta.}
	\label{fig:abordagem_kdm_tese_processo}
	\includegraphics[scale=0.75]{images/Micro_visao_do_doutorado3}
	\fautor
\end{figure}

O passo \ding{183} consiste na disponibilização de refatorações por meio de um metamodelo aqui definido. Esse metamodelo contém metaclasses que permitem armazenar metadados relacionadas com refatorações; por exemplo, o metamodelo permite armazenar os seguintes metadados: o nome da refatoração, sua motivação, autor, pré- e pós-condições e seu mecanismo, etc. É importante salientar que tanto as pré- e pós-condições e os mecanismo das refatorações são disponibilizados no metamodelo por meio de linguagens de transformação e restrições. O objetivo amplo desse metamodelo é permitir e aumentar o reúso de refatorações para um amplo domínio e auxiliar o engenheiro de modernização a definir refatorações representativas em forma de metadados. Em seguida as instâncias desse metamodelo são enviadas para um repositório e são reutilizadas por engenheiros de software por meio de um apoio computacional. 


O apoio computacional (ver Figura~\ref{fig:abordagem_kdm_tese_processo} \ding{184})  também foi implementado para automatizar a atividade de aplicação e reutilização de refatorações em sistemas representados pelo KDM. Para auxiliar o engenheiro de software, as refatorações podem ser aplicadas diretamente em diagramas de classes UML, porém, a refatoração é de fato realizada transparentemente no metamodelo KDM e posteriormente replicada nos diagramas de classes UML. Adicionalmente, após a aplicação de refatorações em sistemas representados pelo KDM é de suma importância manter todos os pacotes/artefatos sincronizados e consistentes. Dessa forma, esse apoio computacional também contém um plug-in responsável por aplicar regras de propagações que são realizadas em instância do metamodelo KDM. O intuito desse plug-in é manter todos os artefatos sincronizados e consistentes de acordo com a refatoração aplicada.


\section{Objetivos}\label{sec:objetivos}

A tese subjacente a este trabalho é de que é possível e benéfico o uso de refatorações para o contexto da ADM, principalmente para o KDM. Nesse contexto, o objetivo geral desta Tese é apresentar uma abordagem para criação e disponibilização de refatorações para o metamodelo KDM, bem como um apoio ferramental que permite aplicá-las em diagramas de classe da UML (ver Figura~\ref{fig:abordagem_kdm_tese_processo}). Mais especificamente pode-se sumarizar os seguintes objetivos:%Para que esse objetivo seja alcançado, os seguintes objetivos específicos devem ser atingidos:

\begin{itemize}

    \item Tornar a criação de refatorações para o metamodelo KDM um processo sistemática e guiado;
    
    \item Viabilizar a disponibilização de refatorações para o metamodelo KDM de forma que possam ser mais facilmente especificadas, disponibilizadas e reusadas;
    
    \item Disseminar conhecimento técnico acerca do desenvolvimento de ferramentas que facilitem o reúso e aplicação de refatorações para o metamodelo KDM em ambientes de modelagem UML; e
    

    \item Propor um metamodelo inicial como uma solução ao \textit{Call for Proposals} do ADM \textit{Refatoring} do OMG. Esse metamodelo possui um conjunto de metaclasses que definem meta-atributos específicos para representar informações (metadados) de refatoração, auxiliando assim o compartilhamento das refatorações de forma intuitiva entre os engenheiros. Além disso, esse metamodelo também possui metaclasses e meta-atributos que representam os mecanismos das refatorações, bem como suas pré- e pós-condições.

	%\item Definir diretrizes para criar refatorações para o metamodelo KDM;
	
	%\item Especificar e criar um metamodelo para auxiliar engenheiros de modernização a compartilhar, criar e reutilizar refatorações no contexto da ADM e KDM;
	
	%\item Elaborar uma linguagem específica de domínio (do inglês - \sigla{DSL}{\textit{Domain-Specific Language}}). Essa DSL possui duas finalidades, a saber: (\textit{i}) auxiliar o engenheiro de modernização a instanciar o metamodelo de refatoração proposto e (\textit{ii}) facilitar a criação de um conjunto de refatorações de forma guiada e automática;
	
	%\item Criar um repositório totalmente integrado com o apoio ferramental para facilitar o compartilhamento e o reúso de refatorações que estão em conformidade com o metamodelo de refatoração proposto; e
	
	%\item Elaborar regras pré-definidas para manter o metamodelo KDM consistente e sincronizado após a aplicar um conjunto de refatorações;
	
    %\item Desenvolver um apoio ferramental totalmente integrado no ambiente de desenvolvimento Eclipse para apoiar a abordagem proposta na Tese.

\end{itemize}




%A tese subjacente a este trabalho é de que é possível e benéfico o uso de refatorações para o contexto da ADM, principalmente para o KDM. %Além disso, pretende-se viabilizar a reutilização e padronizações de refatorações por meio da utilização de um metamodelo de refatorações. Adicionalmente, planeja-se verificar a possibilidade de manter todas as visões do metamodelo KDM sincronizada e consistentes após a aplicação de um conjunto de refatorações. %e de técnicas de propagação de mudanças para manter todas as visões do KDM sincronizadas e consistentes após a aplicação de uma refatoração. 
%Neste contexto, esta tese de doutorado cobre os seguintes aspectos: 


%\begin{itemize}
%	\item Definir diretrizes para criar refatorações para o metamodelo KDM;
	
%	\item Criar regras pré-definidas para manter o metamodelo KDM consistente e sincronizado após a aplicar um conjunto de refatorações;
	
%	\item Especificar e criar um metamodelo para auxiliar engenheiros de modernização a compartilhar, criar e reutilizar refatorações no contexto da ADM e KDM;
	%\item a elaboração de uma linguagem específica de domínio (do inglês - \sigla{DSL}{\textit{Domain-Specific Language}}). Essa DSL possui duas finalidades, a saber: (\textit{i}) auxiliar o engenheiro de modernização a instanciar o metamodelo de refatoração proposto e (\textit{ii}) facilitar a criação de um conjunto de refatorações de forma guiada e automática;
	%\item a definição de um ambiente \emph{Web} para também auxiliar a instanciação do metamodelo de refatoração proposto;
	%\item a concepção de um repositório totalmente integrado com a ferramenta %e com o ambiente \emph{web} 
	%para facilitar o compartilhamento e o reúso de refatorações que estão em conformidade com o metamodelo de refatoração proposto.
	
%	\item Elaborar um apoio ferramental totalmente integrado no ambiente de desenvolvimento Eclipse para apoiar a abordagem proposta na Tese.% que possui três módulos: (\textit{i}) módulo de refatoração; (\textit{ii}) módulo do SRM e (\textit{iii})  %com o objetivo de auxiliar o engenheiro de modernização a aplicar, reutilizar e propagar refatorações de forma gráfica, por meio da utilização de diagramas;
%\end{itemize}


%\section{Contribuições}\label{sec:contribuicoes}

%A principal contribuição dessa Tese de doutorado é entender como refatorações tradicionais, ou seja, refatorações comumente utilizadas em sistemas implementados com o paradigma orientado a objeto, podem ser adaptadas, aplicadas, padronizadas e reutilizadas no contexto da ADM e principalmente para o metamodelo KDM. Em outras palavras, esta pesquisa pode ser entendida como uma incursão inicial para auxiliar o OMG e a ADM na definição de padronizações e soluções ferramentais para facilitar o engenheiro de modernização durante o uso de refatorações para o KDM. Pontualmente, pode-se destacar as principais contribuições dessa tese são:

%\begin{itemize}
%	\item Diretrizes para criar e adaptar refatorações para o metamodelo KDM;
%	\item Regras pré-definidas para manter a instância do metamodelo KDM sincronizado e consistente após a aplicação de refatorações;
%	\item Investigação e definição de um metamodelo para auxilar os engenheiros de modernização a criar, compartilhar e reutilizar refatorações no contexto da ADM e KDM;
%	\item Ambiente para auxiliar o engenheiro de modernização durante a aplicação de refatorações para o metamodelo KDM;
%	\item Definição de uma DSL para auxiliar o engenheiro de modernização a instanciar o metamodelo proposto e facilitar a criação de um conjunto de refatorações de forma guiada e automática;
	%\item a elaboração de um ambiente \emph{Web} para também auxiliar a instanciação do metamodelo de refatoração proposto;
%	\item Concepção de um repositório totalmente integrado com a ferramenta para facilitar o compartilhamento e o reúso de refatorações que estão em conformidade com o metamodelo proposto.
%\end{itemize}
    
\section{Convenções Adotadas nesta Tese}\label{sec:convencoes}

Ao longo desta Tese, \textit{Itálico} é utilizado para dar ênfases, introduzir novos termos e para destacar palavras em inglês. \texttt{Typewriter} é utilizado para operador Java, operador da DSL, palavras-chaves, nome de metaclasses, meta-associação, meta-atributo, nome de métodos, variáveis e URL que aparecem no texto. Símbolos \ding{202}, \ding{203}, \ding{204}, \ding{205} ou \textcircled{a}, \textcircled{b}, \textcircled{c}, \textcircled{d}, são utilizados para chamar a atenção do leitor para informações importantes em figuras e códigos.

\section{Grupo de Pesquisa}

Esse trabalho é uma contribuição para o grupo de pesquisa do Departamento de Ciência de Computação e Estatística do Instituto de Ciências Matemáticas e de Computação (ICMC) da Universidade de São Paulo (campus São Carlos/SP). Além disso, esse trabalho também foi conduzido em parceria com o grupo de pesquisa AdvanSE (\textit{Advanced Research on Software Engineering}), da Universidade Federal de São Carlos (UFSCar). O grupo possui pesquisas em andamento sobre extensões, refatorações, mineração, métricas e validações de arquitetura utilizando a ADM e o metamodelo KDM, as quais o autor desta Tese participa efetivamente.

\section{Estrutura da Tese}

Esta tese está organizada em X\change{mudar} capítulos. No primeiro capítulo estão apresentados o contexto, a motivação, os objetivos, contribuições, convenções adotadas e o grupo de pesquisa do trabalho. 

No Capítulo~\ref{chapter:fundamentacao_teorica} apresenta-se uma revisão dos principais conceitos envolvendo MDD e refatoração com o objetivo de facilitar a compreensão da tese.

No Capítulo~\ref{chapter:adm_kdm} é ilustrada uma contextualização sobre a modernização de sistemas com a utilização dos padrões propostos pelo OMG, ADM e KDM, em que seus conceitos e particularidades são observados. Também é apresentada a ferramenta MoDisco, utilizada nesta Tese. 

No Capítulo~\ref{chapter:mapeamento_sistematico} é apresentado um mapeamento sistemático que foi realizado com o objetivo de identificar e entender soluções já desenvolvidas sobre ADM e KDM. Além disso, nesse mapeamento também são apresentadas as principais constatações e questões em aberto.

Como apresentado na Figura~\ref{fig:abordagem_kdm_tese_processo} a abordagem proposta nesta Tese contém dois principais passos. O passo \ding{182} é apresentado no Capítulo~\ref{chapter:catalogo_refactoring_KDM} onde são destacadas as diretrizes para adaptar e criar refatorações para o metamodelo KDM. O passo \ding{183} é salientado no Capítulo~\ref{chapter:Toward_a_Refactoring_Metamodel_for_KDM}, o qual apresenta um metamodelo para disponibilizar refatorações e promover o reúso de refatorações no contexto da ADM e KDM. No Capítulo~\ref{chapter:Abordagem_de_sincronizacao} é apresentada uma abordagem denominada KDM-SInc que é utilizada para manter uma determinada instância do metamodelo KDM consistente e sincronizado após a aplicação de refatorações. 

O apoio computacional \ding{184} é apresentado no Capítulo~\ref{chapter:ferramenta_kdm_re}. Esse apoio computacional é denominado KDM-RE e é composto por três plug-ins do Eclipse: (\textit{i}) o primeiro consiste em um conjunto de \textit{Wizards} que apoia o engenheiro de software na aplicação das refatorações em diagramas de classe UML; (\textit{ii}) o segundo consiste em um módulo de propagação de mudanças que permite manter modelos internos do KDM sincronizados e; (\textit{iii}) o terceiro consiste em um apoio à importação e reúso de refatorações disponíveis no repositório.


No Capítulo~\ref{chapter:avaliacao} é mostrado o planejamento, execução e análise dos dados de dois experimentos que visou validar a abordagem desenvolvida nesta Tese.  E por ultimo no Capítulo X\change{mudar} são mostradas as conclusões do trabalho com as principais contribuições, limitações, lições apreendidas, publicações e trabalhos futuros.

%A Tese está estruturada de acordo com a Figura~\ref{fig:structure_these_not_final}

%\begin{figure}[h]
%	\centering
	% Requires \usepackage{graphicx}
	%\caption{Macro visão da abordagem proprosta.}
	%\label{fig:abordagem_kdm_tese_processo}
	%\includegraphics[scale=0.9]{images/PhD_Structure_Figure}
	%\fautor
%\end{figure}

\chapter{Desenvolvimento de Software Dirigido a Modelos e Refatorações}
\label{chapter:fundamentacao_teorica}

\section{Considerações Iniciais}


Neste capítulo são apresentados e discutidos os conceitos fundamentais para o entendimento desta tese. Ele está organizado da seguinte forma: na Seção~\ref{Cap2_Sec2_Desenvolvimento_Dirigido_a_Modelos}, os conceitos sobre Engenharia Dirigido por Modelos são salientados; na Seção~\ref{sec:refatoracao} são definidos os conceitos sobre refatoração, bem como refatorações para modelos; em seguida na Seção~\ref{capitulobaclCOnsideracoesFinais} as considerações finais deste capítulo são destacadas.


\section{Engenharia Dirigida por Modelos}\label{Cap2_Sec2_Desenvolvimento_Dirigido_a_Modelos}

De acordo com~\citeonline{Booch:2004:OAD:975416} modelos são abstrações de sistemas que permitem raciocinar e entender o sistema, ignorando detalhes irrelevantes no modelo, enquanto focalizamos nos detalhes mais relevantes. Segundo~\citeonline{Brown_2007, Bezivin02apreliminary}, a utilização de modelos para o desenvolvimento de software não é algo novo. Modelos estão sendo usados há algumas décadas para auxiliar a  concepção e projeto de software, sendo utilizados basicamente nas fases iniciais do desenvolvimento. Por exemplo, modelos como os da UML~\cite{UML:OMG} não fazem parte do software em si, embora sejam importantes para o entendimento e a construção. Os desenvolvedores os criam, mas os descartam, implementando as funções de forma manual e realizando manutenções somente no código-fonte. Desse modo, o conhecimento acerca da solução fica criptografado no código-fonte e dificilmente é reutilizado. Portanto, há necessidade de criar modelos que representem esse conhecimento e que possam ser úteis tanto para a documentação quanto para a construção e manutenção de software.


A partir dessa ideia, a \sigla{MDE}{Engenharia Dirigida por Modelos (do inglês - \textit{Model-Driven Engineering})} surge como uma solução complementar aos processos de desenvolvimento tradicionais~\cite{Lima_2007}. A MDE é uma abordagem que propõe reduzir a distância semântica entre o problema do domínio e a solução/implementação. Nessa abordagem, o desenvolvimento de software ocorre por meio de modelos que protegem os desenvolvedores das complexidades da implementação e de transformações que originam o código-fonte de maneira automatizada a partir das informações contidas nesses modelos. Na literatura é possível também identificar MDE como outros acrônimos, por exemplo, MDD, MDSD (\textit{Model-Driven Software Development}) ou MD*~\cite{Kleppe:2003}, todos esses acrônimos dizem respeito à mesma abordagem. O que é importante ter em mente sobre MDE é que modelos são utilizados no centro do processo de desenvolvimento e manutenção de software. Na MDE, os modelos assumem o papel principal em todo o ciclo de vida de um software. A relevância dos modelos vai além da documentação do software desenvolvido. Na MDE modelos passam a ser utilizados como artefatos principais; eles podem ser compreendidos por computadores e são fundamentais para o desenvolvimento do software, pois podem ser manipulados, refinados e transformados para uma nova versão, até que a partir deles é gerado o código-fonte~\cite{Kleppe:2003, Brown_2007, Ben_Ammar}.

Na MDE o enfoque do desenvolvimento é direcionado aos modelos, ou seja, a modelagem deixa de ser meramente uma forma de planejar o código e passa a ser uma forma de construir o software. O nível de abstração da programação torna-se mais elevado, reduzindo necessidade do desenvolvedor de interagir manualmente com o código-fonte~\cite{Braganca_Machado}. A MDE não tem como objetivo substituir o processo tradicional de desenvolvimento de software~\cite{Kleppe:2003, Brown_2007, Braganca_Machado} e sim contribuir para o seu aprimoramento. O OMG~\cite{ADM:OMG} definiu um modelo de arquitetura para o MDE, conhecido como \sigla{MDA}{\emph{Model-Driven Architecture}}. MDA tem como objetivo promover o uso de modelos no desenvolvimento de software, para fornecer uma solução ao gerenciamento da complexidade do desenvolvimento, manutenção e evolução de sistemas de software e favorecer a interoperabilidade e portabilidade desses sistemas.

O OMG definiu formalmente a MDA como uma abordagem que é bem definida pela ideia de separar a especificação das operações de um sistema dos detalhes de como este sistema usa as potencialidades de sua plataforma. Isso possibilita que ferramentas ofereçam a especificação de um sistema de forma independente de plataforma. De acordo com o OMG os principais objetivos da MDA são a portabilidade, a interoperabilidade e a reusabilidade. Para alcançar esses objetivos a MDA divide o desenvolvimento de software em níveis de abstração~\cite{France_2007, Ben_Ammar}, os níveis são:

\begin{itemize}
	\item Modelo Independente de Computação (do inglês - \sigla{CIM}{\emph{Computation Independent Model}}): descreve o negócio, ou ambiente, no qual o software irá operar. Como a decisão a respeito da informatização é feita por uma pessoa e não por uma ferramenta de transformação, dificilmente a transformação automatizada desse modelo para o de nível seguinte é implementada;
	\item Modelo Independente de Plataforma (do inglês - \sigla{PIM}{\emph{Platform Independent Model}}): nível de análise, em que ocorre a definição das características do software ou domínio. Esse modelo pode ser transformado para um ou mais modelos do nível seguinte;
	\item Modelo Específico de Plataforma (do inglês - \sigla{PSM}{Platform Specific Model}): nível de projeto, considera as tecnologias de implementação, devendo existir uma instância desse modelo para cada plataforma de implantação do software. A partir desse modelo ocorre a transformação para o código-fonte.
\end{itemize}

Com base nos princípios mencionados, pode-se citar as seguintes vantagens da MD*~\cite{Hutchinson_2011, France_2007, Schmidt_2006}:

\begin{enumerate}
	\item Maior facilidade na criação dos modelos das aplicações, pois é utilizada uma linguagem de específica para o domínio do problema;
	\item Maior produtividade e redução do esforço dos desenvolvedores, pois a maior parte do código-fonte das aplicações pode ser gerado a partir dos modelos;
	\item Maximização do tempo de vida útil dos modelos e de outros artefatos, pois como são necessários para a geração de código, eles são menos propensos a serem descartados pelos desenvolvedores nos processos de manutenção;
	\item Flexibilidade de desenvolvimento, pois os modelos independentes de plataforma armazenam a lógica do sistema e são menos sensíveis a essas mudanças;
	\item O conhecimento a respeito do software não fica exclusivamente na mente dos desenvolvedores e no código, fazendo com que os processos de desenvolvimento e de manutenção fiquem menos vulneráveis às oscilações de pessoal.
\end{enumerate}

A MD* não pretende substituir os processos tradicionais de desenvolvimento de software e sim contribuir para seu aprimoramento, porém, de uma maneira racional para mover informações de uma fase de desenvolvimento para outra, trazendo respostas rápidas e eficientes para atender as necessidades inesperadas de novos requisitos.

De acordo com~\citeonline{Kleppe:2003}, um modelo deve ser definido como uma descrição de um (ou de uma parte) sistema expresso em uma linguagem bem definida, isto é, respeitando uma sintaxe e uma semântica. Esta descrição deve ser conveniente para uma interpretação automatizada por computadores. Assim, para criar um modelo deve-se seguir uma sintaxe precisa e uma semântica bem definida a fim de regulamentar a criação de elementos e suas relações. Em MDE esse formalismo pode ser alcançado utilizando uma linguagem de modelagem. Tais linguagens são especificações que contêm os elementos bases para construir modelos, concebida dentro de um domínio limitado e com objetivos específicos. Usualmente uma linguagem de modelagem pode ser gráfica ou textual com notação matemática e deve permitir a definição de modelos sem ambiguidades. O uso de uma linguagem bem definida sintaticamente e semanticamente permite o entendimento e manipulação de modelos por pessoas e computações~\cite{Hutchinson_2011, France_2007, Schmidt_2006}.

Uma linguagem de modelagem geralmente está em conformidade com um metamodelo. Um metamodelo é um modelo que define uma linguagem para representar um modelo. A linguagem de modelagem é simplesmente um modelo do metamodelo. Ela define a estrutura, a semântica e as restrições para uma família de modelos~\cite{Mellor_2004}. Na Figura~\ref{fig:metamodelosCamadas} é apresentado os quatro níveis de modelos da MDE. Note que as relações entre os níveis descritos na Figura~\ref{fig:metamodelosCamadas} são do tipo \aspas{\texttt{instance of}} (instância de) definido por~\citeonline{Bezivin02apreliminary, Brambilla_2012}. Os quatros níveis são descritos a seguir: 

\begin{figure}[htb]
 \caption{Arquitetura de metamodelagem.}
 \label{fig:metamodelosCamadas}
 \centering
 \includegraphics[scale=1]{images/Arquitetura_de_metamodelagem}
\end{figure}

\begin{itemize}
	\item metametamodelo (M3): M3 constitui a base da arquitetura de meta-modelagem. A função primordial deste nível é definir linguagens para especificar metamodelos. Um meta-metamodelo define um modelo de mais alto nível de abstração que o metamodelo, e este primeiro é tipicamente mais compacto que o segundo. \sigla{MOF}{\emph{Meta-Object Facility}}~\cite{MOF} é \sigla{EMF}{\emph{Eclipse-Modeling Framework}}\cite{EMF} são exemplos de meta-metamodelos;
	\item metamodelo (M2): um metamodelo representa uma instância de um meta-metamodelo. A função principal do nível do metamodelo é definir uma linguagem para especificar modelos. Os metamodelos são tipicamente mais elaborados que os meta-metamodelos. Por exemplo, a UML~\cite{UML:OMG} e o KDM~\cite{KDM:ISO} ambos possuem metamodelos que os descrevem estruturalmente;
	\item modelo (M1): um modelo é uma instancia de um metamodelo. A função principal do nível de modelo é definir uma linguagem para descrever um domínio específico;
	\item dados (M0): os objetos de usuários representam os dados finais. A principal responsabilidade dos objetos de usuários é descrever um domínio específico em uma plataforma final.
\end{itemize}

Um objetivo claro da MDA é fornecer um \textit{framework} que integra os padrões existentes do OMG. Os principais padrões do OMG utilizados nesta Tese são:

\begin{itemize}
\item \textit{Meta Object Facility} (MOF): Linguagem abstrata e um \textit{framework} para especificação, construção e gerenciamento de metamodelos independentes de plataforma. Essa especificação contêm um conjunto de construtores que é utilizado para a definição de metamodelos. MOF pode ser utilizado para definir outras linguagens;
\item \textit{Unified Modeling Language} (UML): É uma linguagem para especificação, construção, visualização e documentação de artefatos de software. Essa linguagem permite a modelagem de diferentes aspectos ou pontos de vista de um sistema;
\item \textit{Knowledge Discovery Metamodel} (KDM): Metamodelo utilizado para representar em nível de modelos artefatos de sistemas de software. Capítulo~\ref{chapter:adm_kdm} maiores informações sobre esse metamodelo, bem como suas metaclasses são apresentadas;
\item \textit{XML Metadata Interchange} (XMI): Padrão do OMG para troca de informação baseado em XML. Pode ser utilizado para trocar qualquer informação cujo metamodelo pode ser expresso utilizando MOF. O uso mais comum da XMI é como um formato de intercâmbio de modelos UML, embora também possa ser utilizado para a serialização de modelos de outras linguagens;
\item \textit{ATL Transformation Language} (ATL): É uma implementação da \textit{Query/View/Transformation} (QVT) que é uma especificação híbrida padronizada para transformação de modelos no contexto de metamodelagem MOF. ATL aceita construções declarativas e imperativas, ver Subseção~\ref{sub:atl_transformation_language};
\textit{Object Constratint Language} (OCL): É uma linguagem declarativa para descrever regras que se aplicam aos modelos. A OCL, inicialmente, era apenas uma extensão da UML para especificações formais de modelos. Hoje em dia, a OCL pode ser utilizada para especificar pré- e pós-condições, podendo ser utilizada em qualquer modelo cujo metamodelo seja MOF.
\end{itemize}

\section{Refatoração}\label{sec:refatoracao}

Refatoração pode ser entendida como um processo de redistribuição de funcionalidade com o intuito de melhorar um dado sistema. No contexto do paradigma orientado a objeto, essa redistribuição está totalmente ligada com classes, atributos e operações. A refatoração tem como objetivo permitir a redistribuição de classes, atributos e operações na hierarquia de classes para facilitar futuras atividades de desenvolvimento ou de manutenção. A primeira definição de refatoração foi concebida por~\citeonline{OPDYKE_1992} em sua tese da seguinte forma: \aspas{refatorações são transformações que reestruturam um determinado sistema com o objetivo de melhorar o \textit{design}, evolução e reuso de sistemas desenvolvidos no paradigma orientada a objeto}.

No contexto do paradigma orientado a objeto, refatoração é uma alternativa do conceito de reestruturação. Em outras palavras, refatoração é um termo aplicado ao paradigma orientado a objeto; para outros paradigmas de programação, esse mesmo processo é descrito como reestruturação~\cite{Chikofsky_cross}. De acordo com~\citeonline{Chikofsky_cross}, reestruturação \aspas{consiste no processo de alterar um software, melhorando a sua estrutura interna, de forma que o comportamento externo do código não seja alterado}. Reestruturação e refatoração são técnicas essenciais que são utilizadas para mitigar problemas relacionados à evolução de software~\cite{OPDYKE_1992}. Com o objetivo de aumentar atributos de qualidade dos sistemas, as práticas de refatoração surgiram através do emprego de reestruturação sobre unidades de código preservando o seu comportamento~\cite{Chikofsky_cross,OPDYKE_1992}.

Quando aplicada durante a fase de manutenção de software, a refatoração ajuda a tornar o código mais legível e também tem como objetivo solucionar problemas de códigos mal escritos~\cite{Chikofsky_cross}. A refatoração também pode ser usada no contexto da reengenharia a fim de alterar um sistema específico visando reconstruí-lo em um novo formato. Nesse contexto, a refatoração é necessária para converter código legado ou deteriorado em um formato mais estruturado ou modular, ou para migrar o código para uma diferente linguagem de programação, ou mesmo um diferente paradigma de linguagem.

Em seu livro,~\citeonline{Fowler1999}, apresenta duas definições para refatoração, uma como substantivo e outra como verbo:

\begin{itemize}
	\item Refatoração: uma mudança que é realizada na estrutura interna de um determinado sistema com o objetivo de deixar o sistema mais fácil de entender e fácil de modificar sem alterar o seu comportamento externo; e
	\item Refatorar: reestruturar o software por meio de um conjunto de refatorações sem modificar o seu comportamento externo.
\end{itemize}

As definições apresentadas por~\citeonline{Fowler1999} enfatizam que o propósito da refatoração é fazer com que o software fique mais fácil de entender (melhorar sua compreensão) e de modificar (melhorar sua manutenibilidade). Outra característica de suma importância a ser destacada é que a refatoração em geral deve ser um processo para melhorar o \textit{design} do software. De acordo com~\citeonline{Wake_2003} , refatoração é \aspas{uma arte para melhorar cuidadosamente o \textit{design} de códigos existentes.} O autor também enfatiza que refatoração deve fornecer maneiras de identificar problemas no código e também deve prover soluções para corrigir tais problemas.~\citeonline{Wake_2003} caracteriza refatoração como:

\begin{itemize}
	\item refatoração não inclui nenhuma mudança no sistema - refatoração não deve adicionar novas funcionalidades ao sistema;
	\item refatoração deve ser utilizada para melhorar o código do sistema;
	\item nem toda reestruturação pode ser considerada como refatoração - usualmente refatorações tendem a ser transformações pequenas e seguras. 
\end{itemize}

O primeiro conjunto de refatorações foi proposto por~\citeonline{OPDYKE_1992}, onde o autor definiu 26 refatorações de baixa granularidade. Tais refatorações podem ser resumidas da seguinte forma:

\begin{itemize}
	\item criar um membro  - variável/função/classe: Essas refatorações tem como objetivo criar novas variáveis e/ou funções para uma classe em particular ou criar um nova classe;
	\item deletar um membro - variável/função/classe. Essas refatorações têm como objetivo  deletar membros que não estão sendo utilizados;
	\item renomear um membro - variável/função/classe. Essas refatorações podem ser utilizadas pare renomear membros e fornecerem nomes mais significativos;
	\item mover um membro - variável/função. Essas refatorações são utilizadas para redistribuir um conjunto de variáveis/funções para sub ou superclasses.
\end{itemize}

Similarmente,~\citeonline{Roberts_1999} definiu um conjunto de refatorações que devem ser aplicadas em classes, métodos e atributos. Porém o catálogo mais completo e extensivo de refatorações foi definido por~\citeonline{Fowler1999}. Neste catálogo cada refatoração possui os seguintes tópicos: (\textit{i}) um nome, (\textit{ii}) uma breve descrição, (\textit{iii}) uma motivação para a condução da refatoração, (\textit{iv}) um mecanismo descrevendo como a refatoração deve ser executada e (\textit{v}) um exemplo ilustrando a utilização da refatoração. As refatorações proposta por~\citeonline{Fowler1999} são agrupadas em setes categorias, a saber: (\textit{i}) \textit{Composing Methods}, (\textit{ii}) \textit{Moving Features Between Objects}, (\textit{iii}) \textit{Organizing Data}, (\textit{iv}) \textit{Simplifying Conditional Expressions}, (\textit{v}) \textit{Make Method Calls Simpler}, (\textit{vi}) \textit{Dealing with Generalization} e (\textit{vii}) \textit{Big Refactorings}.

De acordo com~\citeonline{Fowler1999} existem quatro principais motivações para a aplicação de refatoração:

\begin{enumerate}
	\item refatorações quando bem conduzidas tendem a melhorar o \textit{design} do software - dessa forma, refatoração pode auxiliar na prevenção da decadência do software e eliminando código duplicado;
	\item refatorações fazem com que o código-fonte fique mais fácil de entender - código bem legível facilita a comunicação e seu propósito;
	\item refatoração auxilia na identificação de erros - melhorando a estrutura interna do código-fonte erros tendem a ser identificados mais facilmente;
	\item desenvolvimento mais produtivo - uma boa estrutura interna usualmente facilita o desenvolvimento e melhora a produtividade.
\end{enumerate}

Como já salientado refatorações devem preservar o comportamento de um determinado software após a aplicação de \textit{n} refatorações. Dessa forma,~\citeonline{Mens_and_Tourwe_2004,Cinneide_2000} relatam que existem três principais abordagens para auxiliar a preservação (de alguns aspectos) do comportamento do código-fonte. Tais abordagens são: (\textit{i}) abordagem não formal (por exemplo, as refatorações definidas por~\citeonline{Fowler1999}), (\textit{ii}) uma abordagem semiformal~\cite{Roberts_1999} e (\textit{iii}) abordagem completamente formal. No entanto, os autores também argumentam que mesmo com a utilização da última abordagem é impossível garantir totalmente a preservação de comportamento após a aplicação de refatorações~\cite{Mens_and_Tourwe_2004,Cinneide_2000}. 

A ideia de preservação de comportamento no contexto de refatoração foi primeiramente introduzida por~\citeonline{OPDYKE_1992} da seguinte forma: \aspas{Se um programa é chamado duas vezes (antes e depois da refatoração) com o mesmo conjunto de entradas, o resultado deve ser o mesmo}. Essa explicação é plausível e utilizada na literatura~\cite{Roberts_1999, Fowler1999}, mas infelizmente não é suficiente. Por exemplo, considere o seguinte cenário. Se uma classe, um método ou outra estrutura de código é renomeada utilizando a refatoração \texttt{Rename} é desejado que todas as  declarações e utilizações correspondentes também sejam atualizadas. No entanto, suponha o Código-fonte~\ref{lst:example_refactoring_behavior}, se a classe \texttt{Foo} é renomeada para \texttt{Bar} o comportamento do programa é alterado: o Código-fonte~\ref{lst:example_refactoring_behavior} não irá imprimir \aspas{Foo} e sim \aspas{Bar}. Dessa forma, a definição apresentada por~\citeonline{OPDYKE_1992} não é verdadeira para esse cenário.   

\begin{codigo}[caption={[Um simples programa ilustrando porque é errado acreditar que refatoração não muda a saída de um programa.] Simples exemplo do efeito de uma refatoração.},escapeinside={(*@}{@*)}, basicstyle=\footnotesize, language=java, label={lst:example_refactoring_behavior}]{Name}
	public class Foo {
	    public void method(){
	        String className = this.getClass().getName();
	        System.out.println(className);
	    }
	}
\end{codigo}

Outra abordagem para a definição de comportamento é exigir a preservação sintática e semântica de um sistema após a aplicação de refatorações. Obviamente uma refatoração por definição não deveria invalidar a sintaxe de um sistema. Usualmente, a sintaxe e semântica são preservadas por meio de asserções. Asserção é definida por meio de pré- e pós-condições que são executadas antes e após a aplicação de uma refatoração. Pré-condições são asserções que um sistema deve satisfazer para que a refatoração possa ser aplicada de forma segura. Pré-condições podem ser pensadas como condições que caracterizam válidas transformações. Por exemplo, uma possível pré-condição para a refatoração \texttt{Rename Class} seria verificar se o novo nome da classe já existe dentro do pacote que a classe esta definida. ~\citeonline{OPDYKE_1992} foi o primeiro pesquisador a utilizar asserções para garantir que as refatorações aplicadas preservavam a sintaxe e a semântica dos sistemas. É importante observar que as  refatorações apresentadas nesta Tese para o metamodelo KDM foram propostas e adaptas com o intuito de preservar a sintaxe e semântica do sistema. O metamodelo KDM provê meios de garantir que a estrutura do código-fonte foi preservada utilizando o pacote \texttt{Code} e \texttt{Action}.

O processo para a aplicação de um refatoração contêm três principais passos~\cite{Wake_2003}. O primeiro passo consiste na identificação de partes do código que precisam ser refatoradas. O segundo passo consiste na escolha da melhor refatoração para solucionar o problema anteriormente identificado. O terceiro passo consiste na aplicação da refatoração. Seguindo a mesma ideologia e fundamentação proposta por~\citeonline{Fowler1999} e~\citeonline{OPDYKE_1992} existe a possibilidade de aplicar refatorações para modelos. Na subseção a seguir os conceitos e características relacionados com refatorações para modelos são apresentados.


% subsection modelos_e_meta_modelos (end)
\subsection{Transformação e Refatoração de Modelos}\label{sec:transformacoes_de_modelos}

Refatorações para modelos é um tipo especial de transformação de modelos\footnote{Transformação e Refatoração de modelos são utilizadas nesta Tese de forma intercambiáveis.} que têm como principal objetivo melhorar a estrutura do modelo e também preservar suas características internas. Refatorações para modelos é uma área de estudo relativamente nova quando comparada com refatorações tradicionais, ou seja, aquelas aplicadas em código-fonte. De acordo com a literatura, refatorações para modelos é uma área mais desafiadora do que refatorações tradicionais, uma vez que modelos usualmente possuem múltiplas visões que precisam permanecer sincronizados e consistentes após a aplicação de refatorações. Por exemplo, na literatura é possível identificar trabalhos que apresentam o estado da arte~\cite{Tom_2008_2008}, taxonomias~\cite{Maddeh_2010} e desafios~\cite{mens_03_refactoring, Mens07RefacTools, Van_Der_Straeten_2009, mens2003refactoring_novo_rafa} em relação à refatorações para modelos. 

Tais autores afirmam que a transformação em modelo desempenha um papel fundamental em abordagens que utilizam os princípios de MDE, pois permite a manipulação de modelo de forma totalmente automática. Uma transformação consiste na geração automática de um modelo alvo tendo como base um modelo fonte, sendo que essa transformação é definida por meio de um conjunto de regras de transformações~\cite{Mens_2006}. 

Nos trabalhos de~\citeonline{Mens_2006, Czarnecki_2006} e ~\citeonline{Biehl_2010} os autores buscam identificar e classificar as transformações de modelos. Algumas das classificações apresentadas por tais autores são citadas de forma resumida, a seguir:

\begin{itemize}
	\item Vertical ou Horizontal: Os modelos fonte e alvo podem estar em um ou mais níveis de abstração. Uma transformação horizontal mantêm modelos fonte e alvos no mesmo nível de abstração. Na transformação vertical, existe uma mudança de nível de abstração nos modelos. Esta mudança pode ser tanto para aumentar quanto par diminuir o nível de abstração;
	\item  Endógenas ou Exógenas: Nas transformações Endógenas os modelos envolvidos são expressos na mesma linguagem de modelagem. Nas transformações Exógenas os modelos que participam da transformação são de linguagens diferentes;
	\item Bidirecionais: Uma transformação bidirecional pode tanto gerar modelos alvos utilizando como base modelos fontes, quanto gerar modelos fontes utilizando modelos alvos. Em contrapartida, transformação unidirecional existe apenas um fluxo de execução. 
\end{itemize}

As transformações do tipo classificadas como endógenas ou exógenas~\cite{Brambilla_2012} são apresentadas na Figura~\ref{fig:model_transformations}. Como pode ser observado na Figura~\ref{fig:model_transformations} transformações em modelos do tipo endógenas apenas um modelo e um metamodelo são utilizados, por outro lado, transformações do tipo exógenas os metamodelos alvo e fonte são diferentes. Usualmente, refatorações em modelo são um exemplo de transformações do tipo endógenas, e uma transformação que tem como objetivo transformar uma linguagem para uma outra linguagem é um exemplo de transformações do tipo exógenas. Transformações em modelos que utilizam apenas um modelo como entrada e geram o mesmo modelo como saída com algumas modificações, essas transformações são classificadas como \aspas{\emph{in-place}}, enquanto que transformações em modelos que utilizam como entrada um modelo e tem como objetivo gerar um outro modelo como saída são consideradas \aspas{\emph{out-place}}. 

Transformações endógenas são mais interessantes quando apenas um subconjunto do modelo será afetado pela transformação~\cite{Brambilla_2012}. Por exemplo, em um editor de modelos transformações endógenas podem ser utilizadas para definir pequenas mudanças para automatizar tarefas repetitivas durante o desenvolvimento de modelos. Outra aplicação interessante de transformações endógenas é a condução de refatoração em nível de modelos. Um dos principais objetivos dessa tese de doutorado são a criação e utilização de refatoração em modelos, assim, maior ênfase será concentrada em transformações de modelos do tipo endógenas no restante dessa subseção.


\begin{figure}[ht]
\centering
\caption{Diferentes tipos de transformações em modelos.}
\subfigure[\emph{exogenous} \aspas{\emph{out-place}}]{%
\includegraphics[scale=0.7]{images/transformacaoModeloA}
\label{fig:subfigure1}}
\quad
\subfigure[ \emph{endogenous} \aspas{\emph{in-place}}]{%
\includegraphics[scale=0.7]{images/transformacaoModeloB}
\label{fig:subfigure2}}
\label{fig:model_transformations}
 \fadaptada{Brambilla_2012}
\end{figure}

%\begin{figure}[htb]
% \caption{Diferentes tipos de transformações em modelos: (a) \emph{exogenous} \aspas{\emph{out-place}} vs. (b) \emph{endogenous} \aspas{\emph{in-place}}}
% \label{fig:model_transformations}
% \centering
% \includegraphics[scale=0.45]{images/transformacoes.png}
% \fadaptada{Brambilla_2012}
%\end{figure}

Transformações endógenas usualmente são implementadas utilizando técnicas de \aspas{reescrita de grafo}, ou como também é conhecido \aspas{transformação de grafo}~\cite{Ehrig_2006}. Na teoria dos grafos, \aspas{reescrita de grafo} é um conjunto de regras para reescrever um determinado grafo, por exemplo, dado $p: \emph{left-hand side} (LHS) \rightarrow \emph{right-hand side} (RHS)$, sendo $LHS$ o grafo usado como padrão (no lado esquerdo) e $RHS$ o grafo de substituição (no lado direito da regra). Mais precisamente, o lado esquerdo, LHS, representa todas às pré-condições que devem ser satisfeitas antes da execução das regras de transformações. Similarmente, o lado direito, RHS, contêm todas as pós-condições. As ações que serão executadas pelas regras de transformações são implicitamente definidas tanto no grafo LHS quanto no grafo RHS. A execução de um conjunto de regra de transformação produz os seguintes efeitos: (\textit{i}) todos os elementos que apenas estão contidos no grafo LHS são deletados; (\textit{ii}) todos os elementos que apenas estão contidos no grafo RHS são adicionados; e (\textit{iii}) todos os elementos que estão contidos em ambos os grafos, LHS e RHS, são preservados~\cite{Ehrig_2006}. 

Neste contexto, \aspas{reescrita de grafo} é útil para auxiliar na definição de transformações de modelos e metamodelos. Por exemplo, de acordo com~\citeonline{Lehnert_2012, Fluri_2006}, técnicas de \aspas{reescrita de grafo} podem ser aplicadas em qualquer metamodelo e modelos que implementam o padrão MOF, ou seja, KDM, UML, entre outros. Qualquer instância de um metamodelo que implemente o padrão MOF pode ser representado como um grafo da seguinte forma: (\textit{i}) vertices podem ser entendidos como: \texttt{EPackage}, \texttt{EClass}, \texttt{EDataType}, \texttt{EEnum}, \texttt{EAnotation}, \texttt{EOperation}, \texttt{EAttribute} e \texttt{EEnumLiteral}; (\textit{ii}) arestas pode ser representadas em metamodelo como: \texttt{EReference}, \texttt{Inheritance}, \texttt{EAnnotationLink}. Assim, pode-se definir e realizar evoluções, simulações, refatorações de modelos por meio de técnicas de \aspas{reescrita de grafo}. 

% Dessa forma, as gramáticas de um grafo consistem de um conjunto de regras de transformações e um grafo inicial (geralmente referenciado como grafo \aspas{hospedeiro}) onde as regras de transformações são aplicadas. Usualmente as regras de transformações consistem de um grafo denominado \emph{left-hand side} (LHS) e um grafo \emph{right-hand side} (RHS). O grafo LHS tem como intuito representar todas as pré-condições para antes do conjunto de regras de transformações serem aplicadas no modelos. Similarmente, o grafo RHS contêm todas as pós-condições. As ações que serão executadas pelas regras de transformações são implicitamente definidas tanto no grafo LHS quanto no grafo RHS. Mais precisamente, a execução de um conjunto de regra de transformação produz os seguintes efeitos: (\textit{i}) todos os elementos que apenas estão contidos no grafo LHS são deletados; (\textit{ii}) todos os elementos que apenas estão contidos no grafo RHS são adicionados; e (\textit{iii}) todos os elementos que estão contidos em ambos os grafos, LHS e RHS, são preservados. 

Comumente, transformações em modelos são desenvolvidas utilizando linguagens especializadas, denominadas de linguagens de transformação de modelos. Diversas linguagens de transformação de modelos têm sido propostas atualmente~\cite{Allilaire_06, Biehl_2010}. Cada linguagem tipicamente fornece um conjunto de características que a torna mais apropriada para o tipo de transformação almejada. No trabalho de~\citeonline{Biehl_2010}, são citadas várias linguagens de transformação de modelos, o que mostra uma dimensão do número de linguagens para transformação de modelos existentes e disponíveis para o usuário atualmente. Algumas das linguagens citadas são: ATL~\cite{ATL_eclipse,Jouault_2008}, \sigla{QVT}{\textit{Query/View/Transformation}}~\cite{QVT:OMG}, EMF Henshin~\cite{EMF_Henshin}, SmartQVT~\cite{SmartQVT}, ModelMorf~\cite{ModelMorf}, Kermeta~\cite{kermeta}, \sigla{ETL}{\textit{Epsilon Transformation Language}}~\cite{ETL_eclipse}, \sigla{OAW}{OpenArchitectureWare}~\cite{OpenArchitectureWare}, VIATRA~\cite{viatra}, AndroMDA~\cite{andromda} e Fujaba transformations~\cite{fujaba}.

Nos trabalhos~\cite{Biehl_2010, Mens_2006, Allilaire_06}, os autores buscam levantar características importantes tanto para classificar as transformações de modelos, quanto as linguagens de transformação de modelos  para realização destas transformações. Similarmente,~\citeonline{transformation_huber} busca avaliar diferentes ferramentas e linguagens de transformação de modelos. O estudo conclui que nenhuma ferramenta é melhor do que a outra, mas que uma linguagem pode ser mais adequada para um problema específico do que outras linguagens. Entre as várias características utilizadas pelos autores na classificação das linguagens de transformação, a característica de maior importância é quanto ao paradigma da linguagem. Segundo~\citeonline{Mens_2006}, a maior distinção entre os mecanismo de transformação de modelos é quanto ao seu paradigma. Os principais paradigmas das linguagens de transformações de modelos são descritos, a seguir:

\begin{itemize}
	\item Imperativo: Linguagens imperativas especificam um fluxo de controle sequencial e fornecem meios para descrever a forma como a linguagem de transformação de modelo supostamente deve ser executada~\cite{Mens_2006}. As construções e conceitos de linguagens de transformações imperativas são semelhantes às linguagens de programação de propósito geral, como Java ou C;
	\item Declarativo: Linguagens declarativas não oferecem um fluxo de controle explícito. Em vez de como a transformação deve ser executada, o foco é sobre o que deve ser mapeado pela transformação~\cite{Mens_2006}. Transformações de modelos declarativas descrevem a relação entre os metamodelos fonte e alvo, onde esta relação pode ser interpretada bidirecionalmente. Em geral, tais linguagens são compactas e as descrições de transformações são geralmente curtas e concisas~\cite{Biehl_2010, Mens_2006};
	\item Híbrido: Linguagens de transformação híbridas oferecem tanto as construções de linguagem imperativa quanto as construções de linguagem declarativa;
	\item Transformação Direta: Linguagens de programação de uso geral e bibliotecas para ler e gravar os dados dos modelos são utilizadas para implementar as transformações de modelos~\cite{transformation_huber}. A vantagem da transformação direta é que os programadores não precisam aprender uma nova linguagem. Mas por outro lado, as implementações tendem a se tornar maiores~\cite{Biehl_2010}.
\end{itemize}


%Na literatura é possível identificar um conjunto de linguagens especificadas para auxiliar a realização de transformações em modelos. Por exemplo, VIATRA [16], AGG [17], Henshin [18], ATOM [19], ETL, ATL, QVT. De acordo com Livro MDD ATL é a linguagem de transformações em modelos mais utilizada tanto academicamente quanto industrialmente. Dessa forma, nesta tese de doutorado optou-se por utilizar a ATL para a definição de transformações em modelos. Na subseção a seguir é apresentado como a ATL pode ser utilizada para a realização de transformações em modelos.

\subsection{ATLAS \emph{Transformation Language} (ATL)} % (fold)
\label{sub:atl_transformation_language}

A ATLAS \emph{Transformation Language} (ATL)~\cite{ATL_eclipse} é uma linguagem de transformação de modelo híbrida, ou seja, a linguagem contêm uma mistura de construções declarativas e imperativas. O uso do estilo declarativo é encorajado por vários autores~\cite{Allilaire_06, Jouault_2005, Jouault_2008}, pois permite uma implementação mais objetiva e mais simples. No entanto, a definição de transformações complexas utilizando apenas construções declarativas pode ser uma tarefa difícil. Nesse caso, os desenvolvedores podem recorrer aos recursos imperativos da linguagem~\cite{Allilaire_06}.

A ATL possui uma sintaxe abstrata definida utilizando um metamodelo. Isso significa que cada transformação definida em ATL é de fato um modelo. Uma transformação ATL pode ser decomposta em três partes: um \textit{header}, \textit{helpers} e um conjunto de \textit{rules}. O \textit{header} (cabeçalho) é utilizado para declarar informações gerais, tais como o nome do módulo (nome da transformação que deve coincidir com o nome do arquivo .atl), os metamodelos de entrada e de saída  e a importação de bibliotecas necessárias. Os \textit{helpers} são sub-rotinas que são usados para evitar a redundância de código. Pode-se imaginar um \textit{helper} como um método igual tem-se em linguagens de programação. Já as \textit{rules} (regras) são as principais definições das transformações ATL, porque elas descrevem como os elementos de saída (em conformidade com o metamodelo de saída) são produzidos a partir de elementos de entrada (em conformidade com o metamodelo de entrada). Elas são constituídas por ligações, cada uma expressando um mapeamento entre um elemento de entrada e um elemento de saída~\cite{ATL_eclipse}.

O funcionamento da ATL se dá da seguinte forma. Primeiro o código ATL deve ser compilado e, em seguida, executado pelo mecanismo de transformação ATL. A ATL oferece suporte dedicado para rastreabilidade. A ordem de execução das regras é determinada automaticamente, com exceção das \textit{Lazy Rules}, que precisam ser chamadas explicitamente no código da ATL. Os \textit{helpers} fornecem construções imperativas às transformações. A ATL também contêm um modulo denominada ATL Refining que suporta transformações do tipo \emph{endogenous} \aspas{\emph{in-place}}.

A ATL foi escolhida para implementação deste trabalho considerando vários aspectos. A ATL está integrada na plataforma Eclipse, o que provê uma série de recursos padrões para o desenvolvimento (\textit{syntax highlighting} e \textit{debugger}). A ATL é parte do projeto \sigla{M2M}{Model-To-Model} da ferramenta Eclipse e possui um grupo de discussão ativo, constatemente atualizado, vários exemplos e diversos estudos de casos aplicados até mesmo na indústria\footnote{\texttt{\texttt{https://www.eclipse.org/forums/index.php?t=thread&frm_id=241}}} utilizam a ATL. Por se tratar de uma ferramenta de fácil uso, a partir de premissa de conhecimento de linguagem e do metamodelo, traz como vantagem ao processo: baixo custo, por ser uma ferramenta livre, e alta flexibilidade, por facilitar grandes alterações na transformação diretamente usando a interface do editor de regras ATL~\cite{Salem_2008}. Além disso, a ATL é uma das linguagens de transformações mais madura no contexto da MDE~\cite{bruneliere_2010}.
































%\subsection{Refatorações para Modelos}\label{sec:refactoringModel_section}


%Refatorações para modelos também têm-se a preocupação de preserva o comportamento após a aplicação de refatorações da mesma forma que refatorações tradicionais. A abordagem mais popular para definir a preservação de comportamento no contexto de modelos é por meio de restrições. Restrições são asserções que o modelo deve satisfazer antes e após a aplicação de refatoração. Tais asserções são representadas e definidas em nível de modelo por meio de pré- e pós-condições que devem ser validadas antes de executar a refatoração ou validados após a aplicação de refatorações. OCL é a linguagem mais utilizada na literatura para definir asserções para modelos.

\section{Considerações Finais}\label{capitulobaclCOnsideracoesFinais}

Neste capítulo foi apresentada uma revisão dos principais conceitos envolvendo engenharia dirigida por modelos e refatoração que são relevantes para a proposta desta tese. 

Foram discutidas e apresentadas todas as etapas que devem ser realizadas para a condução da engenharia dirigida por modelos, ou seja, todos os níveis CIM, PIM e PSM foram apresentados. Em seguida, foi apresentada a definição e diferença de metametamodelo, metamodelo, modelo e dados. Posteriormente, transformações em modelos foram apresentadas e discutidas, salientando as principais classificações relacionadas a transformações encontradas na literatura - vertical ou horizontal; endógenas ou exógenas. Ainda em relação à transformação de modelos, algumas das principais linguagens utilizadas para realizar a transformação em modelos também foram apresentadas. Porém, apenas a linguagem ATL foi discutida com maiores informações. Os principais conceitos relacionados com refatoração também foram apresentados.

No próximo capítulo são apresentados os principais conceitos envolvendo Modernização Orientada à Arquitetura e KDM.


\chapter{Modernização Orientada à Arquitetura e KDM}
\label{chapter:adm_kdm}

\section{Considerações Iniciais}

Este capítulo apresenta uma abordagem conceitual da modernização de sistemas e os esforços da OMG para a realização de uma padronização em nível de modelo desses sistemas. O escopo do trabalho é delimitado e se discute especificadamente sobre Modernização Orientada à Arquitetura (do inglês - \textit{Architecture-Driven Modernization} - ADM) e \textit{Knowledge Discovery metamodel} (KDM). ADM é o processo de modernização apoiado por modelos da OMG que utiliza os conceitos da MDE. Neste capítulo, são abordados os principais pontos dessa modernização, bem como é apresentada uma descrição mais detalhada do seu principal metamodelo, que é o KDM. Esse capítulo fornece a base necessária para realizar e entender refatorações em nível de modelo, mais especificadamente no metamodelo KDM.

\section{Modernização Orientada à Arquitetura} 

Eu só quero ver se é rapido a sincronização.


O crescente interesse na MDE motivou a \sigla{OMG}{\textit{Object Management Group}} a lançar a iniciativa denominada Modernização Dirigida à Arquitetura (do inglês - \textit{Architecture-Driven Modernization} (ADM)), cujo objetivo foi estabelecer metamodelos padronizados para auxiliar todo o  processo da reengenharia de software. Tal iniciativa foi motivada devido ao alto número de projetos reengenharia de software que não obtiveram sucesso~\cite{Sneed_2005, Demeyer2}. Como resultado desse esforço os conceitos da ADM foram criados, os quais possuem como objetivo revitalizar/modernizar softwares existentes\footnote{No contexto desse documento softwares existentes e sistemas legados são utilizados de forma intercambiáveis} com a utilização de metamodelos padronizados empregando os princípios da abordagem Arquitetura Dirigida à Modelo (do inglês - \sigla{MDA}{\textit{Model-Driven Architecture}}) (ver Seção~\ref{Cap2_Sec2_Desenvolvimento_Dirigido_a_Modelos}). %No contexto da ADM, todos os modelos são homogêneos, permitindo assim a criação de transformações \textit{Model-To-Model} M2M 

De acordo com a OMG~\cite{OMG_OMG} o objetivo da ADM não é substituir o processo tradicional da reengenharia de software, pelo contrário, a ADM tem como objetivo auxiliar e melhorar o processo de reengenharia de software por meio da utilização dos princípios da MDA. A ADM consiste em uma adaptação do modelo de ferradura tipicamente conhecido em reengenharia de software (i.e, o modelo de ferradura, basicamente contêm dois lados, esquerdo, direito e uma \aspas{ponte} ligando os dois lados). Na Figura~\ref{fig:horse_shoe} é apresentado o modelo de ferradura adaptado para a ADM. É importante observar que essa figura contêm todas as fases e \aspas{palavras-chaves} tradicionais que são encontradas na reengenharia de software tradicional e em MDA, tais como: Modelo Especifico de Plataforma (do inglês - \sigla{PSM}{\textit{Platform-Specific Model}}), Modelo Independente de Plataforma (do inglês - \sigla{PSM}{\textit{Platform-Independent Model}}) e Modelo Computacional Independente (do inglês - \sigla{CIM}{\textit{Computation-Independent Model}}). As fases tradicionais da reengenharia de software adaptadas para a ADM são:


\begin{itemize}
 	\item \textbf{Engenharia Reversa}: Esta fase têm como entrada um sistema que será modernizado, posteriormente esse sistema é transformado em um PSM. Além disso, o PSM é utilizado como entrada para a geração do PIM, que no contexto dessa tese consiste em uma instância do metamodelo denominado KDM (ver Seção~\ref{sec:knowledge_discovery_meta_model}) que será explicado com mais detalhes posteriormente;
 	\item \textbf{Reestruturação}: Nesta fase um conjunto de reestruturação/refatoração podem ser aplicadas sobre uma instância do metamodelo KDM por meio de transformações de modelo para modelo (do inglês - \sigla{M2M}{\emph{Model-To-Model}});
 	\item \textbf{Engenharia Avante}: Nesta fase um novo código-fonte do sistema modernizado é gerado automaticamente por meio de transformações de modelo para código (do inglês - \sigla{M2C}{\emph{Model-To-Code}}). 
 \end{itemize} 

 \begin{figure}[htb]
 \caption{Modelo de ferradura adaptada para a ADM}
 \label{fig:horse_shoe}
 \centering
 \includegraphics[scale=0.8]{images/horseshoes.pdf}
 \fadaptada{ADM:OMG}
\end{figure}

Como ressaltado anteriormente, durante todo o processo da ADM todos os modelos (i.e., PSM, PIM e CIM) podem estabelecer transformações/refatorações entre si, como ilustrado na Figura~\ref{fig:horse_shoe}. Geralmente tais transformações são executadas por meio de linguagens de transformações. Como salientado no Capítulo~\ref{chapter:fundamentacao_teorica}, Seção~\ref{sec:transformacoes_de_modelos} usualmente essas transformações são implementadas utilizando diferentes linguagens de transformações que podem ser declarativas, imperativas ou híbridas. 

É importante também destacar que a ADM não tem como intuito apenas seguir todos os princípios da abordagem MDA~\cite{ADM:OMG}. Um dos principais objetivos da ADM é definir um conjunto de metamodelos, padronizados, para lidar com diferentes desafios que são encontrados durante a reengenharia de software. Dessa forma, em Novembro de 2003, a \sigla{ADMTF}{\textit{Architecture-Driven Modernization Task Force}} criou uma \sigla{RFP}{\textit{Request-for-Proposal}} que por sua vez descrevia um conjunto de metamodelos, tais metamodelos são: (\textit{i}) \sigla{KDM}{\textit{Knowledge Discovery metamodel}}, maiores informações sobre esse metamodelo é apresentado na seção~\ref{sec:knowledge_discovery_meta_model}, (\textit{ii}) \sigla{SMM}{\textit{Structured Metrics metamodel}} que é um metamodelo para representar e definir métricas e resultados de medições, (\textit{iii}) \sigla{ADMPR}{\textit{ADM Pattern Recognition}}, que facilita a busca de padrões em um softare, (\textit{iv}) \sigla{ADMVS}{\textit{ADM Visualization Specification}}, que tem como objetivo representar visualmente metadados de uma aplicação representada em KDM, (\textit{v}) \sigla{ADMRS}{\textit{ADM Refactoring Specification}}, que almeja definir um metamodelo padronizado para especificar e definir refatorações utilizando outros metamodelos da ADM, como por exemplo o KDM. O estado atual de cada metamodelo pode ser visto na Tabela~\ref{tab:todos_os_meta_modelos_da_ADM}~\cite{ADM:OMG}. É de suma importância destacar que alguns metamodelos ainda encontram-se em fase de desenvolvimento e outros já foram finalizadas e disponíveis pela OMG.

% Please add the following required packages to your document preamble:
% \usepackage{multirow}
\begin{table}[]
\centering
\caption{Estado atual dos metamodelos da ADM.}
\label{tab:todos_os_meta_modelos_da_ADM}
\begin{tabular}{|l|l|l|l|}
\hline
metamodelo                                         & Situação           & Versão & Data          \\ \hline
ADMPR (ADM Pattern Recognition)                     & Em desenvolvimento &        &               \\ \hline
ADMRS (ADM Refactoring Specification)               & Em desenvolvimento &        &               \\ \hline
ADMVS (ADM Visualization Specification)             & Em desenvolvimento &        &               \\ \hline
ASTM (Abstract Syntax Tree Metamodel)               & Disponível         & 1.0    & Janeiro/2011  \\ \hline
KDM (Knowledge Discovery Metamodel)                 & Disponível         & 1.3    & Agosto/2011   \\ \hline
\multirow{2}{*}{SMM (Structured Metrics Metamodel)} & Disponível         & 1.0    & Janeiro/2012  \\ \cline{2-4} 
                                                    & Em desenvolvimento & 1.1    & Novembro/2013 \\ \hline
\end{tabular}
\end{table}

É importante destacar que a abordagem proposta nesta Tese, se concentra no metamodelo KDM. Consequentemente, é de suma importância o entendimento desse metamodelo, por isso, esse metamodelo é mais detalhado neste capítulo.
KDM é um metamodelo que pode ser utilizado para representar todos os artefatos de um determinado software existente, por exemplo, o KDM contêm metaclasses especificas para representar desde de código-fonte até a arquitetura de um determinado software. O KDM é um metamodelo de representação intermediária comum para sistemas existentes e seus ambientes operacionais. Utilizando esse metamodelo para sistemas existentes é possível trocar representações do sistema em modelo entre plataformas e linguagens com a finalidade de analisar, padronizar e transformar/refatorar os sistemas existentes~\cite{ADM:OMG}. A ideia por trás do KDM é que a comunidade comece a criar analisadores sintáticos (do inglês \textit{parsers}) para diferentes linguagens de programação, que transforme os códigos-fontes em instâncias do metamodelo KDM. Como resultado, qualquer técnica, ferramenta, abordagem, etc, que utilize KDM como o artefato de entrada pode ser considerado uma técnicas, ferramenta, abordagem independente de linguagem e de plataforma. Por exemplo, um catálogo de refatoração para o KDM~\cite{durelli_catalogo, durelli_VEM_ferramenta} (ver Capítulo~\ref{cap:refactoring_catalogue}) pode ser utilizado para refatorar vários sistemas independentemente da linguagem de programação. Maiores informações sobre o KDM, bem como seus pacotes, metaclasses e metarelacionamentos são apresentados a seguir.

\section{Knowledge Discovery metamodel}
\label{sec:knowledge_discovery_meta_model}

\sigla{KDM}{Knowledge Discovery metamodel} é um metamodelo para representar existentes artefatos de software, seus elementos, associações, e ambientes operacionais. De acordo com~\cite{KDM:specification, PerezCastillo:2011jo, ADMCHAPTERR} o KDM tem como principal objetivo permitir que engenheiros de software criem ferramentas para auxiliar a modernização de software que sejam independente de plataforma e linguagem. Além disso, o KDM facilita e assegura a interoperabilidade e troca de dados entre diferentes ferramentas. 

Um problema tradicional facilmente identificado em várias ferramentas que lidam com a reengenharia de software é que tais ferramentas analisam diversos artefatos de um determinado software (por exemplo, código-fonte, banco de dados, \textit{scripts}, etc.) para obter conhecimentos explícitos com o intuito de realizar transformações/refatorações~\cite{rosenberg, Canfora2011}. Como consequência cada ferramenta gera e analisa tais conhecimentos de forma implícita, ou seja, os conhecimento gerados são restritos à uma especifica linguagem de programação, e/ou a uma plataforma. Como resultado tais restrições podem criar dificuldades com relação a interoperabilidade entre diferentes ferramentas. Consequentemente, o KDM fornece uma estrutura que tem com objetivo facilitar a troca de dados entre diversas ferramentas. Além disso, o KDM possui um conjunto de metaclasse e uma estrutura padronizada que fornece meios para especificar desde artefatos físicos até lógicos de um determinado sistema. Em virtude dessa padronização, todas as técnicas/abordagem/ferramentas que utilizam o KDM como entrada podem ser consideradas independente de plataforma e linguagem, aumentando assim, a interoperabilidade e o reúso. 


Pode-se sumarizar os principais objetivos do KDM como~\cite{ADM:OMG}:

\begin{itemize}
	\item KDM representa artefatos de um sistema legado como entidades, relacionamentos e atributos;
	\item KDM suporta uma  variedade de plataforma e linguagem;
	\item KDM define uma terminologia unificada para artefatos de sistemas legados;
	\item KDM representa estruturas lógicas e físicas de sistemas legados;
	\item KDM permite a modificação/refatoração de sistemas legados utilizando os princípios da MDA;
	\item KDM facilita o rastreamento de mudança entre artefatos;
	\item KDM facilita a sincronização de estruturas lógicas e físicas de um determinado sistema legado;
	\item KDM contêm metaclasses para representar desde código-fonte até metaclasses para representar elementos arquiteturais de um determinado sistema legado.
\end{itemize}

Em 2012 o KDM tornou-se \sigla{ISO}{\emph{International Standards Organization}}~\cite{KDM:ISO} como uma estrutura que facilita a troca de dados entre as diversas ferramentas. O KDM é definido via \sigla{MOF}{\emph{Meta-Object Facility}}~\cite{MOF}. Além disso, o KDM estabelece o formato de troca de dados via \sigla{XMI}{\emph{XML Metadata Interchange}}, o qual é denominado KDM XMI \emph{schema}. 


De acordo com~\citeonline{KDM:specification} e~\citeonline{PerezCastillo:2011jo} o KDM possui o objetivo de cobrir um amplo escopo, para abranger um conjunto grande e diversificado de aplicações, plataformas e linguagens de programação. Um dos principais objetivos do KDM é fornecer a capacidade de troca de metadados entre diversas ferramentas e, assim, facilitar a cooperação entre fornecedores para integrar e aumentar a interoperabilidade de diferentes abordagens, técnicas, algoritmos, etc. A fim de alcançar essa interoperabilidade e, especialmente, a integração de informações sobre diferentes facetas de um determinado sistema a partir de múltiplas ferramentas, o KDM define vários níveis de conformidade, aumentando assim a probabilidade de que duas ou mais ferramentas apoiaram o mesmo metamodelo. Além disso, o KDM também é estruturado de forma modular, seguindo o princípio da separação de interesse, com a capacidade de representar partes heterogêneas de um sistema. A separação de interesses no contexto do metamodelo KDM são alcançadas por meio de pacotes, como apresentado na Figura~\ref{kdm:domain}. Cada pacote está em um nível de conformidade e possui o objetivo de definir um ponto de vista arquitetural do sistema. Em outras palavras, cada pacote do KDM constitui uma determinada ontologia para descrever e representar a grande maioria dos artefatos de sistemas de software existentes. Por exemplo, os pacotes \texttt{Code} e \texttt{Action} definem metaclasses para representar o código-fonte de um sistema, tais como, variáveis, procedimentos/métodos/funções, chamadas para procedimentos/métodos/funções. Similarmente o pacote \texttt{Structure} contêm metaclasses para representar elementos arquiteturais do sistema, tais como, componentes, camadas, sub-componentes, etc. O pacote \texttt{Conceptual} possui  metaclasses para definir regras de negócio do sistema.


\begin{figure}[htb]
 \caption{Pacotes e nível de conformidade do metamodelo KDM.}
 \label{kdm:domain}
 \centering
 \includegraphics[scale=1]{images/kdmLevels_pacotes.pdf}
 \fadaptada{KDM:specification}
\end{figure}

Da perspectiva de um engenheiro de software, essa separação de interesse do KDM, por meio de pacote, significa que o engenheiro só precisa se preocupar com os pacotes do KDM que considerar necessários para as suas atividades de modernização, por exemplo, uma determinada abordagem pode necessitar apenas do pacote \texttt{Code} e \texttt{Action}, enquanto uma outra abordagem utilize apenas o pacote responsável por definir elementos arquiteturais. Se essas abordagens forem evoluídas ao longo do tempo e necessitarem de outros pacotes do KDM, os respectivos pacotes podem ser adicionado ao repertório da abordagem/ferramenta, conforme necessário.

Como observado na Figura~\ref{kdm:domain} o KDM possui três níveis de conformidade, nível 0, nível 1 e nível 2. Informações sobre cada nível é apresentado a seguir:

\begin{itemize}
    \item \textbf{Nível 0}: nesse nível de conformidade são definidos os seguintes pacotes do KDM: (\textit{i})     \texttt{Core}, (\textit{ii}) \texttt{kdm}, (\textit{iii}) \texttt{Source}, (\textit{iv}) \texttt{Code} e (\textit{v}) \texttt{Action}. Mais importante, esse nível de conformidade representa um denominador comum que pode servir como uma base para a interoperabilidade entre diferentes categorias de ferramentas que utilizem o metamodelo KDM. Para que uma ferramenta esteja em conformidade com o Nível 0 ela deve fornecer completo suporte para todas as metaclasses que foram definidas nos pacotes \texttt{Core}, \texttt{kdm}, \texttt{Source}, \texttt{Code} e \texttt{Action};
    \item \textbf{Nível 1}: nesse nível os pacotes definidos no Nível 0 são estendidos para representar outros artefatos de um determinado sistema. Especificamente esse nível define os seguintes pacotes: (\textit{i}) \texttt{Build}, (\textit{ii}) \texttt{Structure}, (\textit{iii}) \texttt{Data}, (\textit{iv}) \texttt{Conceptual}, (\textit{v}) \texttt{UI}, (\textit{vi}) \texttt{Event} e (\textit{vii}) \texttt{Platform}. Para que uma ferramenta esteja em conformidade com o Nível 1 ela deve fornecer suporte para pelo menos um do pacotes devido nesse nível;
    \item \textbf{Nível 2}: esse nível é a união de todos os pacotes definidos no nível anterior. Para que uma ferramenta esteja em conformidade com o Nível 2 ela deve fornecer suporte para todos os pacotes devido no Nível 1.
\end{itemize}


Posteriormente todos os pacotes do KDM são organizados em quatro camadas de abstração. Tais camadas são:

\begin{itemize}
    \item \sigla{CI}{Camada de Infraestrutura}: do inglês \textit{Infrastructure Layer};
    \item \sigla{CEP}{Camada de Elementos de Programa}: do inglês \textit{Program Elements Layer};
    \item \sigla{CRTE}{Camada de Recurso de Tempo de Execução}: do inglês \textit{Runtime Resource Layer};
    \item \sigla{CA}{Camada de Abstração}: do inglês \textit{Abstraction Layer}.
\end{itemize}


%(\textit{i}) \sigla{CI}{Camada de Infraestrutura} (do inglês \textit{Infrastructure Layer}), (\textit{ii}) \sigla{CEP}{Camada de Elementos de Programa} (do inglês \textit{Program Elements Layer}), (\textit{iii}) \sigla{CRTE}{Camada de Recurso de Tempo de Execução} (do inglês \textit{Runtime Resource Layer}) e (\textit{iv} ) \sigla{CA}{Camada de Abstração} (do inglês \textit{Abstraction Layer}). 

%Cada camada é posteriormente organizada em pacotes, como pode ser observado na Figura~\ref{fig:kdm_layer}. Por sua vez, cada pacote define um conjunto de metaclasses cujo propósito é representar o conhecimento de específicos artefatos de um determinado sistema. 

%
\begin{figure}[htb]
 \caption{Camadas e pacotes do KDM.}
 \label{fig:kdm_layer}
 \centering
 \includegraphics[scale=0.8]{images/kdm_layers.pdf}
 \fadaptada{KDM:specification}
\end{figure}
%
A CI contêm três pacotes, são eles: (\textit{i}) \texttt{Core}, (\textit{ii}) \texttt{\aspas{kdm}} e (\textit{iii}) \texttt{Source}. Os dois primeiros pacotes, \texttt{Core} e \texttt{\aspas{kdm}}, representam a infraestrutura básica para outros pacotes do KDM, ou seja, definem metaclasses e relacionamentos básicos. O pacote \texttt{Source} define o \texttt{Inventory Model}, o qual representa artefatos de software e mantêm a rastreabilidade entre eles. 

A CEP possui dois pacotes: (\textit{i}) \texttt{Code} e (\textit{ii}) \texttt{Action}. Esses pacotes coletivamente definem o \texttt{Code Model}, o qual contêm metaclasses para representar artefatos em nível de implementação. O pacote \texttt{Code} contêm um conjunto de metaclasses para representar a estrutura de um determinado programa e seus relacionamentos, enquanto que o pacote \texttt{Action} possui metaclasses para descrever o comportamento e o fluxo de dados de um programa.

A CRTE contêm quatro pacotes: (\textit{i}) \texttt{Data}, (\textit{ii}) \texttt{Platform}, (\textit{iii}) \texttt{Event} e (\textit{iv}) \texttt{UI}. Coletivamente tais pacotes representam a estrutura e comportamento de recursos de tempo de execução do sistema. Tais pacotes são diretamente instanciados por meio da definição de recursos, abstração do \texttt{Code Model}, ou são manualmente instanciado pelo engenheiro de modernização. A rastreabilidade entre os elementos abstraídos e os elementos físicos (por exemplo, código-fonte) são mantidos pelo meta-atributo denominado \texttt{implementation}. Finalmente, a CA contêm três pacotes: (\textit{i}) \texttt{Conceptual}, (\textit{ii}) \texttt{Structure} e (\textit{iii}) \texttt{Build}. Esses pacotes possuem metaclasses para representar o maior nível de abstração de um sistema, por exemplo, a estrutura do sistema, regras de negócios, documentações do sistema, etc.

\begin{figure}[htb]
 \caption{Representação das quatro camadas do KDM de um determinado sistema.}
 \label{fig:kdm_layer_sistema_completo}
 \centering
 \includegraphics[scale=0.6]{images/kdm_Layer.pdf}
 \fautor
\end{figure}

Como salientado anteriormente cada camada do KDM representa diferentes perspectivas de conhecimento sobre os artefatos de sistemas. Tais camadas e/ou pacotes do KDM podem ser criados automaticamente, semi-automaticamente, ou manualmente por meio da aplicação de várias técnicas de extração de conhecimento, análises e transformação. Por exemplo, na Figura~\ref{fig:kdm_layer_sistema_completo} é apresentado uma arquitetura de software que esta dividida em várias camadas do KDM, e consequentemente utiliza vários pacotes e metaclasses para representar todos os artefatos do software. 

Essa figura pode ser analisada verticalmente e horizontalmente; verticalmente é possível identificar as quatros camadas do KDM apresentadas anteriormente: CI \ding{182}, CEP \ding{183}, CRTE \ding{184} e CA \ding{185}. Tais camadas possuem abstrações (metaclasses) dos artefatos heterogêneos que servem para funções distintas e que estão representados horizontalmente. Por exemplo, em \emph{Data Tier} é definido diferentes tipos de artefatos de bancos de dados, XML, CVS e DAT. Os componentes que realizam a conectividade com os bancos de dados (por exemplo, \sigla{ODBC}{\emph{Open Database Connectivity}} e \sigla{JDBC}{\emph{Java Database Connectivity}}) estão em \emph{Data Access tier}. Em \emph{Processing Logic tier} estão definidos os componentes que implementam regras de negócio. Além disso, também são definidos os componentes que interagem com componentes de interface com o usuário. Em \emph{Proxy tier} estão definidos artefatos que fazem iterações com elementos externos, por exemplo SOAP, DCOM, CORBA, JNI, ou RMI. Finalmente, \emph{Presentation tier} é responsável por produzir e controlar componentes gráficos, tais como, formulários, páginas \emph{web} ou relatórios. 

Uma característica importante de ser observada e ressaltada na Figura~\ref{fig:kdm_layer_sistema_completo} é que todos as camadas do KDM interagem, significando que todas as camadas são conectadas de alguma forma\footnote{Essa conectividade entre os elementos de cada camada são mantidos por um conjunto de meta-atributo, por exemplo, o meta-atributo \texttt{implementation}.}, como consequência se uma mudança/refatoração é realizada em uma camada especifica, a mudança/refatoração deve ser propagada para outras camadas com o intuito de manter todas as camadas sincronizadas e consistentes preservando assim a estrutura estática do KDM e o comportamento do código representado em KDM.

Na Tabela~\ref{tab:visao_geral_kdm} é apresentada uma visão detalhada de todas as camadas do metamodelo do KDM. Além disso, essa tabela também apresenta as técnicas que podem ser utilizadas para instanciar cada pacote do KDM. 

\begin{longtable}{ | m{2.5cm} | m{5.9cm}| m{3.5cm} | m{3.5cm} | } 
 \caption{ Visão geral das camadas do KDM e técnicas utilizadas para instanciar todos os pacotes do KDM.\label{tab:visao_geral_kdm}}\\
 
 \hline
 \multicolumn{4}{| c | }{Início da Tabela~\ref{tab:visao_geral_kdm}}\\
 \hline
 \textbf{Camadas} & \textbf{Pacotes do KDM}  & \textbf{Artefatos de Entrada} &\textbf{Técnicas}\\
 \hline
 \endfirsthead
 
 \hline
 \multicolumn{4}{|c|}{Continuação da Tabela~\ref{tab:visao_geral_kdm}}\\
 \hline
 \textbf{Camadas} & \textbf{Pacotes do KDM}  & \textbf{Artefatos de Entrada} &\textbf{Técnicas}\\
 \hline
 \endhead
 
 \hline
 \endfoot
 
 \hline
 \multicolumn{4}{| c |}{Fim da Tabela~\ref{tab:visao_geral_kdm}}\\
 \hline\hline
 \endlastfoot
 
 \hline CI & Pacotes: \emph{Code} e \emph{Action} (\emph{Code Model}) \newline - Representa aspecto estrutural e comportamental do código-fonte, ou seja, contêm metaclasses para representar todos os elementos do código classes/módulos, funções/métodos, variáveis/propriedades, etc )  & \begin{enumerate}
  \item Aplicação de Software; e
  \item Repositório de arquivos.
  \end{enumerate} & \begin{enumerate}
  \item Sistema de Arquivo; e
  \item Repositories traversal.
  \end{enumerate} \tabularnewline
\hline CEP & Pacotes: \emph{Code} e \emph{Action} (\emph{Code Model}) \newline - Representa aspecto estrutural e comportamental do código-fonte, ou seja, contêm metaclasses para representar todos os elementos do código classes/módulos, funções/métodos, variáveis/propriedades, etc ) & \begin{enumerate}
 	\item Código-fonte;
 	\item Definição da API;
 	\item \emph{Schemas} do Banco de Dados; e
 	\item Definição de recursos. 
 \end{enumerate} & \begin{enumerate}
 	\item \emph{Parsing} da árvore sintatica abstrata (AST); e
 	\item Padrões de reconhecimento.
 \end{enumerate} \tabularnewline
\hline 
\multirow{4}{2.5cm}{
CRTE
} &  \emph{Platform Model} \newline - Representa recursos utilizados pelo software em tempo de exeução, suas composições e seus comportamentos & \begin{enumerate}
	\item \emph{Inventory Model;}
	\item \emph{Code Model};
	\item Definições dos recursos; e
	\item Arquivos de configuração.
\end{enumerate} & \begin{enumerate}
	\item \emph{Parsing} a árvore sintatica abstrata;
	\item Padrões de reconhecimento;
	\item Análise de dependência; e
	\item Clusterização.
\end{enumerate} \tabularnewline
\cline{2-4} 
 & \emph{Data Model} \newline - Representa a estrutura dos elementos de persistência que um determinado sistema possui (tabelas, visões, colunas). & \begin{enumerate}
	\item \emph{Schemas} do Banco de Dados; e
	\item Definições dos recursos.
\end{enumerate} & \begin{enumerate}
	\item \emph{Parsing} da árvore sintatica abstrata (AST); e
 	\item Padrões de reconhecimento.
\end{enumerate} \tabularnewline
\cline{2-4} 
 &  \emph{UI Model} \newline - Representa os elementos gráficos do sistema, seus layout, fluxo de controle, etc & \begin{enumerate}
 	\item Definições dos recursos;
 	\item \emph{Inventory Model}; e
 	\item \emph{Code Model}.
 \end{enumerate} & \begin{enumerate}
 	\item \emph{Parsing} da árvore sintatica abstrata (AST); e
 	\item Análise e instanciação manual das metaclasses.
 \end{enumerate} \tabularnewline
\cline{2-4} 
 & \emph{Event Model} \newline - Representa as aspectos comportamentais do sistema, modelo de transições de estado, etc. & \begin{enumerate}
 	\item Definições dos recursos;
 	\item Arquivos de configuração;
 	\item Code Model;
 	\item Data Model;
 	\item Platform Model; e
 	\item UI Model.
 \end{enumerate} & \begin{enumerate}
 	\item Padrões de reconhecimento; e
 	\item Análise e instanciação manual.
 \end{enumerate} \tabularnewline
\hline 
\multirow{3}{2.5cm}{CA} & \emph{Structure Model} \newline - Representa a composição de elementos arquiteturais do sistema e seus relacionamentos.  & \begin{enumerate}
	\item \emph{Inventory Model};
	\item \emph{Code Model};
	\item \emph{Data Model};
	\item \emph{Platform Model}; e
	\item \emph{UI Model}.
\end{enumerate} & \begin{enumerate}
	\item Análise de dependência;
	\item Clusterização; 
	\item Análise e instanciação manual
\end{enumerate}\tabularnewline
\cline{2-4} 
 & \emph{Conceptual Model} \newline - Representa comportamento e fluxo de cenário, termos de regra de negócio, fatos e regras implementadas pelo sistema.& \begin{enumerate}
 	\item \emph{Code Model};
 	\item \emph{Data Model};
 	\item \emph{Platform Model};
 	\item \emph{UI Model};
 	\item Documentação;
 	\item Banco de dados.
 \end{enumerate} & \begin{enumerate}
 	\item Padrões de reconhecimento;
 	\item Definição manual.
 \end{enumerate}\tabularnewline
\cline{2-4} 
 &  \emph{Build Model} \newline - Representa os fatos sobre o processo de implantação do sistema: \emph{input/output}, ferramentas utilizadas para desenvolver o código-fonte, etc & \begin{enumerate}
 	\item \emph{Inventory Model};
 \end{enumerate} & \begin{enumerate}
 	\item Definição manual;
 	\item Itens do inventório;
 \end{enumerate}\tabularnewline
\hline
 \end{longtable}

 Nas próximas subseções são apresentados os principais pacotes que compõem o contexto para o desenvolvimento deste trabalho.

\subsection{Pacote Code}\label{codePackage}

O pacote $\mathtt{Code}$ define um conjunto de metaclasses cujo propósito é representar unidades de programa em nível de implementação e as suas associações. O pacote também inclui metaclasses que representam elementos de programa comuns suportados por várias linguagens de programação, como tipo de dados, classes, procedimentos, macros, protótipos e \textit{templates}.


%The Code package defines a set of metamodel elements whose purpose is to represent implementation level program elements and their associations. It is determined by one or more programming languages used in the design of the given existing software system. Code package includes metaclasses, which represent common program elements supported by various programming languages, such as data types, data items, classes, procedures, macros, prototypes, and templates.

Em uma determinada instância do KDM, cada elemento do pacote~$\mathtt{code}$ representa alguma construção em uma linguagem de programação, determinada pela linguagem de programação utilizada no sistema. Na Figura~\ref{fig:CodeModel} um trecho do~$\mathtt{CodeModel}$\footnote{O diagrama de classes do $\mathtt{CodeModel}$ mostrado aqui só representa o conjunto de meta classes e os seus respectivos relacionamentos lógicos, para informações completas, verifique a especificação do KDM~\cite{KDM:specification}.} é retratado.

%In a given KDM instance, each instance of the code metamodel element represents some programming language construct, determined by the programming language of the existing software system. Each instance of a code metamodel element corresponds to a certain region of the source code in one of the artifacts of the existing software system. Figure~\ref{fig:CodeModel} the \texttt{CodeModel}\footnote{The \texttt{CodeModel} class diagram presented herein shows just a set of the metaclasses and their logical relationship, for complete information please see the KDM specification} is depicted, it represents parts of the KDM infrastructure. 

\begin{figure}[!ht]
	\centering
	% Requires \usepackage{graphicx}
	\includegraphics[scale=0.8]{images/codeModel}
	\caption{Diagrama de classes - $\mathtt{CodeModel}$}
	\label{fig:CodeModel}
	\fadaptada{KDM:specification}
\end{figure}

A metaclasse \texttt{CodeModel} representa um contêiner para outras instâncias de elementos do tipo~$\mathtt{Code}$. Como pode ser observado, o pacote $\mathtt{Code}$ \ding{205} depende dos outros pacotes $\mathtt{kdm}$ \ding{202}, $\mathtt{Source}$ \ding{203} e $\mathtt{Core}$ \ding{204}. A metaclasse $\mathtt{CodeModel}$ \ding{206} é um modelo que possui coleções de fatos sobre o sistema de software, correspondentes ao domínio $\mathtt{Code}$. Ela possui uma associação  $\mathtt{codeElement:AbstractCodeElement[0..*]}$ permitindo a adição de novos elementos de código, por exemplo, métodos, atributos, etc. A metaclasse  $\mathtt{AbstractCodeRelationship}$ \ding{207} representa qualquer relacionamento determinado por em uma linguagem de programação. Por sua vez, a metaclasse $\mathtt{ComputationalObject}$ representa os elementos determinados pela linguagem de programação, que descreve certos objetos computacionais em tempo de execução, por exemplo, métodos e variáveis.

%This metamodel element is a container for other \texttt{code} element instances. As can be observed in Figure~\ref{fig:CodeModel} the \texttt{Code} package \ding{205} depends on the \texttt{kdm} \ding{202}, \texttt{Source} \ding{203}, and \texttt{Core} \ding{204} packages. The meta-class \texttt{CodeModel} \ding{206} is the specific KDM model that owns collections of facts about the existing software system such that these facts correspond to the \texttt{Code} domain, its superclass is \texttt{KDMModel}. It has as association \texttt{codeElement:AbstractCodeElement[0..*]} meaning that one shall arrange code elements (e.g., methods, fields, etc) into one or more code models. The \texttt{AbstractCodeRelationship} \ding{207} is an abstract meta-class representing any relationship determined by a programming language, it is also used to constrain the subclasses of \texttt{KDMRelationship} (see Figure~\ref{fig:CodeModel} \ding{208}) in the \texttt{Code} model. The meta-class \texttt{ComputationalObject} represents the named elements determined by the programming language, which describe certain computational objects at the runtime, for example, methods, and variables.

O pacote~$\mathtt{Code}$ consiste em um total de 24 metaclasses; tais metaclasses são um arranjo de abstrações para representar toda, ou a grande maioria, da estrutura estática de um terminado código-fonte dado um linguagem de programação, seja ela procedural ou orientada a objetos~\cite{KDM:specification}. Na Tabela~\ref{tab:meta_classes_pacoteCODE} algumas metaclasses são apresentadas. Como pode ser observado algumas metaclasses podem ser diretamente elucidadas e mapeadas, como por exemplo, \aspas{classes} e \aspas{interfaces} construções facilmente encontradas em linguagens orientadas a objetos podem ser facilmente mapeada para as metaclasses denominada $\mathtt{ClassUnit}$ e $\mathtt{InterfaceUnit}$, respectivamente. Um mapeamento mais completo entre elementos estruturais e metaclasses do KDM pode ser identificado em~\citeonline{bruno_marinho_dissertacao}. Uma representação dessas metaclasses, bem como seus relacionamentos são apresentados em diagrama de classe na Figura~\ref{fig:classUnit_e_InterfaceUnit}. 

\begin{figure}[!ht]
	\centering
	% Requires \usepackage{graphicx}
	\includegraphics[scale=0.8]{images/ClassUnit_InterfaceUnit}
	\caption{Diagrama de classes elucidando as metaclasses \texttt{ClassUnit} e \texttt{InterfaceUnit}}
	\label{fig:classUnit_e_InterfaceUnit}
	\fadaptada{KDM:specification}
\end{figure}

%The whole \texttt{Code} package consists of $24$ metaclasses\footnote{Note that not all the meta-class are shown in Figure~\ref{fig:CodeModel}} and contains all the abstract elements for modeling the static structure of the source code. In Table~\ref{tab:mappingCodeToKDM} is depicted some of them. This table identifies KDM metaclasses possessing similar characteristics to the static structure of the source code. Some metaclasses can be direct mapped, such as class from object-oriented language, which can be easily mapped to the \texttt{ClassUnit} meta-class from KDM. For instance, the meta-class \texttt{Package} is a subtype for \texttt{Module} that logical collections of program elements, as directly supported by some programming languages, such as Java. 

\begin{table}[h]
\centering
\caption{metaclasses para modelagem de estruturas estáticas do código-fonte}
\label{tab:meta_classes_pacoteCODE}
\begin{tabular}{|l|l|}
\hline
Elemento do Código-Fonte & metaclasses do KDM \\ \hline
Classe                   & \texttt{ClassUnit}           \\ \hline
Interface                & \texttt{InterfaceUnit}       \\ \hline
Método                   & \texttt{MethodUnit}          \\ \hline
Atributo                 & \texttt{StorableUnit}        \\ \hline
Variável Local           & \texttt{MemberUnit}          \\ \hline
Parâmetro                & \texttt{ParameterUnit}       \\ \hline
Associação               & \texttt{KDMRelationShip}     \\ \hline
\end{tabular}
\end{table}

%\begin{table}[!h]
	%\caption{Meta classes para modelagem de estruturas estáticas do código fonte}
%	\label{tab:mappingCodeToKDM}
%	\centering
%	\includegraphics[scale=1]{images/tabela_comparativo_KDM_code_com_source_code}
%\end{table}

$\mathtt{ClassUnit}$ e $\mathtt{InterfaceUnit}$ representam \aspas{classes} e \aspas{interfaces} que são definidas por usuários de linguagens orientadas a objeto. Ambas metaclasses possuem caracteristicas e relacionamentos similares, como observado na Figura~\ref{fig:classUnit_e_InterfaceUnit} \ding{202} e \ding{203}. Uma das diferenças que pode ser destacada é que a metaclasse ~$\mathtt{ClassUnit}$ contém um meta-atributo~$\mathtt{isAbstract:Boolean}$, o qual é utilizado para especificar se uma classe é ou não abstrata. $\mathtt{ClassUnit}$ e $\mathtt{InterfaceUnit}$ podem conter um coleção de elementos que seja do tipo ~$\mathtt{CodeItem}$, por exemplo,~$\mathtt{StorableUnit}$ ou ~$\mathtt{MethodUnit}$. Além disso, tais metaclasses  possuem uma meta-associação denominada ~$\mathtt{codeElement:CodeItem[0..*]}$ que é utilizada para agrupar todos os membros da classe, por exemplo, construtores, métodos, atributos, etc. Na Figura~\ref{fig:StorableUnit_MethodUnit} é apresentado os meta-atributos e metarelacionamentos das metaclasses \texttt{StorableUnit} \ding{204}, \texttt{MethodUnit} \ding{205}, \texttt{ParameterUnit} \ding{207} e \texttt{MemberUnit} \ding{206}.

\begin{figure}[!ht]
	\centering
	% Requires \usepackage{graphicx}
	\includegraphics[scale=0.8]{images/StorableUnit_MethodUnit2}
	\caption{Diagrama de classes elucidando as metaclasses \texttt{StorableUnit}, \texttt{MethodUnit}, \texttt{ParameterUnit} e \texttt{MemberUnit}}
	\label{fig:StorableUnit_MethodUnit}
	\fadaptada{KDM:specification}
\end{figure}


%As stated before, the meta-class \texttt{ClassUnit} represents user-defined classes in object-oriented languages. A class datatype is a named datatype that represents a class: an ordered collection of named elements, each of which can be another \texttt{CodeItem}, such as a \texttt{StorableUnit} or a \texttt{MethodUnit}. The meta-class \texttt{ClassUnit} contains a meta-attribute \texttt{isAbstract:Boolean} used to specify if the class is abstract or not. \texttt{ClassUnit} also has an meta-association named \texttt{codeElement:CodeItem[0..*]} that is used to group all class's members, e.g., fields, constructor, methods, etc. 

%Similarmente, a meta classe~$\mathtt{InterfaceUnit}$ representa o conceito comum a várias linguagens de programação. Ela é uma subclasse de~$\mathtt{DataType}$, assim como~$\mathtt{ClassUnit}$.~$\mathtt{InterfaceUnit}$ também possui uma meta associação chamada~$\mathtt{codeElement:CodeItem[0..*]}$ que representa tipos de dados tal como~$\mathtt{MethodUnit}$.

%Similarly, the meta-class \texttt{InterfaceUnit} represents the interface concept common to various programming languages. It is also a subclass of \texttt{Datatype} as \texttt{ClassUnit}. \texttt{InterfaceUnit} also has an meta-association named \texttt{codeElement:CodeItem[0..*]} that represent data types as well as \texttt{MethodUnits}.

$\mathtt{StorableUnit}$ representa um atributo em um sistema de software - um objeto computacional para que diferentes valores do mesmo tipo de dados possa ser associado em vezes diferentes. Ele é usado para representar as variáveis globais e locais. Ele contém uma meta atributo~$\mathtt{String}$ usado para definir o nome das variáveis.~$\mathtt{StorableUnit}$ também tem a associação~$\mathtt{type:DataType[1]}$, a qual é herdada da metaclasse~$\mathtt{DataElement}$, e é utilizado para especificar o tipo da variável (\textit{int, char, boolean, numeric}, etc). Ele também tem uma enumeração ~$\mathtt{kind:StorableUnit}$, que descreve várias propriedades comuns de um ~$\mathtt{StorableUnit}$ relacionado com o seu ciclo de vida, por exemplo, sua visibilidade (\textit{private, public, protected}, etc).

%\texttt{StorableUnit} represents a variable of existing software system - a computational object to which different values of the same datatype can be associated at different times. It is used to represent both global and local variables. It contains a meta-attribute \texttt{name:String} used to set the name of the variables. \texttt{StorableUnit} also has the association \texttt{type:Datatype[1]} used to specify the variable's type. It also has a enumeration \texttt{kind:StorableKind}, it describes several common properties of a \texttt{StorableUnit} related to their life-cycle, visibility, and memory type. 

~$\mathtt{MethodUnit}$ como o próprio nome sugere representa métodos que são identificados em ~$\mathtt{ClassUnit}$ ou~$\mathtt{InterfaceUnit}$. Ele também é usado para representar construtores e destrutores. Possui como meta-atributos~$\mathtt{name:String}$,~$\mathtt{kind:MethodKind}$ e ~$\mathtt{export:ExportKind}$. O primeiro é usado para descrever o nome de um método. O segundo meta-atributo é uma enumeração que define especificações adicionais do tipo de método, ou seja, é possível especificar se a instância do método é um construtor, destrutor, ou um método normal. O último representa a visibilidade do método (\textit{private, public, protected}, etc).

%\texttt{MethodUnit} represents member functions owned by either \texttt{ClassUnit} or \texttt{InterfaceUnit}. It is also used to represent user-defined operators, constructors, and destructors. It owns as meta-attributes \texttt{name:String}, \texttt{kind:MethodKind}, and \texttt{export: ExportKind}. The fist one is used to describe to name of a method. The second meta-attribute is an enumeration that defines additional specification of the kind of method, i.e., it is possible to specify if the method's instance is a constructor, destructor, abstract, etc. The last one represents the visibility of the method, i.e., \texttt{public}, \texttt{private}, \texttt{protected}.

A fim de entender como o KDM é utilizado para representar estruturas um determinado programa, no Código-fonte~\ref{lst:example_kdm_instance} é mostrado um exemplo simplificado escrito em Java. O correspondente KDM, embora simplificado, é apresentado na Figura~\ref{fig:kdm_instance_Java}. Por questões de simplicidade e para facilitar o entendimento, essa figura ilustra a instância do KDM em forma de um diagrama de objetos; é possível notar que este diagrama representa o código-fonte como uma árvore, na qual cada nó representa uma metaclasses do KDM. Como pode ser visto na Figura~\ref{fig:kdm_instance_Java}, a metaclasse raiz é ~$\mathtt{Segment}$, que é um recipiente para um conjunto significativo de fatos sobre um sistema de software existente. Cada~$\mathtt{Segment}$ pode incluir uma ou mais instâncias de modelos do KDM, como~$\mathtt{CodeModel}$ e~$\mathtt{StructureModel}$.

%In order to fully understand how KDM is used to represent the source code of a specific program, in Listing~\ref{lst:example_kdm_instance} is shown a simplified example in Java. The corresponding, though simplified KDM instance is depicted in Figure~\ref{fig:kdm_instance_Java}. It illustrates a KDM instance as a UML object diagram for the sake of simplicity, note that this diagram represents the source code as a tree of nodes containing some KDM's metaclasses. As can be seen in Figure~\ref{fig:kdm_instance_Java} the root meta-class is \texttt{Segment}, which is a container for a meaningful set of facts about an existing software system. Each \texttt{Segment} may include one or more KDM model instances, such as \texttt{CodeModel} and \texttt{StructureModel}. As stated earlier, the \texttt{CodeModel} is the specific KDM container for other \texttt{code} element instances (see Figure~\ref{fig:CodeModel}). 

Analisando tanto o Código-fonte~\ref{lst:example_kdm_instance} quanto a Figura~\ref{fig:kdm_instance_Java} é evidente que cada estrutura estática do código-fonte tem uma metaclasse específica em KDM para representá-la. Por exemplo, a declaração \textit{package} \textit{model} na Linha 1 do Código-fonte~\ref{lst:example_kdm_instance} \ding{202} é representada em KDM pela metaclasse ~$\mathtt{package}$, como visto na Figura~\ref{fig:kdm_instance_Java} \ding{202}. Posteriormente, como apresentado no Código-fonte~\ref{lst:example_kdm_instance} \ding {203} uma classe~$\mathtt{Car}$ é declarada. Essa classe herda caracteristicas da classe~$\mathtt{Vehicle}$, no Java isso é feito por meio da palavra-chave~$\mathtt{extends}$ seguido do nome de uma classe, conforme por ser observado no Código-fonte~\ref{lst:example_kdm_instance} \ding{204} e \ding{205}. A metaclasse~$\mathtt{Extends}$ representa o conceito de herança em KDM. Como mostrado na Figura~\ref{fig:kdm_instance_Java} \ding {204} a meta classe~$\mathtt{Extends}$ possui duas associações,~$\mathtt{to}$ e~$\mathtt{from}$, o primeiro representa a classe pai (\textit{super class}), e o último a classe filha(\textit{sub-class}). Neste contexto, a classe~$\mathtt{Car}$ e classe pai de~$\mathtt{Vehicle}$, como mostrado na Figura~\ref{fig:kdm_instance_Java} \ding{205}. Finalmente, o atributo~$\mathtt{name}$ e o método~$\mathtt{getName()}$ (ver Código-fonte~\ref{lst:example_kdm_instance} \ding {206} e \ding {207}, respectivamente) são mapeados para os correspondentes elementos do KDM,~$\mathtt{StorableUnit}$ e~$\mathtt{MethodUnit}$ (ver Figura ~\ref{fig:kdm_instance_Java} \ding {206} e \ding {207}).

%Analyzing both the Listing~\ref{lst:example_kdm_instance} and the Figure~\ref{fig:kdm_instance_Java} it is evident that each static structure of the source code has a meta-class in KDM to represent it. For instance, the package model in Line 1 of Listing~\ref{lst:example_kdm_instance} \ding{202} is represented in KDM by the meta-class named \texttt{Package}, see Figure~\ref{fig:kdm_instance_Java} \ding{202}. In addition, the class \texttt{Car} is declared, see Listing~\ref{lst:example_kdm_instance} \ding{203}. It also inherit from class \texttt{Vehicle}, in the Java this is accomplished by using the keyword \texttt{extends} following of a class, see Listing~\ref{lst:example_kdm_instance}  \ding{204} and \ding{205}. The meta-class \texttt{Extends} represents inheritance in KDM models. As shown in Figure~\ref{fig:kdm_instance_Java} \ding{204} the meta-class \texttt{Extends} has two association, \texttt{to} and \texttt{from}, the former represents the parent class (super class), and the latter represents the child class (sub-class). In this context, the child is the class \texttt{Car} and the parent class is \texttt{Vehicle}, which is depicted in Figure~\ref{fig:kdm_instance_Java} \ding{205}. Finally, the variable \texttt{name}, and the method \texttt{getName()} (see Listing~\ref{lst:example_kdm_instance} \ding{206}, and \ding{207}) are mapped to corresponding instances of the KDM elements, \texttt{StorableUnit}, and \texttt{MethodUnit} (see Figure~\ref{fig:kdm_instance_Java} \ding{206}, and \ding{207}). 


\noindent\begin{minipage}{.53\textwidth}
	\begin{codigo}[caption={[Parte de código Java para ilustrar como o KDM é usado para representar o código fonte.] Simples código em java.},escapeinside={(*@}{@*)}, basicstyle=\footnotesize, label={lst:example_kdm_instance}]{Name}
	(*@\ding{202}@*) package model;
	(*@\ding{203}@*) public class Car (*@\ding{204}@*) extends  
	(*@\ding{205}@*) Vehicle{
	(*@\ding{229}@*)(*@\ding{206}@*) private String name;
	(*@\ding{229}@*)(*@\ding{207}@*) public String getName(){
	(*@\ldots @*)
	}
	}
	\end{codigo}
\end{minipage}\hfill
\begin{minipage}{.45\textwidth}
	\centering
	% Requires \usepackage{graphicx}
	\includegraphics[scale=0.6]{images/kdm_instance_java_correspoding_2_with_extends}
	\captionof{figure}{Instância KDM correspondente ao Código-fonte~\ref{lst:example_kdm_instance}}
	\label{fig:kdm_instance_Java}
\end{minipage}


Apenas metaclasses utilizadas para representar estruturas e construções estáticas foram apresentadas. No entanto, o KDM contêm um pacote que permite a representação de construções dinâmicas, em outras palavras, o pacote \textit{Action} contêm metaclasses cuja finalidade é permitir e representar comportamento em nível de execução. Na seção a seguir mais informações sobre esse pacote é apresentado.

\subsection{Pacote Action}\label{sec:actionPackage}

O pacote~$\mathtt{Action}$ define um conjunto de metaclasses, cujo propósito é o de representar descrições de comportamento em nível de implementação estabelecido por linguagens de programação, por exemplo, declarações, operadores, condições e as suas associações. A Figura ~\ref{fig:actionModel} \ding {202} mostra o pacote~$\mathtt{Action}$ e algumas de suas meta classes. Como pode ser observado, este pacote estende o pacote~$\mathtt{Code}$ (ver a Figura~\ref {fig:actionModel} \ding {208}).

%The \texttt{Action} package defines a set of metaclasses whose purpose is to represent implementation-level behavior descriptions determined by programming languages, for example statements, operators, conditions, features, as well as their associations, for example control and data flow. The Figure~\ref{fig:actionModel} \ding{202} shows the\texttt{Action} package and some of its metaclasses. As can be observed it extends the KDM \texttt{Code} package (see Figure~\ref{fig:actionModel} \ding{208}). 

\begin{figure}[!ht]
	\centering
	% Requires \usepackage{graphicx}
	\includegraphics[scale=0.8]{images/ActionModel_Class_Diagram}
	\caption{Diagrama de classes ilustrando o pacote~$\mathtt{Action}$.}
	\label{fig:actionModel}
	\fadaptada{KDM:specification}
\end{figure}

O pacote~$\mathtt{Action}$ é composto por 11 diagramas de classe, ele também depende dos pacotes~$\mathtt{Core}$, ~$\mathtt{kdm}$ e ~$\mathtt{Source}$, e  principalmente do pacote~$\mathtt{Code}$. No entanto, o pacote~$\mathtt{Action}$ segue o padrão uniforme para os modelos KDM e estende o KDM com metaclasses específicas relacionadas com o comportamento do nível de implementação. O pacote~$\mathtt{Action}$ se desvia de um padrão uniforme para os modelos KDM porque ele não define um modelo KDM separado, mas estende o pacote~$\mathtt{Code}$.  Por isso, cada metaclasse do pacote ~$\mathtt{Action}$ é uma subclasse de~$\mathtt{AbstractCodeElement}$, conforme destacado na Figura~\ref{fig:actionModel}. O pacote~$\mathtt{Action}$ define a maioria das metaclasses que tem como objetivo representar comportamentos para as construções estáticas definidas no pacote~$\mathtt{Code}$. Assim, ambos pacotes constituem a CEP, como mostrado na Figura~\ref{fig:kdm_layer}.

%The \texttt{Action} package consists of 11 class diagrams, it also depends on the \texttt{Core}, \texttt{kdm}, \texttt{Source}, and mainly \texttt{Code}. However, the \texttt{Action} package follows the uniform pattern for KDM models and extends the KDM with specific metaclasses related to implementation-level behavior. The \texttt{Action} package deviates from a uniform pattern for KDM models because it does not define a separate KDM model, but rather extends the \texttt{Code}, which is presented in Section~\ref{codePackage}. Therefore each \texttt{Action} metaclasses is a subclass of \texttt{AbstractCodeElement}, as highlighted in Figure~\ref{fig:actionModel}. \texttt{Action} package defines most of the relationship types to the \texttt{Code} model. Together, \texttt{Action}, and \texttt{Code} packages constitute the Program Elements Layer of KDM as depicted in Figure~\ref{fig:all_kdm_layers}. 

A metaclasse~$\mathtt{AbstractionActionRelationship}$ apresentada na Figura~\ref{fig:actionModel} \ding{206} é a metaclasse pai usada para representar várias relações que se originam a partir de um~$\mathtt{ActionElement}$. Além disso, essa metaclasse~$\mathtt{AbstractionActionRelationship}$ possui metaclasses específicas; algumas delas estão representadas na Figura~\ref{fig:actionModel}, por exemplo, as metaclasses~$\mathtt{Calls}$ \ding {203},~$\mathtt{Creates}$ \ding {204},~$\mathtt{ControlFlow}$ \ding {205} e~$\mathtt{EntryFlow}$ \ding{207}.

%The meta-class \texttt{AbstractActionRelationship} presented in Figure~\ref{fig:actionModel} \ding{206} is the parent class used to represent various KDM relationships that originate from an \texttt{ActionElement}. The meta-class \texttt{AbstractActionRelationship} owns specific metaclasses. Some of them are depicted in Figure~\ref{fig:actionModel}, for example, the metaclasses \texttt{Calls} \ding{203}, \texttt{Create} \ding{204}, \texttt{ControlFlow} \ding{205}, and \texttt{EntryFlow} \ding{207}.

O relacionamento~$\mathtt{Calls}$ corresponde a uma chamada para um procedimento, um método estático, um método não-estático de uma instância particular de um objeto, um método virtual, ou um elemento de interface.~$\mathtt{Calls}$ possui duas associações, são elas: ~$\mathtt{ActionElement[1]}$ e ~$\mathtt{ControlElement[1]}$. O primeiro representa o elemento de ação a partir do qual a relação chamada origina, a segunda associação representa o elemento alvo.

%\texttt{Calls} relationship corresponds to ``invoke'' operation on a procedure type. It can represent a call to a procedure, a static method, a non-static method of a particular object instance, a virtual method, or an interface element. \texttt{Calls} has two associations, they are: \texttt{from:ActionElement[1]} and \texttt{to:ControlElement[1]}. The first one represents the action element from which the call relation originates. The second association represents the target \texttt{ControlElement}.

A metaclasse~$\mathtt{Creates}$ representa uma associação entre um elemento de ação que \aspas{cria} uma nova instância de um determinado elemento de dados. Por exemplo, em Java essa metaclasse corresponde a palavra-chave \textit{new} utilizada para instânciar um novo objeto. $\mathtt{Creates}$ também possui duas associações:~$\mathtt{ActionElement[1]}$ e~$\mathtt{DataType[1]}$. Similar a meta classe~$\mathtt{Calls}$, a primeira associação representa o elemento que possui o relacionamento e a segunda representa o elemento de dados (objeto) que é instanciado pelo~$\mathtt{ActionElement}$.

%The meta-class \texttt{Create} represents an association between an action element that ``creates'' a new instance of a certain data element to the corresponding datatype according to the semantics of the programming language of the existing software system. It also has association, \texttt{from:ActionElement[1]} and \texttt{to:Datatype[1]}. Similarly, to the meta-class \texttt{Calls}, the first association represents the element that owns the \texttt{Creates} relationship. The second one illustrates the \texttt{DataElement} that is instantiated by the \texttt{ActionElement}. 

O~$\mathtt{ControlFlow}$ é um elemento de modelagem genérica que representa relação de fluxo de controle entre dois~$\mathtt{ActionElements}$. Além disso, é uma submetaclasse com elementos de modelagem mais específicas. O~$\mathtt{EntryFlow}$ é um elemento de modelagem que representa um fluxo inicial de controle em um elemento KDM. O relacionamento~$\mathtt{EntryFlow}$ é usado de uma maneira uniforme para descrever os pontos de entrada para outros elementos de código KDM.

%The \texttt{ControlFlow} is a generic modeling element that represents control flow relation between two \texttt{ActionElements}. It is further subclassed with more specific modeling elements. The \texttt{EntryFlow} is a modeling element that represents an initial flow of control into a KDM element. The \texttt{EntryFlow} relationship is used in a uniform way for describing entry points to other KDM code elements. 

A fim de compreender como o pacote~$\mathtt{Action}$ é usado no KDM, no Código-fonte~\ref{lst:example_kdm_instance_2} é mostrado um simples método implementado em Java. Neste método é criada uma instância de~$\mathtt{Car}$ e seu método de acesso é invocado. Uma possível correspondente instância do KDM simplificada é apresentada na Figura~\ref{fig:kdm_instance_Java_action}. Nota-se que, o diamante em destaque na cor cinza e anexado com três pontos (\ldots) ilustra que outras meta classes não são mostrados com o intuito de simplificar a figura.

%In order to comprehend  how the \texttt{Action} package is used in KDM, in Listing~\ref{lst:example_kdm_instance_2} shows a simple method implemented in Java. In this method is created an instance of \texttt{Car} and an accessor method of it is called, ie., the \texttt{getName()}. The corresponding simplified KDM instance is depicted in Figure~\ref{fig:kdm_instance_Java_action}. Please note that, the diamond highlighted in grey and attached with three dots (\ldots) illustrates that other metaclasses are not shown in order to simplify the figure.  

\noindent\begin{minipage}{.43\textwidth}
	\begin{codigo}[caption={[Pedaço de código Java para ilustrar como o pacote~$\mathtt{Action}$ funciona.] Método \texttt{e1} ilustrando como o Pacote \texttt{Action} funciona.}, escapeinside={(*@}{@*)}, basicstyle=\footnotesize, label={lst:example_kdm_instance_2}]{Name}
	(*@\ldots @*)
	public void e1 (*@\ding{202}@*)(){
	Car myCar (*@\ding{203}@*) = new Car() (*@\ding{204}@*);
	(*@\ding{229}@*)myCar.getName() (*@\ding{205}@*);
	}
	(*@\ldots @*)
	\end{codigo}
\end{minipage}\hfill
\begin{minipage}{.65\textwidth}
	\centering
	% Requires \usepackage{graphicx}
	\includegraphics[scale=0.6]{images/actionInstanceKDM_2}
	\captionof{figure}{Instância KDM correspondente ao Código-fonte~\ref{lst:example_kdm_instance_2}}
	\label{fig:kdm_instance_Java_action}
\end{minipage}

As três primeiras metaclasses mostradas nesta hierarquia são~$\mathtt{MethodUnit}$,~$\mathtt{BlockUnit}$ e~$\mathtt{Signature}$ conforme destacado na  Figura~\ref{fig:kdm_instance_Java_action} \ding{202}. Essas três metaclasses basicamente representam uma declaração e a assinatura de um determinado método, no caso \texttt{e1()}. Mais especificamente, a metaclasse~$\mathtt{MethodUnit}$ é usada para representar o método~$\mathtt{e1()}$ como mostrado tanto no Código-fonte~\ref{lst:example_kdm_instance_2} \ding{202} e na Figura~\ref{fig:kdm_instance_Java_action} \ding{202}.~$\mathtt{BlockUnit}$ representa blocos lógicos e físicos relacionados de~$\mathtt{ActionElement}$, ou seja, o escopo do método representado por \{...\}. Por sua vez,~$\mathtt{Signature}$ representa a assinatura do método, isto é, essa metaclasse representa além do nome do método todos os parâmetros, o retorno do método, exceções, etc.	

%The first three metaclasses shown in this hierarchy are the \texttt{MethodUnit}, \texttt{BlockUnit}, and \texttt{Signature} (see Figure~\ref{fig:kdm_instance_Java_action} \ding{202}). These metaclasses are used to represent a simple method declaration. More specifically, the meta-class \texttt{MethodUnit} is used to represent the method \texttt{e1} as shown in both Listing~\ref{lst:example_kdm_instance_2} \ding{202} and Figure~\ref{fig:kdm_instance_Java_action} \ding{202}. \texttt{BlockUnit} represents logically and physically related blocks of \texttt{ActionElement}, i.e., the area between the braces, \texttt{\{\ldots\}}. In turn, \texttt{Signature} represents the concept of a method signature, i.e., it also can be used to represent the name of the method, all the parameters, the return of the method, etc.

Na Linha 12 do Código-fonte~\ref{lst:example_kdm_instance_2} \ding{203} uma variável chamada~$\mathtt{myCar}$ é declarada. As metaclasses que representam esta declaração podem ser visualizadas na Figura~\ref{fig:kdm_instance_Java_action} \ding{203}. A metaclasse~$\mathtt{ActionElement}$ representa o significado das operações, por exemplo, um declaração da variável. A metaclasse~$\mathtt{StorableUnit}$ representa a própria variável, \texttt{myCar}. Ainda na Linha 10 \ding{204} a instância da classe~$\mathtt{Car}$ é criada usando a palavra-chave~$\mathtt{new}$. Nota-se que na Figura~\ref{fig:kdm_instance_Java_action} \ding{204} três metaclasses são utilizadas para representar o operador~$\mathtt{new}$. Primeiramente, a metaclasse~$\mathtt{ActionElement}$ é usada para ilustrar o significado da operação, neste caso, a instância da classe \texttt{Car}. A metaclasse~$\mathtt{Calls}$ é usada para ilustrar a instanciação de um objeto, neste caso, o objeto~$\mathtt{Car}$. Adicionalmente, a metaclasse \texttt{Calls} posssui duas ~$\mathtt{to}$ e~$\mathtt{from}$, as quais representam a chamada para o construtor de~$\mathtt{Car}$ e representa o alvo~$\mathtt{ActionElement}$, respectivamente. Em seguida, a metaclasse~$\mathtt{Creates}$ representa a nova instância de~$\mathtt{Car}$.

%In Line 10 of the Listening~\ref{lst:example_kdm_instance_2} \ding{203} a variable named \texttt{myCar} is declared. The corresponding metaclasses representing this variable declaration can be visualized in Figure~\ref{fig:kdm_instance_Java_action} \ding{203}. The meta-class \texttt{ActionElement} represents the meaning of the operations, i.e., ``variable declaration''. The \texttt{StorableUnit} represents the variable itself. Still in Line 10 \ding{204} the instance of the class \texttt{Car} is created by using the keyword \texttt{new}. Note that in Figure~\ref{fig:kdm_instance_Java_action} \ding{204} three metaclasses are used to illustrate the operator \texttt{new}. Firstly, the meta-class \texttt{ActionElement} is used to illustrate the meaning of the operation, in this case ``class instance creation'' . Secondly, the meta-class \texttt{Calls} is used to illustrate the instantiation of an object, herein \texttt{Car} object. Its owns two association, i.e., \texttt{to} and \texttt{from} which illustrates the call to the constructor of \texttt{Car} and represents the target \texttt{ActionElement}, respectively. Thirdly, the meta-class \texttt{Creates} represents the new instance of the \texttt{Car}.

Na linha 11 do Código-fonte~\ref{lst:example_kdm_instance_2} \ding{205} um método acessor é invocado. Como pode ser visto na Figura~\ref{fig:kdm_instance_Java_action} \ding{205}, três metaclasses são utilizadas no KDM para representar esta linha. Em primeiro lugar, é criado uma metaclasse~$\mathtt{ActionElement}$ que representa a declaração em si. Em seguida, outro~$\mathtt{ActionElement}$ é criado para representar a invocação de método. Finalmente, outra metaclasse~$\mathtt{Calls}$ é instânciada para representar a chamada do método~$\mathtt{getName()}$.

%Line 11 of the Listening~\ref{lst:example_kdm_instance_2} \ding{205} the accessor method is called. As can be seen in Figure~\ref{fig:kdm_instance_Java_action} \ding{205}, three metaclasses are used in the KDM to represent this line. Firstly, an \texttt{ActionElement} that represents the statement itself is created. Secondly, another \texttt{ActionElement} is created to represent the method invocation. Finally, the meta-class \texttt{Calls} is used to illustrate the call to the \texttt{getName()}. 


\subsection{KDM Structure Package}\label{sec:actionPackage}

O KDM define metaclasses que representam componentes arquiteturais, como subsistemas, camadas, componentes, etc., e definem a rastreabilidade desses elementos para outras metaclasses do KDM para o mesmo sistema por meio do pacote~$\mathtt{Structure}$. Ele define um ponto de vista arquitetural para um domínio estrutural. As visões de arquitetura com base no ponto de vista definido pelo pacote~$\mathtt{Structure}$ representam a forma como os elementos estruturais do sistema de software estão relacionados com os módulos definidos em código, que correspondem ao pacote~$\mathtt{Code}$. O pacote~$\mathtt{Structure}$ é mostrado na Figura ~\ref{fig:structureModel} como um diagrama de classes.

\begin{figure}[t]
	\centering
	% Requires \usepackage{graphicx}
	\caption{Diagrama de classes do pacote~$\mathtt{Structure}$\label{fig:structureModel}}
	\includegraphics[scale=0.8]{images/StructurePackageFigure}
	\fadaptada{ADM:OMG}
\end{figure}

%The last KDM package explained herein is the \texttt{Structure} package. According to~\cite{OMG_ADM} it defines metaclasses that represent architectural components of existing software systems, such as subsystems, layers, components, etc. and define traceability of these elements to other KDM facts for the same system.

%The \texttt{Structure} package defines an architecture viewpoint for the structure domain. The architectural views based on the viewpoint defined by the \texttt{Structure} model represent how the structural elements of the software system are related to the modules defined in the code views that correspond to the code architectural viewpoint defined by the \texttt{Code} model. 

 Usando suas meta classes é possível relacionar todos os elementos estruturais do sistema, juntamente com os elementos computacionais, isto é, pode-se especificar os elementos estruturais do sistema. Na Figura~\ref{fig:structureModel} é mostrado que o pacote~$\mathtt{Structure}$ e suas meta classes são usadas em combinação com os pacotes~$\mathtt{Code}$,~$\mathtt{Data}$,~$\mathtt{Platform}$,~$\mathtt{UI}$ e~$\mathtt{Inventory}$. 

%The \texttt{Structure} package is shown in Figure~\ref{fig:structureModel} as a class diagram. Using its metaclasses it is possible to relate all the structural elements of the system along with the computational elements, i.e., one can specify the structural elements of the system. Figure~\ref{fig:structureModel} shows that the \texttt{Structure} package and its metaclasses are used in combination with \texttt{Code} package, \texttt{Data} package, \texttt{Platform} package, \texttt{UI} package and \texttt{Inventory} package. Specifically, \texttt{Structure} package corresponding to this architectural viewpoint represent how the structural elements of the software system are related to the modules defined in the code model that correspond to the code architectural viewpoint, defined by the \texttt{Code} package.  

O modelo~$\mathtt{Structure}$ possui uma coleção de elementos estruturais, como pode ser visto na Figura~\ref{fig:structureModel} \ding{202}, isso é representado por meio de uma associação. Pacotes (do modelo~$\mathtt{Code}$) são os elementos folha do modelo~$\mathtt{Structure}$, representando a divisão de um sistema em módulos~$\mathtt{Code}$ discretos, com partes não sobrepostas. A meta classe~$\mathtt{SoftwareSystem}$ fornece um ponto de encontro para todos os pacotes do sistema direta ou indiretamente através de outra associação chamada de~$\mathtt{AbstractStructureElement[0..*]}$. Os pacotes podem ainda ser agrupados nas meta classes~$\mathtt{SubSystem}$,~$\mathtt{Layer}$,~$\mathtt{Component}$ and~$\mathtt{ArchitectureView}$.

%The \texttt{Structure} model owns a collection of structural elements, as can be seen in Figure~\ref{fig:structureModel} \ding{202} this is represented by means of an association. Packages (from the \texttt{Code} model) are the leaf elements of the \texttt{Structure} model, representing a division of a system`s \texttt{Code} Modules into discrete, non-overlapping parts. The meta-class \texttt{SoftwareSystem} provides a gathering point for all the system`s packages directly or indirectly through other association \texttt{structureElement:AbstractStructureElement[0..*]}. The packages may be further grouped into the metaclasses \texttt{Subsystem}, \texttt{Layer}, and \texttt{Component}, or \texttt{ArchitectureView}.

A meta classe~$\mathtt{AbstractStructureElement}$ (conforme a Figura~\ref{fig:structureModel} \ding {203}) representa uma parte arquitetural, relacionada com a organização do sistema de software existente em módulos e possui quatro associações. A primeira associação representa os elementos pertencentes ao modelo e é chamada de $\mathtt{structureElement}$ $\mathtt{:AbstractStructureElement[0..*]}$. Em seguida, há a associação~$\mathtt{structureRelationship}$ $\mathtt{:AbstractStructureRelationship[0..*]}$, ela é usada para representar todas os relacionamentos em nível arquitetural. A associação $\mathtt{aggregated:}$ $\mathtt{AggregatedRelationship[0..*]}$ representa uma relação abstrata entre dois elementos do KDM, dentro dela é possível definir relações concretas. A última associação da meta classe~$\mathtt{AbstractStructureElement}$ é o~$\mathtt{implementation:KDMEntity[0..*]}$. Esta associação é usada para especificar os elementos computacionais (do pacote~$\mathtt{Code}$, ou seja, $\mathtt{Package}$, $\mathtt{ClassUnit}$, $\mathtt{InterfaceUnit}$, etc) que representam o elemento estrutural. 

%The meta-class \texttt{AbstractStructureElement} (see Figure~\ref{fig:structureModel} \ding{203}) represents an architectural part, related to the organization of the existing software system into modules. It has four associations. The first association is the \texttt{structureElement:AbstractStructureElement[0..*]} meaning elements owned by the model. Following there is the association \texttt{structureRelationship:AbstractStructureRelationship[0..*]}. This association is used to represent all the relationship related to architectural part. The association \texttt{aggregated: AggregatedRelationship[0..*]} represents an abstract relationship between two elements of the KDM. Inside it is possible to define concrete relationships. The last association of the meta-class \texttt{AbstractStructureElement} is the \texttt{implementation:KDMEntity[0..*]}. This association is used to specify the computational elements (from the \texttt{Code} package, i.e., \texttt{Package}, \texttt{ClassUnit}, \texttt{InterfaceUnit}, etc) that represent the structural element. The meta-class \texttt{AbstractStructureElement} can be specified as: \texttt{Subsystem}, \texttt{Layer}, and \texttt{Component}, or \texttt{ArchitectureView}.

\noindent \begin{minipage}{.47\textwidth}
	\centering
	% Requires \usepackage{graphicx}
	\includegraphics[scale=0.7]{images/StructureExample}
	\captionof{figure}{Exemplo de uma Arquitetura.}
	\label{fig:kdm_structureExample}
\end{minipage}\hfill
\begin{minipage}{.55\textwidth}
	\centering
	% Requires \usepackage{graphicx}
	\includegraphics[scale=0.67]{images/StructureKDMINstance}
	\captionof{figure}{Instância KDM correspondente a Figura~\ref{fig:kdm_structureExample}}
	\label{fig:kdm_instance_StructureExample}
\end{minipage}

Na Figura~\ref{fig:kdm_structureExample} é descrita uma possível arquitetura mostrada para ilustrar como o KDM pode ser utilizado para representar elementos arquiteturais. Pode ser observado que esta figura é dividida em três níveis para ilustrar como o pacote~$\mathtt{Structure}$ está relacionado com o pacote~$\mathtt{Code}$. O nível mais baixo representa o código-fonte, artefatos físicos. L1 e L2 representam pacotes em código-fonte - cada caixa dentro dos pacotes representa as suas classes e interfaces, também é possível perceber que essas classes e interfaces são relacionados uns com os outros de alguma maneira. No meio há meta classes do pacote~$\mathtt{Code}$, o que significa que as instâncias dessas meta -classes são usadas para representar os artefatos de baixo nível, ou seja, instâncias de~$\mathtt{Package}$ são usadas para representar L1 e L2 e instâncias de~$\mathtt{ClassUnit}$ e~$\mathtt{InterfaceUnit}$ são usados para representar as classes e interfaces, respectivamente. Finalmente, no nível superior a arquitetura é mostrada. Todos os elementos arquiteturais são representados com a seguinte padronização: a meta classe que representa o elemento arquiteturais, ':' seguido pelo seu nome. A arquitetura apresentada é dividida da seguinte forma: no ponto mais alto de abstração há um~$\mathtt{SoftwareSystem}$ (~$\mathtt{S1}$) \ding {202}, que é dividido em duas camadas, ~$\mathtt{Layer}$~$\mathtt{L1}$ \ding {203} e ~$\mathtt{Layer}$~$\mathtt{L2}$ \ding {204}. Tais camadas representam elementos arquiteturais correspondentes aos pacotes L1 e L2 representadas no nível mais baixo. A~$\mathtt{Layer}$~$\mathtt{L1}$ pode acessar os elementos da~$\mathtt{Layer}$~$\mathtt{L2}$, essa restrição \aspas{pode acessar} é representada pela meta classe~$\mathtt{AggregatedRelationship}$ \ding {205}. Além disso, a~$\mathtt{Layer}$~$\mathtt{L2}$ contém dois componentes,~$\mathtt{C1}$ \ding {206} e~$\mathtt{C2}$ \ding {207}. Finalmente, o~$\mathtt{Component}$~$\mathtt{C1}$ fornece recursos através de uma interface para o~$\mathtt{Component}$~$\mathtt{C2}$.

%In Figure~\ref{fig:kdm_structureExample} depicts a possible architecture is shown to illustrate how KDM can be used to represent architectural elements. As can be noted this figure is split in three levels. It is split in these levels to illustrate how the KDM \texttt{Structure} package is related to the \texttt{Code} package. The lowest level represents the source-code, physical artifacts, e.g., L1 and L2 both represent packages in the source-code - each box inside the packages represents its classes and interfaces and and it is also possible to see that these classes and interfaces are related with each other somehow. At the middle there are metaclasses from the KDM \texttt{Code} package, which means that instances of these metaclasses are used to represent the low level artifacts, i.e., instance of \texttt{Package} are used to represent the L1 and L2 and instance of \texttt{ClassUnit} and \texttt{InterfaceUnit} are used to represent classes and interface, respectively. Finally, at the top level the architecture is shown. All architectural elements are represented using the following standardization:  the meta-class that represent the architectural element, ':' followed by its name. The presented architecture is divided as follows: at the highest point of abstraction there is a \texttt{SoftwareSystem} (\texttt{S1}) \ding{202}, which is further split into two layers, \texttt{Layer} \texttt{L1} \ding{203} and \texttt{Layer}  \texttt{L2} \ding{204}. Such layers represent architectural elements corresponding to the packages L1 and L2 depicted at the lowest level. The \texttt{Layer} \texttt{L1} can access the elements of \texttt{Layer} \texttt{L2}, this constraint ``can access'' is represented by the meta-class \texttt{AggregationRelationship} \ding{205}. In addition, \texttt{Layer} \texttt{L2} contains two components, \texttt{Component} \texttt{C1} \ding{206} and \texttt{Component} \texttt{C2} \ding{207}. Finally, \texttt{Component} \texttt{C1} provides resource through an interface to the \texttt{Component} \texttt{C2}. 

O correspondente, porém simplificado da instância KDM é mostrado na Figura~\ref{fig:kdm_instance_StructureExample}. Os diamantes destacadas em cinza e em anexo com três pontos (\ldots) ilustram que algumas meta classes não são mostrados de forma a simplificar a figura. Todos os elementos arquiteturais são subclasses de~$\mathtt{StructureModel}$. As camadas são representados pela meta classe~$\mathtt{Layer}$, como pode ser visto na Figura~\ref{fig:kdm_instance_StructureExample} \ding{203} e \ding{204}. Da mesma forma, os componentes são representados pela meta classe~$\mathtt{Component}$. A meta classe mais importante a se destacar nesta figura é $\mathtt{AggregatedRelationship}$. Ela representa a relação entre a $\mathtt{Layer}$ $\mathtt{L1}$ e a $\mathtt{Layer}$ $\mathtt{L2}$. Ela possui meta atributos que tem como objetivo fornecer informações sobre o relacionamento. Por exemplo, o meta atributo~$\mathtt{density}$ ilustra o número de relações primitivas entre estas camadas. Na Figura~\ref{fig:kdm_instance_StructureExample}, o meta atributo~$\mathtt{density}$ possui o valor 1 (um). Outros dois meta atributos são o~$\mathtt{from}$ e o~$\mathtt{to}$, que representam os elementos arquiteturais de origem e destino, respectivamente. Eles são usados para especificar que a~$\mathtt{Layer}$~$\mathtt{L1}$ em~$\mathtt{SoftwareSystem}$~$\mathtt{S1}$ pode acessar a~$\mathtt{Layer}$~$\mathtt{L2}$ também em~$\mathtt{SoftwareSystem}$~$\mathtt{S1}$ de alguma forma. Finalmente, o meta atributo~$\mathtt{relation}$ representa como a~$\mathtt{Layer}$~$\mathtt{L1}$ pode acessar a~$\mathtt{Layer}$~$\mathtt{L2}$, neste contexto, através de herança usando a meta classe~$\mathtt{Extends}$. Na Figura~\ref{fig:relationship_example_1} é mostrado como as relações entre dois elementos arquiteturais são considerados neste estudo.

\begin{figure}[!ht]
	\centering
	\includegraphics[width=3.3in]{images/relationshipExample1.pdf}
	\caption{Relacionamento entre dois elementos arquiteturais}
	\label{fig:relationship_example_1}
\end{figure}

%The corresponding although simplified KDM instance is shown in Figure~\ref{fig:kdm_instance_StructureExample}. The diamonds highlighted in grey and attached with three dots (\ldots) illustrates that some metaclasses are not shown in order to simplify the figure. All architectures elements are subclass of \texttt{StructureModel}. The layers are represented by the meta-class \texttt{Layer}, as can be seen in Figure~\ref{fig:kdm_instance_StructureExample} \ding{203} and \ding{204}. Similarly, the components are represented by the meta-class \texttt{Component}. The most important meta-class to highlight in this figure is \texttt{AggregationRelationship}. It represents the relationship between the \texttt{Layer} \texttt{L1} and \texttt{Layer} \texttt{L2}. It also owns meta-attributes that aims to provide information about the relationship. For instance, the meta-attribute \texttt{density} illustrate the number of primitive relationships between these layers. In Figure~\ref{fig:kdm_instance_StructureExample} the meta-attribute \texttt{density} was the value 1. Another two meta-attributes are \texttt{from} and \texttt{to}, the target and the source architecture element, respectively. They are used to specify that the \texttt{Layer} \texttt{L1} in \texttt{SoftwareSystem} \texttt{S1} access the \texttt{Layer} \texttt{L2} also in \texttt{SoftwareSystem} \texttt{S1} somehow. Finally, the meta-attribute \texttt{relation} represents how the \texttt{Layer} \texttt{L1} access the \texttt{Layer} \texttt{L2}, in this context, through inheritance using the meta-class \texttt{Extends}. Figure~\ref{fig:relationship_example_1} shows as the relationships between two architectural elements are considered in this study.

Em \ding{202} os elementos arquiteturais são apresentados, camadas~$\mathtt{Controller}$ e~$\mathtt{Model}$. Como visto anteriormente, um relacionamento em nível arquitetural acontece entre dois elementos em nível estrutural (camadas, componentes, subsistemas, etc). Por exemplo, em \ding{203} é mostrando um~$\mathtt{aggregatedRelationship}$ contendo todos os possíveis relacionamentos \ding{204} entre dois elementos (chamadas de métodos, herança, etc). A seta \ding{206} representa o fluxo de entrada e saída dos relacionamentos, em outras palavras, esses relacionamentos representam as possíveis restrições iniciando na camada~$\mathtt{Controller}$  para a camada~$\mathtt{Model}$. Finalmente, a densidade \ding{205} é setada como oito, que representa o número total de relacionamentos possíveis.

%In \ding{202} the architectural elements are presented,~$\mathtt{Controller}$ and~$\mathtt{Model}$ layer. As seen before, a relationship in architectural level happens between two elements on structural level (e.g layer, component, subsystem, etc), in this example, in \ding{203} is shown an $\mathtt{aggregatedRelationship}$ containing all possible relationships \ding{204} between two elements (calls, implements, etc). Besides, the a	rrow \ding{206} represent the entry and exit flow of the relationships, in other words, theses relationships represents the possible accesses starting from~$\mathtt{Controller}$  to the~$\mathtt{Model}$ layer. Finally, density \ding{205} is set to eight, which is the total number of possible relationships.




\section{Ferramenta de apoio ao KDM}

Um dos trabalhos mais importantes publicados no contexto da ADM é o de~\citeonline{Bruneliere_2010MODISCO, Bruneliere_2014} que propõe uma ferramenta chamada MoDisco. MoDisco é uma framework genérico e extensível para a abordagem de Engenheria Reversa dirigida a modelos e foi implementada no \sigla{IDE}{\textit{Integrated Development Environment}} Eclipse como um \textit{plugin}. Mais especificadamente, MoDisco é construido utilizando o \textit{Eclipse Modeling Framework} (EMF). Basicamente essa ferramenta é capaz de recuperar o código-fonte legado, base de dados e outros artefatos legado e representá-los com o metamodelo KDM. Um dos principais objetivos da ferramenta MoDisco é ser adaptável para diferentes cenários, facilitando assim a sua utilização por uma base de usuários potencialmente maior~\cite{Bruneliere_2014}. Inicialmente criado como um modelo experimental de investigação pela Equipe AtlanMod (\sigla{EMN}{\textit{Ecole des mines de Nantes}} \& \sigla{INRIA}{\textit{Institut National de Recherche en Informatique et en Automatique}}), o projeto evoluiu para uma solução industrializados graças à colaboração com a empresa MIA-Software. Este trabalho conjunto ativo resultou em um conjunto eficiente e utilizável de ferramentas para a descoberta, consulta e manipulação de modelos de software para auxiliar toda a atividade de engenharia reversa.

Como ilustrado na Figura~\ref{fig:modisco_allArtefacts} MoDisco tem como objetivo representar uma grande variedade de artefatos (por exemplo, código-fonte, banco de dados, arquivos de configuração, documentação, etc.) de um sistema legado. 

\begin{figure}[!ht]
	\centering
	\includegraphics[scale=0.55]{images/modiscoAllArtefacts.png}
	\caption{Visão geral de um projeto MoDisco.}
	\label{fig:modisco_allArtefacts}
	\fadaptada{Bruneliere_2014}
\end{figure}

Contudo, uma das limitações dessa ferramenta é o suporte e a aplicação de refatorações. Observe que, naturalmente, MoDisco não é capaz de aplicar refatorações de forma automática, pois a maioria das refatorações necessitam de interação do usuário para fornecer as informações necessárias. No contexto desta Tese foi utilizado a ferramenta MoDisco para recuperar as informações do código-fonte legado escrito em Java. Sem o auxílio dessa ferramenta todo o sistema legado, escrito em Java, deveria ser transformada em uma instância do KDM de forma manual, o que poderia atrasar esta pesquisa, uma vez que toda a manipulação do KDM foi possível por causa da existência do MoDisCo e do seu suporte em Java para manipular o metamodelo KDM. Por exemplo, não é possível para o MoDisco adivinhar quais refatorações devem ser aplicadas e em quais elementos; tais informações devem ser fornecidas por um usuário.

\section{Considerações Finais}\label{sec:consideracoes_finais}

O foco deste capítulo direciona-se à análise do panorama atual da literatura que trata sobre a modernização de sistemas legados levando em consideração a padronização proposta pela OMG. Foram mostrados os principais conceitos da ADM, KDM e seus pacote e camadas, necessários ao entendimento dessa Tese e fundamentais ao desenvolvimento da proposta aqui desenvolvida.

Observou-se também que o metamodelo KDM por intermédio de suas camadas, pacotes e metaclasses permite a criação de modelos que representam um sistema legado em diversas visões. O objetivo da OMG ao criar esse metamodelo propoe uma padronização da reengenharia de software, fornecendo modelos que ajudem no processo de reeengeharia de sistemas.

Neste capítulo também foi apresentado sobre a ferramenta MoDisco, uma das principais ferramentas que trabalha com ADM e KDM. Essa ferramenta foi utilizada no contexto deste trabalho para dar suporte à recuperação do modelo KDM a partir de código-fonte legado escrito em Java.

\chapter{Mapeamento Sistemático sobre ADM e KDM}\label{chapter:mapeamento_sistematico}
\section{Considerações Iniciais}

Quando se conduz uma revisão de literatura sem o pré-estabelecimento de um protocolo de revisão, há um direcionamento por interesses pessoais, o que leva a resultados pouco confiáveis. Nesse contexto, pesquisadores vêm utilizando uma técnica denominada de \sigla{MS}{Mapeamento Sistemático} para auxiliar o pesquisador a conduzir uma revisão bibliográfica de forma totalmente sistemática, com o intuito de evitar que trabalhos importantes fiquem fora de suas pesquisas. Um MS é caracterizado por ser um meio de avaliar e interpretar todas as pesquisas disponíveis, referentes a uma questão de pesquisa, tema, área ou fenômeno de interesse. O MS visa expor uma avaliação justa de um tema de pesquisa, usando uma metodologia confiável e rigorosa~\cite{Petersen_2008, Kitchenham_2010, Petersen_20151}.


De acordo com~\citeonline{Petersen_20151}, o MS é projetado para dar uma visão geral de uma área de investigação por meio da classificação e contagem de contribuições em relação a um conjunto de categorias de classificação~\cite{Petersen_2008, Kitchenham_2010}. Em outras palavras, trata-se de realizar uma vasta pesquisa na literatura, a fim de identificar quais temas já foram abordados, quais temas ainda não foram abordados e quais são novas possíveis pesquisas~\cite{Kitchenham_2010}. Existe também a técnica de \sigla{RS}{Revisão Sistemática}, a qual compartilha algumas características com o MS, como no que diz respeito à busca e seleção do estudo. Segundo~\citeonline{Petersen_20151}, tais técnicas são diferentes em termos de objetivos e, portanto, usam diferentes abordagens para a análise de dados, pois a RS visa sintetizar evidências, também considerando a força da evidência, e o MS preocupa-se principalmente com a estruturação de uma determinada área de pesquisa~\cite{Petersen_20151}. Segundo~\citeonline{Kitchenham_2010}, MS implica na forma mais adequada para se identificar, avaliar e interpretar toda uma área de pesquisa para um tema em particular. 

Resume-se, então, que um MS configura um alicerce para novas atividades de pesquisa acerca de um determinado tema. Diante disso, foi realizado um MS sobre ADM e KDM~\cite{durelli_systematic_mapping}; a motivação para realizar esse MS é identificar os temas que são mais investigados, bem como os temas que ainda não foram pesquisados no contexto da abordagem ADM e do metamodelo KDM. Embora a ADM seja uma abordagem relativamente nova, o OMG afirma que ela é uma importante abordagem, pois combina dois dos principais campos da Engenharia de Software: MDE (ver Capítulo~\ref{chapter:fundamentacao_teorica}, Seção~\ref{Cap2_Sec2_Desenvolvimento_Dirigido_a_Modelos}) e reengenharia de software. Desde a criação da ADM, muitos esforços têm enfatizado a modernização de sistemas por meio dessa abordagem. Assim, se faz necessário a condução de uma investigação mais sistemática dos temas englobados por tal área de pesquisa. Nota-se que este capítulo é uma extensão do seguinte artigo: \textit{A Mapping Study on Architecture-Driven Modernization}~\cite{durelli_systematic_mapping}.


Este capítulo está organizado da seguinte forma: Seção~\ref{sec:metodologia_pesquisa} descreve a metodologia de como o MS foi conduzido;  Subseção~\ref{subsec:estrategia_de_busca} apresenta a estratégia de busca utilizada nesse MS; a Subseção~\ref{subsec:fonte_de_estudo_e_selecao}, descreve as fontes de estudos utilizadas no MS, além disso, também apresenta como foi realizada a seleção dos estudos; a Subseção~\ref{subsec:definindo_esquema_de_classificacao} apresenta o esquema de classificação considerado no MS; na Subseção~\ref{subsec:extracao_e_sintese_do_dados}, a extração e a síntese dos dados são discutidas; na Subseção~\ref{subsec:mapeamento_e_dis}, um gráfico de bolha resultante do MS é apresentado, dando ênfase nas principais descobertas desse MS. Ainda nessa subseção, as questões de pesquisas são discutidas; a Seção~\ref{subsec:principais_constatações_e_questões_Em_Aberto} apresenta as principais constatações e questões em aberto, relacionadas à ADM e ao KDM; a Seção~\ref{subsec:ameaças_a_validade} apresenta as ameaças à validade do MS; Seção~\ref{sec:considerações_finais_do_mapeamento_sistematico} descreve as considerações finais deste capítulo.


%Seção X apresenta as principais descobertas desse MS. Na Seção X as ameaças às validades são apresentadas. Na Seção X as considerações finais são destacadas.\change{mudar e colocar a estrutura}

\section{Metodologia de Pesquisa}\label{sec:metodologia_pesquisa}

Como já salientado nas considerações iniciais, um MS fornece um processo sistemático para identificar relevantes pesquisas com o objetivo de responder específicas questões. Durante a condução desse MS, todos os passos propostos por~\citeonline{Petersen_20151, Petersen_2008} foram seguidos. Uma visão geral de todos os passos propostos por tais autores pode ser observada na Figura~\ref{fig:all_steps_MS}. Nota-se que cada passo produz um resultado intermediário e de acordo com tais autores, cinco passos são essenciais: (\textit{i}) \textit{Definition of Research Questions}, (\textit{ii}) \textit{Conducting Search}, (\textit{iii}) \emph{Screening of Papers}, (\textit{iv}) \emph{Keywording using Abstracts} e (\textit{v}) \textit{Data Extraction and Mapping Process}. 

\begin{figure}[h]
 \caption{Processo para a condução de um Mapeamento Sistemático.}
 \label{fig:all_steps_MS}
 \centering
 \includegraphics[scale=0.7]{images/SystematicProcessSteps_new_day_21_01}
 \fadaptada{Petersen_2008}
\end{figure}

\subsection{Estratégia de Busca}\label{subsec:estrategia_de_busca}

O protocolo do MS deve ser definido e é usualmente dividido em dois subpassos: (\textit{i}) a definição das questões de pesquisa e (\textit{ii}) a \textit{string} de busca. 

\sigla{QPs}{Questões de Pesquisas} devem englobar o propósito do MS, o qual foca a identificação e a caracterização do estado atual da ADM e do metamodelo KDM. Dessa forma, como já citado, a motivação para realizar esse MS é identificar os temas que são mais investigados, bem como os temas que ainda não foram investigados no contexto da abordagem ADM e do metamodelo KDM. A partir desses objetivos, delineiam-se as seguintes questões:

\begin{itemize}
\item \textbf{QP$_1$} - Dados os metamodelos da ADM, qual é mais utilizado na literatura? Além disso, dado o metamodelo identificado, qual (is) é (são) o (s) pacote (s) mais e menos utilizado (s)?
\item \textbf{QP$_2$} - Que tipos de estudos são publicados no contexto da ADM?
\item \textbf{QP$_3$} - Quais são as áreas mais e menos investigadas no contexto da ADM? Adicionalmente, quais são os tipos de contribuições que foram publicados até agora?
\end{itemize}

Considerando as QPs estabelecidas, definiram-se os atributos e a amplitude do MS com a técnica \sigla{PICO}{\textit{Population}, \textit{Intervention}, \textit{Comparator} e \textit{Outcomes}}~\cite{Kitchenham_2010}, e identificaram-se os termos a serem utilizados na \textit{string} de busca:

\begin{itemize}
\item Quanto à população: Em Engenharia de Software e no contexto de MS, população diz respeito à uma específica área de pesquisa. No contexto desse MS, a população são artigos publicados na literatura científica sobre algum processo, técnica ou ferramenta que utilize ADM e seus metamodelos;

\item Quanto à intervenção: Em Engenharia de Software, intervenção refere-se à metodologia de software, ferramenta, tecnologia ou procedimento. No contexto, desse MS, a intervenção são abordagens e ferramentas publicadas na literatura científica que utilizam ADM e seus metamodelos;

\item Quanto à comparação: A comparação não é aplicada no contexto desse MS;

\item Quanto aos resultados esperados: Espera-se como resultado uma visão geral dos estudos que foram publicados para a ADM e seus metamodelos, enfatizando estudos primários que descrevem técnicas, abordagens, processos e ferramentas para auxiliar o engenheiro de modernização durante a condução de modernização de sistemas legados com a utilização da abordagem ADM.

\end{itemize}

A partir dos termos identificados, define-se a \textit{string} de busca para a recuperação de estudos. Todos os termos devem ser traduzidos de acordo com o idioma dos artigos que se deseja recuperar (no contexto deste MS, inglês) e associados com sinônimos, conforme sugestões de especialistas. Na Figura~\ref{fig:string_de_busca}, é exposta a \textit{string} de busca que foi utilizada no presente MS.

\begin{figure}[h]
 \caption{\textit{String} de busca definida.}
 \label{fig:string_de_busca}
 \centering
 \includegraphics[scale=0.6]{images/searchStringMS}
 \fautor
\end{figure}

\subsection{Fonte de Estudos e Seleção dos Estudos}\label{subsec:fonte_de_estudo_e_selecao}

As fontes de estudos utilizadas durante o MS foram as bibliotecas digitais da \textit{ACM}, 
\textit{IEEE XPLORE}, \textit{Scopus}, \textit{Web of Science} e \textit{Engineering Village}. Tais bibliotecas digitais foram selecionadas com base na experiência reportada por~\citeonline{dyba_2015}. De acordo com esses autores, tais fontes de estudos são suficientes para identificar estudos primários relevantes. Nota-se que os recursos fornecidos por essas bibliotecas digitais, bem como a sintaxe exata da \textit{string} de busca a ser aplicada variam de uma biblioteca para outra, assim, a \textit{string} de busca apresentada na Figura~\ref{fig:string_de_busca} foi utilizada como base para construir uma \textit{string} de busca semanticamente equivalente e sob medida para cada biblioteca digital. %As \textit{strings} de buscas utilizada em todas as bibliotecas digitais pode ser visualizada na Tabela~\ref{long}.


%\begin{longtable}[!tb]{ | m{2cm} | m{12cm}| }
% \caption{Bibliotecas digitais e \textit{String} de busca adaptadas.\label{long}}\\
 
% \hline
% \multicolumn{2}{| c |}{Início da Tabela}\\
% \hline
% Bibliotecas Digitais & \textit{String} de Busca\\
% \hline
% \endfirsthead
 
% \hline
% \multicolumn{2}{|c|}{Continuação da Tabela~\ref{long}}\\
% \hline
% Bibliotecas Digitais & \textit{String} de Busca\\
% \hline
% \endhead
 
% \hline
% \endfoot
 
% \hline
% \multicolumn{2}{| c |}{Fim da Tabela}\\
% \hline\hline
% \endlastfoot
 
% ACM & ("Knowledge Discovery Metamodel" or "Knowledge-Discovery Metamodel" or "Knowledge-Discovery Meta-model" or "Knowledge Discovery Meta-model" or "Architecture Driven Modernization" or "Architecture-Driven Modernization" or "Model Driven Modernization" or "Model-Driven Modernization" or "Model-driven software modernization" or "ADM Pattern Recognition specification" or "ADM Visualization specification" or "ADM Refactoring specification" or "ADM Transformation specification" or "KDM Metamodel" or "KDM Meta-model" or "Software Metrics Meta-model" or "Software Metrics Metamodel" or "Structured Metrics Meta-Model" or "Structured Metrics Metamodel" or "Abstract Syntax Tree Metamodel" or "Abstract Syntax Tree Meta-model")\\
 %\hline
 %IEEE XPLORE & ((((((((((((((((((((("Knowledge Discovery Metamodel") OR "Knowledge-Discovery Metamodel") OR "Knowledge-Discovery Meta-model") OR "Knowledge Discovery Meta-model") OR "Architecture Driven Modernization") OR "Architecture-Driven Modernization") OR "Model Driven Modernization") OR "Model-Driven Modernization") OR "Model-driven software modernization") OR "ADM Pattern Recognition specification") OR "ADM Visualization specification") OR "ADM Refactoring specification") OR "ADM Transformation specification") OR "KDM Metamodel") OR "KDM Meta-model") OR "Software Metrics Meta-model") OR "Software Metrics Metamodel") OR "Structured Metrics Meta-Model") OR "Structured Metrics Metamodel") OR "Abstract Syntax Tree Metamodel") OR "Abstract Syntax Tree Meta-model")\\
 %\hline
 %Scopus & ("Knowledge Discovery Metamodel"  OR  "Knowledge-Discovery Metamodel"  OR  "Knowledge-Discovery Meta-model"  OR  "Knowledge Discovery Meta-model"  OR  "Architecture Driven Modernization" OR  "Architecture-Driven Modernization"  OR  "Model Driven Modernization"  OR  "Model-Driven Modernization"  OR  "Model-driven software modernization"  OR  "ADM Pattern Recognition specification"  OR  "ADM Visualization specification"  OR  "ADM Refactoring specification"  OR  "ADM Transformation specification"  OR  "KDM Metamodel"  OR  "KDM Meta-model"  OR  "Software Metrics Meta-model"  OR  "Software Metrics Metamodel"  OR  "Structured Metrics Meta-Model"  OR  "Structured Metrics Metamodel"  OR  "Abstract Syntax Tree Metamodel"  OR  "Abstract Syntax Tree Meta-model")\\
 %\hline
 %Web of Science & ("Knowledge Discovery Metamodel" OR "Knowledge-Discovery Metamodel" OR "Knowledge-Discovery Meta-model" OR "Knowledge Discovery Meta-model" OR "Architecture Driven Modernization" OR "Architecture-Driven Modernization" OR "Model Driven Modernization" OR "Model-Driven Modernization" OR "Model-driven software modernization" OR "ADM Pattern Recognition specification" OR "ADM Visualization specification" OR "ADM Refactoring specification" OR "ADM Transformation specification" OR "KDM Metamodel" OR "KDM Meta-model" OR "Software Metrics Meta-model" OR "Software Metrics Metamodel" OR "Structured Metrics Meta-Model" OR "Structured Metrics Metamodel" OR "Abstract Syntax Tree Metamodel" OR "Abstract Syntax Tree Meta-model")\\
 %\hline
 %Engeneering Village & "Knowledge Discovery Metamodel" OR "Knowledge-Discovery Metamodel" OR "Knowledge-Discovery Meta-model" OR "Knowledge Discovery Meta-model" OR "Architecture Driven Modernization" OR "Architecture-Driven Modernization" OR "Model Driven Modernization" OR "Model-Driven Modernization" OR "Model-driven software modernization" OR "ADM Pattern Recognition specification" OR "ADM Visualization specification" OR "ADM Refactoring specification" OR "ADM Transformation specification" OR "KDM Metamodel" OR "KDM Meta-model" OR "Software Metrics Meta-model" OR "Software Metrics Metamodel" OR "Structured Metrics Meta-Model" OR "Structured Metrics Metamodel" OR "Abstract Syntax Tree Metamodel" OR "Abstract Syntax Tree Meta-model"\\
 %\hline
 %Google Scholar & ("Knowledge Discovery Metamodel" or "Knowledge-Discovery Metamodel" or "Knowledge-Discovery Meta-model" or "Knowledge Discovery Meta-model" or "Architecture Driven Modernization" or "Architecture-Driven Modernization" or "Model Driven Modernization" or "Model-Driven Modernization" or "Model-driven software modernization" or "ADM Pattern Recognition specification" or "ADM Visualization specification" or "ADM Refactoring specification" or "ADM Transformation specification" or "KDM Metamodel" or "KDM Meta-model" or "Software Metrics Meta-model" or "Software Metrics Metamodel" or "Structured Metrics Meta-Model" or "Structured Metrics Metamodel" or "Abstract Syntax Tree Metamodel" or "Abstract Syntax Tree Meta-model")\\
 %\hline
 %\end{longtable}

Para determinar quais estudos primários são relevantes para responder às QPs, definiu-se um conjunto de critérios de inclusão.

\begin{itemize}
\item Critérios de Inclusão:
    \begin{itemize}
    \item O estudo primário apresenta pelo menos uma abordagem de modernização que utiliza ADM e seus metamodelos;
    \item O estudo primário descreve uma avaliação empírica da abordagem que utiliza ADM.
    \end{itemize}
\end{itemize}

Similarmente, também foram definidos três critérios de exclusão, a saber:

\begin{itemize}
\item Critérios de Exclusão:
    \begin{itemize}
    \item Artigos que mencionam a ADM e seus metamodelos apenas no \textit{abstract};
    \item Artigos introdutórios para livros e \textit{workshops};
    \item O estudo primário é um artigo pequeno (\textit{short paper}), o qual contém até três páginas.
    \end{itemize}
\end{itemize}

A Scopus foi a biblioteca digital que retornou mais estudos primários, 58\% (150) (ver Figura~\ref{fig:distribuicao_biblioteca_digital}). Por outro lado, foram recuperados 20\% (51) estudos da ACM, 12\% (30) da Engineering Village, 6\% (17) da Web of Science e 4\% (11) da IEEE. 

\begin{figure}[h]
 \caption{Distribuição dos estudos primários de cada biblioteca digital.}
 \label{fig:distribuicao_biblioteca_digital}
 \centering
 \includegraphics[scale=0.9]{images/retornoDasBasesMS}
 \fautor
\end{figure}

Somando todas as bibliotecas digitais, obtiveram-se 259 estudos primários no primeiro passo, como ilustrado na Figura~\ref{fig:todos_os_passos}. Após o primeiro passo (ver Figura~\ref{fig:todos_os_passos}), 82 artigos foram selecionados, sendo possível notar que apenas publicações de conferências e \textit{journals} foram considerados nesse MS. Posteriormente, os critérios de inclusão anteriormente apresentados foram aplicados para os 82 artigos selecionados. Após esse passo, 30 estudos primários foram considerados para serem analisados no MS como mostrado na Figura~\ref{fig:todos_os_passos}.

\begin{figure}[h]
 \caption{Todos os passos conduzidos no MS.}
 \label{fig:todos_os_passos}
 \centering
 \includegraphics[scale=0.7]{images/todosOsPassosMS}
 \fautor
\end{figure}

\subsection{Definindo um Esquema de Classificação}\label{subsec:definindo_esquema_de_classificacao}

O esquema de classificação utilizado no MS foi o esquema proposto por~\citeonline{Petersen_2008, Petersen_20151} que classifica cada publicação entre categorias de acordo com três perspectivas: (\textit{i}) \textbf{Área de Foco}, (\textit{ii}) \textbf{Tipo de Contribuição} e (\textit{iii}) \textbf{Tipo de Pesquisa}. O esquema de classificação resultante é descrito a seguir.

\begin{itemize}
\item \textbf{Área de Foco}: Após ler os estudos primários, foram identificadas cinco principais áreas de foco: 
    
    \begin{itemize}
        \item \aspas{\textbf{Modernização de Software}}: está relacionada com estudos primários que descrevem abordagens que empregam ADM para modernizar sistemas legados para outra plataforma ou arquitetura;
        \item \aspas{\textbf{Extração de \textit{Business Knowledge}}}: descreve estudos primários que apresentam processos, métodos ou abordagens para extrair informações de negócio de sistemas legados;
        \item \aspas{\textbf{Extração de Interesse}}: representa estudos primários que descrevem processos, métodos ou abordagens para extrair interesses transversais de sistemas legados;
        \item \aspas{\textbf{Extensão dos metamodelos da ADM}}: descreve estudos primários que apresentam abordagens, métodos ou processos para estender um determinado metamodelo da ADM;
        \item \aspas{\textbf{Aplicabilidade}}: inclui estudos primários que buscam representar a evidência da utilização da ADM e seus metamodelos na prática, ou seja, artigos que apresentam pesquisas ou relatórios para facilitar o entendimento da ADM e seus metamodelos.
    \end{itemize}
    
    \item \textbf{Tipo de Contribuição}: Similarmente, também foram identificados cinco tipos de contribuições:
        
        \begin{itemize}
            \item \aspas{\textbf{Ferramentas}}: estudos primários que apresentam ferramentas para auxiliar a modernização de sistemas legados utilizando ADM e seus metamodelos;
            \item \aspas{\textbf{Processo}}: estudos primários que descrevem processos para auxiliar a modernização de sistemas legados utilizando ADM e seus metamodelos;
            \item \aspas{\textbf{Transformação de Modelos}}: estudos primários que descrevem o uso de linguagens de transformações para realizar transformações entre os metamodelos da ADM;
            \item \aspas{\textbf{Metamodelos}}: estudos primários que relatam extensão nos metamodelos da ADM para suprir um específico problema, por exemplo, fornecer uma extensão leve para o metamodelo KDM representar o paradigma orientado a aspecto;
            \item \aspas{\textbf{Métricas}}: estudos primários que se concentram em propor ou aplicar métricas para medir a eficácia de ADM e seus metamodelos.
        \end{itemize}
        
        \item \textbf{Tipo de Pesquisa}: reflete a abordagem de pesquisa que foi utilizada no estudo primário, e essa categoria foi criada com base no esquema proposto por~\citeonline{Wieringa_2005}:
        
        \begin{itemize}
            \item \aspas{\textbf{Pesquisa de validação}}: tem por objetivo analisar uma proposta de solução que ainda não foi aplicada na prática. A validação é realizada de uma forma sistemática e pode apresentar qualquer um destes tipos: protótipos, análise matemática, etc.;
            \item \aspas{\textbf{Pesquisa de avaliação}}: em contraste com a pesquisa de validação, pesquisa de avaliação visa examinar uma solução que já foi praticamente aplicada. Estudos nessa categoria investigam a aplicação na prática da solução proposta e, geralmente, os resultados obtidos utilizando estratégias empíricas (por exemplo, experimentos e estudo de casos);
            \item \aspas{\textbf{Proposta conceitual}}: apresenta um arranjo de coisas que já existem, de uma nova maneira. No entanto, isso não resolve precisamente um problema particular. Podem ser incluídos taxonomias, referenciais teóricos, etc.;
            \item \aspas{\textbf{Artigo descrevendo experiência}}: artigos que descrevem sobre a experiência pessoal do autor para um ou mais projetos. O autor geralmente apresenta como o projeto foi feito e o que foi realizado;
            \item \aspas{\textbf{Artigo descrevendo opinião}}: artigos que descrevem a opinião pessoal do autor sobre a adequação ou inadequação de uma técnica ou ferramenta específica.
        \end{itemize}
    
\end{itemize}


\subsection{Extração e Síntese dos Dados}\label{subsec:extracao_e_sintese_do_dados}

Os 30 estudos selecionados na etapa anterior foram analisados. Criou-se, então, um formulário para auxiliar a extração de dados, abordando os seguintes aspectos: (\textit{i}) dados relevantes sobre como a ADM e seus metamodelos são utilizados na literatura, (\textit{ii}) a data de quando a extração do dado foi realizada, (\textit{iii}) o título do estudo primário, (\textit{iv}) os autores do estudo primário, (\textit{v}) o veículo de publicação e (\textit{vi}) um resumo destacando as principais contribuições do estudo primário para posteriormente realizar a classificação. Durante o processo de extração, informações sobre cada estudo primário foram independentemente coletadas por todos os pesquisados que participaram do MS. É importante destacar que a primeira execução desse MS foi realizada em novembro de 2013, posteriormente, foi conduzido novamente em agosto de 2015 com o objetivo de atualizá-lo.

\subsection{Mapeamento e discussão das QPs}\label{subsec:mapeamento_e_dis}

O foco desta seção é apresentar uma visão geral de como a ADM e seus metamodelos são pesquisados e utilizados na literatura, bem como identificar possíveis grupos de evidências (ou seja, onde pode haver margem para uma literatura mais completa) e deserto de evidências (ou seja, onde melhores ou novas pesquisas são necessárias) de pesquisas. Além de apresentar essa visão geral, a presente seção almeja destacar respostas para as QPs definidas anteriormente.

Em vez de utilizar tabelas de frequência, foi produzido um gráfico de bolha para reportar a frequência e distribuição dos estudos primários selecionados de acordo com suas categorias e data de publicação. Argumenta-se que esse gráfico de bolha representa um mapa geral de como a ADM e seus metamodelos são utilizados na literatura. O mapa resultante é apresentado na Figura~\ref{fig:mapa_mapeamento_sistematico}.


\begin{figure}[h]
 \caption{Visão geral da pesquisa sobre ADM e seus metamodelos.}
 \label{fig:mapa_mapeamento_sistematico}
 \centering
 \includegraphics[scale=0.8]{images/MapaMS_port}
 \fautor
\end{figure}

Como pode ser observado, esse gráfico de bolha contém dois vértices, X e Y, os quais possuem bolhas em cada categoria. É visto que o tamanho de cada bolha representa o número de estudos primários que foram classificados em uma categoria específica. Esse gráfico é um resumo visual e fornece uma visão panorâmica que permite identificar quais são as categorias que foram salientadas em pesquisas anteriores, além disso, é possível identificar facilmente lacunas e oportunidades para futuras pesquisas. O gráfico de bolha apresentado na Figura~\ref{fig:mapa_mapeamento_sistematico} contém três facetas: \textbf{Tipo de Contribuição}, \textbf{Área de Foco} e \textbf{Tipo de Pesquisa}. Embora 30 estudos primários foram considerados no presente MS, é importante mencionar que para o gráfico de bolha alguns estudos primários foram agrupados em mais de uma categoria. Por exemplo, para o gráfico de bolha apresentado na Figura~\ref{fig:mapa_mapeamento_sistematico}, a soma dos estudos primários agrupados em cada faceta é maior do que o número de estudos primários selecionados, e isso acontece uma vez que um determinado estudo primário pode ser classificado em diversas facetas e categorias.

Para responder a primeira parte da \textbf{QP$_1$}, foram analisados todos os estudos primários, concentrando-se na identificação de qual metamodelo da ADM tem sido mais utilizado na literatura. Na Figura~\ref{fig:frequencia_kdm_packages}, lado esquerdo, são exibidos os metamodelos da ADM utilizados na literatura. Como pode ser observado, o metamodelo KDM é o mais utilizado, tendo uma frequência de 66.67\%. Em seguida, o segundo metamodelo mais utilizado é o SMM - 10\% dos estudos primários relatam a utilização desse metamodelo. O metamodelo ASTM foi utilizado em apenas 6.66\% dos estudos primários. 16.66\% dos estudos primários não mencionam explicitamente qual metamodelo foi utilizado durante o processo de modernização conduzido, apenas citam e relatam a utilização da ADM.

\begin{figure}[!h]
\caption{Frequência de utilização dos metamodelos da ADM e frequência de utilização dos seus pacotes.}
 \label{fig:frequencia_kdm_packages}
\centering
\begin{minipage}{.5\textwidth}
  \centering
  \includegraphics[scale=0.9]{images/MetamodelosDIstribuition}
\end{minipage}%
\begin{minipage}{.5\textwidth}
  \centering
  \includegraphics[scale=0.9]{images/PacotesKDM}
\end{minipage}
\fautor
\end{figure}

Para responder a segunda parte da \textbf{QP$_1$}, foram analisados quais pacotes são mais e menos utilizados dentro do metamodelo KDM. Na Figura~\ref{fig:frequencia_kdm_packages}, lado direito, é evidente que os pacotes \texttt{Code} e \texttt{Action} são os mais utilizados na literatura, contendo uma frequência de 65\%. Acredita-se que a razão para essa frequência tão alta seja por dois principais motivos: (\textit{i}) tais pacotes representam o código-fonte de um determinado sistema, além disso, a maioria das abordagens identificadas utiliza como entrada o código-fonte de sistemas que almejam modernizar seguindo a ADM e seus metamodelos; (\textit{ii}) outros pacotes do KDM ainda não possuem ferramentas para realizar a instanciação de forma automática, ou seja, existe uma limitação de \textit{parsers} que análisa outros tipos de artefatos de um determinado sistema para criar uma representação mais fiel do sistema, não apenas do código-fonte. O terceiro pacote mais utilizado é o \texttt{Data}, o qual é utilizado para representar dados,  tais como banco de dados, registros, etc. Os pacotes \texttt{Event} e \texttt{UI} foram utilizados 10\%. Outros pacotes do metamodelo KDM não foram explicitamente mencionados nos estudos primários identificados.

Observando a faceta \textbf{Tipo de Pesquisa}, lado direito da Figura~\ref{fig:mapa_mapeamento_sistematico}, é possível responder a \textbf{QP$_2$}. A maioria dos estudos primários identificados foi classificado como \aspas{\textbf{Pesquisa de Avaliação}}, aproximadamente 49\%. Uma pequena porcentagem de estudos primários foi classificada como \aspas{\textbf{Pesquisa de Validação}} - apenas 3.12\%. 12.50\% dos estudos primários identificados descrevem a experiência dos autores (\aspas{\textbf{Artigo Des. Experiência}}) com a utilização da ADM e de seus metamodelos. 18.75\% foram agrupas em \aspas{\textbf{Proposta Conceitual}} e \aspas{\textbf{Artigos Des. Opinião}} teve a frequência de 15.62\%.

Ainda em relação à Figura~\ref{fig:mapa_mapeamento_sistematico} na faceta \textbf{Tipo de Contribuição}, lado esquerdo, é visto que a maioria dos estudos primários identificados apresenta \aspas{\textbf{processos}} para auxiliar os engenheiros de software durante a modernização de sistemas legados. Também foi identificado um total de 15 estudos primários (25\%) que apresentam algum tipo de \aspas{\textbf{transformação de modelos}} utilizando os metamodelos da ADM. Similarmente, 15 estudos primários (25\%) demonstram ferramentas para auxiliar o engenheiro de software durante a condução da modernização utilizando ADM e seus metamodelos. Acredita-se que esses dois últimos resultados foram obtidos uma vez que a maioria dos estudos primários identificados descreve processos de modernização, assim, pesquisadores devem criar um conjunto de transformações de modelos e ferramentas para automatizar parcialmente ou totalmente o processo proposto.

Por outro lado, \aspas{\textbf{metamodelos}} e \aspas{\textbf{métricas}} são as contribuições com menos estudos primários identificados, 5\% cada. Com isso, argumenta-se que os estudos primários que descrevem \aspas{\textbf{processos}} para ajudar a modernização dos sistemas legados por meio da ADM, os estudos que apresentam \aspas{\textbf{transformação de modelos}} entre os metamodelos da ADM (KDM, SMM e ASTM) e os artigos que descrevem ferramentas para automatizar parcialmente ou totalmente o processo da ADM podem ser considerados como grupos de evidências. E os \aspas{\textbf{metamodelos}} (artigos que explicam e/ou apresentam como estender metamodelos da ADM) e \aspas{\textbf{métricas}} (artigos que descrevem como aplicar métricas nos metamodelos da ADM) podem ser considerados como deserto de evidência, evidenciando que novos estudos primários são necessários.

Considerando o centro da Figura~\ref{fig:mapa_mapeamento_sistematico}, faceta \textbf{Área de Foco}, é possível visualizar que a maioria dos estudos primários identificados foi classificado como \aspas{\textbf{Modernização de Software}}, um total de 53.32\%. Em seguida, 21.66\% dos estudos primários foram categorizados em \aspas{\textbf{Extração de \textit{Business Knowledge}}}. \aspas{\textbf{Extração de Interesses}}, \aspas{\textbf{Extensão dos metamodelos da ADM}} e \aspas{\textbf{Aplicabilidade}}, coletivamente, representam uma porcentagem de aproximadamente 25\% dos estudos primários. Como resultado dessa análise, foi possível responder parcialmente a \textbf{QP$_3$}, ou seja, os principais \textbf{Tipo de Contribuição} relacionado à ADM e seus metamodelos disponíveis na literatura e identificados no MS foram destacados. A resposta conclusiva e completa da \textbf{QP$_3$} é apresentada a seguir. Para facilitar o entendimento e a organização deste mapeamento cada \textbf{Área de Foco} identificado no MS é apresentado em uma subseção.


\subsubsection{Modernização de Software} % (fold)
\label{ssub:approach}

\citeonline{6311013} propõem \textbf{GAFEMO}, a qual é uma abordagem para auxiliar a modernização de sistemas legados em sistemas orientados a serviço. Essa abordagem utiliza como entrada o sistema legado e, então, cria uma instância do metamodelo KDM para representar o código-fonte do sistema legado. Posteriormente, os autores definiram um conjunto de transformações para serem aplicadas nessa instância do KDM almejando criar os serviços.


 %Jorge Maratalla et al., propose \textbf{GAFEMO}~\cite{6311013}, which aims to modernize a legacy systems to the service-oriented approach taking advantage of the features provided by gap-analysis techniques. This approach takes as input a legacy system and then creates KDM representations of it. Afterwards, a set of rules are applied in this model to create the services.

%In~\cite{Mazon:2007:MDM:1784489.1784497} the authors propose a modernization approach for the modernization of Data warehouses following the concepts of ADM. The approach automatically performs the following tasks: (\textit{i}) obtain a logical representation of data sources (\textit{ii}) mark this logical representation with MD concepts, and (\textit{iii}) derive a conceptual MD model from the marked model.

%\citeonline{Mazon:2007:MDM:1784489.1784497} apresentam uma abordagem para modernizar

\citeonline{Mazon:2007:MDM:1784489.1784497} definiram uma abordagem para modernizar \textit{data warehouses} seguindo os conceitos da ADM. Essa abordagem automaticamente executa as seguintes tarefas: (\textit{i}) obtém uma representação lógica das fontes de dados; (\textit{ii}) posteriormente, essa representação lógica é anotada e transformada em instância do KDM; (\textit{iii}) em seguida, a instância do KDM é transformada em um modelo de análise de dados multidimensional. Similarmente, \citeonline{Guzman:2007:AAR:1339262.1339532} definem uma abordagem para realizar a análise de sistemas legados e criar funcionalidades para serem expostas como serviços, usando conceitos de \textit{Web Services} juntamente com a ADM.



%In~\cite{Guzman:2007:AAR:1339262.1339532} is defined an approach that is focused on the analysis of legacy systems to discover and create functionalities to be exposed as services using Web Services by means of ADM. 
%It is based in five steps: (\textit{i}) Database reverse engineering: database schema is reversed and a suitable model is built; (\textit{ii}) First service extraction: based on the structure of the database schema, a first service extraction can be undertaken; (\textit{iii}) PIM generation: is obtained from the PSM representation using a model-to-model transformation, CRUD operations are automatically created; (\textit{iv}) Service discovering: abstract objects are identified in the PIM; (\textit{v}) WSDL (Web Service Description Language) generation: using the PIM, a model-to-model transformation and a WSDL  metamodel are generated to expose the services discovered and created in the PIM and the PSM. 

Em~\citeonline{5741334, SMR:SMR582} os autores desenvolveram uma abordagem seguindo as diretrizes da ADM denominada \aspas{CloudMIG}. Essa abordagem pretende fornecer software como serviço (do inglês - \sigla{SaaS}{\textit{Software as a Service}}). %Essa abordagem consiste de seis principais passos: (\textit{i}) extração: inclui a extração da arquitetura e instanciação do sistema em nível do metamodelo KDM, (\textit{ii}) seleção: 
Do mesmo modo, \citeonline{4400179} também desenvolveu uma abordagem para utilizar ADM e seus metamodelos, principalmente o KDM, para analisar sistemas legados, descobrir e criar funcionalidades para serem expostas como serviços, utilizando os conceitos de \textit{Web Services}.

%In~\cite{5741334, SMR:SMR582} is proposed an approach based on ADM named CloudMIG that aims at supporting SaaS (Software as a Service) providers to semi-automatically migrate legacy software systems to the cloud. It is composed of six major steps: (\textit{i}) Extraction: Includes the extraction of architectural and utilization models of the legacy system, the approach uses KDM; (\textit{ii}) Selection: Select an appropriate CEM- compatible cloud profile candidate; (\textit{iii}) Generation: Produces the target architecture and a mapping model; (\textit{iv}) Adaptation: The adaptation activity enables a reengineer to manually adjust the target architecture; (\textit{v}) Evaluation: Realize static analyses and a runtime simulation of the target architecture; (\textit{vi}) Transformation: The actual transformation of the existing system from the generated target architecture to the aimed cloud environment. In~\cite{4400179} the authors propose an approach that uses ADM which is focused on the analysis of legacy systems to discover and create functionalities to be exposed as services using Web Services.

\citeonline{5328801, delCastillo:2009:PRP:1529282.1529753, ICEISPerez:CastilloGCP12} apresentam uma abordagem para modernizar sistemas legados juntamente com o banco de dados relacional. Mais especificamente, essa abordagem obtém três principais modelos seguindo a abordagem ADM. Primeiro, tal abordagem recupera uma instância do pacote \texttt{Code} do metamodelo KDM. Em seguida, uma instância do pacote \texttt{Data} do metamodelo KDM também é recuperada para representar o banco de dados relacional. Essa segunda instância é recuperada com base nas \textit{embedded SQL}, que são encontradas no código-fonte. O objetivo é separar tais SQL para facilitar a modularidade e futuras manutenções. Depois, transformações de modelos são executadas e o sistema legado é modernizado.

 %P\'{e}rez-Castillo et al.,~\cite{5328801, delCastillo:2009:PRP:1529282.1529753, ICEISPerez:CastilloGCP12} present  approaches to modernize legacy systems together with the legacy relational database. This approach recovers the code-to-data linkages and obtains three kinds of models according to the ADM approach: (\textit{i}) The KDM Code Model, which represents the inventory of legacy source code. It has also the points that link the SQL Sentence Models and Database Schema Models. (\textit{ii}) The SQL Sentence Model for modeling a certain SQL query that was embedded in legacy source code. (\textit{iii}) The Database Schema Model, which represents the specific database fragment derived by an SQL Sentence Model. 


~\citeonline{FuentesFernandez2012247} apresentam uma abordagem de modernização denominada XIRUP, interativa e estruturada em quatro fases: (\textit{i}) avaliação preliminar,
(\textit{ii}) compreensão, (\textit{iii}) construção e (\textit{iv}) migração. Essa abordagem de modernização é baseada em componentes, com foco no levantamento inicial de informações-chave, e depende de uma abordagem orientada a modelos, com o uso extensivo da experiência dos projetos anteriores. 

~\citeonline{Mainetti:2012:MMT:2364120.2364182} apresentam uma abordagem que permite desenvolvedores automaticamente modernizar a interface gráfica de um determinado sistema legado para \sigla{RIA}{\textit{Rich Internet Application}}.
%
%
%Mainetti et al.,~\cite{Mainetti:2012:MMT:2364120.2364182} present an approach that allows developers to automatically modernize the client side of legacy systems. In this approach developers can refactor the Graphical User Interface (GUI) of legacy systems during the modernization, taking the opportunities offered by novel interaction paradigms, i.e.,  
%
 Similarmente, \citeonline{Rodriguez-Echeverria:2011:MLW:2186508.2186536} também apresentam uma abordagem de modernização que utiliza os metamodelos da ADM para definir um processo sistemático com o objetivo de transformar aplicações web em RIA.
 
 %In~\cite{Rodriguez-Echeverria:2011:MLW:2186508.2186536} the authors present an approach for the definition of a systematic process for Web Applications (WA) to RIA modernization, by applying ADM principles. The approach presented by the authors consists on generating a RIA client from the legacy WA presentation and navigation layers and its corresponding service-oriented connection layer with the underlying business logic at server side. 
 
 
 
 
~\citeonline{6385130} definem uma abordagem que ajuda na construção de distintas visões arquiteturais de sistemas legados. Assim, os autores criaram um conjunto de algoritmos de agrupamento que é conduzido por meio de visões arquiteturais comuns. Essa abordagem faz
utilização do metamodelo KDM. ~\citeonline{5440163} utilizam os conceitos da ADM para construir uma ferramenta de modernização, visando gerar relatórios de métricas para avaliar os esforços de migração. Os autores desenvolveram um extrator que gera instâncias do metamodelo KDM a partir de código PS-SQL, ou seja, transformaram PS-SQL para instâncias do metamodelo KDM, e, em seguida, relatórios de métricas foram gerados para o metamodelo KDM.

 
 %Boussaidi et al.,~\cite{6385130} propose an approach that makes use of the KDM to reconstruct and document software architectural views of the legacy system. They consider an architectural view to be a way of partitioning a system using a specific set of KDM relevant concepts and relations and they propose clustering algorithms that target specific views mainly a layered view that we call horizontal view and a feature based view that we call vertical view. In~\cite{5440163} ADM is used into practice by building a modernization tool to generate metric reports of legacy Oracle Forms applications to assess migration efforts. The authors devised an extractor that generates KDM models from PL-SQL code (PL/SQL-to-KDM) and a metrics report generator for these KDM models. 

\subsubsection{Extração de \textit{Business Knowledge}}
\label{ssub:Business_Knowledge_Extraction}

~\citeonline{Perez-Castillo:2011:ECS:1982185.1982249,6080834, 6498507,Perez-Castillo:2010:IBP:1875847.1875861} apresentam uma abordagem para recuperar regras de negócio de um determinado sistema legado, com base na ADM e no KDM. Essa abordagem é baseada em um conjunto de transformações: (\textit{i}) vários PSM são recuperados de acordo com específicos artefatos do sistema legado, (\textit{ii}) posteriormente, tais PSM são transformados em uma instância do metamodelo KDM, (\textit{iii}) em seguida, o metamodelo KDM é transformado em um modelo específico para definir regras de negócio. Além disso, os autores realizaram um conjunto de estudo de caso para verificar a eficiência e a eficácia da abordagem~\cite{PerezCastillo20121370}.

~\citeonline{Perez-Castillo:2012:IEL:2231936.2231949} também fornecem uma técnica semiautomática baseada em análise dinâmica, combinada com análise estática para instrumentar o código-fonte, com o objetivo de descobrir e obter processos de negócios em nível do metamodelo KDM. ~\citeonline{Perez-Castillo:2010:IBP:1875847.1875861} apresentam e descrevem com detalhes todas as transformações entre o metamodelo KDM e o metamodelo \sigla{BPMN}{\textit{Business Process Model and Notation}}. 

~\citeonline{lastDAyOFMyLife} apresentam uma abordagem que facilita a compreensão de um determinado software, permitindo a rastreabilidade de regras de negócios e cenários de negócios no sistema de software. A abordagem visa extrair conhecimentos específicos de negócios a partir do conhecimento sobre o sistema de software existente, representados no KDM. ~\citeonline{Fernandez-Ropero:2012:EAB:2367051.2367064} descrevem um conjunto de regras para transformar o metamodelo \textit{Mining} XML, o qual é utilizado para representar a sequência de atividades de negócios executados para o metamodelo KDM. 






%apresentam uma abordagem que facilita 
%Normantas and Vasilecas~\cite{lastDAyOFMyLife} present an approach that facilitates software comprehension by enabling traceability of business rules and business scenarios in software system, i.e., their approach aim to extract business specific knowledge from the knowledge about the existing software system represented within the KDM. Ropero et al.,~\cite{Fernandez-Ropero:2012:EAB:2367051.2367064} describes a set of rules to transform Mining XML (MXML) metamodel, which is common used to represent the sequence of business activities executed by an enterprise system to KDM. The authors takes an MXML model and obtains an equivalent KDM model at the same abstraction level. The proposed set of rules consist of eight declarative transformation rules. 

\subsubsection{Extração de Interesses} % (fold)
\label{ssub:Concern Extracting}

\citeonline{dani_san, dani_san_tool, daniel_san_journal} definem uma abordagem denominada CCKDM para auxiliar a identificação de interesses transversais utilizando uma combinação de bibliotecas de interesses com o algoritmo de mineração de dados K-means. A abordagem possui quatro passos  e dois deles são opcionais. A entrada ao processo é uma instância do metamodelo KDM e as saídas são a mesma instância do KDM com anotações dos interesses transversais identificados e alguns arquivos de registros (\textit{logs}). A identificação é realizada por meio de uma biblioteca de interesses em conjunto com o repositório de elementos KDM. A abordagem é explicada de forma que possa ser replicada por outros pesquisadores que tenham interesse em modificá-la e/ou estendê-la.

\subsubsection{Extensão dos metamodelos da ADM} % (fold)
\label{ssub:extension_of_adm_s_metamodels}

~\citeonline{5773392} definem uma extensão do metamodelo KDM denominada \sigla{COMO}{\textit{Component-Oriented MOdernization}}. De acordo com os autores, KDM suporta apenas parcialmente os conceitos de componentes. Dessa forma, a extensão COMO objetiva suprir tal limitação do metamodelo KDM. Assim, novas metaclasses são definidas para detalhar conceitos específicos relacionados com componentes. Similarmente, ~\citeonline{Perez-Castillo:2012:IEL:2231936.2231949} definem uma extensão para o metamodelo KDM que visa melhorar a representação de arquivos de registros (\textit{logs}). Porém, diferentemente da abordagem COMO, os autores não criam novas metaclasses para o metamodelo KDM, apenas utilizam anotações. ~\citeonline{library7329} define uma extensão para o metamodelo KDM para representar todos os elementos do paradigma de programação orientada a aspectos, ou seja, novas metaclasses como \texttt{Aspect}, \texttt{Advice} e \texttt{Point-cut} podem ser instanciadas para representar conceitos do paradigma de programação orientada a aspecto por meio desse KDM estendido.


\subsubsection{Aplicabilidade} % (fold)
\label{ssub:applicability}

%We also identified a small number of papers that address just the applicability of the ADM and its metamodel. 

~\citeonline{PerezCastillo:2011jo, Perez-Castillo:2012:IEL:2231936.2231949, 6498507} descrevem como utilizar e aplicar  o metamodelo KDM para modernizar sistemas legados. Além disso, os autores também descrevem cada camada do metamodelo KDM e é apresentado um conjunto de exemplos para auxiliar no entendimento de como utilizar os conceitos da ADM e as metaclasses do metamodelo KDM durante as atividades de modernização de sistemas. De acordo com os autores, tais estudos primários podem ser utilizados para auxiliar novos pesquisados a entender e começar a utilizar os conceitos da ADM e do KDM.

%\subsection{Ferramentas identificada no MP}

%Durante a condução do MP foi possível identificar 15 ferramentas como pode ser observado na Figura~\ref{fig:mapa_mapeamento_sistematico}, facetas \textbf{Tipo de Contribuição}. Porém, apenas nove ferramentas foram explicitamente explicadas seu propósito.   
\section{Principais Constatações e Questões em Aberto}\label{subsec:principais_constatações_e_questões_Em_Aberto}

As recentes propostas em ADM têm-se concentrado principalmente na elaboração de abordagens para auxiliar a modernização do sistema legado para outra plataforma/arquitetura. No entanto, se olharmos para o problema global da integração da modernização considerando o contexto da ADM, ainda há espaço para melhorias. Por exemplo, hoje em dia, apenas a ferramenta MoDisco\footnote{https://eclipse.org/MoDisco/} cria uma instância do metamodelo KDM de forma automática. Entretanto, essa ferramenta preocupa-se apenas com os pacotes \texttt{Code} e \texttt{Action}. Dessa maneira, para integrar o metamodelo KDM em um contexto maior, novos \textit{parsers} precisam ser definidos e criados. Embora alguns autores tenham criado algumas iniciativas~\cite{5440163,Bruneliere_2010MODISCO}, essa área ainda é limitada. Nesse contexto, ainda se faz necessário novos esforços para criar \textit{parsers}, para representar todas as camadas do metamodelo KDM de forma automática ou semiautomática.

Foi observado durante o MS que existem algumas principais limitações a serem investidas para facilitar a utilização da ADM e do metamodelo KDM de forma eficaz. Por exemplo, como já salientado no Capítulo~\ref{chapter:fundamentacao_teorica} Seção~\ref{sec:refatoracao}, refatorações são técnicas utilizadas para melhorar a estrutura do software. Atualmente, é evidente que a refatoração é de suma importância para melhorar a qualidade do código-fonte, e, assim, melhorar a sua manutenibilidade.
Embora a ADM e, principalmente, o KDM tenham sido criados para auxiliar todo o processo da modernização de sistemas, até esse momento existe uma ausência de abordagens e ferramentas que auxiliem os engenheiros de modernização a criar e aplicar refatorações de forma consistente para o metamodelo KDM. Dessa forma, usualmente, os engenheiros de modernização precisam desenvolver suas próprias ferramentas para refatorar diversos sistemas representados em nível do metamodelo KDM. Tais soluções geralmente são proprietárias e, consequentemente por meio delas fica difícil reutilizar e prover interoperabilidade entre ferramentas.

Além disso, é sabido que a atividade de refatoração é pertinente a qualquer processo de modernização. Dessa forma, quando um sistema é representado utilizando diferentes visões conceituais para representar níveis de abstração do sistema (por exemplo, visão arquitetural, visão de código-fonte, visão do banco de dados, etc.), um acidente comum que surge durante atividades de refatorações é a dessincronização das instâncias do metamodelo, resultando em visões inconsistentes após a aplicação de uma refatoração. Assim, no contexto do metamodelo KDM existe uma carência de abordagens e apoios ferramentais que auxiliam a sincronizar tais mudanças após a aplicação de um conjunto de refatorações no KDM. Pesquisas recentes sugerem que a aplicação de técnicas de propagação de mudanças pode auxiliar na identificação e atualização de todas as instâncias/visões do KDM, permitindo manter todas as visões/instâncias do metamodelo KDM sincronizadas. 

Na literatura, é possível identificar um conjunto de refatorações já validadas e que são geralmente aplicadas em código-fonte, por exemplo, \textit{Extract Class}, \textit{Move Method}, \textit{Move Attribute}, etc. Essas são apenas alguns exemplos de refatorações úteis que não são facilmente reutilizadas na prática durante a condução de modernização de um determinado sistema. Essa limitação pode ser atribuída devido à ausência de um meio padronizado de disponibilizar refatorações. Embora a ADM forneça um conjunto de metamodelos para auxiliar o engenheiro de modernização a conduzir MDE, até esse momento a ADM não provê instruções para auxiliar o engenheiro a promover o reúso de refatorações juntamente com os seus metamodelos padronizados (por exemplo, KDM) durante o processo de modernização. Essa limitação faz com que o engenheiro de modernização crie suas próprias soluções/refatorações, resultando em um possível atraso no processo de modernização. Contudo, as soluções/refatorações definidas não são facilmente reutilizadas, pois são proprietárias e dependente de linguagem. Uma abordagem promissora é lidar com a refatoração de forma independente da linguagem, aumentando as possibilidades de reutilização de refatorações. Dessa forma, existe uma necessidade de criar um metamodelo para auxiliar o engenheiro de modernização a promover o reúso de metadados relacionados às refatorações. Adicionalmente, esse metamodelo deve ser coerente com as terminologias da ADM e, principalmente, deve trabalhar de forma uniforme com o metamodelo KDM para garantir a independência de linguagem e plataforma proporcionada pelo KDM. 

\section{Ameaças à Validade}\label{subsec:ameaças_a_validade}
Nesta seção as quatro ameaças à validade do MS aqui estruturado são destacadas:

\begin{itemize}
\item \textbf{Seleção dos estudos primários}: Com o objetivo de garantir um processo de seleção imparcial, definiram-se questões de pesquisa e critérios de inclusão e exclusão.
No entanto, não é possível descartar ameaças de uma perspectiva de avaliação da qualidade, uma vez que os estudos foram selecionados sem atribuir qualquer pontuação. Além disso, tentou-se ser o mais abrangente possível, de modo que nenhum limite foi definido em relação à data de publicação dos estudos primários. Nota-se que não foram definidas muitas restrições, pois almeja-se obter uma visão ampla da área de pesquisa;

\item \textbf{Estudo primário não identificado}: Foi conduzido o MS em várias bibliotecas digitais (\textit{ACM}, \textit{IEEE XPLORE}, \textit{Scopus}, \textit{Web of Science} e \textit{Engeneering Village}), todavia, é possível que alguns estudos primários não tenham sidos identificados durante a condução do MS. Para mitigar essa ameaça, foram selecionadas as bibliotecas digitais recomendas por~\citeonline{Kitchenham, Dyba2005rafa};

\item \textbf{Confiabilidade dos colaboradores}: Os revisores desse MS são pesquisadores da área de reutilização de software e não há conhecimento de qualquer viés que possa ter sido introduzido durante as análises;

\item \textbf{Extração dos dados}: Outra ameaça com relação ao MS refere-se a como os dados foram extraídos das bibliotecas digitais. Nem toda informação estava clara o suficiente para responder às perguntas, assim, alguns dados tiveram de ser interpretados.
A fim de garantir a validade, foram analisadas várias fontes de dados. No caso de desacordo entre dois colaboradores, um terceiro colaborador, definido como árbitro, era consultado para garantir um acordo comum.

\end{itemize}


\section{Considerações Finais}\label{sec:considerações_finais_do_mapeamento_sistematico}

Pesquisas na área da ADM podem levar a avanços na modernização de sistemas, resultando em sistemas que são mais sustentáveis, extensíveis e reutilizáveis. Para obter uma visão geral da pesquisa atual nessa área de pesquisa, foi realizado e apresentado neste capítulo um MS. O MS foi conduzido em várias bibliotecas digitais, como \textit{ACM}, \textit{IEEE XPLORE}, \textit{Scopus}, \textit{Web of Science}, \textit{Engeneering Village} e \textit{Google Scholar}. Posteriormente, identificaram-se 30 estudos primários, os quais foram utilizados para extrair informações para responder três QPs. Scopus foi a biblioteca digital que retornou mais estudos primários, 58\% (150), foram recuperados 20\% (51) estudos da ACM, 12\% (30) da Engineering Village, 6\% (17) da Web of Science e 4\% (11) da IEEE. 

O metamodelo KDM é o é mais utilizado, tendo uma frequência de 66.67\%. Em seguida, o metamodelo que é mais utilizado é o SMM - 10\% dos estudos primários relatam a utilização desse metamodelo. Já o metamodelo ASTM foi utilizo em apenas 6.66\% dos estudos primários. 16.66\% dos estudos primários não mencionam explicitamente qual metamodelo foi utilizado durante o processo de modernização conduzido, apenas citam e relatam a abordagem ADM. Também foi identificador que os pacotes \texttt{Code} e \texttt{Action} são os mais utilizados na literatura, contendo uma frequência de 65\%. O terceiro pacote mais utilizado é o \texttt{Data}, o qual é usado para representar dados e persistência. Os pacotes \texttt{Event} e \texttt{UI} foram utilizados 10\%. Outros pacotes do metamodelo KDM não foram explicitamente mencionados nos estudos primários.

A maioria dos estudos primários identificados foram classificados como \aspas{\textbf{Pesquisa de Avaliação}}, aproximadamente 49\%. Uma pequena porcentagem de estudos primários foi classificada como \aspas{\textbf{Pesquisa de Validação}}, apenas 3.12\%. 12.50\% dos estudos primários identificados descrevem a experiência dos autores (\aspas{\textbf{Artigo Des. Experiência}}) com a utilização da ADM e seus metamodelos. 18.75\% foram agrupados em \aspas{\textbf{Proposta Conceitual}} e \aspas{\textbf{Artigos Des. Opinião}} teve a frequência de 15.62\%. A maioria dos estudos primários identificados apresenta \aspas{\textbf{processos}} para auxiliar os engenheiros de software durante a modernização de sistemas legados. Também foram identificados 15 estudos primários (25\%) que apresentam algum tipo de \aspas{\textbf{transformação de modelos}} utilizando os metamodelos da ADM. Similarmente, 15 estudos primários (25\%) apresentam ferramentas para auxiliar o engenheiro de software durante a condução da modernização utilizando ADM e seus metamodelos. \aspas{\textbf{Metamodelos}} e \aspas{\textbf{métricas}} são as contribuições com menos estudos primários identificados, 5\% cada. Observando a faceta \textbf{Área de Foco}, é possível visualizar que a maioria dos estudos primários identificados foram classificados como \aspas{\textbf{Modernização de Software}}, um total de 53.32\%. Em seguida, 21.66\% dos estudos primários foram categorizados em \aspas{\textbf{Extração de \textit{Business Knowledge}}}. \aspas{\textbf{Extração de Interesses}}, \aspas{\textbf{Extensão dos metamodelos da ADM}} e \aspas{\textbf{Aplicabilidade}} coletivamente representam uma porcentagem de aproximadamente 25\% dos estudos primários.


Outra contribuição deste capítulo é o mapa definido na Figura~\ref{fig:mapa_mapeamento_sistematico}. Ao observar esse mapa, tem-se uma visão global da literatura em relação à ADM, - identificando quais categorias foram enfatizadas nas últimas pesquisas, além de lacunas e possibilidades para futuras pesquisas. Adicionalmente, esse mapa também fornece esclarecimentos adicionais sobre as frequências de publicação ao longo do tempo.


%Como já salientado, é evidente que a aplicação de refatoração é de suma importância para melhorar a qualidade do código-fonte, e assim, melhorar a sua manutenibilidade. No entanto, embora a ADM, e principalmente o KDM tenham ambos sido propostos para auxiliar todo o processo da modernização de sistemas, 

Até esse momento, existe uma ausência de abordagens e apoios computacionais que auxiliem os engenheiros de modernização a aplicar refatorações de forma consistente para o metamodelo KDM. Diante disso, usualmente, os engenheiros de modernização precisam desenvolver suas próprias ferramentas para refatorar diversos sistemas. E tais soluções geralmente tendem a ser proprietárias e, como consequência, torna-se difícil a reutilização e a interoperabilidade entre ferramentas. 

Com o intuito de mitigar essa ausência, nos próximos capítulos desta tese, uma abordagem, um metamodelo e um apoio computacional para auxiliar o engenheiro de modernização e o engenheiro de software durante a aplicação, reúso e compartilhamento de refatorações para o KDM são apresentados. Mais especificamente, no Capítulo~\ref{chapter:catalogo_refactoring_KDM} é destacada uma abordagem para criar refatorações para o metamodelo KDM, ou seja, evidenciando como refatorações podem ser criadas para o KDM~\cite{durelli_catalogo, durelli_VEM_ferramenta}. %Além disso, ainda é exposto um mapeamento entre os conceitos do paradigma orientado a objetos para o metamodelo KDM, assim, engenheiros de modernização podem de forma mais fácil adaptar novas refatorações para o KDM.

No Capítulo~\ref{chapter:Toward_a_Refactoring_Metamodel_for_KDM}, é apresentado um metamodelo para disponibilizar e promover o reúso de refatorações no contexto da ADM e do KDM. %No Capítulo~\ref{chapter:Abordagem_de_sincronizacao} é apresentada uma abordagem denominada KDM-SInc que é utilizada para manter uma determinada instância do metamodelo KDM consistente e sincronizado após a aplicação de refatorações.
Um apoio computacional, denominado KDM-RE, é apresentado no Capítulo~\ref{chapter:ferramenta_kdm_re} e é composto por três \textit{plug-ins} do Eclipse: (\textit{i}) o primeiro consiste em um conjunto de \textit{Wizards} que apoia o engenheiro de software na aplicação das refatorações em diagramas de classe UML; (\textit{ii}) o segundo consiste em um apoio à importação e reúso de refatorações disponíveis no repositório; (\textit{iii}) o terceiro consiste em um módulo de propagação de mudanças que permite manter modelos internos do KDM sincronizados;

\chapter{Características das Refatorações para o metamodelo KDM}\label{chapter:catalogo_refactoring_KDM}
\input{tex/cap_catalogo_refactoring}

\chapter{O Metamodelo de Refatorações SRM}
\label{chapter:Toward_a_Refactoring_Metamodel_for_KDM}
\section{Considerações Iniciais}\label{sec:consideracoes_iniciais}
Geralmente refatorações são aplicadas para melhorar a qualidade do software (por exemplo, a extensibilidade, a modularidade, a capacidade de reutilização, a complexidade, manutenção, etc). 
A maioria das pesquisas apontam que usualmente refatorações são aplicadas em nível de código-fonte~\cite{Fowler1999, Demeyer1, Demeyer2, Opdy92b}. Tais pesquisas não estão preocupadas em como promover o reuso, compartilhamento de refatorações e também não estão interessadas em criar refatorações para outros tipos de artefatos, tais como os artefatos providos pela ADM, por exemplo, o metamodelo KDM. 
%
Como já destacado, na literatura é possível identificar um conjunto de refatorações já validadas e que são usualmente aplicadas em código-fonte, por exemplo, \textit{Extract Class}, \textit{Move Method}, \textit{Move Attribute}, etc. Essas são apenas alguns exemplos de refatorações úteis que não são facilmente reutilizadas na prática durante a condução de modernização de um determinado sistema. Essa limitação pode ser atribuída devido a ausência de um meio padronizado de disponibilizar refatorações. 
Uma abordagem promissora é lidar com a refatoração de forma independente da linguagem – aumentando assim as possibilidades de reutilização de refatorações.

Como ressaltado no Capítulo~\ref{chapter:adm_kdm} a ADM fornece um conjunto de metamodelos para auxiliar o engenheiro de software a conduzir MDRE. 
Porém, até esse momento a ADM não provê instruções ou até mesmo um metamodelo para auxiliar o engenheiro a promover o reuso de refatorações juntamente com os seus metamodelos padronizados (por exemplo, KDM) durante o processo de modernização. 
Essa limitação faz com que o engenheiro de modernização crie suas próprias soluções/refatorações, resultando em um possível atraso no processo de modernização. 
No Capítulo~\ref{chapter:catalogo_refactoring_KDM} algumas refatorações bem conhecidas e propostas por~\citeonline{Fowler1999} são adaptadas para serem aplicadas em instâncias do metamodelo KDM. Contudo, em seu estado atual as refatorações adaptadas não podem serem facilmente reutilizadas e compartilhadas entre os modernizadores. 
Com o intuito de suprir tal limitação neste capítulo é apresentado um metamodelo para auxiliar o engenheiro de modernização a promover o reuso de refatorações no contexto do metamodelo KDM. Com a utilização desse metamodelo, informações (metadados) sobre refatorações podem ser reutilizadas de forma independente de linguagem e plataforma. 

O metamodelo aqui proposto tem as seguintes características: (\textit{i}) permite a interoperabilidade de refatorações para um amplo domínio e (\textit{ii}) auxilia o engenheiro de modernização a definir refatorações representativas em forma de metadados. Nota-se que o metamodelo aqui proposto segue a mesma proposta de outros metamodelos definidos na ADM e está totalmente integrado com o metamodelo KDM. Em outras palavras, uma instância do metamodelo de refatoração contêm metadados que representam uma refatoração escrita para ser executada em uma instância do metamodelo KDM. A escolha do metamodelo KDM se deu pois o mesmo é um metamodelo padronizado pela OMG, além de ser um metamodelo PIM, o que significa que quaisquer refatorações aplicadas em uma instância do KDM, podem ser consideradas independentes de linguagem e de plataforma e então ser transformado para em um PSM (ver Capítulo~\ref{chapter:fundamentacao_teorica} e Capítulo\ref{chapter:adm_kdm}). Utilizar refatorações como um metamodelo (de forma independente de linguagem) pode abrir opções para promover o reuso de refatorações. Por exemplo, dado uma instância do metamodelo de refatoração, o engenheiro de modernização poderia utilizar os metadados contidos nessa instância e aplicar a refatoração em qualquer sistema representado pelo KDM. Além disso, o engenheiro de modernização poderia instanciar o metamodelo de refatoração, criar um catálogo de refatorações e disponibilizar para que outros possam reutilizar tais refatorações. Tornando assim possível compartilhar e reutilizar refatorações no contexto da ADM.

%Com o intuito de apresentar o metamodelo de uma maneira mais prática, neste capítulo a construção do mesmo é realizado com o uso do EMF~\cite{EMF}. Porém, ressalta-se que o metamodelo pode ser criado com o uso de outras tecnologias.

As demais seções deste capítulo estão organizadas da seguinte forma: na Seção.... é apresentada a motivação para a criação do metamodelo de refatoração; na Seção ... o metamodelo é apresentado..., na Seção .. são feitas as considerações finais a respeito do metamodelo de refatoração proposto neste capítulo.\change{terminar aqui. Deve colocar todas as seções bem escritas.}

\section{Motivação para a criação de um metamodelo de refatoração} % (fold)
\label{sec:motiva_o_para_a_cria_o_de_um_meta_modelo_de_refatora_o}

Durante o mapeamento sistemático conduzido (ver Capítulo~\ref{chapter:mapeamento_sistematico})~\cite{durelli_systematic_mapping} pôde-se observar na literatura a carência de estudos que definem soluções para especificar e promover o reuso de refatorações no contexto da ADM e do metamodelo KDM. Sem a adequada representação de refatorações para o KDM, a especificação e realização de uma refatoração pode se tornar uma atividade propensa a erros e difícil de reutilizar.

Neste contexto, suponha o seguinte cenário, um engenheiro de modernização define uma determinada refatoração e a compartilha por meio de um repositório. Dessa forma, outros engenheiros podem então navegador nesse repositório e identificar, não apenas essa, mas um conjunto de refatorações. O engenheiro então escolhe uma determinada refatoração no repositório, faz o \textit{download} e a reutiliza em seu sistema representado em nível de uma instância do metamodelo KDM. 

Uma das soluções para a concretização desse cenário seria a criação de um metamodelo para persistir metadados referentes à refatorações. Esse metamodelo deve possuir metaclasses e meta-relacionamentos que permitam representar e especificar metadados de refatorações, por exemplo, o nome da refatoração, sua motivação, os passos para sua realização, e até mesmo o mecanismo e sua pré- e pós-condições. Como ressaltado anteriormente, devido a carência de um metamodelo que possui tais características, esse capítulo têm como principal objetivo definir um metamodelo que permita a representação de metadados relacionadas com refatorações, porém, ainda respeite e siga as diretrizes dos metamodelos definidos na abordagem ADM, por exemplo, o metamodelo aqui definido precisa ser independente de linguagem e plataforma.


Essa ultima características é de suma importância. Assim, o metamodelo aqui apresentado utiliza as metaclasses do metamodelo KDM. Uma instância do metamodelo de refatoração, deve possuir instâncias de metaclasses específicas para definir metadados sobre refatorações (autor, nome, motivação, descrição, etc) e também instâncias de metaclasses do KDM que representam os elementos estruturais (\texttt{ClassUnit}, \texttt{StorableUnit}, \texttt{MethodUnit}, etc) onde a refatoração/transformação será aplicada. Dessa forma, o metamodelo de refatoração faz com que a operação/mecanismo de uma refatoração torne-se independente de plataforma e linguagem de programação. Assim, o metamodelo pode ser facilmente utilizado e aplicado em ferramentas existentes que utilizam como base o KDM aumentando a interoperabilidade de futuras ferramentas que utilizem esse metamodelo de refatoração. Levando em consideração as motivações destacadas, na Seção~\ref{sec:meta_modelo_de_refatora_es_estruturadas_srm_do_ingl_s_structured refactoring meta-model_} é apresentado o metamodelo de refatorações estruturadas (do inglês - Structured Refactoring Metamodel – SRM).  


%Além disso, uma característica de suma importância é que esse metamodelo de refatoração seja baseado nas metaclasses do KDM. Em outras palavras, uma instancia desse metamodelo, deve conter referencias a metaclasses do KDM, tornando assim, a operação/mecanismo de um refatoração independente de plataforma e linguagem. 

\section{Metamodelo de Refatorações Estruturadas} % (fold)
\label{sec:meta_modelo_de_refatora_es_estruturadas_srm_do_ingl_s_structured refactoring meta-model_}

Como pôde ser observado na seção anterior, são várias as motivações para definir um metamodelo para especificar refatorações. Neste sentido, nesta seção é apresentado um metamodelo denominado Metamodelo de Refatorações Estruturadas (do inglês - \sigla{SRM}{Structured Refactoring Metamodel}).

Refatorações são ubíquas durante a manutenção, produção e análise de software. Inúmeras comunidades têm surgido na literatura para criar e definir vários tipos de refatorações, incluindo refatorações de baixa granularidade~\cite{Fowler1999, Demeyer1, Demeyer2}, refatorações arquiteturais, refatorações para o paradigma orientado a aspecto, etc. Neste contexto, existe a necessidade de definir uma forma de promover o compartilhamento dessas refatorações. Assim, nessa seção é apresentado um metamodelo para auxiliar tais comunidades. Esse metamodelo representa uma fundamentação para o compartilhamento de informações (metadados) relacionadas à refatorações, assim, detalhes com informações precisas sobre refatorações podem ser armazenadas e/ou compartilhadas para que outros engenheiros possam reutilizar em seus projetos que utilizem o metamodelo KDM.

Por exemplo, uma comunidade que cria refatorações de baixa granularidade usualmente esta preocupada em definir e especificar refatorações desse tipo. Por outro lado, uma comunidade que especifica refatorações arquiteturais esta interessada apenas em refatoração arquiteturais. Porém, nota-se que ambas comunidades têm em comum a necessidade de especificar como organizar e definir as refatorações. Além disso, tais comunidades devem se preocupar em definir a motivação para a condução da refatoração, especificar o mecanismo da refatoração e descrever tais refatorações de maneira formal e adequada para a sua comunidade. Embora o contexto dessas comunidades seja diferentes ambas necessitam de uma forma padronizada para definir e especificar as refatorações. 


O SRM define um padrão comum para a especificação e descrição de refatorações. Além disso, esse metamodelo têm como princípio ser independente de linguagem de programação com o objetivo de fornecer uma plataforma comum pelo qual arquiteto, pesquisador e modernizador possam expressar refatorações sem se preocuparem com a plataforma ou linguagem de programação. O SRM possui três principais objetivos: (\textit{i}) prover uma forma de compartilhar informações sobre refatorações; (\textit{ii}) promover o reuso de refatorações de forma independente de plataforma e linguagem; e (\textit{iii}) ser uma proposta inicial ao \textit{Call for Proposals} do ADM Refatoring da OMG. O primeiro objetivo é apoiada por um conjunto de metaclasses que definem meta-atributos específicos para representar informações (metadados) de uma refatoração, auxiliando assim o compartilhamento das refatorações de forma intuitiva para os modernizadores. Similarmente, o segundo objetivo também é alcançado por metaclasses que possuem meta-atributos que representam os mecanismos das refatorações, bem como suas pré- e pós-condições. O terceiro objetivo é apoiado por todo o metamodelo SRM.
 
Na Figura~\ref{fig:refactoring_metamodel} o SRM é esquematicamente apresentado. O centro da figura representa a abordagem ADM. Pode-se entender que o SRM está inserido no contexto da ADM da mesma forma que seus outros metamodelos. A esfera externa é dividida em quatro facetas, cada uma contêm o nome de um metamodelo – por sua vez, cada faceta é acoplada por um retângulo contendo o nome de um metamodelo e um conjunto de metaclasses. Como pôde ser observado nessa figura o SRM está inserido no contexto da ADM para preencher a definição de um metamodelo de refatorações. 

Um dos objetivos do SRM é seguir as padronizações propostas pela ADM. Porém, deve-se ressaltar que o SRM é utilizado para especificar a representação de refatorações sem se preocupar com a representação das partes estruturais de uma refatoração, ou seja, os elementos que serão refatorados (classes, métodos, atributos, etc.). Assim, o SRM assume que tais elementos (classes, métodos, atributos, etc.) devem ser representados utilizando outro metamodelo proposto pela ADM, neste contexto o KDM. Como visualizado na Figura~\ref{fig:refactoring_metamodel}, o SRM interage com o metamodelo KDM para utilizar instâncias de metaclasses do KDM para representar os elementos que serão refatorados. Consistente com outros metamodelos definidos pela OMG, o SRM é definido utilizando a padronização de modelagem MOF e a \textit{framework} EMF. Um dos benefícios de utilizar MOF é que o mesmo permite que o metamodelo seja serializado e deserializado sem perder nenhum tipo de informação, ou seja, metamodelos instanciados são representados utilizando uma representação textual padronizada, XMI. Além disso, o SRM é compatível com repositórios MOF para armazenamento e recuperação de várias ferramentas, aumentando assim a interoperabilidade de futuras ferramentas que utilizem esse metamodelo.

\begin{figure}[h]
	\centering
	% Requires \usepackage{graphicx}
	\caption{Integração do SRM com outros metamodelos da ADM.}
	\label{fig:refactoring_metamodel}
	\includegraphics[scale=0.65]{images/SRM2Formatted}
	\fautor
\end{figure}


A Figura~\ref{fig:fases_para_a_construcao_e_uso_do_SRM} apresenta uma macro-visão da engenharia do metamodelo SRM utilizando a notação \sigla{SADT}{\textit{Structured Analysis and Design Technique}}~\cite{Marca_1987}. Como observado um conjunto de elementos foram utilizados como base para o desenvolvimento do SRM: elementos estruturais, linguagens de transformações, linguagens de restrições e catálogos de refatorações. A seguir, na Subseção~\ref{Engenharia_do_Meta_modelo_SRM} esses elementos são apresentados e comentados como foram utilizados para desenvolver o metamodelo SRM.

\begin{figure}[h]
	\centering
	% Requires \usepackage{graphicx}
	\caption{Fases para a construção e uso do SRM.}
	\label{fig:fases_para_a_construcao_e_uso_do_SRM}
	\includegraphics[scale=0.9]{images/SRM_Construcao2}
	\fautor
\end{figure}

\subsection{Engenharia do SRM}\label{Engenharia_do_Meta_modelo_SRM}

Refatorações, suas características e terminologia foram analisadas para que o SRM fosse desenvolvido. Antes de iniciar a criação do metamodelo SRM (ou de qualquer outro metamodelo) é necessário definir os elementos que deverão ser representados e as relações entre eles. Nesse sentido, cinco etapas que compõem a engenharia do SRM são mostradas na Figura~\ref{fig:etapas_da_fase_de_e_do_SRM}: (\textit{i}) Identificação dos Elementos Estruturais, (\textit{ii}) Identificação das Linguagens de Transformações, (\textit{iii}) Identificação das Linguagens de Restrições, (\textit{iv}) Identificação de vocabulário/termos/conceitos e (\textit{v}) Projeto e criação do metamodelo SRM. 

\begin{figure}[h]
	\centering
	% Requires \usepackage{graphicx}
	\caption{Etapas da fase de Engenharia do SRM.}
	\label{fig:etapas_da_fase_de_e_do_SRM}
	\includegraphics[scale=0.75]{images/todasAsFasesDaEngenhariaDOSRM6}
	\fautor
\end{figure}



\subsection{Identificação dos Elementos Estruturais}

Esta etapa possui como objetivo analisar e identificar possíveis elementos estruturais para serem utilizados durante a refatoração. No Capítulo~\ref{chapter:catalogo_refactoring_KDM} Seção~\ref{sec:mapeamento_POO_e_KDM}, a Tabela~\ref{tab:mapemanetoEntreOOPeKDM} apresenta uma relação existente entre os conceitos do POO, bem como algumas instruções de linguagens de programção e o metamodelo KDM. Para fazer com o que o metamodelo SRM seja independente de linguagem e plataforma foi utilizado os elementos (metaclasses) do KDM para representar os elementos estruturais. Como já salientado no Capítulo~\ref{chapter:catalogo_refactoring_KDM}, o metamodelo KDM contêm algumas metaclasses que podem ser diretamente mapeadas a elementos estruturais, tal como, classes (\texttt{ClassUnit}), interfaces (\texttt{InterfaceUnit}), atributos (\texttt{StorableUnit}), métodos (\texttt{MethodUnit}), etc. Essas metaclasses possuem o mesmo objetivo e características dentro do contexto do POO e linguagens de programação. %Entretanto, como o KDM tem como objetivo ser um modelo independente de plataforma, ou seja, tem como intuito representar de forma genérica todas as abstrações e paradigmas de programação, algumas construções de programação não possuem uma metaclasse particular. Por exemplo, iterações e ramificações em KDM são representadas utilizando a mesma metaclasse, \texttt{ActionElement}.

Elementos estruturais no contexto deste capítulo representam as instâncias das metaclasses que serão refatorados, por exemplo, \texttt{ClassUnit}, \texttt{MethodUnit}, \texttt{StorableUnit}, etc. Tais metaclasses foram consideradas boas candidatas para facilitar o reuso e troca de metadados no metamodelo SRM. Por exemplo, considere a refatoração \textit{RenameX}, onde \textit{X} pode ser qualquer elemento estrutural (\texttt{ClassUnit}, \texttt{MethodUnit}, \texttt{StorableUnit}, etc.). Os passos e mecanismo necessários para realizar essa refatoração são iguais ou possuem poucas diferenças não importa o tipo de elemento estrutural que será refatorado. Tendo como base esse exemplo, é possível identificar a necessidade de representar os metadados dos elementos estruturais que serão refatorados no SRM. Como o KDM (ver Capítulo~\ref{chapter:adm_kdm}) foi desenvolvido com o intuito de representar todo o sistema e seus artefatos – o mesmo é um bom candidato para representar os elementos estruturais de uma refatoração. Portanto, nesta etapa as metaclasses do metamodelo KDM foram consideradas boas candidatas para serem os elementos estruturais das refatorações. Dessa forma, como apresentado na Figura~\ref{fig:refactoring_metamodel} ambos metamodelos, SRM e KDM, são utilizados juntamente para a definir refatorações em nível de metadados.

\subsection{Identificação das Linguagens de Transformações}

Nesta etapa são identificadas e analisados os detalhas de um conjunto de linguagens de transformações com o intuito de identificar as informações necessários para promover o reuso do mecanismo das refatorações. 

Conceitualmente, refatorações são definidas por meio de um conjunto de passos que devem ser seguidos para realizar uma determina mudança~\cite{Fowler1999, Demeyer1}. Por outro lado, programaticamente as refatorações são definidos como \aspas{programas} parametrizados que executam um conjunto de transformações seguindo uma ordem lógica. Usualmente, no contexto de modelos, tais transformações são conhecidas como endógenas e são implementadas utilizando técnicas de reescrita de grafo, ou como também é conhecida transformação de grafo (ver Capítulo~\ref{chapter:fundamentacao_teorica} Seção~\ref{sec:transformacoes_de_modelos}). Dessa forma, pode-se caracterizar que técnica de reescrita de grafo é útil para auxiliar na elaboração de transformações de forma independente para \textit{n} metamodelos, ou seja, técnica de reescrita de grafo pode ser aplicada em qualquer metamodelo que implemente o padrão MOF, como por exemplo o metamodelo KDM. 

Comumente, transformações em modelos são definidas utilizando linguagens específicas de transformações de modelos. Diversas linguagens de transformação de modelos têm sido propostas atualmente~\cite{Biehl_2010, Allilaire_06}, entre elas pode-se citar ATL e QVT. Por exemplo, no Código-fonte~\ref{codigo:rename_classUnit_SRM} é apresentado um trecho da refatoração \texttt{Rename ClassUnit} escrito em ATL. Por meio da linguagem de programação ATL facilmente pode-se especificar qual metamodelo esta sendo utilizado na refatoração. Além disso, com o uso da ATL e QVT pode-se automatizar os mecanismos e todos os passos que uma determinada refatoração deve realizar. Por exemplo, no Código-fonte~\ref{codigo:rename_classUnit_SRM} uma determinada instância de \texttt{ClassUnit} que contêm o meta-atributo \texttt{name} igual à \aspas{Fusca} será refatorada para \aspas{Ferrari}. Dessa forma, com o propósito de automatizar os mecanismo e todos os passos de uma refatoração no contexto do metamodelo SRM, tanto ATL e QVT foram consideradas boas candidatas para especificar programaticamente os metadados sobre o mecanismo das refatorações. %ATL e QVT foram escolhidas como linguagens de transformação nesta capítulo considerando vários aspectos. Tais linguagens estão integradas na plataforma Eclipse, o que fornece uma série de recursos padrões para o desenvolvimento (\textit{syntax highlighting} e \textit{debugger}). Ambas linguagens são parte do projeto \textit{Model-To-Model} e possuem um grupo de discussão ativo, constantemente atualizado, vários exemplos e diversos estudos de casos aplicados até mesmo na indústria utilizam tais linguagens.


\begin{codigo}[caption={[Refatoração \textit{Rename ClassUnit}.] Refatoração \textit{Rename ClassUnit}.},escapeinside={(*@}{@*)}, basicstyle=\footnotesize, label={codigo:rename_classUnit_SRM}, language=ATL]{Name}
module renameClassUnit;
create OUT : MM refining IN : MM;
rule renameClassUnit {
	from
		source : MM!ClassUnit (source.name = (*@\aspas{Fusca}@*))
	to 
		target : MM!ClassUnit (
			name (*@$\leftarrow$@*) (*@\aspas{Ferrari}@*)
		)
}
\end{codigo}



%Linguagem de transformação representam as operações/mecanismos de um refatoração. Tais operações/mecanismos são responsáveis por realizar a refatoração propriamente dita no metamodelo. Usualmente  tais transformações são escritas em linguagens imperativas como \textit{Query}/\textit{View}/-\textit{Transformation} (QVT) ou ATL \textit{Transformation Language}.

\subsection{Identificação das Linguagens de Restrições}
Nesta etapa as formas de especificar as restrições são identificadas para definir as pré- e pós-condições de uma determinada refatoração. É importante que o metamodelo SRM permita definir tais restrições e não apenas o mecanismo da refatoração, assim, engenheiros podem utilizar uma determinada instância do metamodelo SRM e verificar quais são as restrições que eles devem respeitar para executar a refatoração de forma correta. Tais restrições no contexto de refatorações são conhecidas como pré- e pós-condições\cite{OPDYKE_1992, Roberts_1999}.

Usualmente antes e após executar uma determinada refatoração algumas restrições precisam ser satisfeitas. Tais restrições usualmente são úteis para verificar se os parâmetros necessários para executar a refatoração foram completamente e corretamente informados, bem como verificar se a refatoração foi aplicada de forma totalmente correta. Além disso, restrições são úteis para garantir que a sintaxe e semântica do modelo sejam preservadas após a aplicação das refatorações, assegurando assim uma possível preservação de comportamento. No contexto de modelos, tais restrições podem ser especificadas utilizando linguagens como a padronização definida pelo OMG denominada OCL e XQuery. Por meio dessas linguagens, é possível verificar, por exemplo, se todos os parâmetros obrigatórios para executar a refatoração foram especificados pelo engenheiro de modernização. Assegurar que a refatoração será aplicada de forma correta é de suma importância para preservar a sintaxe e semântica da instância do metamodelo, como por exemplo, preservar comportamentos, sincronização, etc. 

Com o propósito de automatizar as pré- e pós-condições de uma refatoração no contexto do metamodelo SRM, tanto OCL e XQuery foram considerados boas candidatas para especificar programaticamente as restrições que devem ser satisfeitas e respeitadas para a execução da refatoração.

%Condições prévias de refatoração são propriedades do programa original que deve segurar por um refactoring ser de preservação comportamento.


%Linguagem de restrição no contexto desse capítulo representa pré- e pós-condição de uma refatoração. Usualmente no contexto de modelos, restrições podem ser escritas utilizando a \textit{Object Constraint Language} (OCL).

\subsection{Identificação de vocabulário/termos/conceitos}

Nesta etapa é realizada a identificação de vocabulário, termos e conceitos comuns que são utilizadas dentro da comunidade de refatoração. Durante a criação de metamodelos é de suma importância entender o domínio que o metamodelo representa. metamodelos definem abstrações (termos), notações e relacionamentos para representar um determinado domínio. Assim, nesta etapa tanto os vocabulários, termos e conceitos definidos por~\citeonline{OPDYKE_1992} e \citeonline{Fowler1999} foram analisados para a identificação de abstrações para facilitar a criação do metamodelo SRM. Durante a análise pode-se observar e identificar alguns termos comumente utilizados durante a definição de uma refatoração. Por exemplo, todas as refatorações descritas e definidas por~\citeonline{OPDYKE_1992} e \citeonline{Fowler1999} seguem os seguintes termos: 

\begin{itemize}
\item Refatoração: o nome da refatoração
\item Autor: autor da refatoração;
\item Catálogo: o catalogo no qual a refatoração pertence;
\item Biblioteca de refatoração: onde um conjunto de catálogos podem ser incluídos;
\item Descrição: informando uma típica situação onde a refatoração deveria ser aplicada;
\item Motivação: informando a motivação para a realização da refatoração;
\item Operação: descrevendo os passos que devem ser realizados para executar a refatoração;
\item Parâmetros: informações necessários para executar a operação da refatoração;
\item Restrições: Asserções utilizadas para garantir a semântica e sintaxe após a aplicação da refatoração;
\begin{itemize}
\item Pré-condição: asserção que deve ser verdadeira antes de executar a operação da refatoração;
\item Pós-condição: asserção que deve ser verdadeira após executar a operação da refatoração;
\end{itemize}
\end{itemize}

Utilizando os termos identificados nos trabalhos de~\citeonline{OPDYKE_1992} e ~\citeonline{Fowler1999} foi possível criar o metamodelo SRM. 

\subsection{Projeto e criação do Metamodelo SRM}

Nesta etapa o SRM é especificado a partir das informações extraídas nas etapas anteriores. Utilizando as informações extraídas foi possível identificar terminologias, bem como palavras-chaves que geralmente estão relacionadas com refatoração. Utilizando tais terminologias e palavras-chaves foi possível realizar a especificação do SRM. O SRM pode ser definido como uma quadrupla, como observado na Definição~\ref{def:SRM}: 


\begin{definicao}\label{def:SRM}
    \textit{O SRM é uma quadrupla $SRM = (SRM_{mC}, SRM_{mA}, SRM_{e}, SRM_{mR})$ onde $SRM_{mC} $ representa um conjunto de metaclasses, $SRM_{mA}$ representa um conjunto de meta-atributos, $SRM_{e}$ representa um conjunto de enumerações e $SRM_{mR}$ representa associações.}
\end{definicao}

Formalmente pode-se definir o metamodelo como:

\begin{itemize}
	\item Todas as metaclasses \textit{mC} $\in SRM_{mC}$ tem um nome que representa o seu significado;
	\item Todos os meta-atributos \textit{mA} $\in SRM_{mA}$ contêm um nome, um tipo e uma cardinalidade. Além disso, cada mA está associado a uma metaclasse;
	\item Todas as enumerações \textit{e} $\in SRM_{mA}$ contêm um nome e um valor;
	\item Todas as meta-associações \textit{mR} $\in SRM_{mR}$ é um conjunto R = $E_{1}$, $E_{2}$  , onde  $E_{1}$ e $E_{2}$ são associações de R. Posteriormente, cada R, contêm um nome. Ambos $E_{1}$  e $E_{1}$ possuem uma cardinalidade e são associados a uma metaclasse de $SRM_{mC}$.
\end{itemize}

Na Figura~\ref{fig:meta_modelo_SRM} é apresentado o metamodelo SRM. O SRM contêm 12 metaclasses e três enumerações. Um descrição detalhada de cada elementos desse metamodelo é apresentado a seguir:

\begin{figure}[h]
	\centering
	% Requires \usepackage{graphicx}
		\caption{Metamodelo SRM.}
	\includegraphics[scale=0.65]{images/refactoring_metamodel}
	\label{fig:meta_modelo_SRM}
	\fautor
\end{figure}

\begin{itemize}
\item \texttt{RefactoringModel} representa a metaclasse raiz do metamodelo.

\begin{itemize}
	\item \textbf{Associações}
		\begin{itemize}
			\item \texttt{author:Author[0..1]}: representa o autor de uma refatoração; 
			\item \texttt{libraries:RefactoringLibrary[0..*]}: representa um conjunto de biblioteca de refatorações que uma instância da metaclasse \texttt{RefactoringModel} possui..
		\end{itemize}
\end{itemize}

\item \texttt{Author} representa o autor de uma refatoração. Essa metaclasse contêm dois meta-atributos.

\begin{itemize}
	\item \textbf{Meta-atributos}
		\begin{itemize}
			\item \texttt{name}: utilizado para definir o nome do autor;
			\item \texttt{lastName}: utilizado para definir o sobrenome do autor.
		\end{itemize}	
\end{itemize} 

\item \texttt{RefactoringLibrary} utilizado para descrever uma biblioteca de refatorações.

\begin{itemize}
	\item \textbf{Attributes}
		\begin{itemize}
			\item \texttt{name}: utilizado para descrever o nome da biblioteca de refatoração;
			\item \texttt{shortDescription}: representa uma breve descrição sobre a biblioteca de refatoração;
			\item \texttt{description}: representa uma completa descrição sobre a biblioteca de refatoração.
		\end{itemize}	
\end{itemize} 

\begin{itemize}
	\item \textbf{Associação}
		\begin{itemize}
			\item \texttt{catalogs:Catalog[0..*]}: um conjunto de catálogos que contem refatorações.
		\end{itemize}	
\end{itemize} 

\item \texttt{Catalog} metaclasse utilizada para representar um catalogo de refatorações.

\begin{itemize}
	\item \textbf{Meta-atributos}
		\begin{itemize}
			\item \texttt{name}: representa o nome do catálogo. 
		\end{itemize}	
\end{itemize} 

\begin{itemize}
	\item \textbf{Associações}
		\begin{itemize}
			\item \texttt{author:Author[0..1]}: representa o autor do catalogo;
			\item \texttt{refactorings:Refactoring[0..*]}: conjunto de todas as refatorações que um catalogo contêm.
		\end{itemize}	
\end{itemize} 

\item \texttt{Refactoring} representa uma das principais metaclasses do SRM.

\begin{itemize}
	\item \textbf{Meta-atributos}
		\begin{itemize}
			\item \texttt{name}: utilizado para identificar a refatoração e ajuda a construir um vocabulário comum para os desenvolvedores de software;
			\item \texttt{motivation}: descreve o motivo pelo qual a refatoração deve ser realizada – lista também as circunstâncias na qual a refatoração deve ser utilizada;
			\item \texttt{summary}: informa quando e onde uma determinada refatoração deve ser utilizada. Também é útil para auxiliar o engenheiro de software a identificar uma refatoração relevante em uma determinada situação. 
		\end{itemize}	
\end{itemize} 

\begin{itemize}
	\item \textbf{Associações}
		\begin{itemize}
			\item \texttt{operation:Operation[1]}: deve a ação que será executa, representa o mecanismo da refatoração;
			\item \texttt{preCondition:PreCondition[1]}: representa uma pré-condição que deve ser satisfeita antes da execução da operação/refatoração;
			\item \texttt{postCondition:PostCondition[1]}: representa uma pós-condição que tem como intuito verificar a corretude da refatoração;
			\item \texttt{parameters:Parameter[0..*]}: um conjunto de parâmetros que são utilizados para realizar a refatoração. Tais parâmetros podem ser metaclasses do KDM;
			\item \texttt{chainOfRefactoring:Refactoring[0..*]}: um conjunto de refatorações que quando combinados podem realizar refatorações complexas, ou seja, \textit{macro-grained refactoring};
			\item \texttt{classification:Classification[0..*]}: define a classificação de uma refatoração.
		\end{itemize}	
\end{itemize} 

\item \texttt{Operation} também representa uma das principais metaclasses do SRM. Essa metaclasse contêm metadados do código responsável por realizar a transformação/refatoração.

\begin{itemize}
	\item \textbf{Meta-atributos}
		\begin{itemize}
			\item \texttt{language}: especifica a linguagem que será escrito o código responsável por realizar a transformação/refatoração. Valores válidos são: ``ATL'' e ``QVT'';
			\item \texttt{body}: especifica a transformação/refatoração com base na linguagem selecionada, ``ATL'' ou ``QVT''.
		\end{itemize}	
\end{itemize} 

\item \texttt{PreCondition} define uma pré-condição para ser executada antes de operação/refatoração.

\begin{itemize}
	\item \textbf{Meta-atributos}
		\begin{itemize}
			\item \texttt{context}: especifica o classificador para qual a pré-condição será definido;
			\item \texttt{language}: especifica a linguagem que será escrito a pré-condição. Valor válido é: ``OCL'' ou ``XQuery'';
			\item \texttt{body}: especifica a OCL ou ``XQuery'' que representa a pré-condição.
		\end{itemize}	
\end{itemize} 

\item \texttt{PostCondition} define uma pós-condição para ser executada após a operação/refatoração.

\begin{itemize}
	\item \textbf{Meta-atributos}
		\begin{itemize}
			\item \texttt{context}: especifica o classificador para qual a pós-condição será definido;
			\item \texttt{language}: especifica a linguagem que será escrito a pós-condição. Valor válido é: ``OCL'' ou ``XQuery'';
			\item \texttt{body}: especifica a OCL ou ``XQuery'' que representa a pós-condição.
		\end{itemize}	
\end{itemize} 

\item \texttt{Parameter} define um conjunto de parâmetros necessários para executar a refatoração. Essa metaclasse utiliza uma estrutura similar a tabela \textit{hash} para definir os parâmetros.

\begin{itemize}
	\item \textbf{Meta-atributos}
		\begin{itemize}
			\item \texttt{key}: representa o nome do parâmetro;
			\item \texttt{value}: representa o tipo do parâmetro. Esse tipo deve ser tipos primitivos (\textit{int}, \textit{string}, \textit{double}, \textit{float}, etc.) ou metaclasses do metamodelo KDM.
		\end{itemize}	
\end{itemize} 

\item \texttt{Classification} define a classificação da refatoração.

\begin{itemize}
	\item \textbf{Associações}
		\begin{itemize}
			\item \texttt{level}: representa se a refatoração é fine ou macro grained refactoring;;
			\item \texttt{kdmPack}: define qual pacote do KDM é necessário para executar a refatoração..
		\end{itemize}	
\end{itemize} 

\item \texttt{Level} utilizado para definir se a refatoração é de granularidade baixa ou alta..

\begin{itemize}
	\item \textbf{Meta-atributos}
		\begin{itemize}
			\item \texttt{kind}: especifica o level da refatoração..
		\end{itemize}	
\end{itemize} 

\item \texttt{KDMPackage} define qual pacote do KDM é necessário para executar a refatoração.

\begin{itemize}
	\item \textbf{Meta-atributos}
		\begin{itemize}
			\item \texttt{kind}: representa qual pacote do KDM é necessário para executar a refatoração.
		\end{itemize}	
\end{itemize} 

\end{itemize}

Três enumerações também foram definidas, como pode ser observado na Figura~\ref{fig:meta_modelo_SRM} delimitada por um retângulo de linha pontilhada. A primeira enumeração é \texttt{Language}, que é utilizada para especificar a linguagem da operação/refatoração, valores válidos são: ``ATL'' e ``OCL''. A segunda enumeração é \texttt{LevelKind}, a qual é utilizada para definir o level da refatoração. Finalmente, \texttt{KDMPackageKind} é utilizado para definir qual pacote do metamodelo KDM é (são) utilizado(s) durante a execução da refatoração. 


\section{Considerações Finais}
\label{sec:consideracoes_finais}

Refatorações são de suma importância durante a manutenção, produção e análise de software. Inúmeras comunidades têm surgido na literatura para criar e definir vários tipos de refatorações, incluindo refatorações de baixa granularidade~\cite{Fowler1999, Demeyer1, Demeyer2}, refatorações arquiteturais, refatorações para o paradigma orientado a aspecto, etc. 

Neste contexto, há uma grande necessidade da definição de um padrão para auxiliar e promover o compartilhamento dessas refatorações, tanto dentro como entre estas comunidades. Como uma iniciativa para suprir tal limitação neste capítulo foi apresentado o metamodelo SRM para auxiliar o engenheiro a promover o reuso de refatorações. O SRM está inserido no contexto da ADM para preencher a definição de um metamodelo de refatorações. Assim, com a utilização desse metamodelo, metadados sobre refatorações podem ser reutilizadas de forma independente de linguagem e plataforma.

O SRM contêm 12 metaclasses e três enumerações e foi desenvolvimento utilizando o EMF. Para a representação dos elementos estruturais que são utilizados durante uma refatoração o metamodelo KDM foi utilizada. O mecanismo/operação da refatoração são representados utilizando a linguagem de transformação ATL. As restrições que devem ser satisfeitas antes e após a condução da refatoração são representadas por meio de OCL. A fim de utilizar plenamente as vantagens dos metamodelo SRM, os engenheiros de modernização precisam ter um bom conhecimento de linguagem de programação avançada. Na verdade os engenheiros devem estar familiarizados como as semânticas das refatorações (por exemplo, qual(is) é (são) o(s) pré-requisito(s) para a execução de uma refatoração) e como/onde utilizar e programar tais refatorações. A instanciação de uma refatoração utilizando o SRM é bastante verbosa, complexa e propensa a erros, pois exige conhecimento avançadas de refatoração e habilidades de programação em relação a API Ecore, uma vez que o SRM foi desenvolvido utilizando o EMF. Para facilitar a utilização do metamodelo SRM no Capítulo~\ref{chapter:ferramenta_kdm_re} é apresentado a ferramenta denominada KDM-RE que contêm um módulo que defini uma DSL para facilitar a instanciação e reutilização de do metamodelo SRM. No Capítulo~\ref{chapter:avaliacao} é discuto um experimento que foi realizado para avaliar as refatorações e a ferramenta KDM-RE.

No próximo capítulo é apresentado uma abordagem denominada KDM-SInc para realizar a propagação de mudança e preservação de comportamento após a aplicação de refatorações em instâncias do metamodelo KDM. 




\chapter{Uma Abordagem para Manter o KDM Sincronizado Após a Aplicação de Refatorações}\label{chapter:Abordagem_de_sincronizacao}
\section{Considerações Iniciais}

Usualmente, durante o desenvolvimento e modernização de software seguindo as diretrizes e passos da abordagem MDE, o software geralmente é modelado e representado utilizando diferentes instâncias de metamodelos para representar as visões e todos os artefatos de um sistema. Em outras palavras, geralmente existem metamodelos para abstrair e representar todos os artefatos do sistema, tais como: metamodelos para o código-fonte, metamodelos para representar e abstrair o banco de dados, metamodelos para representar e abstrair a arquitetura do sistema, etc. Como apresentado no Capítulo~\ref{chapter:adm_kdm} o metamodelo KDM é capaz de agrupar todos esses artefatos em um único metamodelo, sendo assim, possível representar diferentes visões/artefatos e seus relacionamentos de um determinado sistema em uma única instância do metamodelo KDM. Porém, conforme o engenheiro aplica um conjunto de refatorações durante o processo deem uma determinada visão do metamodelo KDM mudanças são realizadas. Além disso, tais mudanças podem necessitar que subsequentes mudanças sejam realizadas para que outras visões/artefatos do metamodelo KDM fiquem consistentes e sincronizados.

Uma premissa fundamental é manter todas as visões/artefatos do metamodelo KDM sincronizados durante todo o processo de modernização do software. Dessa forma, quando as visões/artefatos representados em nível de modelos são alterados, é de extrema importância realizar um conjunto de propagação de mudança por todas as visões/artefatos para mantê-los atualizados e sincronizados, espelhando assim, a alteração em todas as visões/artefatos do software. Usualmente, como apresentado nos Capítulo~\ref{chapter:fundamentacao_teorica}, Seção~\ref{sec:refatoracao} e Capítulo~\ref{chapter:catalogo_refactoring_KDM} essas alterações podem ser realizadas por meio de refatorações, as quais são atividades centrais durante o processo de modernização (ref)\change{aqui tbm}. Porém, quando um software é representado utilizando diferentes instâncias de metamodelos, um acidente comum que pode ocorrer durante a atividade de refatoração é a dessincronização dessas instâncias, fazendo com que as visões/artefatos que representam o sistema fiquem inconsistente após a atividade de refatoração. Uma forma de resolver esse problema é aplicar técnicas de propagação de mudança, cujo objetivo é identificar e atualizar todas as instâncias dependentes dos elementos que foram refatorados. No entanto, a maioria das propostas de propagação de mudança foram desenvolvidas para propagarem mudanças em diferentes metamodelos, além disso, usualmente tais metamodelos são de diferentes fornecedores dificultando o entendimento e a programação de mudança (ref). 
Diante deste contexto, neste capítulo é apresentado uma abordagem para realizar a propagação de mudança e preservação de comportamento após a aplicação de refatorações no metamodelo KDM. Utilizando a abordagem aqui definida, modernizadores podem se concentrar apenas no desenvolvimento das refatorações ou reutiliza-las por meio do metamodelo SRM (ver Capítulo~\ref{chapter:Toward_a_Refactoring_Metamodel_for_KDM}), sem terem que se preocuparem com a propagação de mudanças para outras visões/artefatos do metamodelo KDM. 

É importante destacar que o fluxo da abordagem inicia-se considerando que o modernizador almeja aplicar um conjunto de refatorações em um sistema que está já representado por meio de uma instância do metamodelo KDM. Essa instância do metamodelo KDM deve estar o mais completo possível, ou seja, represente todas as visões/artefatos do sistema, desde o código-fonte até os elementos arquiteturais do sistema. Após o modernizador aplicar uma determinada refatoração, a abordagem, a qual contêm três principais passos, efetivamente é iniciada. De forma resumida pode-se descrever os três passos da abordagem da seguinte forma. 

O primeiro passo da abordagem realiza uma comparação (do inglês - \textit{diff}) entre a instância refatorada do metamodelo KDM com a instância do metamodelo KDM original, ou seja, a instância do metamodelo KDM antes do modernizador aplicar a refatoração. Como resultado, esse passo cria uma lista que contêm todas as instâncias das metaclasses do KDM que sofreram uma modificação durante a refatoração quando comparado com a instância do KDM original. Em seguida, o segundo passo utiliza como entrada a lista gerada para ser utilizada como parâmetro para um algoritmo de mineraçã o e identificação de dependências. Esse algoritmo tem como objetivo identificar todas as instâncias das metaclasses do KDM que possuem dependência com as metaclasses refatoradas. Como resultado, esse algoritmo também cria uma lista, a qual é utilizada no terceiro passo. O terceiro passo utiliza a lista criada pelo algoritmo para realizar um conjunto de transformações em nível de modelo. Tais transformações foram pré-definidas e representam a propagação de mudança por todas as visões do KDM. É importante destacar que a abordagem foi implementada com a preocupação de ser uma forma genérica e desacoplada. Assim, a abordagem pode ser aplicada em um grande conjunto de refatorações fazendo com que o modernizador não tenha que se preocupar com a propagação de mudança para outras visões/artefatos do KDM. 

As demais seções deste capítulo estão organizadas da seguinte forma:\change{terminar aqui. Deve colocar todas as seções bem escritas.}


\section{A Abordagem KDM-SInc}\label{sec:kdm_sinc}

Um problema critico durante a modernização de software diz respeito a propagação de mudança - por exemplo,  dado um conjunto de refatorações que são aplicadas durante a modernização de software é importante identificar quais são as mudanças que precisam ser realizadas para manter a consistência e sincronia de todos os artefatos do sistema. Dessa forma, propagação de mudança é uma técnica de extrema importância durante a elaboração de processo de modernização de software. O engenheiro de modernização têm que ter a certeza que a refatoração foi corretamente propagada e que o software não contêm nenhuma inconsistência. Embora muitas abordagens de propagações de mudanças possam ser identificadas na literatura, a propagação de mudanças ainda é um desafio técnica significativo durante a manutenção e modernização de software~\cite{Tom_2008_roadmap}. Além disso, a maioria das abordagens de propagação de mudanças existentes têm como principal artefato o código-fonte~\cite{Vaclav_methodology, Deursen07model_drivensoftware}. Similarmente também é possível identificar algumas abordagens que dão suporte para a propagação de mudança para a UML~\cite{Egyed_2008,Liu02rule, Briand_2006}. Porém, até o momento nenhuma abordagem ou iniciativa foi criada para o metamodelo KDM. Para suprir essa limitação nessa seção a abordagem denominada KDM-SInc é apresentada. Essa abordagem têm como objetivo propagar mudanças por todas as visões/artefatos do metamodelo KDM para mantê-lo atualizado e sincronizado após a aplicação de uma refatoração. A intenção é criar uma abordagem que mantenha uma determinada instância do metamodelo KDM consistênte e sincronizada entre todas as visões/artefatos do metamodelo KDM após a aplicação de uma determinada refatoração. 

Na Figura~\ref{fig:kdm_sinc} é apresentado uma visão geral da abordagem KDM-SInc. Como pode ser observado a abordagem KDM-SInc contêm três principais passos, os quais estão contidos em um módulo de propagação (caixa cinza). Antes de iniciar o módulo de propagação uma atividade de refatoração deve ser realizada como apresentado na Figura~\ref{fig:kdm_sinc} lado esquerdo caixa branca. A atividade de refatoração esta fora do escopo da abordagem KDM-SInc, assim, é de responsabilidade do engenheiro de modernização criar e/ou reutilizar refatoração para o metamodelo KDM e aplica-la em uma instância do metamodelo KDM. A única restrição da abordagem KDM-SInc é que duas versões da instância do metamodelo KDM seja utilizada como entrada para a abordagem - uma versão que representa a instância do metamodelo KDM antes da aplicação das refatorações (\aspas{instância original}) e outra versão que representa uma instância do metamodelo KDM após a aplicação de \textit{n} refatorações (\aspas{instância refatorada}). No contexto desse capítulo é importante entender que uma \aspas{instância original} do metamodelo KDM, aquela que ainda não foi refatorada, também é demonidade de \aspas{KDM esquerdo}, similarmente, a \aspas{instância refatorado} do metamodelo KDM pode ser entendida como \aspas{KDM direito}. 

\begin{figure}[h]
	\centering
	% Requires \usepackage{graphicx}
	\caption{Visão Geral da Abordagem KDM-SInc.}
	\label{fig:kdm_sinc}
	\includegraphics[scale=0.5]{images/ApproachLifeCicle2}
	\fautor
\end{figure}

Após a aplicação de um conjunto de refatorações o passo [A] pode ser iniciado. Nesse passo uma comparação (\textit{diff}) entre a instância original e a instância refatorada é realizada. Como resultado desse passo uma lista é crida. Essa lista contêm todas as instâncias das metaclasses do KDM que sofreram alguma modificação durante a refatoração quando comparado com a instância do KDM original. Além disso, essa lista também especifica qual(is) foi(ram) a(s) modificação(ões) realizada(s). Por exemplo, se na versão refatorada (\aspas{KDM direito}) uma nova instância da metaclasse \texttt{ClassUnit} foi adicionada a lista irá conter duas importantes informações: (\textit{i}) a instância da metaclasse \texttt{ClassUnit} e (\textit{ii}) qual operação foi realizada, nesse exemplo \texttt{add} \texttt{ClassUnit}. Essas duas informações são importantes para identificar o que foi alterado (neste caso uma \texttt{ClassUnit} foi adicionada) e qual operação foi realizada - assim, é possível identificar quais propagações devem ser realizadas nas outras visões do KDM.

Em seguida o passo [B] é iniciado, o qual identifica todas as metaclasses que precisam ser sincronizadas/atualizadas após a aplicação da refatoração. Nesse passo utiliza o algoritmo de busca em profundidade (do inglês - \sigla{DFS}{\textit{Depth-First Search}}). Para a abordagem KDM-SInc algoritmo DFS foi alterado para utilizar como entrada a lista criada no passo [A]. Além disso, o algoritmo DFS também utiliza como entrada a instância refatorado do KDM (\aspas{KDM direito}). Utilizando a instância refatorado o algoritmo DFS identifica e cria uma lista contendo todas as metaclasses que possuem dependência com as metaclasses que efetivamente foram refatoradas.

Posteriormente o passo [C] pode ser iniciado para realizar a propagação das mudanças na instância do metamodelo KDM. Como entrada esse passo utiliza todas as metaclasses que possuem dependência com as metaclasses que foram refatoradas (lista criada no passo [B]). Após o término do passo [C] todas as visões/artefatos da instância do metamodelo KDM estão sincronizadas e consistentes.

É importante salientar que os três passos da abordagem KDM-SInc são executados várias vezes até que não haja mais elementos que precisem ser atualizados/sincronizados. Esse ciclo é necessário uma vez que uma determinada instância do metamodelo KDM pode ainda exigir propagações em outros artefatos/visões, por isso, cada ciclo da abordagem KDM-SInc se concentra apenas no próximo nível de propagação. A condição de parada da abordagem KDM-SInc é quando o algoritmo, definido no passo [B], retornar uma lista vazia, indicando que não há mais elementos que precisam ser modificados.

Para auxiliar a elaboração do passo [A] o \textit{framework} EMFCompare\footnote{https://www.eclipse.org/emf/compare/} foi estendido para comparar instâncias do metamodelo KDM. O passo [B] é tecnicamente apoiado por um motor de busca cuja parte central é o algoritmo DFS juntamente com um conjunto de expressões definidas em XPath que são executadas em uma instância do metamodelo KDM para obter todos os pacotes do KDM. Finalmente, o passo [C] é apoiado por um motor de propagação, o qual utiliza um conjunto de transformações pré-definidas em ATL para executar as propagações. Todas as propagações foram definidas com base nas operações atômicas (\texttt{add}, \texttt{delete} e \texttt{change}) apresentadas no Capítulo X \change{Mudar aqui}. Dessa forma, quando um conjunto de refatorações são executadas um conjunto de propagações bem definidas podem ser executadas no contexto do metamodelo KDM. Maiores detalhes sobre da passo da abordagem KDM-SInc são apresentados nas próximas seções.

\subsection{Identificando \textit{Diff} entre Instâncias do Metamodelo KDM}\label{sec:diff_entre_kdm}

Nessa seção o primeiro passo da abordagem KDM-SInc é apresentado. Como já salientado o primeiro passo da abordagem KDM-SInc utiliza o \textit{framework} EMFCompare. Esse \textit{framework} foi escolhido pois o mesmo pode ser facilmente adaptado e estendido. Especificadamente o passo [A] da abordagem KDM-SInc realizada três sub-passos: (\textit{i}) \textit{Matching}, (\textit{ii}) \textit{Diffing} e (\textit{iii}) Análise dos \textit{Diffs} como apresentado na Figura~\ref{fig:diff_emf_compare}. 

Com pode ser observado na Figura~\ref{fig:diff_emf_compare} o primeiro sub-passo, \textit{matching}, necessita de duas instâncias do metamodelo KDM - uma instância original (\aspas{KDM esquerdo}) denominada \textbf{versão 1} na Figura~\ref{fig:diff_emf_compare} e uma instância refatorada (\aspas{KDM direito}) \textbf{versão 2} na Figura~\ref{fig:diff_emf_compare}. Dado as duas instâncias do metamodelo KDM, os correspondentes elementos nas duas versões do metamodelo KDM são identificados. Os correspondentes elementos são identificados por meio de identificadores únicos tais como XMI IDs. Por exemplo, na Figura~\ref{fig:diff_emf_compare} pode-se observar que a instância da metaclasse \texttt{ClassUnit} Artista apresentada na \textbf{versão 1} corresponde a instância da metaclasse \texttt{ClassUnit} Artista na \textbf{versão 2}. Para as instâncias das metaclasses \texttt{ClassUnit} Ator e \texttt{Extends} na \textbf{versão 2}, no entanto, nenhum elemento correspondente foi identificado na \textbf{versão 1}. Note que para cada elemento correspondente identificado, ou não identificado, um elemento \textit{match} é criado que será utilizado no sub-passo seguinte.
\begin{figure}[h]
	\centering
	% Requires \usepackage{graphicx}
	\caption{Visão Geral da Execução do Primeiro Passo da Abordagem KDM-SInc.}
	\label{fig:diff_emf_compare}
	\includegraphics[scale=0.8]{images/matching_diffing_analise_3}
	\fautor
\end{figure}
No segundo sub-passo, \textit{Diffing}, todos os correspondentes elementos identificados são examinados para identificar diferenças em seus meta-atributos. Para cada diferença identificada um objeto \textit{diff} é criado, o qual descreve com precisão cada diferença entre os correspondentes elementos. Por exemplo, ainda considerando a Figura~\ref{fig:diff_emf_compare}, quando a instância da metaclasse \texttt{ClassUnit} Artista da \textbf{versão 1} e \textbf{versão 2} são examinadas é possível observar que o meta-atributo \texttt{isAbstract} possui o valor \textit{false} na \textbf{versão 1}, enquanto que na \textbf{versão 2} o mesmo meta-atributo o valor é \textit{true} - representa a operação \texttt{change}. Instâncias de metaclasses que não contêm elementos correspondentes em ambas as versões são consideradas adicionadas ou deletadas (\texttt{add} e \texttt{delete}) - a operação é identificada dependendo da direção, por exemplo, se uma instância de uma metaclasse apenas existe do lado direito (\aspas{KDM direito}) essa instância foi adicionada, por outro lado, se uma instância apenas existe do lado esquerdo (KDM esquerdo) essa instância foi deletada. Na Figura~\ref{fig:diff_emf_compare} é possível identificar que duas instâncias foram adicionadas - uma instância da metaclasse \texttt{ClassUnit} denominada Ator e uma instancia da metaclasse \texttt{Extends}.

Em seguida o terceiro sub-passo, Análise dos \textit{Diffs} é executado. Nesse sub-passo todos os objetos \textit{diffs} criados anteriormente são examinados para criar uma lista de dependência contendo as instâncias das metaclasses alteradas e quais operações foram realizadas. No exemplo apresentado na Figura~\ref{fig:diff_emf_compare} a lista criada contêm três dependências. A primeiro dependência informa que o meta-atributo \texttt{isAbstract} da metaclasse \texttt{ClassUnit} Artista da \textbf{versão 1} foi alterado (\texttt{change}) de \textit{false} para \textit{true} na \textbf{versão 2}. A segunda dependência ilustra que uma instância da metaclasse \texttt{ClassUnit} Ator foi adicionada (\texttt{add}) na \textbf{versão 2} e a terceira dependência representa que uma instância da metaclasse \texttt{Extends} foi adicionada na \textbf{versão 2}.


\subsection{Identificando Pontos para Executar a Propagação}

Nessa seção o segundo passo da abordagem KDM-SInc é apresentado. Esse passo resume-se basicamente na adaptação do algoritmo DFS para identificar todas as metaclasses que precisam ser sincronizadas/atualizadas após a aplicação da refatoração. Esse algoritmo utiliza como entrada a lista criada no passo anterior. Como as instâncias do metamodelo KDM são persistidas utilizando a padronização XMI o algoritmo precisar de uma forma para buscar as dependências nesse XMI. Assim, esse passo utiliza expressões em XPath que são executadas na instância do metamodelo KDM para obter todos os pacotes do KDM. Por exemplo, na Figura~\ref{fig:xpath_queries} é apresentado algumas expressões definidas em XPath que é utilizada antes da aplicação do algoritmo DFS. A primeira expressão retorna a metaclasse \texttt{Segment} que é o elemento inicial de qualquer instância do metamodelo KDM. As outras expressões ilustradas nas linhas 2-5 representam os outros pacotes do metamodelo KDM. Os elementos retornados nas expressões XPath são também utilizadas como entrada para o algoritmo DFS.

\begin{figure}[h]
	\centering
	% Requires \usepackage{graphicx}
	\caption{Expressões definidas em XPath para obter os pacotes do KDM.}
	\label{fig:xpath_queries}
	\includegraphics[scale=0.68]{images/queiresANDATLSBESNew}
	\fautor
\end{figure}

\begin{algoritmo}[h]
     \SetAlgoLined
     \KwIn{DFS (G, u) onde \textit{G} é uma instância do  KDM, \textit{u} é a metaclasse inicial obtida pela expressão XPath, ou seja, \texttt{Segment}}
     \KwOut{Uma coleção de metaclasses que precisam ser sincronizadas}
     \Begin{
     \ForEach{$outgoing$ edge e = (u, v) of u} {
	\If(\tcp*[f]{Garante que o maior número seja $n_1$}){vertex v as has not been visited}{
			\If{vertex v contain implementation = true }{
				
				\ForEach{$implementations$ element}{
				verify all elements in implementation
				}
				Mark vertex v as visited (via edge e).
				Recursively call DFS (G, v).
			}
			
				}				
			}		
	
	}
     \caption{Algoritmo DFS.}
     \label{alg:death1}
   \end{algoritmo}

\section{Considerações Finais}


\chapter{A Ferramenta KDM-RE}\label{chapter:ferramenta_kdm_re}
\section{Considerações Iniciais}

%Embora a refatoração dirigida a modelo tenha alcançado bastante reconhecimento e aceitação na literatura~\cite{Moghadam_2012,Maneerat_2011,Fourati_2011,Einarsson_2012,Steimann_2015,Akiyama_2011, Jensen_2010,Arendt_2012,Millan_2009,Tom_2008_2008}, ainda se fazem necessárias pesquisas nessa área~\cite{durelli_systematic_mapping, revisao_sistematica_uml_refactoring}. Apesar da ADM e principalmente do KDM terem sido propostos para auxiliar todo o processo da modernização de sistemas, até este momento existe uma carência de abordagens e ferramentas que auxiliem os engenheiros de software a aplicar refatorações de forma consistente para o metamodelo KDM. Neste contexto, no Capítulo~\ref{chapter:catalogo_refactoring_KDM} é descrito uma abordagem para auxiliar a criação de refatorações para o metamodelo KDM~\cite{durelli_catalogo}. %O intuito da criação de refatorações para o metamodelo KDM é facilitar a condução da modernização de um determinado sistema legado utilizando a abordagem ADM.


%Como já salientado nesta tese, o OMG e a ADM fornecem um conjunto de metamodelos para auxiliar o engenheiro de modernização durante as atividades de modernização. Porém, hoje em dia, a ADM não provê um metamodelo para auxiliar o engenheiro de modernização a promover o reúso de refatorações juntamente com os seus metamodelos padronizados durante o processo de modernização. Isso faz com que o engenheiro de modernização crie suas próprias soluções/refatorações. Com o intuito de suprir tal limitação, no Capítulo~\ref{chapter:Toward_a_Refactoring_Metamodel_for_KDM}, foi apresentado o metamodelo SRM para auxiliar o engenheiro de modernização a promover o reúso de refatorações no contexto do metamodelo KDM. Com a utilização do SRM, informações (metadados) sobre refatorações podem ser reutilizadas de forma independente de linguagem e plataforma.

%No Capítulo~\ref{chapter:Abordagem_de_sincronizacao} foi apresentado uma abordagem denominada KDM-SInc para manter instâncias do metamodelo KDM sincronizadas e consistentes após à aplicação de refatorações no KDM. Essa abordagem contém três principais passos: (\textit{i}) comparação (do ingles - \textit{diff}) entre a instância refatorada do metamodelo KDM e a instância do metamodelo KDM original, ou seja, a instância do metamodelo KDM antes do modernizador aplicar refatorações; (\textit{ii}) algoritmo de mineração para identificar todas as instâncias das metaclasses do KDM que possuem dependência com as metaclasses refatoradas e; (\textit{iii}) transformações pré-definidas em nível de modelo são aplicadas para propagar a mudança por todas as visões do KDM.

Este capítulo apresenta uma ferramenta denominada \sigla{KDM-RE}{\textit{Knowledge Discovery Model-Refactoring Environment}}. % para automatizar os conceitos apresentados nos Capítulos~\ref{chapter:catalogo_refactoring_KDM} e~\ref{chapter:Toward_a_Refactoring_Metamodel_for_KDM}. 
%
%Visando automatizar os conceitos apresentados nos Capítulos~\ref{chapter:catalogo_refactoring_KDM} e~\ref{chapter:Toward_a_Refactoring_Metamodel_for_KDM} no presente capítulo, é apresentado um apoio computacional denominado \sigla{KDM-RE}{\textit{Knowledge Discovery Model-Refactoring Environment}}. 
%
%
A ferramenta KDM-RE é um protótipo para auxiliar o processo de modernização e ela utiliza internamente os seguintes metamodelos: (\textit{i}) UML, (\textit{ii}) KDM e (\textit{iii}) SRM. Como pré-requisito para utilizar a ferramenta KDM-RE e iniciar o processo de modernização ilustrado na Figura~\ref{fig:abordagem_kdm_tese_processo}~\ding{204}, o engenheiro de software deve primeiramente converter o sistema a ser modernizado para uma instância do metamodelo KDM. E em seguida, a instância do KDM deve ser convertida para uma instância do metamodelo UML. Após realizar essas conversões à instância do metamodelo UML pode então ser visualizada por meio de diagrama de classes da UML, o qual é utilizado como interface gráfica durante o processo de modernização. Dessa forma, o sistema a ser modernizado pode ser visualizado graficamente facilitando a identificação de problemas estruturais. Durante o processo de modernização o engenheiro de software pode interagir com o diagrama de classes e escolhe um conjunto de refatorações a ser aplicadas. Assim que o engenheiro de software escolhe uma determinada refatoração, um \textit{Wizard} é executado, onde o engenheiro deve fornecer informações para a correta execução da refatoração. As informações providas nesse \textit{Wizard} são enviadas para uma linguagem de transformação que é responsável por realizar a refatoração na instância do metamodelo KDM. Além disso, KDM-RE também utiliza internamente o metamodelo SRM (ver Capítulo~\ref{chapter:Toward_a_Refactoring_Metamodel_for_KDM}), o qual permite a potencialização do reúso de refatorações em nível de modelo. Assim, refatorações podem ser reutilizadas e compartilhadas por meio da KDM-RE.


A ferramenta KDM-RE foi implementada como um conjunto de \textit{plug-ins} para o Ambiente de Desenvolvimento Eclipse IDE\footnote{\texttt{https://eclipse.org/home/index.php}}. Cada \textit{plug-in} representa um módulo da KDM-RE responsável por demonstrar a viabilidade dos conceitos descritos nos Capítulos~\ref{chapter:catalogo_refactoring_KDM} e~\ref{chapter:Toward_a_Refactoring_Metamodel_for_KDM}. Por exemplo, KDM-RE dá apoio à refatorações criadas por meio da abordagem mostrada no Capítulo~\ref{chapter:catalogo_refactoring_KDM}. Além disso, o formato exigido pela ferramenta para disponibilização das refatorações é o metamodelo mostrado no Capítulo~\ref{chapter:Toward_a_Refactoring_Metamodel_for_KDM}. Dessa forma, observando a Figura~\ref{fig:abordagem_kdm_tese_processo}~\ding{204}, nota-se que a ferramenta mostrada neste capítulo é a infraestrutura base para que toda a abordagem opere adequadamente. Nota-se que esse Capítulo é uma extensão do seguinte artigo: \textit{KDM-RE: A Model-Driven Refactoring Tool for KDM}~\cite{durelli_VEM_ferramenta}.


As seções deste capítulo estão organizadas da seguinte forma: na Seção~\ref{sec:construcao_da_kdm_re}, o desenvolvimento da KDM-RE é mostrado; na Seção~\ref{sec:modulo_de_refatoracao_kdm_re}, é apresentado o módulo responsável por aplicar refatorações no metamodelo KDM; na Seção~\ref{label:sec_modulo_do_srm}, é apresentado o módulo responsável por instanciar o metamodelo SRM por meio de uma DSL, e também é apresentado como reutilizar programaticamente as refatorações persistidas no metamodelo SRM; na Seção~\ref{sec:modulo_de_sincronizacao_kdm_re}, é apresentado o módulo responsável por manter a instância do metamodelo KDM sincronizada e consistente após a aplicação de refatorações. Algumas ferramentas relacionados são comentadas na Seção~\ref{sec:trabalhos_relacionados_ferramentas_kdm_re}. As considerações finais desse capítulo são apresentadas na Seção~\ref{sec:consideracoes_final_kdm_re}.

\section{Construção da KDM-RE}\label{sec:construcao_da_kdm_re}

A KDM-RE foi construída de modo a ser utilizada em conjunto com os demais recursos oferecidos pelo ambiente de desenvolvimento Eclipse IDE. Os \textit{plug-ins} da KDM-RE são organizados em módulos de acordo com a sua funcionalidade. Os três principais módulos são:

\begin{itemize}
\item \textbf{Módulo de Refatoração}, que é responsável pela aplicação de refatorações de forma transparente em instâncias do metamodelo KDM;

\item \textbf{Módulo do SRM}, que é responsável pela instanciação e reúso do metamodelo SRM;

\item \textbf{Módulo de Sincronização}, que é responsável por manter consistente e programar mudanças após a aplicação de refatorações em instâncias do metamodelo KDM.

\end{itemize}

A Figura~\ref{fig:arquitetura_ferramenta_kdm_re}, apresenta uma visão lógica da arquitetura da ferramenta KDM-RE. Como observado, a KDM-RE contém três camadas: (\textit{i}) IDE, (\textit{ii}) KDM-RE e (\textit{iii}) UI. A primeira camada da KDM-RE agrupa os recursos do Eclipse IDE para a criação dos três principais módulos da ferramenta KDM-RE. EMF, ATL, OCL, MoDisco, Papyrus, XText, EMFCompare e KDM são utilizadas durante a criação do KDM-RE.

\begin{figure}[h]
	\centering
	% Requires \usepackage{graphicx}
	\caption{Arquitetura da Ferramenta KDM-RE.}
	\label{fig:arquitetura_ferramenta_kdm_re}
	\includegraphics[scale=0.75]{images/arquitetura_KDM-RE}
	\fautor
\end{figure}

Na segunda camada, os módulos da KDM-RE são definidos. O módulo de refatoração utiliza EMF, ATL, OCL e MoDisco para criar um \textit{plug-in} onde o engenheiro de software pode aplicar refatorações em instâncias do metamodelo KDM. O módulo de sincronização utiliza o \textit{framework} EMFCompare e ATL. Por sua vez, o módulo do SRM utiliza EMF e XText. XText é utilizado no módulo do SRM para definir uma DSL para auxiliar a instanciação do metamodelo SRM. A última camada, UI, é responsável pela interface gráfica da ferramenta KDM-RE. Essa camada possui \textit{Wizards}, editores e visões para a aplicação, definição e reutilização das refatorações no contexto do KDM. Nas seções seguintes, os três principais módulos da ferramenta KDM-RE são apresentados e discutidos.


\section{Módulo de Refatoração}\label{sec:modulo_de_refatoracao_kdm_re}

Nesta seção, o módulo de refatoração é apresentado e ele foi implementado para prover suporte às refatorações criadas por meio das diretrizes apresentadas no Capítulo~\ref{chapter:catalogo_refactoring_KDM}. Na literatura, é possível identificar um conjunto de técnicas e linguagens específicas para auxiliar a condução e especificação de transformação de modelos~\cite{Biehl_2010, Mens_2006, Allilaire_06}. Nesta tese, a linguagem de transformação ATL~\cite{ATL_eclipse,Jouault_2008} foi escolhida para definir e aplicar refatorações no metamodelo KDM. Similarmente, as pre- e pós-condições foram implementadas em OCL. KDM-RE programaticamente executa as refatorações implementadas em ATL por meio da ATL EMF \textit{Transformation Virtual Machine}. As pre- e pós-condições são executadas na KDM-RE por meio da API Desden OCL\footnote{\texttt{http://www.dresden-ocl.org/}}. Essa API facilita a aplicação de OCL em qualquer metamodelo definido em EMF. Como consequência, KDM-RE é capaz de suportar a detecção de violações semânticas estáticas em instâncias do metamodelo KDM.


Como já salientado no Capítulo~\ref{chapter:catalogo_refactoring_KDM}, as refatorações foram definidas para serem executadas no contexto de instâncias do metamodelo KDM. Dessa forma, é necessário, primeiramente, transformar um determinado sistema em instância do metamodelo KDM. Para tal, foi integrada a ferramenta MoDisco na KDM-RE como apresentado na Figura~\ref{fig:kdm_modisco_discovery}. Primeiramente, o engenheiro de software deve clicar com o botão direito em um projeto Java e escolher a opção \texttt{Knowledge Discovery} e, depois, escolher o menu \texttt{Discovery KDM MODEL}. MoDisco irá, então, transformar o código escrito na linguagem de programação Java para uma instância do metamodelo KDM. 

\begin{figure}[h]
	\centering
	% Requires \usepackage{graphicx}
	\caption{KDM \textit{Discovery}.}
	\label{fig:kdm_modisco_discovery}
	\includegraphics[scale=0.65]{images/kdm_discovery_kdm_re}
	\fautor
\end{figure}

Após a criação da instância do metamodelo KDM, o engenheiro de software pode aplicar as refatorações. A ferramenta KDM-RE permite que as refatorações sejam aplicadas por meio de duas interfaces gráficas. A primeira é uma extensão do editor gráfico da ferramenta MoDisco~\cite{Bruneliere_2014}, que representa a instância do metamodelo em formato de árvore. A segunda interface gráfica é uma extensão do editor gráfico da ferramenta Papyrus\footnote{\texttt{https://eclipse.org/papyrus/}} que é uma ferramenta \sigla{CASE}{\textit{Computer-Aided Software Engineering}}. Por meio da segunda interface, o engenheiro de software pode aplicar refatorações utilizando um diagrama de classe UML. Nota-se que nessa segunda interface, de forma transparente, as refatorações são de fato aplicadas em instâncias do KDM, e o diagrama de classe UML é utilizado apenas como uma ponte entre as informações (nome de classe, atributos, métodos, etc.) e as refatorações.

A primeira interface gráfica fornece uma visão de árvore da instância do metamodelo KDM (\textit{model browser}) como apresentado na Figura~\ref{fig:modisco_modeol_browser}. No lado esquerdo dessa figura, algumas das metaclasses (\texttt{ClassUnit}, \texttt{MethodUnit} e \texttt{StorableUnit}, etc.) instanciadas do metamodelo KDM são apresentadas. O lado direito é onde as refatorações são aplicadas pelo engenheiro de software. E para aplicar as refatorações, o engenheiro de software deve clicar com o botão direito em cima de uma determinada instância de metaclasse, por exemplo, \texttt{ClassUnit}, \texttt{MethodUnit}, \texttt{StorableUnit}, etc., assim, o menu \texttt{Refactoring KDM} irá aparecer como ilustrado na Figura~\ref{fig:kdm_re_refactoring_arvore}. Além disso, utilizando-se esse menu, o engenheiro de software pode interagir com a instância do metamodelo KDM e escolher qual refatoração deve ser executada. Após o engenheiro de software clicar no menu \texttt{Refactoring KDM} e escolher uma determinada refatoração, o processo será iniciado. 

Para cada refatoração, um determinado \textit{RefactoringWizard} é executado. Esse \textit{Wizard} irá guiar o engenheiro de software durante a aplicação da refatoração. Em cada refatoração, a instância do metamodelo KDM deve ser analisada para identificar e obter os metadados das metaclasses que serão afetadas pela refatoração. Esses metadados são utilizados tanto na ATL (refatoração), quanto na OCL (pré- e pós-condições). 


\begin{figure}[h]
	\centering
	% Requires \usepackage{graphicx}
	\caption{MoDisco \textit{Model Browser}.}
	\label{fig:modisco_modeol_browser}
	\includegraphics[scale=0.6]{images/kdm-re_modisco}
	\fautor
\end{figure}

\begin{figure}[h]
	\centering
	% Requires \usepackage{graphicx}
	\caption{KDM-RE \textit{Refactoring Browser}.}
	\label{fig:kdm_re_refactoring_arvore}
	\includegraphics[scale=0.55]{images/novoMenuPopupKDM_RE}
	\fautor
\end{figure}


Por exemplo, suponha que o engenheiro de software identificou que uma instância da metaclasse \texttt{ClassUnit} está realizando o trabalho que deveria ser feito por duas instâncias (ver Figura~\ref{fig:kdm_re_wizard_extract_class}). Assim, o engenheiro de software deve aplicar a refatoração \texttt{Extract ClassUnit}. O primeiro passo é selecionar a instância da metaclasse \texttt{ClassUnit} para realizar a refatoração. Em seguida, deve-se selecionar a opção \texttt{Extract ClassUnit} no menu \texttt{Refactoring KDM}. Automaticamente, KDM-RE irá criar um \textit{Wizard} para auxiliar o engenheiro de software, como ilustrado na Figura~\ref{fig:kdm_re_wizard_extract_class}. Utilizando esse \textit{Wizard}, o engenheiro de software define o nome da nova instância da metaclasse \texttt{ClassUnit}. Além disso, esse \textit{Wizard} permite visualizar todos as instâncias das metaclasses \texttt{StorableUnit} e \texttt{MethodUnit} que serão extraídas para a nova instância a ser criada. O \textit{Wizard} também permite especificar se instâncias da metaclasse \texttt{MethodUnit} devem ser criadas para representar os métodos assessores (\textit{getters} e \textit{setters}). 

Uma característica importante da KDM-RE é a opção de visualizar previamente o resultado da refatoração. Assim, caso o engenheiro de software almejar visualizar o efeito da refatoração antes de efetivamente realizá-la, ele poderá selecionar o botão \texttt{Preview}. Após clicar no botão \texttt{Preview}, uma visão de comparação será criada como apresentado na Figura~\ref{fig:previa_resultado_extractClassUnit}. Essa visão de comparação contém duas principais partes. A parte superior representa quais instâncias foram deletadas, movidas e adicionadas de forma textual. Na parte inferior, é possível visualizar graficamente a diferença entre as duas instâncias do metamodelo KDM, ou seja, a instância não refatorada (original) e a instância refatorada. O lado direito representa a instância do metamodelo KDM após a aplicação da refatoração \texttt{Extract ClassUnit} e o lado esquerdo representa a instância do metamodelo KDM antes da aplicação da refatoração. 

Ao focar a Figura~\ref{fig:previa_resultado_extractClassUnit}, vê-se a instância direita do metamodelo KDM (instância refatorada) uma nova instância da metaclasse \texttt{ClassUnit} chamada \texttt{Document} foi criada e uma instância da metaclasse \texttt{StorableUnit} (\aspas{\texttt{CPF}}) foi movida para a nova \texttt{ClassUnit} \texttt{Document}. Instâncias da metaclasse \texttt{MethodUnit} também foram criadas para representar os métodos assessores. A instância da metaclasse denominada \texttt{Pessoa} agora possui uma instância da metaclasse \texttt{StorableUnit}, denominada \texttt{document} que representa um link entre as duas instâncias de \texttt{ClassUnit}.

\begin{figure}[h]
	\centering
	% Requires \usepackage{graphicx}
	\caption{\texttt{Extract ClassUnit} \textit{Wizard}.}
	\label{fig:kdm_re_wizard_extract_class}
	\includegraphics[scale=0.5]{images/extractClassEasierToExplainerEMFCOmpare}
	\fautor
\end{figure}

\begin{figure}[h]
	\centering
	% Requires \usepackage{graphicx}
	\caption{Prévia do resultado da refatoração \texttt{Extract ClassUnit}.}
	\label{fig:previa_resultado_extractClassUnit}
	\includegraphics[scale=0.5]{images/previaRefatoracaoExtractClassUnitEMFCOmpare}
	\fautor
\end{figure}


Após especificar todas as entradas necessárias e visualizar o efeito que a refatoração irá resultar na instância do metamodelo KDM, o engenheiro de software deve clicar no botão \textit{Finish}. A KDM-RE executa a API Dresden OCL para verificar as pré-condições. Caso as pré-condições forem satisfeitas, a refatoração \texttt{Extract ClassUnit} é executada efetivamente. A execução da refatoração é totalmente realizada com base na ATL. As entradas informadas pelo engenheiro de software no \textit{Wizard} são enviadas para ATLs pré-definidas e, em seguida, são executadas programaticamente pela KDM-RE por meio da ATL EMF \textit{Transformation Virtual Machine}\footnote{\texttt{https://wiki.eclipse.org/ATL/EMFTVM}}. Posteriormente, a KDM-RE  executa a API Dresden OCL para verificar as pós-condições. 


Embora a primeira interface gráfica seja útil para aplicar refatorações diretamente em instâncias do metamodelo KDM, ela não é intuitiva. Dessa forma, a segunda interface gráfica almeja deixar as refatorações mais fáceis e intuitivas de serem aplicadas. Por exemplo, embora~\citeonline{Fowler1999} tenha criado um catálogo de refatorações para ser utilizado em código-fonte, mais de 60\% das refatorações (44 de 72) são ilustradas utilizando modelos, mais especificamente diagramas de classes da UML. Além disso, ~\citeonline{Zhang_2005, Boger_2003} afirmam que algumas refatorações (por exemplo, \texttt{Extract Method}) são mais naturais quando executadas diretamente no código-fonte e outras refatorações, como: \texttt{Rename Class}, \texttt{Pull Up Method}, \texttt{Push Down Method}, etc., podem ser aplicadas tanto em código-fonte, quanto em modelos; já as refatorações que lidam com herança, tais como \texttt{Extract Class}, \texttt{Extract Interface}, \texttt{Replace Inheritance with Delegation}, são mais intuitivas quando aplicadas diretamente em nível de modelo, tais como o diagrama de classe da UML. 

Diante disso, a KDM-RE foi integrada com a ferramenta CASE Papyrus para utilizar o diagrama de classe da UML. Para utilizar essa segunda interface, primeiramente, é necessário transformar a instância do metamodelo KDM para uma instância da UML. Assim, as refatorações definidas na KDM-RE podem ser aplicadas diretamente em diagramas da UML; por exemplo, pode-se aplicar refatorações por meio do diagrama de classe. É importante observar que, embora as refatorações sejam aplicadas graficamente por meio do diagrama de classe da UML, todas as refatorações (transformações) são aplicadas transparentemente na instância do metamodelo KDM e não na instância da UML - apenas informações são extraídas do diagrama de classe da UML e são enviados como entrada para as refatorações pré-definidas em ATL. %Tecnicamente essa segunda interface é implementada como uma extensão da ferramenta CASE Papyrus.


Os passos para utilizar a segunda interface são quase os mesmos da primeira interface. Porém, um passo a mais se faz necessário para utilizar a segunda interface. Deve-se gerar uma instância do metamodelo UML tendo como base uma instância do metamodelo KDM. A geração da instância do metamodelo UML é totalmente apoiada por um \textit{plug-in} denominado \texttt{DiscoverUmlModelFromKdmModel} do MoDisco. Esse \textit{plug-in} utiliza regras ATL para criar uma instância do metamodelo UML, tendo como base outra instância do metamodelo KDM. Por exemplo, na Figura~\ref{fig:kdmToUML}, são apresentadas duas instâncias dos metamodelos UML (lado esquerdo) e KDM (lado direito) após a transformação, respectivamente.

\begin{figure}[!h]
	\centering
	% Requires \usepackage{graphicx}
	\caption{Instância UML gerada a partir do KDM.}
	\label{fig:kdmToUML}
	\includegraphics[scale=0.5]{images/kdmToUML}
	\fautor
\end{figure}

Após a criação da instância do metamodelo UML, o próximo passo é utilizar a ferramenta CASE Papyrus para exibir a nova instância do metamodelo UML por meio do diagrama de classe, como apresentado na Figura~\ref{fig:kdmToUML_diagrama_de_classe}. Por meio desse diagrama, a KDM-RE permite que o engenheiro de software realize refatorações. Por exemplo, na Figura~\ref{fig:refatoracao_papyrus_KDM_iteragir} é ilustrado que, primeiramente, o engenheiro de software deve clicar com o botão direito em cima do elemento que almeja refatorar (nesse caso a classe \texttt{Secretary}), escolher a opção \texttt{Refactoring Model} e, depois, decidir qual refatoração aplicar (nesse exemplo: \texttt{Extract ClassUnit}). Em seguida, um \textit{RefactoringWizard} similar ao apresentado na Figura~\ref{fig:kdm_re_wizard_extract_class} é executado e ele também irá guiar o engenheiro de software durante a aplicação da refatoração. Da mesma forma como na primeira interface, na segunda interface o engenheiro de software pode solicitar também a realização de uma prévia da refatoração. O resultado dessa solicitação será uma interface similar a apresentada na Figura~\ref{fig:previa_resultado_extractClassUnit}.

\begin{figure}[!h]
	\centering
	% Requires \usepackage{graphicx}
	\caption{Diagrama de Classe da UML gerada a partir do KDM.}
	\label{fig:kdmToUML_diagrama_de_classe}
	\includegraphics[scale=0.5]{images/refatoracao_UML_papyrus}
	\fautor
\end{figure}

As informações fornecidas pelo engenheiro de software no \textit{Wizard} são enviadas para ATLs pré-definidas e, em seguida, são executadas programaticamente pela KDM-RE por meio da ATL EMF \textit{Transformation Virtual Machine}. O resultado da refatoração altera a instância do metamodelo KDM e é replicado automaticamente no diagrama de classe da UML; portanto, o engenheiro de software pode visualizar graficamente no diagrama de classe UML o resultado da refatoração.

\begin{figure}[!h]
	\centering
	% Requires \usepackage{graphicx}
	\caption{Refatorações por meio do Diagrama de Classe.}
	\label{fig:refatoracao_papyrus_KDM_iteragir}
	\includegraphics[scale=0.5]{images/kdm_uml_papyrus_refactoring_extract_class_new}
	\fautor
\end{figure}

\section{Módulo do SRM}\label{label:sec_modulo_do_srm}

Nessa seção, é apresentado um módulo desenvolvido para fornecer suporte ao metamodelo SRM apresentado no Capítulo~\ref{chapter:Toward_a_Refactoring_Metamodel_for_KDM} (Figura~\ref{fig:meta_modelo_SRM}). Esse módulo implementa o metamodelo SRM utilizando EMF por meio do meta-metamodelo Ecore. Nesse módulo, também foi definida uma linguagem específica de domínio (DSL) para facilitar a instanciação do metamodelo SRM. A gramática dessa DSL é apresentada no Capítulo~\ref{chapter:Toward_a_Refactoring_Metamodel_for_KDM}, mais detalhadamente nos Códigos-fontes~\ref{lst:dsl_part_1}, \ref{lst:dsl_part_2}, \ref{lst:dsl_part_3}, \ref{lst:dsl_part_4} e \ref{lst:dsl_part_5}.


Esse módulo fornece uma forma de compartilhar e reutilizar refatorações por meio de instâncias do metamodelo SRM. Para permitir o reúso e o compartilhamento de refatorações, um repositório remoto foi criado. Esse repositório remoto é dedicado para executar solicitações RESTful. Instâncias do metamodelo SRM são enviadas e recebidas por meio da API RESTful. Isso é possível, pois as instâncias do SRM são arquivos persistidos no formato XMI. \sigla{JPA}{Java Persistence API} e o banco de dados MySQL foram utilizados para realizar as persistências das instâncias do metamodelo SRM.

%Para utilizar as vantagens dos metamodelo SRM, os engenheiros de modernização precisam ter conhecimento de linguagem de programação avançada. Eles devem estar familiarizados como as semânticas das refatorações (por exemplo, qual(is) é (são) o(s) pré-requisito(s) para a execução de uma refatoração) e como/onde utilizar e programar tais refatorações. Além disso, a instanciação de uma refatoração utilizando o metamodelo SRM é bastante verbosa, complexa e propensa a erros, pois exige conhecimento avançadas de refatoração e habilidades de programação em relação a API Ecore/EMF. Como salientado no Capítulo~\ref{chapter:Toward_a_Refactoring_Metamodel_for_KDM} o metamodelo SRM foi desenvolvido para permitir a reutilização de metaclasses já definidas no metamodelo KDM. Mais especificadamente os elementos estruturais que são utilizados em uma refatoração são representados por metaclasses previamente já definidas no metamodelo KDM, tais como: \texttt{ClassUnit}, \texttt{InterfaceUnit}, \texttt{Package}, \texttt{StorableUnit}, etc.

%No Código-fonte~\ref{cod:instancia_do_SRM} é apresentado um trecho onde é feita a instanciação em memória de algumas metaclasses definidas no metamodelo SRM. Note que nesse código-fonte a instanciação das metaclasses do metamodelo SRM é um processo verboso e propenso a erros. Para instanciar o metamodelo SRM o engenheiro deve saber utilizar a API do \textit{framework} EMF. Metaclasses são instanciadas em EMF utilizando o conceito de fabrica (em inglês - \textit{Factory}). Cada fabrica de uma metaclasse possui um método \texttt{createX}, onde X representa o nome da metaclasse do metamodelo. Para criar uma instancia válida do metamodelo SRM o engenheiro de modernização deve seguir um conjunto de passos. Por exemplo, como pode ser observado na linha 1 do Código-fonte~\ref{cod:instancia_do_SRM} uma instância da metaclasse \texttt{Author} é criada por meio da interface \texttt{RefactoringModelFactory} e do método \texttt{createAuthor()}. Em seguida, todos os meta-atributos da metaclasse \texttt{Author} devem ser especificados como apresentado nas linhas 2 e 3 do Código-fonte~\ref{cod:instancia_do_SRM}. Nas linhas 4-14 outras metaclasses do metamodelo SRM são instanciadas. 


%Como apresentado e salientado no Capítulo~\ref{chapter:Toward_a_Refactoring_Metamodel_for_KDM} a instanciação de uma refatoração por meio do SRM pode ser um processo demorado e suscetível a erro. Para diminuir a quantidade de código-fonte, esforço obrigatórios e competência necessárias para instanciar refatorações utilizando o metamodelo SRM, a KDM-RE provê uma DSL que auxilia a instanciação de refatorações sistematicamente. Na parte superior a esquerda da Figura~\ref{fig:DSL_SRM} é possível visualizar o relacionamento entre a DSL criada e as metaclasses do metamodelo SRM. 

Por meio das gramáticas apresentadas nos Códigos-fontes~\ref{lst:dsl_part_1}, \ref{lst:dsl_part_2}, \ref{lst:dsl_part_3}, \ref{lst:dsl_part_4}, \ref{lst:dsl_part_5_1} e \ref{lst:dsl_part_5}, Xtext gera um editor textual para o ambiente de desenvolvimento Eclipse IDE. Esse editor textual provê \textit{highlighting} de sintaxe, \textit{autocomplete} de código e navegação de código. Na parte superior, à esquerda da Figura~\ref{fig:DSL_SRM}, é possível visualizar trechos da DSL resultante. Além disso, essa figura ilustra o relacionamento entre a DSL criada e as metaclasses do metamodelo SRM, ou seja, representa que a sintaxe da DSL está em conformidade com as metaclasses do SRM.

%\begin{codigo}[caption={[Instanciação do metamodelo SRM programaticamente.] Instanciação do metamodelo SRM.},escapeinside={(*@}{@*)}, basicstyle=\footnotesize, label={cod:instancia_do_SRM}, language=Java]{Name}
%Author author = RefactoringModelFactory.eINSTANCE.createAuthor();
%author.setName("Rafael");
%author.setLastName("Durelli");
%RefactoringModel rM = RefactoringModelFactory.eINSTANCE.createRefactoringModel();
%rM.setAuthor(author);
%RefactoringLibrary library = RefactoringModelFactory.eINSTANCE.createRefactoringLibrary();
%lib.setName("Fowler's refactorings");
%lib.setDescription("Contains some Fowler's refactorings such as ExtractClass, RenameElements, PushMethod, PushAttribute, etc.");
%lib.setShortDescription("Fine grained Refactorings");
%rM.getLibraries().add(lib);
%Catalog cat = RefactoringModelFactory.eINSTANCE.createCatalog();
%cat.setAuthor(author);
%cat.setName("Fowler's catalog");
%libr.getCatalogs().add(cat);
%...
%\end{codigo}

\begin{figure}[!h]
	\centering
	% Requires \usepackage{graphicx}
	\caption{DSL para auxiliar a instanciação do SRM.}
	\label{fig:DSL_SRM}
	\includegraphics[scale=0.58]{images/MetaModelEDSL3}
	\fautor
\end{figure}

Cada trecho de código da DSL representa uma instância de uma metaclasse do metamodelo SRM, ou seja, cada declaração da DSL está em conformidade com uma metaclasse do SRM. Por exemplo, a palavra-chave \texttt{refactoringModel} \texttt{author} entre \aspas{\{} e \aspas{\}} representa a instanciação da metaclasse \texttt{Author} do SRM que possui dois meta-atributos: \texttt{name} e \texttt{lastName}. A DSL criada para auxiliar a instanciação do SRM foi desenvolvida utilizando Xtext\footnote{\texttt{https://www.eclipse.org/Xtext/}}. Xtext é um \textit{framework} do Eclipse\footnote{\texttt{https://www.eclipse.org}} que facilita a definição de gramática\footnote{Gramáticas representam a definição formal de uma sintaxe textual concreta. Consistem em um conjunto de regras de produção para definir como o \textit{textual input} (ou seja, sentenças) é representado. Basicamente, as regras de produção podem ser representadas utilizando \sigla{BNF}{\textit{Backus–Naur Form}}, por exemplo, \textit{S ::= P1 ... Pn}, essa gramática define um símbolo \textit{S} por um conjunto de expressões \textit{P1 ... Pn}.} 
com a utilização de um metamodelo que foi definido utilizando EMF. Xtext tem como principal objetivo automatizar e agilizar o processo de desenvolvimento de DSLs. Além disso, a sintaxe da DSL segue as terminologias e conceitos definidos no metamodelo SRM para facilitar a utilização da DSL e o entendimento do metamodelo SRM.

%Em Xtext a gramática segue uma notação similar ao \textit{Backus–Naur Form} (BNF) chamada de regras do \textit{parser}. Tais regras representam a sintaxe concreta da DSL. Note que para facilitar o entendimento da DSL, trechos da mesma são mostradas em listagens de códigos separados, bem como símbolos para explanar o propósito de uma terminada linha da gramática. No Código-fonte~\ref{lst:dsl_part_1} é ilustrado o primeiro trecho da gramática concreta da DSL desenvolvida. A gramática começa com a definição do nome da DSL (SRM) (ver Código-fonte~\ref{lst:dsl_part_1} \ding{182}). Em sequência é definido os metamodelos que devem ser importados para serem utilizados durante a criação da DSL: o metamodelo SRM\ding{183} e o Ecore\ding{184}.


%\begin{lstlisting}[language=Xtext, frame=single, basicstyle={\scriptsize}, mathescape=true, label={lst:dsl_part_1}, caption={Gramática da DSL - parte 1}]
%$\textrm{\ding{182}}$ grammar refactoring.xtext.SRM with org.eclipse.xtext.common.Terminals 
%$\textrm{\ding{183}}$ import platform:/resource/refactoring/model/SRM.ecore
%$\textrm{\ding{184}}$ import http://www.eclipse.org/emf/2002/Ecore as ecore
%RefactoringModel: 
%	$\textrm{\ding{185}}$ `refactoringModel' name = ID `{'
%	$\textrm{\ding{186}}$ author = Author
%	$\textrm{\ding{187}}$ libraries += RefactoringLibrary$^{*}$;
%	`}'
%\end{lstlisting}


%Em seguida é criado a primeira regra. Essa regra começa com a definição da metaclasse \texttt{RefactoringModel}. O corpo da regra começa logo após os \aspas{\texttt{:}}. Primeiramente para o entendimento da regra, é importante destacar que literais de \textit{string} (em Xtext os literais podem ser expressos com aspas simples ou duplas) definem palavras-chave da DSL. Como pode ser observado no Código-fonte~\ref{lst:dsl_part_1} é esperado a palavra-chave \texttt{refactoringModel}\ding{185} seguido por um \texttt{ID} e \aspas{\{}. A gramática que rege o objeto \texttt{ID} é definida como uma sequência ilimitada de maiúsculas e minúsculas, números e o carácter de sublinhado, embora possa não começa por um dígito. A gramática que representa o nó \texttt{ID}\ding{182} pode ser visualizada no Código-fonte~\ref{lst:dsl_part_2}. 

%\begin{lstlisting}[language=Xtext, frame=single, basicstyle=\scriptsize, mathescape=true, label={lst:dsl_part_2}, caption={Gramática da DSL - parte 2}]
%	$\textrm{\ding{182}}$ terminal ID: (`a'..`z' | `A'..`Z'|`_')(`a'..`z' | `A'..`Z'|`_'|`0'..`9')*;
%\end{lstlisting}

%Ainda no Código-fonte~\ref{lst:dsl_part_1} a expressão \texttt{author=Author}\ding{186} especifica que pode-se instanciar uma instancia da metaclasse \texttt{Author}. A expressão \texttt{(libraries += RefactoringLibrary)$^{*}$}\ding{187} descrita no Código-fonte~\ref{lst:dsl_part_1} especifica que pode-se instanciar várias instâncias da metaclasse \texttt{RefactoringLibrary}. O operador estrela, \aspas{\texttt{*}}, ilustra que o número de elementos (nesse caso \texttt{RefactoringLibrary}) é arbitrário; em particular, ele pode ser qualquer número \texttt{>=} 0. Operador \texttt{+=} por sua vez representa que a propriedade \texttt{libraries} será uma lista do tipo \texttt{RefactoringLibrary}.

%\begin{lstlisting}[language=Xtext, frame=single, basicstyle=\scriptsize, mathescape=true, label={lst:dsl_part_3}, caption={Gramática da DSL - parte 3}]
%Author:
%	$\textrm{\ding{182}}$ `author' `{'
%	$\textrm{\ding{183}}$ `name' `:' name = ID  
%		$\textrm{\ding{229} \ding{184}}$ `lastName' `:' lastName = ID; 
%`}'
%RefactoringLibrary:
%	$\textrm{\ding{185}}$ `refactoringLibraries' `{'
%	$\textrm{\ding{186}}$ `name' `:' name = ID  
%		$\textrm{\ding{229}}$ `shortDescription' `:' shortDescription = STRING
%		$\textrm{\ding{229}}$ `description' `:' description = STRING
%		$\textrm{\ding{229}}$ $\textrm{\ding{187}}$ catalogs += Catalog$^{*}$
%`}'
%\end{lstlisting}

%A definição das regras que regem as metaclasses \texttt{Author} e \texttt{RefactoringLibrary} são apresentadas no Código-fonte~\ref{lst:dsl_part_3}. A regra para a definição de \texttt{Author} começa com a definição da palavra-chave \texttt{author} seguida por um \aspas{\{}\ding{182}. Em seguida a palavra-chave \texttt{name} é esperada, seguido por \aspas{\texttt{:}} \ding{183}. Posteriormente a palavra-chave \texttt{lastName} também é esperada, seguido por \aspas{\texttt{:}} \ding{184}. Na linha 6 do Código-fonte~\ref{lst:dsl_part_3} começa a definição da regra da metaclasse \texttt{RefactoringLibrary}. A regra para a definição de \texttt{RefactoringLibrary} começa com a definição da palavra-chave \texttt{refactoringLibraries} seguida por um \aspas{\{}\ding{185}. Em seguida, deve-se especificar a palavra-chave \texttt{name} e \texttt{:}. Posteriormente, as palavras-chaves \texttt{shortDescription} e \texttt{description} são especificadas nas linhas 9 e 10, respectivamente. A expressão descrita na Linha 11 representa que pode haver qualquer número de instâncias da metaclasse \texttt{Catalog}.

%\begin{lstlisting}[language=Xtext, frame=single, basicstyle=\scriptsize, mathescape=true, label={lst:dsl_part_4}, caption={Gramática da DSL - parte 4}]
%Catalog:
%	$\textrm{\ding{182}}$`catalog' `{' 
%		$\textrm{\ding{229} \ding{183}}$`name' `:' name=ID
%		$\textrm{\ding{229} \ding{184}}$`author' `:' author=[Author]
%		$\textrm{\ding{229} \ding{185}}$refactorings += Refactoring$^{*}$
%	`}'
%Refactoring:
%	`refactoring' `{' 
%		$\textrm{\ding{229}}$`name' `:' name = ID
%		$\textrm{\ding{229}}$`motivation' `:' motivation = STRING
%		$\textrm{\ding{229}}$`summary' `:' summary = STRING
%		$\textrm{\ding{229} \ding{186}}$ operation = Operation?
%		$\textrm{\ding{229} \ding{187}}$ preCondition = PreCondition?
%		$\textrm{\ding{229} \ding{188}}$ postCondition = PostCondition?
%		$\textrm{\ding{229} \ding{189}}$ classification = Classification
%		$\textrm{\ding{229} \ding{190}}$(`containedRefactoring' `:' chainOfRefactoring+=Refactoring)$^{*}$
%	`}'
%\end{lstlisting}

%O Código-fonte~\ref{lst:dsl_part_4} representa as sintaxes concretas das metaclasses \texttt{Catalog} e \texttt{Refactoring}. A sintaxe concreta da metaclasse \texttt{Catalog} começa com a palavra-chave \texttt{catalog} seguida por um \aspas{\{}\ding{182}. Em seguida o nome do catálogo de refatoração deve ser especifica por meio da palavra-chave \texttt{name} \ding{183}. Posteriormente, deve-se especificar uma instância da metaclasse \texttt{Author}, informando quem é o autor desse catalogo de refatoração, ver Código-fonte~\ref{lst:dsl_part_4}\ding{184}. Na Linha 5 do Código-fonte~\ref{lst:dsl_part_4} deve-se informar várias instâncias da metaclasse \texttt{Refactoring}, essa sintaxe representa as refatorações que compõem esse catalogo de refatoração. Nas linhas 7-16 a definição da sintaxe concreta para a definição de uma refatoração por meio do metamodelo SRM é apresentada. Inicialmente, uma refatoração deve possuir um nome, conforme ilustrado na linha 9 do Código-fonte~\ref{lst:dsl_part_4}. Posteriormente, a motivação, bem como o resumo da refatoração também devem ser especificados, conforme apresentado nas linhas 10 e 11 do Código-fonte~\ref{lst:dsl_part_4}. As linhas 12, 13, 14 e 15 informam que uma metaclasse do tipo \texttt{Operation}\ding{186}, \texttt{PreCondition}\ding{187}, \texttt{PostCondition}\ding{188} e \texttt{Classification} \ding{189} devem ser instanciadas, respectivamente. Na linha 16 representa a sintaxe da DSL para especificar um conjunto de refatorações que quando combinadas podem realizar refatorações complexas.

%\begin{lstlisting}[language=Xtext, frame=single, basicstyle=\scriptsize, mathescape=true, label={lst:dsl_part_5}, caption={Gramática da DSL - parte 5}]
%Operation: 
%	$\textrm{\ding{182}}$`operation' `{'
%		$\textrm{\ding{229}}$`language' `:' language=Language
%		`body' `:' `{'
%			$\textrm{\ding{229} \ding{184}}$body = STRING
%		`}'
%	`}'
%PreCondition: 
%	$\textrm{\ding{185}}$`preCondition' `{'
%		$\textrm{\ding{229}}$`context' `:' context=STRING
%		$\textrm{\ding{229}}$`language' `:' language=Language
%		`body' `:' `{' 
%			$\textrm{\ding{229}}$body=STRING	
%		`}'
%	`}'
%PostCondition: 
%	$\textrm{\ding{186}}$`postCondition' `{'
%		$\textrm{\ding{229}}$`context' `:' context=STRING
%		$\textrm{\ding{229}}$`language' `:' language=Language
%		`body' `:' `{' 
%			$\textrm{\ding{229}}$body=STRING	
%		`}'
%	`}'
%enum Language: 
%	ATL | OCL | XQuery
%\end{lstlisting}

\begin{figure}[!h]
	\centering
	% Requires \usepackage{graphicx}
	\caption{Instanciando a DSL.}
	\label{fig:creatingSRMDSL}
	\includegraphics[scale=0.7]{images/creatingSRMDSL}
	\fautor
\end{figure}


\begin{figure}[!h]
	\centering
	% Requires \usepackage{graphicx}
	\caption{Editor Textual para instanciar o metamodelo SRM.}
	\label{fig:editor_SRM_metamodel_ECORE}
	\includegraphics[scale=0.65]{images/dsl_SRM_ECORE}
	\fautor
\end{figure}

\begin{figure}[!h]
	\centering
	% Requires \usepackage{graphicx}
	\caption{Menu para enviar instâncias do metamodelo SRM.}
	\label{fig:editor_SRM_metamodel_ECORE_menu}
	\includegraphics[scale=0.65]{images/SRM_Upload_Image}
	\fautor
\end{figure}

%No Código-fonte~\ref{lst:dsl_part_5} as sintaxes concretas das metaclasses \texttt{Operation}, \texttt{PreCondition} e \texttt{PostCondition} são definidas. A sintaxe concreta da metaclasse \texttt{Operation} inicia com a palavra-chave \texttt{operation} seguida por um \aspas{\{}\ding{182}. Em seguida, deve-se especificar qual a linguagem que a operação/refatoração será escrita. \texttt{Language} é uma enumeração (\textit{enum}) que está representado na linha 24 do Código-fonte~\ref{lst:dsl_part_5}. A linha 5 representa a sintaxe concreta para especificar o \aspas{corpo} da operação/refatoração propriamente dita \ding{184}. Na linha 8 a sintaxe concreta da metaclasse \texttt{PreCondition} é definida. A sintaxe concreta inicia com a palavra-chave \texttt{preCondition} seguida por um \aspas{\{}\ding{182}, \texttt{context}, \texttt{language} e \texttt{body}. A sintaxe concreta da metaclasse \texttt{PostCondition} é definida nas linhas 16 até 23. Note que a sintaxe é similar à sintaxe definida na metaclasse \texttt{PreCondition}. A partir das gramáticas da DSL apresentadas XText gera um editor textual no ambiente de desenvolvimento Eclipse IDE. Esse editor provê para a KDM-RE \textit{highlighting} de sintaxe, \textit{code completion} e \textit{code navigation}. O editor resultante pode ser observado na Figura~\ref{fig:editor_SRM_metamodel_ECORE}.


%Identificar e reutilizar artefatos em nível de modelos é uma limitação hoje em dia~\cite{Rocco_2015}. Essa limitação faz com que modernizadores geralmente tenham que recriar soluções já existentes, tais como: metamodelos, transformações, etc, diminuindo assim a produtividade e benefícios defendidos pelo MDE. Dessa forma, para facilitar e promover o compartilhamento e o reúso de refatorações por meio do metamodelo SRM é necessário uma forma que permita que modernizadores possam enviar, analisar e reutilizar as refatorações instanciadas para um artefato central. Assim, outros modernizadores podem contribuir com a criação de um conjunto de metadados sobre refatorações para que outros modernizadores possam pesquisar, identificar e reutilizar refatorações em seu projeto durante a modernização. 

%Neste contexto, um repositório foi criado para persistir instâncias do metamodelo SRM, onde tais instâncias representam metadados sobre refatorações que podem ser pesquisas e reutilizadas por outros modernizadores. Na Figura~\ref{fig:repositorio} é apresentado o diagrama de entidade e relacionamento do repositório. É importante observar que cada tabela apresentado nessa figura representa uma metaclasse do metamodelo SRM. Por exemplo, a metaclasse \texttt{Refactoring} apresentada na Figura~\ref{fig:meta_modelo_SRM} equivale a entidade \texttt{Refactoring} ilustrada na Figura~\ref{fig:repositorio}. 

%\begin{figure}[!h]
%	\centering
%	\caption{Diagrama de Entidade e Relacionamento do Metamodelo SRM.}
%	\label{fig:repositorio}
%	\includegraphics[scale=0.5]{images/ERD_refactoring}
%	\fautor
%\end{figure}

Para utilizar a DSL, primeiramente, o engenheiro de modernização precisa criar um arquivo com a extensão \aspas{.mydsl}. A KDM-RE fornece um \textit{wizard} para criar esse arquivo, como apresentado na Figura~\ref{fig:creatingSRMDSL}. Por meio desse \textit{wizard}, o engenheiro de modernização deve especificar um nome para o arquivo e também deve especificar a extensão do arquivo como \aspas{.mydsl}. É importante que o engenheiro de modernização especifique a extensão correta do arquivo, ou seja, \aspas{.mydsl}, a qual é utilizada para o módulo do SRM identificar que o arquivo criado é, na verdade, a DSL definida na gramática apresentada nos Códigos-fontes~\ref{lst:dsl_part_1}-~\ref{lst:dsl_part_5}. No exemplo apresentado na Figura~\ref{fig:creatingSRMDSL}, o arquivo foi definido como \texttt{srmInstance.mydsl}. Dessa forma, a KDM-RE automaticamente irá criar um editor para auxiliar o engenheiro de modernização a especificar uma refatoração utilizando as sintaxes da DSL. Após a criação do arquivo \texttt{srmInstance.mydsl}, o engenheiro de modernização deve começar a especificar uma determinada refatoração, como apresentado na Figura~\ref{fig:editor_SRM_metamodel_ECORE}. 


Adicionalmente, a KDM-RE fornece uma maneira de armazenar e compartilhar instâncias do metamodelo SRM. O principal objetivo é fazer com que refatorações definidas utilizando o metamodelo SRM sejam reutilizadas em projetos que utilizam o metamodelo KDM. Assim, após definir uma instância do metamodelo SRM utilizando a DSL, por meio do editor apresentado na Figura~\ref{fig:editor_SRM_metamodel_ECORE}, o engenheiro de modernização deve enviar os metadados de uma determinada instância do SRM para um repositório remoto. Esse processo é realizado por meio de um menu denominado \texttt{Upload SRM's Metadata to repository}, e, para interagir com esse menu, o engenheiro deve clicar com o botão direito no editor textual da DSL e escolher \texttt{SRM's Metadata}, como apresentado na Figura~\ref{fig:editor_SRM_metamodel_ECORE_menu}. Transparentemente, a KDM-RE irá converter a sintaxe e a semântica da DSL em um arquivo XMI, como ilustrado no Código-fonte~\ref{lst:xml_srm_convertido}. Note que nesse código-fonte os códigos em ATL e OCL (que representam a refatoração e as pré- e pós-condições) foram omitidos para facilitar o entendimento do arquivo XMI. 

Cada marcação apresentada nesse XMI representa uma metaclasse do metamodelo SRM. Por exemplo, \aspas{\textbf{<refactoringModel>}}, \aspas{\textbf{<libraries>}}, \aspas{\textbf{<catalogs>}} e \aspas{\textbf{<refactorings>}} estão em conformidades com as metaclasses \texttt{RefactoringModel}, \texttt{RefactoringLibrary}, \texttt{Catalog} e \texttt{Refactoring} do SRM, respectivamente. Adicionalmente, cada marcação contém atributos que representam os meta-atributos de cada metaclasse do SRM.

\begin{lstlisting}[language=XML, frame=single, basicstyle={\scriptsize}, mathescape=true, label={lst:xml_srm_convertido}, caption={Arquivo XMI representando a instância do SRM.}]
<?xml version="1.0" encoding="ASCII"?>
<refactoringModel:RefactoringModel xmi:version="2.0" xmlns:xmi="http://www.omg.org/XMI" xmlns:xsi="http://www.w3.org/2001/XMLSchema-instance" xmlns:refactoringModel="http://refactoringModel/1.0">
  <libraries name="FineGrainedRefactoring" shortDescription="contains a set of refactorings" description="refactorings">
    <catalogs name="Catalog" author="//@author">
      <refactorings name="ExtractCLass" motivation="Motivation" summary="Summary">
        <source key="Class" value=//@ClassUnit/>
        <target key="Class" value=//@ClassUnit/>
        <elements key="Class" value=//@ClassUnit key "Attribute" value=//@StorableUnit/>
        <preCondition context="TO BE DEFINED" language="OCL" body="TO BE DEFINED"/>
        <postCondition context="TO BE DEFINED" language="OCL" body="TO BE DEFINED"/>
        <operation body="TO BE DEFINED"/>
      </refactorings>
    </catalogs>
  </libraries>
  <author name="Rafael" lastName="Durelli"/>
</refactoringModel:RefactoringModel>
\end{lstlisting}


Posteriormente, o arquivo XMI é lido, enviado e armazenado em um repositório remoto para ser manipulado. A KDM-RE também fornece uma maneira de visualizar todas as instâncias do SRM disponíveis no repositório remoto. Essa opção é realizada por meio do menu \texttt{SRM's Metadata} e, em seguida, \texttt{Fetch SRM's Metadata from repository} (ver Figura~\ref{fig:editor_SRM_metamodel_ECORE_menu}). Após clicar no menu \texttt{Fetch SRM's Metadata from repository}, a Figura~\ref{fig:download_kDM_re_repository} é apresentada, a qual fornece a visualização de todas as instâncias do metamodelo SRM disponíveis para serem reutilizadas. Nesse figura pode ser observado que existem seis instâncias do metamodelo SRM: \texttt{PullUpMethodUnit}, \texttt{ExtractClassUnit}, \texttt{PushDownStorableUnit}, \texttt{PushDownMethodUnit}, \texttt{InLineClassUnit} e outra \texttt{ExtractClassUnit}. KDM-RE permite que o engenheiro de software visualize a refatoração para cada instância do metamodelo SRM. Por exemplo, se o engenheiro almejar visualizar a refatoração escrita em ATL, ele deverá, então, selecionar uma determinada instância do SRM e clicar no botão \texttt{VIEW}. Assim, a refatoração escrita em ATL será apresentada em uma área de texto, como ilustrado na parte inferior da Figura~\ref{fig:download_kDM_re_repository}. Após escolher uma determinada instância do metamodelo SRM, o botão \texttt{DOWNLOAD} deve ser clicado para realizar a transferência da instância do metamodelo SRM e reutilizá-la em seu projeto. 

\begin{figure}[!h]
	\centering
	% Requires \usepackage{graphicx}
	\caption{Visão das instâncias do metamodelo SRM disponíveis no repositório.}
	\label{fig:download_kDM_re_repository}
	\includegraphics[scale=0.6]{images/DOWNLOAD_KDM_RE}
	\fautor
\end{figure}


\section{Módulo de Sincronização}\label{sec:modulo_de_sincronizacao_kdm_re}

Usualmente, durante o desenvolvimento e modernização de software seguindo as diretrizes e passos da abordagem MDE, o software geralmente é modelado e representado utilizando diferentes instâncias de metamodelos para representar as visões e todos os artefatos de um sistema. Em outras palavras, geralmente existem metamodelos para abstrair e representar todos os artefatos do sistema, tais como: metamodelos para o código-fonte, metamodelos para representar e abstrair o banco de dados, metamodelos para representar e abstrair a arquitetura do sistema, etc. Como apresentado no Capítulo~\ref{chapter:fundamentacao_teorica}, o metamodelo KDM é capaz de agrupar todos esses artefatos em um único metamodelo, sendo assim, possível de representar diferentes visões/artefatos e seus relacionamentos de um determinado sistema em uma única instância do metamodelo KDM. Porém, conforme o engenheiro de software aplica um conjunto de refatorações em uma determinada instância do metamodelo KDM mudanças são realizadas. Geralmente, tais mudanças podem necessitar que subsequentes alterações sejam realizadas para que outras visões/artefatos do metamodelo KDM fiquem consistentes e sincronizados.

Uma premissa fundamental é manter todas as visões/artefatos do metamodelo KDM sincronizadas durante todo o processo de modernização do software. Dessa forma, quando as visões/artefatos representadas em nível de modelos forem alteradas, é de extrema importância realizar um conjunto de propagação de mudança por todas as visões/artefatos para mantê-las atualizadas e sincronizadas, espelhando, assim, a alteração em todas as visões/artefatos do software. Usualmente, como apresentado nos Capítulo~\ref{chapter:fundamentacao_teorica}, Seção~\ref{sec:refatoracao} e Capítulo~\ref{chapter:catalogo_refactoring_KDM}, essas alterações podem ser realizadas por meio de refatorações, as quais são atividades centrais durante o processo de modernização. Porém, quando um software é representado utilizando diferentes abstrações e visões em nível de modelos, um acidente comum que pode ocorrer durante a atividade de refatoração é a dessincronização dessas visões, fazendo com que as visões/artefatos que representam o sistema fiquem inconsistentes após a atividade de refatoração. Uma forma de resolver esse problema, é aplicar técnicas de propagação de mudança, cujo objetivo é identificar e atualizar todas as instâncias dependentes dos elementos que foram refatorados. % No entanto, a maioria das propostas de propagação de mudança foram desenvolvidas para propagarem mudanças em diferentes metamodelos, além disso, usualmente tais metamodelos são de diferentes fornecedores dificultando o entendimento e a programação de mudança (ref). 
Diante desse contexto, a ferramenta KDM-RE, possui um módulo de sincronização para realizar a propagação de mudança e preservação de comportamento após a aplicação de refatorações em instâncias do metamodelo KDM. Utilizando esse módulo, engenheiros de software podem se concentrar apenas na aplicação das refatorações ou reutilizá-las por meio do metamodelo SRM (ver Capítulo~\ref{chapter:Toward_a_Refactoring_Metamodel_for_KDM}), sem se preocuparem com a propagação de mudanças para outras visões/artefatos do metamodelo KDM. 

É importante destacar que o fluxo desse módulo de sincronização inicia-se considerando que o engenheiro de software almeja aplicar um conjunto de refatorações em um sistema que já esteja representado por meio de uma instância do metamodelo KDM. Essa instância deve ser a mais completa possível, ou seja, represente todas as visões/artefatos do sistema, desde o código-fonte até os elementos arquiteturais do sistema\footnote{Na verdade, é importante que mais de uma visão/artefato seja representada utilizando o metamodelo KDM, seja código-fonte, banco de dados, elementos estruturais, etc.}. Após o engenheiro de software aplicar uma determinada refatoração, o módulo de sincronização, a qual contém três principais passos, efetivamente será iniciado. De forma resumida, os três passos do módulo da seguinte são descritos a seguir. 

O primeiro passo realiza uma comparação (do inglês - \textit{diff}) entre a instância refatorada do metamodelo KDM com a instância do metamodelo KDM original, ou seja, a instância do metamodelo KDM antes do engenheiro de software aplicar a refatoração. Como resultado, esse passo cria uma lista que contém todas as instâncias das metaclasses do KDM que sofreram uma modificação durante a refatoração, quando comparado com a instância do KDM original. Em seguida, o segundo passo utiliza como entrada a lista gerada para ser utilizada como parâmetro para um algoritmo de mineração e identificação de dependências. Esse algoritmo visa identificar todas as instâncias das metaclasses do KDM que possuem dependência com as metaclasses refatoradas. Como resultado, esse algoritmo também cria uma lista, a qual é utilizada no terceiro passo. O terceiro passo utiliza a lista criada pelo algoritmo para realizar um conjunto de transformações em nível de modelo. Tais transformações foram pré-definidas e representam as propagações de mudanças por todas as visões do KDM. É importante destacar que o módulo de sincronização da KDM-RE foi implementada com a preocupação de ser uma forma genérica e desacoplada, assim, pode ser aplicado em um grande conjunto de refatorações, fazendo com que o engenheiro de software não tenha que se preocupar com a propagação de mudança para outras visões/artefatos do KDM. 

%Para exemplificar os três principais passos do módulo de sincronização essa
%As demais seções deste capítulo estão organizadas da seguinte forma: na Seção~\ref{sec:kdm_sinc} a abordagem KDM-SInc é descrita, na Subseção~\ref{sec:diff_entre_kdm} o primeiro passo da abordagem KDM-SInc é apresentado, o segundo passo é apresentado na SubSeção~\ref{subsec:identificandoPontoParaExecutarApropagacao}, e na Subseção~\ref{subsec:aplicar_propagacao_KDM-SInc}. Na Seção~\ref{sec:consideracoes_finals_kdm_sinc} as considerações finais deste capítulo são apresentadas.


%\section{A Abordagem KDM-SInc}\label{sec:kdm_sinc}

Como já mencionado, um problema crítico durante a modernização de software diz respeito à propagação de mudança, por exemplo, dado um conjunto de refatorações que são aplicadas durante a modernização de software, é importante identificar quais são as mudanças que precisam ser realizadas para manter a consistência e sincronia de todos os artefatos do sistema. Dessa forma, propagação de mudança é uma técnica de extrema importância durante a elaboração de processo de modernização de software. O engenheiro de software tem que ter a certeza de que a refatoração foi corretamente propagada e que o software não possui nenhuma inconsistência. Embora muitas abordagens de propagações de mudanças possam ser identificadas na literatura, a propagação de mudanças ainda é um desafio durante a manutenção e modernização de software~\cite{Tom_2008_roadmap}. Além disso, a maioria das abordagens de propagação de mudanças existentes tem como principal artefato o código-fonte~\cite{Vaclav_methodology, Deursen07model_drivensoftware}. Similarmente, também é possível identificar algumas abordagens que dão suporte para a propagação de mudança para o metamodelo UML~\cite{Egyed_2008,Liu02rule, Briand_2006}. Porém, até o momento nenhuma iniciativa foi criada para o metamodelo KDM. Para suprir essa limitação, a KDM-RE possui um módulo de sincronização, o qual visa propagar mudanças por todas as visões/artefatos do metamodelo KDM para mantê-lo atualizado e sincronizado após a aplicação de uma refatoração. A intenção é criar um apoio computacional que mantenha uma determinada instância do metamodelo KDM consistente e sincronizada entre todas as visões/artefatos do metamodelo KDM após a aplicação de uma determinada refatoração. 

\begin{figure}[h]
	\centering
	% Requires \usepackage{graphicx}
	\caption{Visão Geral do Módulo de Sincronização.}
	\label{fig:kdm_sinc}
	\includegraphics[scale=0.65]{images/AbordagemKDM_SInc}
	\fautor
\end{figure}


Na Figura~\ref{fig:kdm_sinc}, é apresentada uma visão geral do módulo de sincronização, o qual contempla três principais passos. Antes dos passos do módulo se iniciarem, uma atividade de refatoração deve ser realizada. Essa atividade está representada na caixa cinza da Figura~\ref{fig:kdm_sinc} \ding{202} e é apoiada pelo módulo de refatoração apresentado na Seção~\ref{sec:modulo_de_refatoracao_kdm_re}. A atividade de refatoração esta fora do escopo do módulo de sincronização, assim, é de responsabilidade do engenheiro de software aplicar e/ou reutilizar refatoração para o metamodelo KDM e executa-lá em uma instância do metamodelo KDM. A única restrição do módulo de sincronização da KDM-RE é que duas versões da instância do metamodelo KDM sejam utilizadas como entrada - uma versão que represente a instância do metamodelo KDM antes da aplicação das refatorações (\aspas{instância original}) e outra versão que represente uma instância do metamodelo KDM após a aplicação de \textit{n} refatorações (\aspas{instância refatorada}). Note que uma \aspas{instância original} do metamodelo KDM é aquela que ainda não foi refatorada, também denominada de \aspas{KDM esquerdo}, e, similarmente, a \aspas{instância refatorada} do metamodelo KDM pode ser entendida como \aspas{KDM direito}. 

Após a aplicação de um conjunto de refatorações, o primeiro passo do módulo de sincronização é iniciado. Nesse passo uma comparação (\textit{diff}) entre a instância original e a instância refatorada é realizada. Como resultado desse passo uma lista é criada, contendo todas as instâncias das metaclasses do KDM que sofreram alguma modificação durante a refatoração quando comparada com a instância do KDM original. Além disso, essa lista também especifica qual(is) foi(ram) a(s) modificação(ões) realizada(s). Por exemplo, se na versão refatorada (\aspas{KDM direito}) uma nova instância da metaclasse \texttt{ClassUnit} foi adicionada, a lista irá conter duas importantes informações: (\textit{i}) a instância da metaclasse \texttt{ClassUnit} e (\textit{ii}) qual operação foi realizada nesse exemplo \texttt{add} \texttt{ClassUnit}. Essas duas informações são importantes para identificar o que foi alterado e qual operação foi realizada (nesse caso, uma \texttt{ClassUnit} foi adicionada); assim, é possível identificar quais propagações devem ser realizadas nas outras visões/pacotes do metamodelo KDM.

Em seguida, o segundo passo do módulo de sincronização é iniciado, identificando todas as metaclasses que precisam ser sincronizadas/atualizadas após a aplicação da refatoração. Utiliza-se, aqui, o algoritmo de busca em profundidade (do inglês - \sigla{DFS}{\textit{Depth-First Search}}). O algoritmo DFS foi alterado para utilizar os seguintes parâmetros como entrada: (\textit{i}) a lista criada no primeiro passo e (\textit{ii}) a instância refatorada do KDM (\aspas{KDM direito}). Utilizando a instância refatorada, o algoritmo DFS identifica e cria uma lista que contém todas as metaclasses que possuem dependência com as metaclasses que efetivamente foram refatoradas.

Posteriormente, o último passo pode ser iniciado para realizar a propagação de mudanças na instância do KDM. Como entrada, esse passo utiliza todas as metaclasses que possuem dependência com as metaclasses que foram refatoradas (lista criada no passo anterior). As propagações de mudanças são um conjunto de regras pré-definidas e realizadas de acordo com a instância alterada (\texttt{Package}, \texttt{ClassUnit}, \texttt{MethodUnit}, \texttt{StorableUnit}, etc.) e sua operação atômica (\texttt{add}, \texttt{delete} e \texttt{change}). Após o término desse último passo, todas as visões/artefatos da instância do KDM estão sincronizadas e consistentes.

%É importante salientar que os três passos do módulo de sincronização são executados várias vezes até que não haja mais elementos que precisem ser atualizados/sincronizados. Esse ciclo é necessário uma vez que uma determinada instância do metamodelo KDM pode ainda exigir propagações em outros artefatos/visões, por isso, cada ciclo da abordagem KDM-SInc se concentra apenas no próximo nível de propagação. A condição de parada da abordagem KDM-SInc é quando o algoritmo DFS, definido no segundo, retornar uma lista vazia, indicando que não há mais elementos que precisam ser modificados.

Para auxiliar a elaboração do primeiro passo do módulo de sincronização, o \textit{framework} EMFCompare\footnote{https://www.eclipse.org/emf/compare/} foi estendido para comparar instâncias do metamodelo KDM. O segundo passo é apoiado por um motor de busca, cuja parte central é o algoritmo DFS juntamente com um conjunto de expressões definidas em linguagem de buscas que são executadas em uma instância do metamodelo KDM para obter todos os pacotes do KDM. O último passo do módulo de sincronização é apoiado por um motor de propagação, o qual utiliza um conjunto de regras pré-definidas e implementadas em ATL para executar as propagações. Todas as propagações foram definidas com base nas instâncias das metaclasses alteradas juntamente com as operações atômicas (\texttt{add}, \texttt{delete} e \texttt{change}), apresentadas no Capítulo~\ref{chapter:catalogo_refactoring_KDM}. Maiores detalhes sobre cada passo do módulo de sincronização são apresentados a seguir. %Porém, detalhes técnicas da abordagem KDM-SInc são omitidos nesse capítulo. Informações técnicas sobre a abordagem KDM-SInc são salientadas no Capítulo X onde o apoio computacional KDM-RE é apresentado, mais especificadamente na Seção X, onde o módulo de propagação é apresentado.

%Para auxiliar a elaboração do passo [A] o \textit{framework} EMFCompare\footnote{https://www.eclipse.org/emf/compare/} foi estendido para comparar instâncias do metamodelo KDM. O passo [B] é tecnicamente apoiado por um motor de busca cuja parte central é o algoritmo DFS juntamente com um conjunto de expressões definidas em XPath que são executadas em uma instância do metamodelo KDM para obter todos os pacotes do KDM. Finalmente, o passo [C] é apoiado por um motor de propagação, o qual utiliza um conjunto de transformações pré-definidas em ATL para executar as propagações. Todas as propagações foram definidas com base nas operações atômicas (\texttt{add}, \texttt{delete} e \texttt{change}) apresentadas no Capítulo X \change{Mudar aqui}. Dessa forma, quando um conjunto de refatorações são executadas um conjunto de propagações bem definidas podem ser executadas no contexto de instâncias do metamodelo KDM. Maiores detalhes sobre da passo da abordagem KDM-SInc são apresentados nas próximas seções.

\subsection{Identificar \textit{Diff} entre Instâncias do Metamodelo KDM}\label{sec:diff_entre_kdm}

Nessa seção, o primeiro passo do módulo de sincronização é apresentado e discutido. Como já salientado o primeiro passo é apoiado pelo \textit{framework} EMFCompare, o qual foi escolhido, pois pode ser facilmente adaptado e estendido, além de implementar um algoritmo de similaridade de instâncias de metamodelo eficiente. A fim de entender melhor como o primeiro passo do módulo de sincronização funciona, deve-se considerar os seguintes três subpassos: (\textit{i}) \textit{Matching}, (\textit{ii}) \textit{Diffing} e (\textit{iii}) Análise dos \textit{Diffs}, como apresentado na Figura~\ref{fig:diff_emf_compare}. 

\begin{figure}[h]
	\centering
	% Requires \usepackage{graphicx}
	\caption{Visão Geral da Execução do Módulo de Sincronização.}
	\label{fig:diff_emf_compare}
	\includegraphics[scale=0.7]{images/matching_diffing_analise_3}
	\fautor
\end{figure}

Com pode ser observado na Figura~\ref{fig:diff_emf_compare}, o primeiro subpasso, \textit{matching}, necessita de duas instâncias do metamodelo KDM - uma instância original (\aspas{KDM esquerdo}), denominada \textbf{versão 1} na Figura~\ref{fig:diff_emf_compare}, e uma instância refatorada (\aspas{KDM direito}), \textbf{versão 2} na Figura~\ref{fig:diff_emf_compare}. Dadas essas duas instâncias do KDM, os correspondentes elementos nas duas versões do metamodelo KDM são identificados. Em uma instância do KDM, cada elemento possui um identificar único, não volátil e persistente. Portanto, os correspondentes elementos são identificados por meio desses identificadores únicos, como XMI IDs. Por exemplo, ainda analisando a Figura~\ref{fig:diff_emf_compare}, pode-se observar que a instância da metaclasse \texttt{ClassUnit} \aspas{Artista}, apresentada na \textbf{versão 1}, corresponde à instância da metaclasse \texttt{ClassUnit} \aspas{Artista} na \textbf{versão 2}. Para as instâncias das metaclasses \texttt{ClassUnit} \aspas{Ator} e \texttt{Extends} na \textbf{versão 2}, nenhum elemento correspondente foi identificado na \textbf{versão 1}. É visto que para cada elemento correspondente identificado, ou não identificado, um elemento \textit{match} é criado e será utilizado no subpasso seguinte.

No segundo subpasso, \textit{Diffing}, todos os correspondentes elementos identificados são examinados para identificar diferenças em seus meta-atributos. Para cada diferença identificada, um objeto \textit{diff} é criado, o qual descreve com precisão cada diferença identificada entre os correspondentes elementos. Por exemplo, ainda considerando a Figura~\ref{fig:diff_emf_compare}, quando as instâncias da metaclasse \texttt{ClassUnit} \aspas{Artista} da \textbf{versão 1} e \textbf{versão 2} são examinadas, é possível observar que o meta-atributo \texttt{isAbstract} possui o valor \textit{false} na \textbf{versão 1}, e que na \textbf{versão 2} o mesmo meta-atributo apresenta o valor \textit{true}, representando a operação \texttt{change}. Instâncias de metaclasses que não contêm elementos correspondentes em ambas as versões são consideradas adicionadas ou deletadas (\texttt{add} e \texttt{delete}). A operação é identificada dependendo da direção; por exemplo, se uma instância de uma metaclasse apenas existe do lado direito (\aspas{KDM direito}), essa instância foi adicionada, por outro lado, se uma instância apenas existe do lado esquerdo (KDM esquerdo), essa instância foi deletada. Na Figura~\ref{fig:diff_emf_compare}, é possível identificar que duas instâncias foram adicionadas - uma instância da metaclasse \texttt{ClassUnit} denominada Ator e uma instância da metaclasse \texttt{Extends}.

%Detecting refactorings may be realized by tracking the user’s operations. Operation-based conflict detection is very precise, because every change is recorded and the complete operation sequence may be replayed in the correct order.

Subsequentemente, o terceiro subpasso, Análise dos \textit{Diffs}, é executado, no qual todos objetos \textit{diffs} criados anteriormente são examinados para criar uma lista de dependência contendo as instâncias das metaclasses alteradas e quais operações foram realizadas. No exemplo apresentado na Figura~\ref{fig:diff_emf_compare}, a lista criada possui três dependências. A primeira dependência informa que o meta-atributo \texttt{isAbstract} da metaclasse \texttt{ClassUnit} Artista da \textbf{versão 1} foi alterado (\texttt{change}) de \textit{false} para \textit{true} na \textbf{versão 2}. A segunda dependência, ilustra que uma instância da metaclasse \texttt{ClassUnit} Ator foi adicionada (\texttt{add}) na \textbf{versão 2}, e a terceira dependência representa que uma instância da metaclasse \texttt{Extends} foi adicionada na \textbf{versão 2}.


\subsection{Identificar Pontos para Executar a Propagação}\label{subsec:identificandoPontoParaExecutarApropagacao}

Nessa seção, o segundo passo do módulo de sincronização é apresentado. Esse passo resume-se basicamente na adaptação do algoritmo DFS para identificar todas as metaclasses que precisam ser sincronizadas/atualizadas após a aplicação da refatoração. Esse algoritmo utiliza como entrada a lista criada no passo anterior. Como as instâncias do metamodelo KDM são persistidas utilizando a padronização XMI, o algoritmo precisa de uma forma para buscar as dependências nesse XMI. Assim, esse passo utiliza expressões em XPath~\cite{kay2011xslt} que são executadas na instância do metamodelo KDM para obter todos os pacotes do KDM. Por exemplo, na Figura~\ref{fig:xpath_queries}, são apresentadas algumas expressões definidas em XPath, utilizadas antes da aplicação do algoritmo DFS. A primeira expressão retorna a metaclasse \texttt{Segment} que é o elemento inicial de qualquer instância do metamodelo KDM. As outras expressões ilustradas nas linhas 2-5 representam os outros pacotes do metamodelo KDM. Os elementos retornados nas expressões XPath são também utilizados como entrada para o algoritmo DFS.

\begin{figure}[h]
	\centering
	% Requires \usepackage{graphicx}
	\caption{Expressões definidas em XPath para obter os pacotes do KDM.}
	\label{fig:xpath_queries}
	\includegraphics[scale=0.68]{images/queiresANDATLSBESNew}
	\fautor
\end{figure}

O Algoritmo~\ref{alg:death1} ilustra como o DFS identifica todas as metaclasses que precisam ser sincronizadas/atualizadas após a aplicação da refatoração. A Figura~\ref{fig:dfsalg} apresenta como é o funcionamento do algoritmo DFS. Cada nó representa uma instância de uma metaclasse do metamodelo KDM e os vértices representam os relacionamentos entre as instâncias das metaclasses, por exemplo, o nó \texttt{A} representa uma instância da metaclasse \texttt{Segment} e os nós \texttt{K}, \texttt{H}, \texttt{E} e \texttt{B} representam instâncias das metaclasses \texttt{CodeModel}, \texttt{StructureModel}, \texttt{ConceptualModel} e \texttt{DataModel}, respectivamente. Mais especificamente, o algoritmo funciona da seguinte forma: primeiro é necessário escolher um ponto inicial de partida, no caso do módulo de sincronização o ponto inicial é a instância da metaclasse \texttt{Segment}, metaclasse raiz de qualquer instância do metamodelo KDM. Depois, a instância da metaclasse \texttt{Segment} deve ser visitada, adicionada em uma pilha e marcada como visitada. Posteriormente, o algoritmo visita outra instância de outra metaclasse que ainda não foi visitada e verifica se ela possui uma associação do tipo \texttt{implementation}. Caso afirmativo, o algoritmo deve verificar se essa associação possui referência para algum elemento identificado na lista gerada no primeiro passo; se a associação apresentar um elemento, ele deverá ser adicionado em outra pilha e marcado como visitado. Todo esse processo continua até que o algoritmo alcance a última metaclasse instanciada no KDM. 

\begin{algoritmo}[h]
     \SetAlgoLined
     \KwIn{DFS (G, u) onde \textit{G} é uma instância do  KDM, \textit{u} é a metaclasse inicial obtida pela expressão XPath, ou seja, \texttt{Segment}}
     \KwOut{Uma coleção de metaclasses, as quais precisam ser sincronizadas}
     \Begin{
     \ForEach{$outgoing$ edge e = (u, v) of u} {
	\If{vertex v as has not been visited}{
			\If{vertex v contain implementation = true }{
				
				\ForEach{$implementations$ element}{
				verify all elements in implementation
				}
				Mark vertex v as visited (via edge e).
				Recursively call DFS (G, v).
			}
			
				}				
			}		
	
	}
     \caption{Algoritmo DFS.}
     \label{alg:death1}
   \end{algoritmo}

\begin{figure}[h]
	\centering
	% Requires \usepackage{graphicx}
	\caption{Funcionamento do Algoritmo DFS.}
	\label{fig:dfsalg}
	\includegraphics[scale=0.3]{images/algWorks2}
	\fautor
\end{figure}

Em seguida, o algoritmo ainda verifica se a instância da metaclasse \texttt{Segment} possui alguma instância adjacente que ainda não foi marcada como visitada. Caso o algoritmo identifique uma instância de metaclasse adjacente ainda não visitada todo o processo é iniciado novamente, sempre verificando a associação \texttt{implementation}. Quando o algoritmo finalmente alcançar a última instância da metaclasse, ou seja, todas as instâncias de metaclasse do KDM foram visitadas e verificadas corretamente, o algoritmo criará uma lista contendo todas as instâncias das metaclasses afetadas na refatoração. 

\subsection{Aplicar Propagação}\label{subsec:aplicar_propagacao_KDM-SInc}

Nesta seção, o terceiro passo do módulo de sincronização é apresentado. Esse passo objetiva realizar as mudanças e propagações necessárias para manter uma determinada instância do metamodelo KDM sincronizada e consistente. A sincronização é importante para o metamodelo KDM, uma vez que ele possui metaclasses que contêm conexões diretas com outras metaclasses de outras visões/artefatos do KDM. Assim, manter a instância do metamodelo KDM sincronizado e consistente após a aplicação de uma refatoração é importante. 

No contexto desta tese, como apresentado no Capítulo~\ref{chapter:catalogo_refactoring_KDM}, as refatorações que são criadas para o metamodelo KDM são de baixa granularidade e aplicadas diretamente na camada \texttt{Code} do metamodelo KDM. Contudo, uma determinada refatoração pode demandar outras modificações que deveriam ser realizadas em outras camadas/visões do metamodelo KDM para mantê-lo consistente e sincronizado. Por exemplo, na refatoração \textit{Rename Package}, o nome de um determinado pacote é alterado de PacoteX para PacoteY se uma instância da metaclasse \texttt{Layer}\footnote{Metaclasse definida no pacote \texttt{Structure} do metamodelo KDM para representar camadas em âmbito arquitetural.} for utilizada para representar o pacote em âmbito arquitetural, essa mesma instância da metaclasse \texttt{Layer} também deve ser renomeada. 

Esse passo utiliza um conjunto de regras pré-definidas que são iniciadas de acordo a(s) refatoração(ões) aplicada(s) na instância do metamodelo KDM. Mais detalhadamente, todas as propagações especificadas nesse passo são pré-definidas para serem disparadas após a aplicação de específicas refatorações. Todas as propagações são definidas com base nas mudanças realizadas em uma determinada instância de metaclasses do metamodelo KDM. Além disso, todas as regras pré-definidas foram implementadas em ATL. No Apêndic~\ref{apendice:regras_propagacao} nas Tabelas~\ref{tab:propagacaoes_kdm_sinc_package},~\ref{tab:propagacaoes_kdm_sinc_classUnit},~\ref{tab:propagacaoes_kdm_sinc_StorableUnit} e \ref{tab:propagacaoes_kdm_sinc_method} todas as regras de programação pré-definidas são explicadas.  

\subsection{Exemplo de execução do Módulo de Sincronização}

Como apresentado no Capítulo~\ref{chapter:catalogo_refactoring_KDM}, a refatoração \texttt{Extract ClassUnit} pode ser criada por um conjunto de duas operações atômicas: \texttt{add} e  \texttt{delete}. Dessa forma, o módulo de sincronização identifica que a refatoração \texttt{Extract ClassUnit} executou um conjunto de operações atômica (\texttt{add} e \texttt{delete}) e cria uma lista para ser utilizada no passo seguinte. No contexto da refatoração \texttt{Extract ClassUnit}, apresentada na Figura~\ref{fig:kdm_re_wizard_extract_class}, as seguintes operações foram realizadas: 

\begin{itemize}

\item \texttt{add} uma instância de \texttt{ClassUnit} denominada \texttt{Document};

\item \texttt{delete} uma instância de \texttt{StorableUnit} denominada CPF da \texttt{ClassUnit} \texttt{Cliente};

\item \texttt{add} uma instância de \texttt{StorableUnit} na \texttt{ClassUnit} \texttt{Document};

\item \texttt{add} uma instância de \texttt{StorableUnit} do tipo \texttt{Document} na \texttt{ClassUnit} \texttt{Cliente};

\item \texttt{add} duas instâncias de \texttt{MethodUnit} para representar os métodos assessores na \texttt{ClassUnit} \texttt{Cliente}.
\end{itemize}

Depois, o módulo de sincronização executa o algoritmo DFS para identificar todas as metaclasses que precisam ser sincronizadas/atualizadas após a aplicação da refatoração. Esse algoritmo utiliza como entrada a lista criada no passo anterior. Expressões definidas em XPath são utilizadas para navegar em todas as visões da instância do metamodelo KDM. Ao término da execução desse algoritmo, ele irá criar uma lista que contém todas as instâncias das metaclasses afetadas na refatoração. 
   

O terceiro passo do módulo de sincronização objetiva realizar as mudanças e propagações necessárias para manter uma determinada instância do metamodelo KDM sincronizada e consistente. Utilizando a lista de operações realizadas na refatoração e a lista que possui os elementos afetados na refatoração, o módulo de programação utiliza um conjunto de regras pré-definidas para realizar a propagação. Todas as propagações estão apresentadas nas Tabelas~\ref{tab:propagacaoes_kdm_sinc_package},~\ref{tab:propagacaoes_kdm_sinc_classUnit},~\ref{tab:propagacaoes_kdm_sinc_StorableUnit} e~\ref{tab:propagacaoes_kdm_sinc_method} do Apêndice~\ref{apendice:regras_propagacao}. Para a refatoração \texttt{Extract ClassUnit} utilizada como exemplo, as propagações que serão realizadas são:

\begin{figure}[!h]
	\centering
	% Requires \usepackage{graphicx}
	\caption{Instância antes e após a execução do Módulo de Sincronização.}
	\label{fig:efeitoPropagacaoKDMSINC}
	\includegraphics[scale=0.6]{images/propagacaoKDMEfeito}
	\fautor
\end{figure}

\begin{itemize}
    \item \texttt{add} uma instância de \texttt{RelationalTable} com o meta-atributo \texttt{name} denominado \texttt{Document};
    
    \begin{itemize}
        \item \texttt{add} uma instância da metaclasse \texttt{UniqueKey} na instância \texttt{RelationalTable} criada;
    \end{itemize}
    
    \item \texttt{delete} uma instância de \texttt{ColumnSet} denominada CPF da \texttt{RelationalTable} \texttt{Pessoa};
    
    \item \texttt{add} uma instância de \texttt{ColumnSet} com o meta-atributo \texttt{name} denominado CPF na instância da metaclasse \texttt{RelationalTable} \texttt{Document};
    
    \item \texttt{add} uma instância de \texttt{ColumnSet} com o meta-atributo \texttt{name} denominado \texttt{document} na \texttt{RelationalTable} \texttt{Pessoa}.

    %\item \texttt{add} duas instância de \texttt{MethodUnit} para representar os métodos assessores na \texttt{ClassUnit} \texttt{Cliente}.
    
\end{itemize}

Adicionalmente, a KDM-RE fornece uma forma de visualizar graficamente se as propagações foram realmente executadas. Por exemplo, a Figura~\ref{fig:efeitoPropagacaoKDMSINC} representa duas instâncias do metamodelo KDM. A instância apresentada do lado esquerdo é denominada \textbf{versão 1} e equivale à instância antes (original) de aplicar o módulo de sincronização. A instância do lado direito é intitulada de \textbf{versão 2} e representa uma instância do metamodelo KDM sincronizada e propagada de acordo com os passos e regras definidas pela abordagem KDM-SInc.% nas Tabelas~\ref{tab:propagacaoes_kdm_sinc_package},~\ref{tab:propagacaoes_kdm_sinc_classUnit},~\ref{tab:propagacaoes_kdm_sinc_StorableUnit} e~\ref{tab:propagacaoes_kdm_sinc_method}.





%Como observado a instância apresentada do lado direito da Figura~\ref{fig:efeitoPropagacaoKDMSINC} representa a instância do metamodelo KDM sincronizada e propagada. 


%A sincronização é importante para o metamodelo KDM uma vez que o mesmo possui metaclasses que contêm conexões diretas com outras metaclasses de outras visões/artefatos do KDM. Assim, manter a instância do metamodelo KDM sincronizado e consistente após a aplicação de uma refatoração é importante. 

%No contexto desta Tese, como apresentado no Capítulo~\ref{chapter:catalogo_refactoring_KDM} as refatorações que são definidas e adaptadas para o metamodelo KDM são refatorações de baixa granularidade e que são aplicadas diretamente na camada \texttt{Code} do metamodelo KDM. Porém, uma determinada refatoração pode demandar outras modificações que deveriam ser realizadas em outros camadas/visões do metamodelo KDM para mantê-lo consistente e sincronizado. Por exemplo, considere a refatoração \textit{Rename Package} - o nome de um determinado pacote é alterado de PacoteX para PacoteY, se uma instância da metaclasse \texttt{Layer}\footnote{Metaclasse definida no pacote \texttt{Structure} do metamodelo KDM para representar camadas em nível arquitetural.} é utilizada para representar o pacote em nível arquitetural, então essa mesma instância da metaclasse \texttt{Layer} também deve-se ser renomeada. 

%Diferentemente da atividade de refatoração apresentada no Capítulo~\ref{chapter:catalogo_refactoring_KDM} onde o engenheiro de modernização precisa escolher qual refatoração aplicar na instância do metamodelo KDM ou reutilizar um refatoração utilizando o metamodelo SRM apresentado no Capítulo~\ref{chapter:Toward_a_Refactoring_Metamodel_for_KDM}, esse passo utiliza um conjunto de regras pré-definidas que são iniciadas de acordo a(s) refatoração(ões) aplicada(s) na instância do metamodelo KDM. Mais especificadamente, todas as propagações especificadas nesse passo são pré-definidas para serem disparadas após a aplicação de específicas refatorações. Algumas propagações são realizadas sem a interferência do engenheiro de modernização, no entanto, existem propagações que precisam de informações importantes para realizar consistentemente a propagação. Assim, algumas propagações podem ser realizadas de forma totalmente automática, enquanto outras precisam de algumas informações antes de executar a propagação propriamente dita.

%Esse passo da abordagem KDM-SInc utiliza um conjunto de regras pré-definidas para propagar todas as mudanças em uma instância do metamodelo KDM com o intuito de mantê-lo consistente e sincronizado com todas as visões/artefatos. Todas as propagações são definidas com base nas mudanças realizadas em um determinada instância de metaclasses do metamodelo KDM. A abordagem KDM-SInc 

%Por exemplo, considere a Figura~\ref{fig:extractClassRefactoring} onde o engenheiro de modernização está aplicando a refatoração \texttt{Extract ClassUnit} em uma instância de \texttt{ClassUnit} denominada \texttt{Pessoa}. O primeiro passo do módulo de sincronização irá realiazar uma comparação entre a instância refatorada e a instância original do metamodelo KDM. Como apresentado no Capítulo X\change{mudar} a refatoração \texttt{Extract ClassUnit} é uma refatoração composta, assim, é necessário aplicar um conjunto de operações atômicas: \texttt{add}, \texttt{delete} e \texttt{change}.

\section{Trabalhos Relacionados}\label{sec:trabalhos_relacionados_ferramentas_kdm_re}

Nesta seção, são descritos os principais trabalhos relacionados com ferramentas que aplicam refatorações em modelos encontradas na literatura. Foram identificadas algumas ferramentas que nortearam o desenvolvimento da KDM-RE, assim, nesta seção, são mostradas as principais semelhanças e diferenças encontradas.

\citeonline{arendt2013tool} em seu artigo propõem uma ferramenta denominada EMF Refactor para aplicar refatorações em modelos EMF. Na Figura~\ref{fig:arquitetura_emf_refactor}, é apresentada a arquitetura da ferramenta EMF Refactor. Como pode ser observado, a EMF Refactor também foi desenvolvimento para ser utilizado no ambiente de desenvolvimento Eclipse. EMF Refactor também utiliza o \textit{framework} EMF Compare para mostrar os efeitos da refatoração. De acordo com~\citeonline{arendt2013tool}, a EMF Refactor permite realizar a identificação de \textit{bad smells} e, em seguida, o usuário pode aplicar refatorações. Da mesma forma que a KDM-RE, um conjunto de refatorações propostas por~\citeonline{Fowler1999} foi implementado na EMF Refactor. Por exemplo, na Figura~\ref{fig:wizard_emfCOmpare} é possível visualizar quais refatorações o usuário pode aplicar por meio da EMF Refactor. 

\begin{figure}[h]
	\centering
	% Requires \usepackage{graphicx}
	\caption{Arquitetura da EMF Refactor.}
	\label{fig:arquitetura_emf_refactor}
	\includegraphics[scale=0.9]{images/EMFRefactorArchitecture}
	\fadaptada{hostesupporting}
\end{figure}


\begin{figure}[h]
	\centering
	% Requires \usepackage{graphicx}
	\caption{\textit{Wizard} para aplicar refatoração do EMF Refactor.}
	\label{fig:wizard_emfCOmpare}
	\includegraphics[scale=0.7]{images/EMFRefactorWizard}
	\fdireta{hostesupporting}
\end{figure}

Existem duas principais diferenças entre a KDM-RE e a ferramenta EMF Refactor. A primeira é que a ferramenta EMF Refactor não se preocupa em sincronizar as instâncias de um determino metamodelo após a aplicação das refatorações; diferentemente da KDM-Re, a qual define um módulo de sincronização exclusivo para essa característica. A segunda diferença está relacionada com a identificação de qual(is) refatoração(ões) aplicar. Por meio da EMF Refactor, é possível identificar quais refatorações precisam ser realizadas, ou seja, a ferramenta permite identificar quais são os \textit{bad smells}; KDM-RE no seu estado atual não permite a identificação de \textit{bad smells}.


\begin{figure}[h]
	\centering
	% Requires \usepackage{graphicx}
	\caption{Arquitetura da ferramenta Refactory.}
	\label{fig:refactory}
	\includegraphics[scale=0.2]{images/refactoryArchitecture}
	\fdireta{Reimann_2015}
\end{figure}

Outra ferramenta relacionada à KDM-RE é a Refactory\footnote{\texttt{http://www.modelrefactoring.org/index.php/Refactoring}}. A arquitetura da ferramenta Refactory é apresentada na Figura~\ref{fig:refactory}. Da mesma forma que a KDM-RE, Refactory também contém três camadas: (\textit{i}) \textit{Platform Layer}, (\textit{ii}) \textit{Core Layer} e (\textit{iii}) \textit{UI layer}. A camada \textit{Core} contém três módulos: (\textit{i}) \textit{Quality Smell Framework}, (\textit{ii}) \textit{Refactoring Framework} e (\textit{iii}) \textit{Co-Refactoring Framework}. Note que, Refactory também é baseada no EMF e pode refatorar qualquer metamodelo definido pelo EMF. Além disso, de acordo com~\citeonline{reimann2010role}, Refactory, foi criada para permitir refatorações genéricas. Os autores afirmam que, realizar uma refatoração, por exemplo, \textit{extract method} em Java é o mesmo mecanismo para um sistema representado por instância da UML. As refatorações são todas implementas em QVT e as asserções são implementadas em OCL. As principais semelhanças com a KDM-RE são: (\textit{i}) Refactory também permite aplicar refatorações graficamente por meio de diagramas da UML, (\textit{ii}) refatorações são executadas utilizando QVT, uma linguagem de transformação similar a ATL, (\textit{iii}) restrições (pré- e pós-condições) são implementas em OCL e (\textit{iv}) Refactory também não identifica \textit{bad-smells}. As principais diferenças são: (\textit{i}) Refactory não se preocupa em sincronizar/propagar o metamodelo após a aplicação de refatorações e (\textit{ii}) Refactory não utiliza o metamodelo KDM para aplicar as refatorações.



~\citeonline{mohamed2011m} propõem uma abordagem e uma ferramenta denominada M-Refactor. M-Refactor possui duas principais funcionalidades: (\textit{i}) identificar \textit{bad-smells} em instâncias do metamodelo UML e (\textit{ii}) aplicar refatorações em instâncias do metamodelo UML. A identificação dos \textit{bad-smells} é realizada em diagramas de classe e diagramas de sequência da UML. Os \textit{bad-smells} identificados são representados graficamente como apresentado na Figura~\ref{fig:m_refactor}. Após a identificação e visualização dos \textit{bad-smells}, o usuário pode aplicar as refatorações. As principais semelhanças entre KDM-RE e M-Refactor são: (\textit{i}) ambas utilizam diagramas da UML para aplicar as refatorações, (\textit{ii}) M-Refactor também utiliza linguagens de transformações para executar as refatorações e (\textit{iii}) linguagens de restrições (pré- e pós-condições) também são especificadas na M-Refactor. As principais diferenças entre M-Refactor e KDM-RE são: (\textit{i}) M-Refactor pode ser utiliza para identificar \textit{bad-smells} em instâncias do metamodelo UML, (\textit{ii}) M-Refactor não implementa nenhum mecanismo para propagar/sincronizar a instância do metamodelo UML após a aplicação de um conjunto de refatorações e (\textit{iii}) M-Refactor não utiliza o metamodelo KDM como base para aplicar as refatorações.

%Da mesma forma que a KDM-RE, M-Refactor também permite aplicar refatorações diretamente em diagramas de classe da UML. Porém, M-Refactor não se preocupa em sincronizar outras representações da UML como a KDM-RE.

\begin{figure}[h]
	\centering
	% Requires \usepackage{graphicx}
	\caption{Visão geral da ferramenta M-Refactor.}
	\label{fig:m_refactor}
	\includegraphics[scale=0.6]{images/m_refactor}
	\fdireta{mohamed2011m}
\end{figure}

\begin{figure}[h]
	\centering
	% Requires \usepackage{graphicx}
	\caption{Arquitetura da Ferramenta MOOSE.}
	\label{fig:moose}
	\includegraphics[scale=0.32]{images/MOOSEArchitecture}
	\fdireta{ducasse2000moose}
\end{figure}


Outra ferramenta relacionada é a MOOSE~\cite{tichelaar2000meta, ducasse2000moose}. A arquitetura da ferramenta MOOSE é apresentada na Figura~\ref{fig:moose}. Como pode ser observado, MOOSE aceita como entrada diversos tipos de linguagem de programação (Smalltalk, Java, C++, etc) e, em seguida, MOOSE transforma tais linguagens em um metamodelo denominado FAMIX~\cite{tichelaar2000meta}. Após realizar essa transformação, refatorações podem ser aplicadas. De acordo com os autores, MOOSE permite a aplicação de refatorações em nível de modelo de forma independente utilizando o metamodelo FAMIX~\cite{tichelaar2000meta}. MOOSE utiliza a ferramenta Refactoring Browser~\cite{roberts1997refactoring} e, assim, permite que as refatorações aplicadas em instâncias do metamodelo FAMIX sejam automaticamente replicas no código-fonte Smalltalk, porém, essa função apenas funciona quando a linguagem de entrada é a Smalltalk. Da mesma forma que a KDM-RE, MOOSE, permite a aplicação de refatorações de forma independente de linguagem, além disso, KDM-RE e MOOSE permitem a aplicação de refatorações por meio de diagramas. As principais diferenças são: (\textit{i}) MOOSE permite identificar \textit{bad-smells} utilizando o metamodelo FAMIX, (\textit{ii}) MOOSE não implementa nenhum mecanismo de propagação/sincronização após a aplicação um conjunto de refatorações, isso se deve ao fato do metamodelo FAMIX ser sucinto e representar apenas construções do POO, (\textit{iii}) linguagem de transformação também não é utiliza na MOOSE, (\textit{iv}) restrições (pré- e pós-condições) também não são apoiadas na MOOSE e (\textit{v}) MOOSE utiliza o metamodelo FAMIX, um metamodelo proprietário.

\begin{figure}[h]
	\centering
	% Requires \usepackage{graphicx}
	\caption{Visão geral da Ferramenta RACOoN.}
	\label{fig:racoon}
	\includegraphics[scale=0.52]{images/fig_racoon}
	\fdireta{van2006model}
\end{figure}

RACOoN é um \textit{plug-in} desenvolvido por~\citeonline{van2006model}, ver Figura~\ref{fig:racoon}. Utilizando esse \textit{plug-in} regras de transformação em modelo podem ser implementadas, carregadas e em seguida serem executadas em diagramas da UML. RACOoN é uma ferramenta totalmente manual e permite que o usuário definida qual refatoração almeja aplicar, sendo de inteira responsabilidade do usuário definir a refatoração corretamente. Inconsistências encontradas durante a aplicação de múltiplas refatorações são apresentadas para o usuário juntamente com um conjunto de soluções. As principais semelhanças entre KDM-RE e RACOoN são: (\textit{i}) RACOoN também não identifica \textit{bad-smells}, (\textit{ii}) as refatorações podem ser aplicadas graficamente por meio de diagramas da UML, (\textit{iii}) linguagem de transformação também é utilizada para implementar as refatorações e (\textit{iv}) linguagem de restrição é utilizada para implementar as pré- e pós-condições. As principais diferenças são: (\textit{i}) nenhum mecanimos de propagação/sincronização é implementa na RACOoN e (\textit{ii}) RACOoN utiliza o metamodelo UML não KDM.


%RACOoN e KDM-RE são similares, porém, RACOoN foi definida para diagramas da UML e KDM-RE para o metamodelo KDM.

~\citeonline{Boger_2003} definiram uma extensão da ferramenta Refactoring Browser~\cite{roberts1997refactoring} e criaram a primeira ferramenta para aplicar refatorações em modelos UML, a qual é denominada Refactoring Browser for UML. Algumas semelhanças com a KDM-RE podem ser destacadas: (\textit{i}) Refactoring Browser for UML também não identifica \textit{bad-smells}, (\textit{ii}) as refatorações podem ser aplicadas graficamente por meio de diagramas da UML e (\textit{iii}) a linguagem de restrição OCL também é utilizada na Refactoring Browser for UML. As principais diferenças são: (\textit{i}) nenhum mecanimos de sincronização/propagação é implementado na Refactoring Browser for UML, (\textit{ii}) linguagem de transformação (ATL e/ou QVT) não é utilizada na Refactoring Browser for UML e (\textit{iii}) apenas KDM-RE utiliza o metamodelo KDM.


VisTra~\cite{vstolc2010visual} é uma ferramenta visual desenvolvida também como um \textit{plug-in} para o ambiente de desenvolvimento Eclipse, a qual aplica refatorações em diagramas de classes da UML. Essa ferramenta permite que o usuário defina regras de transformação graficamente e, em seguida, a ferramenta gera automaticamente restrições em OCL e regras de transformações em QVT. As principais semelhanças entre KDM-RE e VisTra podem ser resumidas como: (\textit{i}) VisTra também permite aplicar refatorações por meio de diagramas da UML, (\textit{ii}) a linguagem de transformação QVT, linguagem similar a ATL, é utilizada para implementar as refatorações e (\textit{iii}) a linguagem de restrição OCL também é utilizada para definir as pre- e pós-condições. As diferenças entre VisTra e KDM-RE são: (\textit{i}) VisTra pode ser utilizada para identificar \textit{bad-smells}, (\textit{ii}) nenhum mecanismo de propagação/sincronização é implementado na VisTra e (\textit{iii}) VisTra não utiliza o metamodelo KDM.


NEPTUNE~\cite{millan2009ocl} é uma ferramenta que permite verificar e transformar instâncias do metamodelo UML. Essa ferramenta utiliza uma extensão da OCL denominada pOCL para automatizar a detecção de \textit{bad-smells}. Em seguida, NEPTUNE sugere um conjunto de refatorações para reparar os \textit{bad-smells} identificados. As principais semelhanças entre KDM-RE e NEPTUNE podem ser resumidas como: (\textit{i}) NEPTUNE também permite aplicar refatorações por meio de diagramas da UML, (\textit{ii}) a linguagem de transformação QVT, linguagem similar a ATL, é utilizada para implementar as refatorações e (\textit{iii}) a linguagem de restrição OCL também é utilizada para definir as pre- e pós-condições. As diferenças entre NEPTUNE e KDM-RE são: (\textit{i}) NEPTUNE pode ser utilizada para identificar \textit{bad-smells}, (\textit{ii}) nenhum mecanismo de propagação/sincronização é implementado na NEPTUNE e (\textit{iii}) NEPTUNE não utiliza o metamodelo KDM.

\begin{table}[h]
\centering
\caption{Comparação entre a KDM-RE e as ferramentas relacionadas.}
\label{tab:relatedWorks_consideracoes_KDM_re}
\begin{tabular}{| m{8.3cm} |c|c|c|c|c|c|}
\hline
\multicolumn{1}{|c|}{Ferramenta} & B-S & AG & S/P & LT & LR & KDM \\ \hline
KDM-RE  & \ding{55} & \ding{51}   & \ding{51}    & \ding{51}   & \ding{51} & \ding{51}\\ \hline

EMF Refactor~\cite{arendt2013tool}  & \ding{51} & \ding{51}   & \ding{55}    & \ding{55}   & \ding{51} & \ding{55}\\ \hline

Refactory~\cite{reimann2010role} & \ding{55} & \ding{51}   & \ding{55}    & \ding{51}   & \ding{51} & \ding{55}\\ \hline

M-Refactor~\cite{mohamed2011m}  & \ding{51} & \ding{51}   & \ding{55}    & \ding{51}   & \ding{51}& \ding{55}\\ \hline

MOOSE~\cite{ducasse2000moose}  & \ding{55} & \ding{51}   & \ding{55}    & \ding{55}   & \ding{55}& \ding{55}\\ \hline

RACOoN~\cite{van2006model}  & \ding{55} & \ding{51}   & \ding{55}    & \ding{51}   & \ding{51}& \ding{55}\\ \hline

Refactoring Browser for UML~\cite{Boger_2003} & \ding{55} & \ding{51}   & \ding{55}    & \ding{55}  & \ding{51}& \ding{55}\\ \hline

VisTra~\cite{vstolc2010visual} & \ding{51} & \ding{51}   & \ding{55}    & \ding{51}  & \ding{51}& \ding{55}\\ \hline

NEPTUNE~\cite{millan2009ocl} & \ding{51} & \ding{51}   & \ding{55}    & \ding{51}  & \ding{51}& \ding{55}\\ \hline

\end{tabular}
\end{table}

Na Tabela~\ref{tab:relatedWorks_consideracoes_KDM_re} é apresentada uma comparação entre a KDM-RE e as ferramentas relacionadas identificados. Alguns critérios foram definidos durante a comparação, tais critérios são: (\textit{i}) a ferramenta permite identificar \textit{bad-smells}, (\textit{ii}) a ferramenta permite aplicar refatorações graficamente, ou seja, por meio de diagramas, (\textit{iii}) a ferramenta permite a sincronização/propagação da instância do metamodelo após aplicar um conjunto de refatorações, (\textit{iv}) a ferramenta utiliza linguagens de transformação para executar a refatoração, (\textit{v}) a ferramenta utiliza linguagens de restrições para verificar pré- e pós-condições. A Tabela~\ref{tab:relatedWorks_consideracoes_KDM_re} contém sete colunas e algumas foram abreviadas: (\textit{i}) \aspas{B-S} significa \aspas{\textit{\textbf{B}ad-\textbf{S}mells}}, (\textit{ii}) \aspas{AG} significa \aspas{\textbf{A}plicação das refatorações \textbf{G}raficamente}, (\textit{iii}) \aspas{S/P} representa se a ferramenta realiza a \aspas{\textbf{S}incronização/\textbf{P}ropagação} após aplicar um conjunto de refatorações}, (\textit{iv}) \aspas{LT} representa se a ferramenta utiliza \aspas{\textbf{L}inguagens de \textbf{T}ransformações} para aplicar a refatoração, (\textit{v}) \aspas{LR} informa se a ferramenta utiliza \aspas{\textbf{L}inguagens de \textbf{R}estrições} para verificar as pré- e pós-condições e (\textit{vi}) \aspas{KDM} representa se a ferramenta aplica as refatorações no metamodelo KDM. O símbolo \aspas{\ding{51}} representa que o metamodelo define o critério e o símbolo \aspas{\ding{55}} representa que o metamodelo não define o critério.% O símbolo \aspas{+} representa o grau de facilidade de compreensão do metamodelo, sendo que \aspas{+++} representa maior facilidade de compreensão e \aspas{+} representa menor facilidade de compreensão.



Embora todas as ferramentas identificadas apliquem refatorações em nível de modelo, nenhuma realiza as refatorações utilizando o metamodelo KDM. Além disso, nenhuma ferramenta identificada fornece uma forma de reutilizar as refatorações. Dessa forma, acredita-se que a KDM-RE é uma contribuição para a área de pesquisa de refatoração em nível de modelo. Além disso, somente a KDM-RE utiliza o metamodelo KDM como base para as refatorações, o que significa que a KDM-RE é independente de plataforma e de linguagem de programação. Embora todas as ferramentas identificas utilizem o EMF e/ou UML, nenhuma utiliza o metamodelo KDM. Uma das vantagens de utilizar o KDM é que ele possui diversas metaclasses para representar níveis diferentes de um determinado sistema.

\section{Considerações Finais}\label{sec:consideracoes_final_kdm_re}


O presente capítulo foca a ferramenta KDM-RE, a qual foi implementada de modo a ser utilizada em conjunto com os demais recursos oferecidos pelo ambiente de desenvolvimento Eclipse IDE. Dessa forma, a KDM-RE foi desenvolvida utilizando os conceitos de \textit{plug-ins}. Três \textit{plug-ins} foram criados e organizados em módulos de acordo com a sua predominante funcionalidade. Os três principais módulos da KDM-RE são: (\textit{i}) módulo de refatoração, (\textit{ii}) módulo do SRM e (\textit{iii}) módulo de sincronização. 

O primeiro módulo é responsável por criar uma infraestrutura que auxilia o engenheiro de software a aplicar refatorações em nível de modelos no contexto do metamodelo KDM. O engenheiro de software pode aplicar as refatorações no metamodelo KDM por meio de duas interfaces; a primeira denominada \textit{model browser} permite que o engenheiro tenha uma visão de árvore da instância do metamodelo KDM e aplique refatorações. A segunda interface permite que o engenheiro de software aplique refatorações diretamente em diagramas de classe da UML. Embora o engenheiro de software utilize diagrama de classe da UML na segunda interface, as refatorações são aplicadas transparentemente na instância do metamodelo KDM; o diagrama de classe da UML é utilizado apenas para extrair metadados (nome da classe, nome do atributo, tipo do atributo, etc.) que são enviados como entrada para a refatoração pré-definida em ATL.

O segundo módulo é responsável por disponibilizar uma DSL para instanciar o metamodelo SRM apresentado no Capítulo~\ref{chapter:Toward_a_Refactoring_Metamodel_for_KDM}. Além disso, o módulo do SRM fornece uma forma de reutilizar e compartilhar refatorações por meio de instâncias do metamodelo SRM. Esse segundo módulo converte a sintaxe e a semântica da DSL em um arquivo XMI. Cada marcação desse arquivo XMI representa uma instância da metaclasse do metamodelo SRM. Para permitir o reúso e o compartilhamento de refatorações de forma consistente e abrangente, um repositório remoto foi criado, o qual é dedicado para executar solicitações RESTful. Instâncias do metamodelo SRM são enviadas e recebidas por meio da API RESTful. JPA e o banco de dados MySQL foram utilizados para realizar as persistências das instâncias do metamodelo SRM.

Finalmente, o terceiro módulo é responsável por implementar uma forma de manter a instância do metamodelo KDM sincronizada e consistente. Esse módulo visa propagar e sincronizar uma determinada instância do metamodelo KDM após a aplicação de refatorações. Essas propagações são realizadas com base em regras pré-definidas e apresentadas no Apêndice~\ref{apendice:regras_propagacao}. Como o metamodelo KDM possui um conjunto de pacotes para representar diferentes artefatos, é importante manter a instância e todos os outros artefatos sincronizados após a aplicação de refatorações. No Capítulo~\ref{chapter:avaliacao}, uma avaliação é executada para avaliar a abordagem apresentada nesta tese.

\chapter{Avaliação}\label{chapter:avaliacao}
\section{Considerações Iniciais}

Nos Capítulo X, Y, Z foram apresentadas soluções para auxiliar o engenheiro de modernização a aplicar, reutilizar e propagar refatorações no contexto da abordagem ADM e do metamodelo KDM. Além disso, no Capítulo X foi apresentado uma ferramenta, KDM-RE, que automatiza todo o processo de aplicação, reutilização e propagação de mudanças no contexto do metamodelo KDM, deixando somente sob responsabilidade do engenheiro de modernização a identificação de onde aplicar a refatoração. Desse modo, com o uso da ferramenta o engenheiro de modernização tem um ambiente totalmente integrado na IDE Eclipse, onde refatorações podem ser aplicadas e reutilizadas sem se preocupar com a propagação para outras visões/artefatos representados em um determinada instância do metamodelo KDM. 

Com o intuito de verificar se a ferramenta KDM-RE realmente proporciona tais vantagens (facilidade e eficiência) na prática, foi-se realizado um experimento ao longo deste projeto de doutorado. Esse experimento foi planejado e executado seguindo a abordagem definida por Wohlin et\change{mudar aqui}, que é composta por três principais fases: (\textit{i}) definição e planejamento, onde são especificados o contexto, as hipóteses, as variáveis, os participantes, os instrumentos e o modelo do experimento; (\textit{ii}) operação, onde ocorre a preparação e a execução do experimento com os participantes; e (\textit{iii}) análise dos dados, onde os dados coletados durante o experimento são agrupados e analisados por meio de técnicas estatísticas.

Na Seção~\ref{sec:teste_estatisticos} há uma descrição dos testes estatísticos aplicados no experimento realizado durante este projeto de doutorado. Na Seção~\ref{sec:experimento} é descrito o experimento que a ferramenta KDM-RE. Em seguida, na Seção~\ref{sec:consideracoes_finais_experimento} são comentadas as considerações finais desse capítulo.

\section{Testes Estatísticos}\label{sec:teste_estatisticos}

Wohlin et\change{mudar aqui} declaram que os experimentos têm como objetivo responder questões a respeito de um objeto de estudo. Para cada questão são definidas uma ou mais métricas a partir da qual os dados são coletados e um também um conjunto de hipóteses sobre os possíveis resultados são definidos. Esse conjunto de hipóteses é formado por:

\begin{itemize}
\item Uma \textbf{hipótese nula}: Considera que não há uma diferença significativa entre os dados obtidos ao se aplicarem os diferentes tratamentos sobre o objeto de estudo;
\item Uma ou mais \textbf{hipóteses alternativas}: Consideradam os demais possíveis resultados. Por exemplo, a hipótese alternativa 1 considera que a ferramente X é mais eficiente do que a ferramenta Y, enquanto que a hipótese alternativa 2 considerada que a ferramenta X é menos eficiente do que a ferramenta Y.
\end{itemize}

Alguns cálculos podem ser realizados sobre os dados coletados para se obter um resultado e uma das hipóteses alternativas ser aceita. Por exemplo, cálculos de média e de porcentagem podem indicar que, em determinado experimento, o tempo gasto pelos participantes para aplicar refatorações no metamodelo KDM foi menor quando a ferramenta X foi utilizada do que quando a ferramenta Y foi utiliza. Contudo, ainda se faz necessário testes estatísticos para comprovar que esse resultado é significativo, refutando assim, a hipótese nula.

No experimento apresentado nesse capítulo foram aplicado os testes estatísticos \textbf{Shapiro-Wilk} e \textbf{Paried T-Test}. Os testes estatísticos foram aplicados com o apoio da ferramenta R\change{referencia r}. Esse testes são brevemente explicados a seguir.

\subsection{Shapiro-Wilk}

O teste \textbf{Shapiro-Wilk} é aplicado para verificar se um conjunto de dados segue, ou não, uma distribuição normal, que possui graficamente o formato de um sino simétrico em relação à sua média, ver Figura~\ref{fig:shapiro_wilk}. Se o p-valor (resultado) do teste \textbf{Shapiro-Wilk} sobre um conjunto de dados for menor do que 0.05, significa que a chance dos dados seguirem uma distribuição normal é menor do que 5\%. Quando esse resultado ocorre, considera-se, estatisticamente, que os dados não seguem uma distribuição normal(wholn ref\change{colocar ref}). 

\begin{figure}[h]
	\centering
	% Requires \usepackage{graphicx}
	\caption{Representação gráfica de uma distribuição normal de dados.}
	\label{fig:shapiro_wilk}
	\includegraphics[scale=0.9]{images/distribuicao_normal}
	\fautor
\end{figure}

Em geral, as ferramentas de estatísticas utilizam um gráfico de propabilidade, também conhecido como \textit{Q-Q Plot} para representar graficamente a distribuição de um conjunto de dados. Nesse gráfico, quando os dados se posicionam ao redor da linha diagonal, considera-se que os dados seguem uma distribuição normal. Na Figura~\ref{fig:qq_plot_exemple} são apresentados dois exemplos de gráficos de probabilidade, um com distribuição normal e outro não normal.

\begin{figure}[h]
	\centering
	% Requires \usepackage{graphicx}
	\caption{Exemplos de gráficos de probabilidade.}
	\label{fig:qq_plot_exemple}
	\includegraphics[scale=0.6]{images/qq_plot_exemplo}
	\fautor
\end{figure}



% ---
% Finaliza a parte no bookmark do PDF, para que se inicie o bookmark na raiz
% ---
\bookmarksetup{startatroot}% 
% ---

% ----------------------------------------------------------
% ELEMENTOS PÓS-TEXTUAIS
% ----------------------------------------------------------
\postextual

% ----------------------------------------------------------
% Referências bibliográficas
% ----------------------------------------------------------
\bibliography{references}

% ---------------------------------------------------------------------
% GLOSSÁRIO
% ---------------------------------------------------------------------

% Arquivo que contém as definições que vão aparecer no glossário
\input{tex/glossario}
% Comando para incluir todas as definições do arquivo glossario.tex
\glsaddall
% Impressão do glossário
\printglossaries

% ----------------------------------------------------------
% Apêndices
% ----------------------------------------------------------

% ---
% Inicia os apêndices
% ---
\begin{apendicesenv}
    
    \chapter{Teste}
    \begin{center}
        \fbox{\includegraphics[page=1,scale=0.45]{tex/appendix/formulario_experimento2_1.pdf}}
    \end{center}
    \newpage


    \chapter{Formulários dos Experimentos}
    \label{chapter:formulario_experimento2}
    \input{tex/appendix/formulario_caracterizacao_experimento}
    
    \chapter{Documento básico usando a classe \textit{icmc}}
    \label{chapter:documento-basico}
    \input{tex/appendix/documento-basico}
    
    \chapter{Configuração do programa JabRef}
    \label{chapter:configuracao-jabref}
    \input{tex/appendix/configuracao-jabref}

\end{apendicesenv}
% ---


% ----------------------------------------------------------
% Anexos
% ----------------------------------------------------------

% ---
% Inicia os anexos
% ---
\begin{anexosenv}

    \chapter{Páginas interessantes na Internet} 
    \label{chapter:paginas-interessantes}
    \input{tex/annex/paginas-interessantes}

\end{anexosenv}
% ---

\end{document}